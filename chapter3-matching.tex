% +--- CHAPTER --- 
\begin{comment}
    ./texfix.py --fpaths chapter3-matching.tex --outline --asmarkdown --numlines=999 -w
    ./texfix.py --fpaths chapter3-matching.tex --outline --asmarkdown --numlines=999 -w
    ./texfix.py --fpaths chapter3-matching.tex --reformat 
    # http://jaxedit.com/mark/
\end{comment}



\chapter{Identification using a single query}\label{chap:matching}

    This chapter addresses the core problem of animal identification in
      a static context.
    Given, a \emph{single} annotation depicting an unknown animal and a
      \emph{static} database of previously identified annotations, the
      task is to determine the identity of the unknown animal.
    An \glossterm{annotation} is a rectangular region  of interest
      around a specific animal within an image.
    Each known annotation is associated with a \name{} label denoting
      its individual identity.
    A \glossterm{\name} refers to a group of annotations known to be
      the same individual.
    The identification process assigns a \name{} label to an unknown
      annotation either as
    (1) the \name{} label of a matched database annotation or
    (2) a new \name{} label if no matches are found.

    A static context is chosen for this chapter in order to introduce
      and determine the effectiveness of the algorithm that identifies a
      query by ranking the \names{} in the database.
    In the static context, only a single query annotation is used to
      perform identification, all \name{} labels in the database are
      assumed to be correct, and annotations are not added to or removed
      from the database.
    This is done to determine what properties of query annotations,
      configurations of the database, and parameters of the
      identification algorithm have the highest impact on identification
      accuracy.
    In~\cref{chap:application}, the identification algorithm is
      extended for use in a dynamic context.
    In the dynamic context, annotations are added and removed from the
      database, multiple annotations are used to perform identification,
      and the database is assumed to contain errors.

    The identification algorithm is based on the feature
      correspondences between a query annotation and a set of database
      annotations.
    In each annotation a set of patch-based features is detected at
      keypoint locations.
    Then the visual appearance of each patch is described using
      SIFT~\cite{lowe_distinctive_2004}.
    A nearest neighbor algorithm establishes a set of feature
      correspondences between query and annotations in database.
    A scoring mechanism based on Local \Naive{} Bayes Nearest Neighbor
      (LNBNN)~\cite{mccann_local_2012} produces a score for each feature
      correspondence.
    These scores are then aggregated into a single score for each
      \name{} in the database, producing a ranked list of \names{}.
    Identification is performed by applying a classifier (decision
      algorithm) to the scores in this ranked list.
    If the top ranked \name{} has a ``high'' score it is likely to be
      the same individual depicted in the query annotation.
    If the top ranked \name{} has a ``low'' score it is likely that the
      query individual is not depicted in any database annotation.
    An example ranked list returned by the algorithm is illustrated
      in~\cref{fig:rankedmatches}.
    In the baseline algorithm this identification decision is left to a
      user.

    The outline of this chapter is as follows:
    \Cref{sec:annotrepr} discusses the initial processing of an
      annotation which involves image normalization, feature extraction,
      and feature weighting.
    \Cref{sec:baselineranking,sec:sver} describes the baseline matching
      and scoring algorithm.
    The first of these sections focuses on establishing feature
      correspondences, and the second focuses on verifying the
      correspondences.
    \Cref{sec:experiments} provides an experimental evaluation of the
      single image identification algorithm.
    These experiments inform the direction we take in our proposal to
      extend this algorithm.
    \Cref{sec:staticsum} summarizes this chapter.

    \rankedmatches{}

    \section{Annotation representation}\label{sec:annotrepr}
  
    For each annotation in the database we
    (1) normalize the image geometry and intensity,
    (2) compute features,
    (3) and weight the features.
    % Chip Extract
    Image normalization rotates, crops, and resizes an annotation from
      its image.
    This helps to remove background clutter and roughly align the
      annotations in pose and scale.
    The extracted and normalized region is referred to as a
      \glossterm{chip}.
    % Feat Detect
    Then, a set of features ---  a keypoint and descriptor pair --- is
      computed.
    Keypoints are detected at multiple locations and scales within the
      chip, and a texture based descriptor vector is extracted at each
      keypoint.
    % Featweight
    Finally, each feature is assigned a probabilistic weight using a
      foregroundness classifier.
    This helps remove the influence of background features.

    \subsection{Chip extraction}

        % Bounding box + orientation
        Each annotation has a bounding box and an orientation specified
          in a previous detection step.
        For zebras and giraffes, the orientation of the chip is chosen
          such that the top of the bounding box is roughly parallel to
          the back of the animal.

        A chip is a extracted by jointly rotating, scaling, and
          cropping an annotation's parent image using Lanczos
          resampling~\cite{lanczos_applied_1988}.
        The scaling resizes the image such that the cropped chip has
          approximately $450^2$ pixels and it maintains the aspect ratio
          of the bounding box.
        If specified in the pipeline configuration, adaptive histogram
          equalization~\cite{pizer_adaptive_1987} is applied to the chip,
          however this is not used in the experimental evaluation
          presented later in this chapter.

    \subsection{Keypoint detection and description}

        Keypoints are detected within each annotation's chip using a
          modified implementation of the Hessian detector described
          in~\cite{perdoch_efficient_2009} and reviewed
          in~\cref{sec:featuredetect}.
        This produces a set of elliptical features localized in space,
          scale, shape, and orientation.
        Each keypoint is described using the
          SIFT~\cite{lowe_distinctive_2004} descriptor that was reviewed
          in~\cref{sec:featuredescribe}.
        The resulting keypoint-descriptor pairs are an annotation's
          features.
        Further details about the keypoint structure are given
          in~\cref{sec:kpstructure}.

        We choose a baseline feature detection algorithm that produces
          affine invariant keypoints with the gravity vector.
        Affine invariant (\ie{} shape adapted) keypoints detect
          elliptical patches instead of circular ones.
        We choose affine invariant keypoints because the animals we
          identify will be seen from many different viewpoints.
        Because all chips have been rotated into an upright position,
          we assign all keypoints a constant orientation --- this is the
          gravity vector assumption~\cite{perdoch_efficient_2009}.
        However, these baseline settings may not be appropriate for all
          species.

        It is important to select the appropriate level of invariance
          for each species we identify.
        % Shape 
        Our experiments in~\cref{sub:exptinvar} vary several parameters
          related to invariance in keypoint detection.
        To determine if affine invariance is appropriate for animal
          identification we experiment with both circular and elliptical
          keypoints.
        % Orientation 
        We also experiment with different levels of orientation
          invariance.
        The gravity vector assumption holds in the case of rigid
          non-poseable objects (\eg{} buildings), if the image is
          upright.
        Clearly, for highly poseable animals, this assumption is more
          questionable.
        However, full rotation invariance (using dominant gradient
          orientations) has intuitive problems.
        Patterns (like ``V'' and ``$\Lambda$'') that might contribute
          distinguishing evidence that two annotations match, would
          always appear identical under full rotation invariance.
        Ideally orientation selection would be made based on the pose
          of the animal.

        We introduce a simple orientation heuristic to help match keypoints
          from the same animal in the presence of small pose variations.
        Instead of extracting a single keypoint in the direction of gravity or
          multiple keypoints in the directions of the dominant gradient
          orientation we extract 3 descriptors at every keypoint:
        one in the direction of gravity, and the other two offset at
          $\pm15\degrees$.
        This provides a middle ground between rotation invariance and the
          gravity vector.
        Using this heuristic, it will be more likely to extract similar
          descriptors from two annotations of the same animal seen from slightly
          different poses.

        %In our experiments we will investigate replacing the SIFT descriptor
        %  features extracted from a deep convolutional neural network.

    \subsection{Feature weighting}
     
        In animal identification, there will often be many annotations
          containing the same background.
        Photographers may take many photos in a single place and camera
          traps will contribute many images with the same background.
        Without accurate background masking, regions of an annotation
          from different images containing the same background may
          strongly match and outscore matches to correct individuals.
        An example illustrating two different individuals seen in front
          of the same distinctive background is shown
          in~\cref{fig:SceneryMatch}.
        To account for this, each feature is given a weight based on
          its probability of belonging to the foreground --- its
          ``foregroundness''.
        This weight is used indicate the importance of a feature in
          scoring and spatial verification.

        Foregroundness is derived from a species detection algorithm
          developed by Jason Parham~\cite{parham_photographic_2015}.
        The input to the species detection algorithm is the
          annotation's chip, and the output is an intensity image.
        Each pixel in the intensity image represents the likelihood
          that it is part of a foreground object.

        A single feature's foregroundness weight is computed for each
          keypoint in an annotation as follows:
        The region around the keypoint in the intensity image is warped
          into a normalized reference frame.
        Each pixel in the normalized intensity patch is weighted using
          a Gaussian falloff based on the pixel's distance from the
          center of the patch.
        The sum of these weighted intensities is the feature's
          foregroundness weight.
        The steps of feature weight computation are illustrated
          in~\cref{fig:genfeatweight}.

        \SceneryMatch{}

        \genfeatweight{}

    \subsection{Keypoint structure overview}\label{sec:kpstructure}
        %Before we discuss the computation of the we review the structure of
        %  a keypoint.
        The keypoint of a feature is represented as:
        $\kp\tighteq(\pt, \vmat, \ori)$, %
        The vector $\pt\tighteq\ptcolvec$ is the feature's
          $xy$-location.
        The scalar $\theta$ is the keypoint orientation.
        The lower triangular matrix $\vmat\tighteq\VMatII$ encodes the
          keypoint's shape and scale.
        This matrix skews and scales a keypoint's elliptical shape into
          a unit circle.
        A keypoint is circular when $a\tighteq{}d$ and $c\tighteq0$.
        %If $c\tighteq0$, there is no skew and if $a\tighteq{}d$ the keypoint
        %  is circular.
        The keypoint scale is related to the determinant of this matrix
          and can be extracted as: %
        $\sigma = \frac{1}{\sqrt{\detfn{\vmat}}} =
          \frac{1}{\sqrt{ad}}$.
        All of this information can be encoded in a single affine
          matrix.

        \paragraph{Encoding keypoint parameters in an affine matrix}
        It will be useful to construct two transformations that encode
          all keypoint information in a single matrix.
        The first, $\rvmat$, maps a keypoint in an annotation into a
          normalized reference frame --- the unit circle.
        The second transformation, $\inv{\rvmat}$ is the inverse, which
          warps the normalized reference frame back onto the keypoint.
        To construct $\rvmat$, the keypoint is centered at the origin
          $(0, 0)$ using translation matrix, $\mat{T}$.
        Then $\vmat$ is used to skew and scale the keypoint into a unit
          circle.
        Finally, the keypoint's orientation is normalized by rotating
          $-\theta$ radians using a rotation matrix $\mat{R}$.
        \begin{equation}\label{eqn:RVTConstruct}
          %\rvmat=\paren{\invrotMatIII{\paren{-\ori}} \VMatIII \transMATIII{-x}{-y}}
            \rvmat=\mat{R} \vmat \mat{T} = \rotBigMatIII{\paren{-\ori}} \VBigMatIII \transBigMatIII{-x}{-y}
        \end{equation}
        The construction of $\inv{\rvmat}$ is performed similarly.
        % see vtool.keypoint.get_invVR_mats_oris
        \begin{equation}\label{eqn:invTVRConstruct}
             \inv{\rvmat} = \inv{\mat{T}} \inv{\vmat} \inv{\mat{R}} = 
             \transBigMatIII{x}{y}
             \BIGMAT{
                \frac{1}{a}     & 0               & 0\\
                -\frac{c}{a d}  & \frac{1}{d}     & 0\\
                0               & 0               & 1
                }
             \rotBigMatIII{\paren{\ori}}
        \end{equation}

        \paragraph{Extracting keypoint parameters from an affine matrix}
        During the spatial verification step, described
          in~\cref{sec:sver}, keypoints are warped from one image into
          the space of another.
        It will be useful to extract the keypoint parameters from an
          arbitrary keypoint matrix.
        This will allows us to directly compare properties of
          corresponding right side of~\cref{eqn:invTVRConstruct}.
        Given an arbitrary affine matrix $\inv{\rvmat}$ representing
          keypoint $\kp$, we show how the individual parameters $(\pt,
          \scale, \ori)$ can be extracted.
        First consider the components of $\inv{\rvmat}$ by simplifying
          the right side of~\cref{eqn:invTVRConstruct}.
        \begin{equation}\label{eqn:ArbInvRVTMat}
            \inv{\rvmat} = 
            \BIGMAT{
            e & f & x\\
            g & h & y\\
            0 & 0 & 1
            } = 
            %\BIGMAT{
            %\frac{1}{a} \cos{(-\theta )}                                   & \frac{1}{a} \sin{(-\theta )}                                  & x\\
            %-\frac{1}{d} \sin{(-\theta )} - \frac{c}{a d} \cos{(-\theta )} & \frac{1}{d} \cos{(-\theta )} + \frac{c}{a d} \sin{(-\theta )} & y\\
            %0                                                              & 0                                                             & 1
            %}
            \BIGMAT{
            \frac{1}{a} \cos{(\theta )}                                 & -\frac{1}{a} \sin{(\theta )}                                & x\\
            \frac{1}{d} \sin{(\theta )} - \frac{c}{a d} \cos{(\theta )} & \frac{1}{d} \cos{(\theta )} + \frac{c}{a d} \sin{(\theta )} & y\\
            0                                                           & 0                                                           & 1
            }
        \end{equation}
        %%---------
        The position, scale, and orientation can be extract from an
          arbitrary affine keypoint shape matrix $\invvrmat$ as follows:
        \begin{equation}\label{eqn:affinewarp}
            \begin{aligned}
                \pt     &= \VEC{x\\y} \\
                \scale  &= \sqrt{\detfn{\invvrmat}}\\
                \ori    &= \modfn{\paren{-\atantwo{f, e}}}{\TAU}
            \end{aligned}
        \end{equation}

    %\newcommand{\annotscoreop}{\opname{annot\_score}}
%\newcommand{\namescoreop}{\opname{name\_score}}
\newcommand{\annotscoreop}{\opname{K_{\tt annot}}}
\newcommand{\amechscoreop}{\opname{K_{\csum}}}
\newcommand{\fmechscoreop}{\opname{K_{\nsum}}}

\section{Matching against a database of individual animals}\label{sec:baselineranking}

    To identify a query annotation, it is matched against a database of known \names{}.
    \Aan{\name{}} is a set of annotations known to depict the same animal.
    The basic matching pipeline can be summarized in \three{} steps:
    establish feature correspondences \rpipe{} %
    score feature correspondences \rpipe{} %
    aggregate feature correspondence scores across the \names{}.
    Correspondences between a query annotation's features and \emph{all} database annotation features are
      established using an approximate nearest neighbor algorithm.
    This step also establishes a normalizing feature which is used to measure the distinctiveness of a query
      feature.
    Each feature correspondence is scored based on the feature weights established in the previous section and a
      measure of the distinctiveness of the query feature.
    The feature correspondence scores are then aggregated into a \glossterm{\namescore{}} for each \name{} in the
      database.
    The \namescores{} induce a ranking on \names{} in the database where database \names{} with higher ranks are
      more likely to be correct matches.

   \subsection{Establishing initial feature correspondence}\label{sub:featmatch}

        \paragraph{offline indexing}
            Before feature correspondences can be established, an offline algorithm indexes descriptors from all
              database annotations for fast approximate nearest neighbor search.
            All database descriptor vectors are stacked into a single database array of vectors, %
            $\AnyDB$, % should this be removed?
              %
            and these descriptors are indexed by an inverted file.
            The inverted file maps each descriptor in the stacked array back to its original annotation and
              feature.
            This database array is indexed for nearest neighbor search using a forest of
              kd-trees~\cite{silpa_anan_optimised_2008} using the FLANN library~\cite{muja_fast_2009}, which were
              reviewed in~\cref{sec:ann}.
            This allows for the efficient implementation of a \codeobj{neighbor index function}  %
            $\NN(\AnyDB, \desc, \K)$  %
            %$\NN(\desc, \K)$  %
            that returns the indices in $\AnyDB$ of the $\K$ approximate nearest neighbors of a query feature's
              descriptor $\desc$.
            %that returns the database indices of the $\K$ approximate
            %  nearest neighbors of a query feature's descriptor
            %$\desc$.

        \paragraph{Approximate nearest neighbor search}

            Matching begins by establishing multiple feature correspondences between each query feature and
              several visually similar database features.
            For each query descriptor vector $\desc_i \in \X$ the $\K + \Knorm$ approximate nearest neighbors are
              found using the \coderef{neighbor index function}.
            These neighbors sorted by ascending distance are:
            \begin{equation}
                \NN(\AnyDB, \desc_i, \K + \Knorm) \eqv \dotarrIII{j}{\K}{\K + \Knorm}
                %\NN(\desc_i, \K + \Knorm) \eqv \dotarrIII{j}{\K}{\K + \Knorm}
            \end{equation}
            The $\K$ nearest neighbors, $\dotsubarr{\desc}{{j_1}}{{j_\K}}$, are the initial feature
              correspondences to the $i$\th{} query feature.
            The remaining $\Knorm$ neighbors, $\dotsubarr{\desc}{{j_{\K + 1}}}{j_{\Knorm}}$, are candidate
              normalizers for use in LNBNN scoring.

        \paragraph{Normalizer selection}
            A single descriptor $\descnorm_{i}$ is selected from the $\Knorm{}$ candidate normalizers and used in
              computing the LNBNN score for all (up to $\K$) of the $i$\th{} query descriptor's correspondences in
              the database.
            The purpose of a normalizing descriptor is to estimate the local density of descriptor space, which
              can be interpreted as a measure of the query descriptor's distinctiveness \wrt{} the database.
            The normalizing descriptor is chosen as the most visually similar descriptor to the query that is not
              a correct match.
            In other words, the query descriptor's normalizer should be from an individual different from the
              query.
            The intuition is there will not be any features in the database that are close to distinctive
              features in the query except for the features that belong to the correct match.

            The selection process described in the original formulation of LNBNN is to simply choose the $\K +
              1$\th{} nearest neighbor, which amounts to setting $\Knorm=1$.
            The authors of LNBNN find that there is no benefit to using a higher value of
              $\Knorm$~\cite{mccann_local_2012}.
            However, this does not account for the case when the $\K + 1$\th{} nearest neighbor belongs to the
              same class as one of the nearest $\K$ neighbors.
            Therefore, we employ a slightly different selection process.
            To motivate our selection process, consider the case when there are more than $\K$ images of the same
              individual from the same viewpoint in the database and a distinctive feature from a new annotation of
              that individual is being scored.
            In this case $\K$ correspondences will be correctly established a distinctive the query feature and
              $\K$ database features.
            However, if the normalizer is chosen as the $\K + 1$ neighbor, then these correspondences will be
              inappropriately downweighted.
              
            % probably need to reword.
            Consider the example in~\cref{fig:knorm}.
            In this case there are two examples \Cref{sub:knorma,sub:knormb} of the query images in the database.
            The figure shows the case where $\K=3$ and $\Knorm=1$.
            Even though there is an incorrect match, the LNBNN scores of the correct matches are an order of
              magnitude higher than the score for the incorrect match.
            Now, consider the case where the number of correct matches in the database is greater than $\K$ by
              setting $\K=1$.
            In this case the normalizing descriptor is the ``same'' feature as the query feature and the nearest
              match drops from $0.066$ to $0.007$.

            \knorm{}

            To avoid this case, a normalizing feature is carefully chosen to reduce the possibility that it
              belongs to a potentially correct match.
            More formally, the normalizing descriptor is chosen to be the descriptor with the smallest distance
              to the query descriptor that is not from the same \name{} as any of the chosen correspondences.
            Let $\nid_j$ be the \name{} associated with the annotation containing descriptor $\desc_j$.
            Let %
            %$\multiset{N}_i \eqv \curly{\nid_j \quad \forall j \in \NN(\desc_i, \K + \Knorm)}$
            $\multiset{N}_i \eqv \curly{\nid_j \where j \in \NN(\AnyDB, \desc_i, \K + \Knorm)}$
              %\NN(\desc_i, \K + \Knorm)}$
              %
            be the set of \names{} matched by the $i$\th{} query feature.
            The descriptor that normalizes all matches of query descriptor $\desc_i$ is:
              \begin{equation}
                  \descnorm_{i} \eqv 
                  \argmin{\desc_j \in \dotsubarr{\desc}{{j_{\K + 1}}}{j_{\Knorm}}}
                  \elltwosqrd{\desc_j - \desc_i} \where \nid_j \notin \multiset{N}_i
              \end{equation}

    \subsection{Feature correspondence scoring}
        Each feature correspondence is given a score representing how likely it is to be a correct match.
        While the L2-distance between query and database descriptors is useful ranking feature correspondences
          based on visual similarity, the distinctiveness of the match is more useful for ranking the query
          annotation's similarity to a database annotation~\cite{lowe_distinctive_2004,
          arandjelovic_dislocation_2015, mccann_local_2012}.
        However, highly distinctive matches from other objects --- like background matches --- do not provide
          relevant information about a query annotation's identity and should not contribute to the final score.
        Therefore, each feature correspondence is scored using a mechanism that combines both distinctiveness and
          likelihood that the object belongs to the foreground.
        For each feature correspondence $m = (i, j)$ with query descriptor $\desc_i$ and matching database
          descriptor $\desc_j$, several scores are computed which are then combined into a single feature
          correspondence score $s_{i,j}$.

        \paragraph{LNBNN score}\label{sec:lnbnnscore}

            Using the normalizing feature, $\descnorm_{i}$, LNBNN compares a query feature's similarity to the
              match and query feature's similarity to the normalizer.
            This serves as an estimate of local feature density and measures the distinctiveness of the feature
              correspondence.
            A match is distinctive when the query-to-match distance is much smaller than the query-to-normalizer
              distance, \ie{} the local density of descriptor space around the query is sparse.
            The LNBNN score of a feature match is computed as follows:
            \begin{equation}\label{eqn:lnbnn}
                \fs_{\LNBNN} \eqv \frac{\elltwo{\desc_i - \descnorm_{i}} - \elltwo{\desc_i - \desc_j}}{Z}
            \end{equation}
            All descriptors used in this calculation are L2-normalized to unit length --- \ie{} to sit on the
              surface of a unit hypersphere.
            The $Z$ term normalizes the score to ensure that it is in the range $\rangeinin{0, 1}$.
            If descriptor vectors have only non-negative components (as in the case of
              SIFT~\cite{lowe_distinctive_2004}) then the maximum distance between any two L2-normalized
              descriptors is $Z\tighteq\sqrt{2}$.
            If descriptors vectors have negative components (like those that might extracted from a deep
              convolutional neural network~\cite{zagoruyko_learning_2015}) then the maximum distance between is
              $Z\tighteq2$.

        \paragraph{Foregroundness score}
            To reduce the influence of background matches, each feature correspondence is assigned a score based
              on the foregroundness of both the query and database features.
            The geometric mean of the foregroundness of query feature, $w_i$, and database feature, $w_j$, drives
              the score to $0$ if either is certain to be background.
            % show python -m ibeis --db testdb3 --query 325 -y
            \begin{equation}
                \fs_{{\tt fg}} \eqv \sqrt{w_i w_j}
            \end{equation}

        \paragraph{Final feature correspondence score}
            The final score of the correspondence $(i, j)$ captures both the distinctiveness of the match as well
              as the likelihood that the match is part of the foreground.
              \begin{equation}\label{eqn:featscore}
                  \fs_{i,j} \eqv \fs_{{\tt fg}} \fs_{\LNBNN} 
              \end{equation}

    \subsection{Feature score aggregation}\label{subsec:namescore}

        So far, each feature in a query annotation has been matched to several features in the database and a
          score has been assigned to each of these correspondences based on its distinctiveness and foregroundness.
        The next step in the identification process is to aggregate the scores from these patch-based
          correspondences into a single \glossterm{\namescore} for each \name{} in the database.
        Note that this \name-based definition of scoring is a key difference between animal identification and
          image retrieval, where a score is assigned to each image in the database.
        In animal identification the analogous concept is an \glossterm{\annotscore} --- a score assigned to each
          annotation in the database~\cite{philbin_object_2007}.
        This distinction between a score from a query annotation to a database annotation is important because
          the goal of the application is to classify a new query annotation as either a known \name{} or as a new
          \name{}, not to determine which annotations are most similar.

        This subsection presents two mechanisms to compute \namescores{}.
        The first mechanism is \csumprefix{} and computes a \namescore{} in two steps.
        This mechanism aggregates feature correspondences scores into an \annotscore{} for each annotation in the
          database.
        Then the \annotscores{} are aggregated into a score for each \name{} in the database.
        The second mechanism is \nsumprefix{}.
        This mechanism aggregates feature correspondences scores matching multiple database annotations directly
          into a \namescore{}.
        These mechanisms are respectively similar to the image-to-image distance and the image-to-class distance
          described in~\cite{boiman_defense_2008}.

        \paragraph{The set of all feature correspondences}
        All scoring mechanisms presented in this subsection are based on aggregating scores from features
          correspondences.
        The set of all feature correspondences for a query annotation $\X$ is expressed as:
        \begin{equation}
            %\Matches \eqv \{(i, j) \where \desc_i \in \X \AND j \in \NN(\desc_i, \K)\}
            \Matches \eqv \{(i, j) \where \desc_i \in \X \AND j \in \NN(\AnyDB, \desc_i, \K)\}
        \end{equation}

        % FIXME: SAME AS IMAGE-TO-IMAGE
        \paragraph{Annotation scoring}
            An \annotscore{} is a measure of similarity between two annotations.
            An \annotscore{} between a query annotation and a database annotation is defined as the sum of the
              feature correspondence scores matching to the features from that database annotation.
            %However, the \annotscore{} will allow us to compare our
            %  techniques against other instance recognition techniques.
            Let $\daid_j$ be the database annotation containing feature $j$.
            Let
            %
            $\Matches_{\daid} \eqv \curly{(i, j) \in \Matches \where \daid_j = \daid}$
            %
            denote all of the correspondences to a particular database annotation.
            The \annotscore{} between the query annotation $\qaid$ and database annotation $\daid$ is:
            \begin{equation}
                \annotscoreop(\qaid, \daid) \eqv \sum_{(i, j) \in \Matches_{\daid}} \fs_{i, j}
            \end{equation}

        % FIXME: SAME AS IMAGE-TO-CLASS
        \paragraph{Name scoring (1) --- \csumprefix{}} %

            The \cscoring{} mechanism aggregates \annotscores{} into \namescores{} by taking as the score highest
              scoring annotation for each \name{}.
            In our experiments we refer to this version of \namescoring{} as \csum{}.
            Let $\nid$ be the set of database annotations with the same \name{}.
            The \cscore{} between a query annotation and a database \name{} is the maximum over all annotations
              scores in that \name{}:
            \begin{equation}
                \amechscoreop(\qaid, \nid) 
                \eqv
                \max_{\daid \in \nid}
                \paren{
                    \annotscoreop\paren{\qaid, \daid}
                }
                %\opname{annot\_score}(\qaid, \daid) \eqv \sum_{m_a \in \Matches_{\daid}} \fs_{i, j}
            \end{equation}

         \paragraph{Name scoring (2) --- \nsumprefix{}} %
            The \cscoring{} mechanism accounts for the fact that animals will be seen multiple times, but it does
              not take advantage of complementary information available when \aan{\name{}} has multiple
              annotations.
            The following aggregation mechanism combines scores on a feature level to correct for this.
            It allows each query feature at a specific location to vote for a given \name{} at most once.
            Thus, when a query feature (or multiple query features at the same location) corresponds to database
              features from multiple views of the same animal, only the best correspondence for that feature will
              contribute to the score.
            In our experiments we refer to this version of \namescoring{} as \nsum{}.

            The first step of computing \aan{\namescore{}} for a specific \name{} is grouping the feature
              correspondences.
            Two feature correspondences are in the same group if the query features have the same location and
              the database features belong to the same \name{}.
            The next step is to choose the highest scoring correspondence within each group.
            The sum of the chosen scores is the score for \aan{\name{}}.
            This procedure is illustrated in~\cref{fig:namematch}.

            \newcommand{\MatchesGroup}{\Matches^{G}}

            Formally, consider two feature correspondences $\mI\tighteq(\iI, \jI)$ and $\mII\tighteq(\iII,
              \jII)$.
            Let $\pt_{\iI}$ and $\pt_{\iII}$ be the $xy$-location of the query feature in a correspondence.
            Let $\nid_{\jI}$ and $\nid_{\jII}$ be the \name{} of the database annotations containing the matched
              features.
            The group that contains feature correspondence $\mI$ is defined as:
            \begin{equation}
                \MatchesGroup_{\mI} \eqv \curly{\mII \in \Matches  \where
                \paren{
                    \paren{\pt_{\iI} \eq \pt_{\iII}} \AND 
                    \paren{\nid_{\jI} \eq \nid_{\jII}}
                }
            }
            \end{equation}
            The correspondence with the highest score in each connected component is flagged as chosen.
            Ties are broken arbitrarily.
            \begin{equation}
                \ischosen(\mI) \eqv 
                \bincase{
                \paren{
                    \fs_{\mI} > \fs_{\mII} 
                    \quad \forall \mII  \in \MatchesGroup_{\mI}
                } 
                \OR
                \card{\MatchesGroup_{\mI}} \eq 1
                }
            \end{equation}

            Let $\Matches_{\nid} \eqv \{(i, j) \in \Matches \where
              \nid_j = \nid\}$ denote all of the correspondences to a particular
              \name{}.
            The \nscore{} of \aan{\name{}} is:
            \begin{equation}
                \fmechscoreop(\qaid, \nid) 
                \eqv 
                \sum_{m \in \Matches_{\nid}} \ischosen(m) \; \fs_m
            \end{equation}

            \namematch{}

    
\section{Spatial verification}\label{sec:sver}

    The basic matching algorithm treats each annotation as an orderless
      set of feature descriptors (with a small exception in name scoring,
      which has used a small amount of spatial information).
    This means that many of the initial feature correspondences will be
      spatially inconsistent.
    Spatial verification removes these spatially inconsistent feature
      correspondences.
    Determining which features are inconsistent is done by first
      estimating an affine transform between the two annotations.
    Then a projective transform is estimated using the inliers to the
      affine transform.
    Finally any correspondences that do not agree with the projective
      transform transformation are removed~\cite{fischler_random_1981,
      philbin_object_2007}.
    We have reviewed related work in spatial verification
      in~\cref{subsec:sverreview}.

    %In our problem, the animals are seen in a wide variety of poses,
    %  and projective transforms may not always be sufficient to capture
    %  all correctly corresponding features.
    %Yet, without strong spatial constraints on matching, many
    %  background features will be spatially verified.
    %For now, we proceed with standard techniques for spatial
    %  verification and evaluate if more sophisticated methods are needed.

    \subsection{Shortlist selection}
        Standard methods for spatial verification are defined on the
          feature correspondences between two annotations (images).
        Normally, a shortlist of the top ranked annotations are passed
          onto spatial verification.
        However, in our application we rank \names{}, which may have
          multiple annotations.
        In our baseline approach we simply apply spatial verification
          to the top $N_{\tt{nameSL}}=50$ \names{} and the top
          $N_{\tt{annotSL}}=6$ annotations within those \names{}.

    \subsection{Affine hypothesis estimation}
        Here, we will compute an affine transformation that will remove
          a majority of the spatially inconsistent feature
          correspondences.
        Instead of using random sampling of the feature correspondences
          as in the original RANSAC
          algorithm~\cite{hartley_multiple_2003}, we estimate affine
          hypotheses using a deterministic method similar
          to~\cite{philbin_object_2007, chum_homography_2012}.
        Given a set of matching features between annotation $\annotI$
          and $\annotII$, the shape, scale, orientation, and position of
          each pair of matching keypoints are used to estimate a
          hypothesis affine transformation.
        Each hypothesis transformation warps keypoints from annotation
          $\annotI$ into the space of $\annotII$.
        Inliers are estimated by using the error in position, scale,
          and orientation between each warped keypoint and its
          correspondence.
        The transformation with the most inliers determines the final
          affine transform.

        % vmat = V here
        % V maps from ellipse to u-circle
        \newcommand{\AffMat}{\mat{A}}
        \newcommand{\HypothSet}{\set{A}}
        \newcommand{\AffMatij}{\mat{A}_{i, j}}
        \newcommand{\HypothAffMat}{\hat{\mat{A}}}

        \subsubsection{Enumeration of affine hypotheses}
            %The deterministic set of hypothesis transformations mapping from
            %  query annotation $\annotI$ to database annotation $\annotII$ is
            %  computed for each feature correspondence in a match from  to an
            %  annotation.
            Let $\Matches_{\annotII}$ be the set of all correspondences
              between features from query annotation $\annotI$ to
              database annotation $\annotII$.
            For each feature correspondence $(i, j) \in
              \Matches_{\annotII}$, we construct a hypothesis
              transformation, $\AffMatij$ using the matrices $\rvmat_{i}$
              and $\inv{\rvmat_{j}}$, which where defined
              in~\cref{eqn:RVTConstruct} and~\cref{eqn:invTVRConstruct}.
            The first transformation $\rvmat_{i}$, warps points from
              $\annotI$-space into a normalized reference frame.
            Then the second transformation, $\inv{\rvmat_{j}}$, warps
              points in the normalized reference frame into
              $\annotII$-space.
            Formally, the hypothesis transformation is defined as
              $\AffMatij \eqv \inv{\rvmat_{j}}\rvmat_{i}$, and the set of
              hypothesis transformations is:
            \begin{equation}
                \HypothSet \eqv \curly{ \AffMatij \where (i, j) \in \Matches_{\annotII} }
            \end{equation}

        \subsubsection{Measuring the affine transformation error}
            The transformation $\AffMatij$ perfectly aligns the
              corresponding $i$\th{} query feature with the $j$\th{}
              database feature in the space of the database annotation
              ($\annotII$).
            If the correspondence is indeed correct, then we can expect
              that other corresponding features will be well aligned by
              the transformation.
            The next step is to determine how close the other
              transformed features from the query annotation ($\annotI$)
              are to their corresponding features in database annotation
              ($\annotII$).
            This can be measured using the error in distance, scale,
              and orientation for every correspondence.
            The following procedure is repeated for each hypothesis
              transform %
            $\AffMatij \in \HypothSet$.
            Note that the following description is in the context of
              the $i$\th{} query feature and the $j$\th{} database
              feature, and the $i,j$ suffix is omitted for clarity.
            In this context, the suffixes $\idxI$ and $\idxII$ will be
              used to index into features correspondences.

            Let $\set{B}_{\idxI} = \curly{\invvrmatI \where (\idxI,
              \idxII) \in \Matches_{\annotII}}$ be the set of keypoint
              matrices in the query annotation with correspondences to
              database annotation $\annotII$.
            Given a hypothesis transform $\AffMat$, each query keypoint
              in the set of matches
            %(mapping from the normalized reference frame to feature space)
            $\invvrmatI \in \set{B}_{\idxI}$, is warped into
              $\annotII$-space:
            \begin{equation}
                \warp{\invvrmatI} = \AffMat \invvrmatI
            \end{equation}
            %---
            The warped position $\warp{\ptI}$, scale $\warp{\scaleI}$,
              and orientation $\warp{\oriI}$, can be extracted from
              $\warp{\invvrmatI}$ using~\cref{eqn:affinewarp}.
            The warped query keypoint properties in $\annotII$-space
              and can now be directly compared to the keypoint properties
              of their database correspondences.
            %Each warped point is checked for consistency with its
            %  correspondence's $\ptII$, scale $\scaleII$, and orientation
            %  $\oriII$, in $\annotII$.
            The absolute distance in position, scale, and orientation
              between the $\idxI$\th{} query feature and the
              $\idxII$\th{} database feature with respect to hypothesis
              transformation $\AffMat$ is measured as follows:
            \begin{equation}\label{eqn:inlierdelta}
                \begin{aligned}
                    \Delta \pt_{\idxI, \idxII}     & \eqv  \elltwo{\warp{\ptI} - \ptII}\\
                    \Delta \scale_{\idxI, \idxII}  & \eqv  \max(
                        \frac{\warp{\scaleI}}{\scaleII},
                        \frac{\scaleII}{\warp{\scaleI}}) \\
                    \Delta \ori_{\idxI, \idxII}    & \eqv  \min(
                        \modfn{\abs{\warp{\oriI} - \oriII}}{\TAU},\quad 
                        \TAU - \modfn{\abs{\warp{\oriI} - \oriII}}{\TAU})
                \end{aligned}
            \end{equation}

        \subsubsection{Selecting affine inliers}
            %Valid inliers are those matches that have all absolute differences
            %  within a certain spatial distance threshold $\xythresh$, orientation
            %  threshold $\orithresh$, and scale threshold $\scalethresh$.
            %  %$\xythresh$ is specified as a percentage of the matched chip's
            %  %  diagonal length.
            Any keypoint match $(\idxI, \idxII) \in
              \Matches_{\annotII}$  is considered an inlier \wrt{}
              $\AffMat$ if its absolute differences in position, scale,
              and orientation are all within a spatial distance threshold
              $\xythresh$, scale threshold $\scalethresh$, and
              orientation threshold $\orithresh$.
            This is expressed using the function $\isinlierop$, which
              determines if match is an inlier:
             %\begin{equation}
              %\label{eqn:inlierchecks}
              %    \begin{gathered}
              %    %\begin{aligned}
              %        \txt{isinlier}(\kp_1, \kp_2) \rightarrow \elltwo{\pt_1' - \pt_2} < \xythresh \AND \\
              %  %-----
              %        {\frac{\scale_1'}{\scale_2} < \scalethresh \txt{ if }
              %        \paren{\scale_1' > \scale_2} \txt{ else }
              %        \frac{\scale_2}{\scale_1'} < \scalethresh}  \AND\\
              %  %-----
              %        \txt{minimum}(
              %        \modfn{\abs{\ori_1' - \ori_2}}{\TAU},
              %        \TAU - \modfn{\abs{\ori_1' - \ori_2}}{\TAU}) < \orithresh
              %    %\end{aligned}
              %    \end{gathered}
            %\end{equation}
            \begin{equation}\label{eqn:inlierchecks}
                \begin{gathered}
                %\begin{aligned}
                    \isinlierop((\idxI, \idxII), \AffMat)  \eqv  
                        \Delta \pt_{\idxI, \idxII} < \xythresh \AND 
                        \Delta \scale_{\idxI, \idxII} < \scalethresh \AND 
                        \Delta \ori_{\idxI, \idxII} < \orithresh
                %\end{aligned}
                \end{gathered}
            \end{equation}
        The set of inlier matches for a hypothesis transformation
          $\AffMat$ can then be written as:
        \begin{equation}\label{eqn:affinliers}
            \Matches_{\AffMat} \eqv \curly{m \in \Matches_{\annotII} \where \isinlierop(m, \AffMat)}
        \end{equation}
        The best affine hypothesis transformation, $\HypothAffMat$,
          maximizes the weighted sum of inlier scores.
        % FIXME
        \begin{equation}
            \HypothAffMat \eqv \argmax{\AffMat \in \HypothSet} 
                \sum_{(\idxI, \idxII) \in \Matches_{\AffMat}} \fs_{\idxI, \idxII}
        \end{equation}

    \subsection{Homography refinement}
        Matches that are inliers to the best hypothesis affine
          transformation, $\HypothAffMat$, are used in the least squares
          refinement step.
        This step is only executed if there are at least $4$ inliers to
          $\HypothAffMat$, otherwise all correspondences between features
          in query annotation $\annotI$ to features in database
          annotation $\annotII$ are removed.
        The refinement step estimates a projective transform from
          $\annotI$ to  $\annotII$.
        To avoid numerical errors the $xy$-locations of the
          correspondence are normalized to have a mean of $0$ and a
          standard deviation of $1$ prior to estimation.
        A more comprehensive explanation of estimating projective
          transformations using point correspondences can be found
          in~\cite[311--320]{szeliski_computer_2010}.

        Unlike in the affine hypothesis estimation where many
          transformations are generated, only one homography
          transformation is computed.
        Given a set of inliers to the affine hypothesis transform
          $\Matches_{\HypothAffMat}$, the least square estimation of a
          projective homography transform is:
        \begin{equation}
            \HmgMatBest \eqv \argmin{\HmgMat} \sum_{(i, j) \in
              \Matches_{\HypothAffMat}} \elltwosqrd{\HmgMat \pt_{i} - \pt_{j}}
        \end{equation}

        Similar to affine error estimation, we will identify the subset
          of inlier features correspondences $\Matches_{\HmgMatBest}
          \subseteq \Matches_{\annotII}$.
        A correspondence is an inlier if the query feature is
          transformed to within a certain spatial distance threshold
          $\xythresh$, orientation threshold $\orithresh$, and scale
          threshold $\scalethresh$ of its corresponding database feature.
        For convenience, let $\tohmg{\cdot}$ transform points into
          homogeneous coordinates.
        For each feature correspondence $(\idxI, \idxII) \in
          \Matches_{\annotII}$, the query feature position, $\ptI$, is
          warped from $\annotI$-space into $\annotII$-space.
        \begin{equation}
            \warp{\ptI} = \unhmg{\HmgMatBest \tohmg{\ptI}}
        \end{equation}
        However, because projective transformations are not guaranteed
          to preserve the structure of the affine keypoints, warped
          scales and orientations cannot be estimated with the method
          previously shown in~\cref{eqn:affinewarp}.

        \subsubsection{Estimation of warped shape parameters}
        Because we cannot warp the shape of an affine keypoint using a
          projective transformation, we instead estimate the warped scale
          and orientation for the $\idxI$\th{} query feature using a
          reference point.
        Given a single feature match $(\idxI, \idxII) \in
          \Matches_{\annotII}$, we associate a reference point $\refptI$
          with the query location $\ptI$, scale $\scaleI$ and orientation
          $\oriI$.
        The reference point is defined to be $\scaleI$ distance away
          from the feature center at an angle of $\oriI$ radians in
          $\annotI$-space.
          \begin{equation}
            \refptI = \ptI + \scale_1 \BVEC{\sin{\oriI} \\ -\cos{\oriII}}
          \end{equation}

        To estimate the warped scale and orientation, first the reference
          point is warped from $\annotI$-space into $\annotII$-space.
        \begin{equation}
            \warp{\refptI} = \unhmg{\HmgMatBest \tohmg{\refptI}}
        \end{equation}
        %-----------
        Then we compute the residual vector $\ptres$ between the warped point
          and the warped reference point:
        \begin{equation}
            %\Delta \warp{\refptI} = \BVEC{\Delta \warp{\inI{x}} \\ \Delta \warp{\inI{y}}} = \warp{\ptI}- \warp{\refptI}.
            \ptres = \BVEC{\xres \\ \yres} = \warp{\ptI}- \warp{\refptI}.
        \end{equation}
        The warped scale is estimated using the length of the residual
          vector, and the warped orientation is estimated using the angle
          of the residual vector.
        In summary, the warped location, scale, and orientation of the
          $\idxI$\th{} query feature is:
        \begin{equation}\label{eqn:homogwarp}
            \begin{aligned}
                \warp{\ptI}      &\eqv \unhmg{\HmgMatBest \tohmg{\refptI}} \\
                 \warp{\scaleI}  &\eqv \elltwo{\ptres}\\
                %\warp{\oriI}    &= \atantwo{\yres, \xres} - \frac{\TAU}{4}.
                %\warp{\oriI}    &= \atantwo{\yres, \xres} - \frac{\pi}{2}.  % is this adjustment right?
                \warp{\oriI}     &\eqv \atantwo{\yres, \xres}
            \end{aligned}
        \end{equation}

        \subsubsection{Homography inliers}
        The rest of homography inlier estimation is no different than
          affine inlier estimation.
        \Cref{eqn:inlierdelta} is used to compute the errors $( %
        \Delta \pt_{\idxI, \idxII}, %
        \Delta \scale_{\idxI, \idxII}, %
        \Delta \ori_{\idxI, \idxII})$
        %
        between the warped query location, scale, and orientation,
          $(\warp{\ptI}, \warp{\scaleI}, \warp{\oriI})$, %
        and the corresponding database location, scale, and
          orientation, %
        $({\ptII}, {\scaleII}, {\oriII})$.
        The final set of homograph inliers is given as:
        \begin{equation}\label{eqn:homoginliers}
            \Matches_{\HmgMatBest} \eqv \curly{m  \in \Matches_{\annotII} \where \isinlierop(m, \HmgMatBest)}
        \end{equation}

        Spatial verification results in a reduced set of inlier feature
          correspondences from the query annotation to the database
          annotations.
        The \namescoring{} mechanism from~\cref{subsec:namescore} is
          then applied to these inlier feature correspondences.
        This final per-name score is the output of the identification
          algorithm and used to form a ranked list that is presented to a
          user for review.

     \sver{}

    
\section{Experiments}\label{sec:experiments}

    This section presents an experimental evaluation of the identification algorithm using annotated images of
      plains zebras, Grévy's zebras, and Masai giraffes.
    The input to each experiment is
    (1) a dataset,
    (2) a subset of query and database annotations from the database,
    (3) a pipeline configuration.
    The datasets are described in~\cref{sub:datasets}.
    The subsets of query and database annotations are carefully chosen to measure the accuracy of the algorithm
      under different conditions and to control for time, quality, and viewpoint.
    The pipeline configuration is a set of parameters --- \eg{} the level of feature invariance, the number of
      nearest neighbors, and the \namescoring{} mechanism --- given to the identification algorithm.
    We will vary these pipeline parameters in order to measure their effect on the accuracy of the ranking
      algorithm.

    For each query annotation, the identification algorithm returns a ranked list of \names{} with a score for each
    name. The accuracy of identification is measured in two ways: (1) database ranking and (2) score separability.
    % ALGORITHM ACCURACY PRIMARY MEASURE 
    The first and primary measure of identification accuracy is the percentage of queries in the returned list with
    the \groundtrue{} name ranked $1\st$.
    % ALGORITHM ACCURACY SECONDARY MEASURE 
    In some experiments the percentage of queries with their \groundtrue{} result ranked $2\nd$, $3\rd$, $4\th$,
    and $5\th$ is also presented. These auxiliary ranking measures are also valuable because users will often
    review more than just the top ranked results, but a user will rarely review past the $5\th$ rank. The second
    measure of identification accuracy is the returned score's separability --- \ie{} how well separated the scores
    of \groundtrue{} and \groundfalse{} matches are. This will help to determine when it is possible to choose
    correct matches without a manual decision, further automating the identification process.
    %apply a binary classifier to results to
    %  further automate the identification process.
    The separability of scores is measured by first recording the score of the \groundtrue{} \name{} and the
    highest scoring \groundfalse{} \name{} for each query and then plotting the receiver operator characteristic
    (ROC) curve and measuring the area under the curve (AUC).
    %Other measures that are reported include the percentage of true positive
    %  and false positive results that fall within certain categories (\eg
    %  temporal windows).

    The outline of this section is as follows.
    First, \Cref{sub:datasets} describes the datasets and how they were generated.
    Then, \Cref{sub:exptbase} establishes the baseline performance of the algorithm using the default pipeline
      configuration and a subset of the data that controls for time, viewpoint, number of exemplars, and quality.
    \Cref{sub:exptfeatmatchscore} tests the effect of the foregroundness weight on identification accuracy.
    \Cref{sub:exptinvar} investigates the effect of the level of feature invariance and viewpoint.
    \Cref{sub:exptscoremech} compares the \csumprefix{} and the \nsumprefix{} \namescoring{} mechanism.
    \Cref{sub:exptk} varies the $\K$ parameter (the number of nearest neighbors used in establishing feature
      correspondences) and investigates the relationship between $\K$ and database size in terms of both the number
      of annotations and the number of exemplars per name.
    \Cref{sub:exptfail} discusses the failure cases of the algorithm.
    \Cref{sub:exptsep} presents an evaluation of the score separability for the pipeline configuration with the
      highest accuracy determined for each species.
    Finally,~\cref{sub:exptsum} summarizes this section.


    \subsection{Datasets}\label{sub:datasets}

        All of the images in the datasets used in these experiments were taken by photographers in the field.
        Each dataset is labeled with groundtruth in the form of annotations with name labels.
        Annotations (bounding boxes) have been drawn to localize animals within the image.
        A unique \name{} label has been assigned to all annotations with the same identity.
        Some of this groundtruth labeling was generated independently.
        However, large portions of the datasets were labeled with assistance from the matching algorithm.
        While this may introduce some bias in the results, there was no alternative because the amount of time
          needed to independently label a large dataset is prohibitive.

        There are two important things to note before we describe each dataset.
        First, in order to control for challenging factors in the images such as quality and viewpoint some
          experiments sample subsets of the datasets we describe here.
        Second, note that there do exist labeling errors in some datasets.

        \DatabaseInfo{}

        \timedist{}

        The number of names, annotations, and their distribution within each database are summarized in the
          following tables.
        In these tables the term \glossterm{singleton} refers to \names{} that only contain one annotation, and
          \glossterm{multiton} refers to \names{} that have more than one annotation.
        We make this distinction between singletons and multitons because multitons are names that have known
          groundtruth matches.
        \Cref{tbl:DatabaseStatistics} summarizes the number of annotations per database.
        \Cref{tbl:AnnotationsPerNameMultiton} summarizes the number of annotations for multiton \names{}.
        \Cref{tbl:AnnotationsPerQuality} summarizes the quality labels of the annotations.
        \Cref{tbl:AnnotationsPerViewpoint} summarizes the viewpoint labels of the annotations.
        The name and a short description of each dataset is given in the following list.

        \begin{itemln}
            \item \textbf{\pzmasterI{}} is an aggregated dataset of plains zebras.
            There is variation in how the data was collected and preprocessed.
            Some of the images are cropped to the flank of the animal, while others are cropped to encompass the
              entire body.
            The datasets contributing to \pzmasterI{} were collected in Kenya at several locations including
              Nairobi National Park, Sweetwaters, and Ol Pejeta.
            More than $90\percent$ of the groundtruth generated for this dataset was assisted using the matching
              algorithm.
            This dataset contains many imaging challenges including occlusion, viewpoint, pose, quality, and time
              variation.
            There are some annotations in this dataset without quality or viewpoint labelings and some images
              contain undetected animals.
            This data was collected between 2006 and 2015, but the majority of the data was collected in
              2012--2015.
            The distribution of collection times is shown in~\cref{sub:timedistA}.

            \item \textbf{\gzall{}} is an aggregated dataset of Grévy's zebras.
            The original groundtruth for this dataset was generated independently of the matching algorithm,
              however the matching algorithm revealed several groundtruth errors that have since been corrected.
            The Grévy's dataset was collected in Mpala, Kenya.
            Most of the annotations in this database have been cropped to the animal's flank.
            This dataset contains a moderate degree of pose and viewpoint variation as well as occlusion.
            This data was collected between 2003 and 2012, but the majority was collected in 2011 and 2012.
            The distribution of collection times is shown in~\cref{sub:timedistB}.

            \item \textbf{\girmmasterI{}} is a dataset of Masai giraffes.
            The images were all taken in Nairobi National Park.
            All groundtruth was established using the matching algorithm followed by manual verification.
            This dataset contains a high degree of pose and viewpoint variation, as well as occlusion.
            There are also many \glossterm{photobombs} --- instances where there is more than one giraffe in an
              annotation.
            This data was collected in the \GZC{} between February 20, 2015 and March 2, 2015.
            The distribution of collection times is shown in~\cref{sub:timedistC}.
        \end{itemln}

    \subsection{Baseline experiment}\label{sub:exptbase}
      
        % TODO: compute encounters and use a single annotation per encounter instead of the timectrl metric

        This first experiment determines the accuracy of the identification algorithm using the baseline pipeline
          configuration.
        The baseline pipeline configuration uses affine invariant features oriented using the gravity vector,
          $\K\tighteq4$ as the number of feature correspondences assigned to each query feature, and \nscoring{}
          (\nsum{}).
        In this test we control for several biases that may be introduced by carefully selecting a subset of our
          datasets.
        We only use annotations that
        (1) are known (\ie{} have been assigned a name),
        (2) are assigned the species primary viewpoint (left for plains zebras and Masai giraffes and right for
          Grévy's zebras),
        (3) have not been assigned a quality of ``junk'' or ``poor''.
        Furthermore, to account for the fact that some \names{} contain more annotations than others, we
          constrain our data selection such that there is only one \groundtrue{} \exemplar{} in the database for
          each query annotation.

        We test two configurations of databases.
        The first is denoted as \ctrl{} and uses only the aforementioned constraints.
        The second is denoted as \timectrl{} and we further restrict the annotation selection by enforcing that
          each query and its corresponding \groundtrue{} \exemplar{} are separated in time by at least $6$ hours.
        This allows us control for correct results that may be due to incidentally identifying a near duplicate
          image.
        For the smaller Masai giraffes database we relax this constraint to $1$ hour.

        The results are presented in the form of a cumulative rank histogram in~\cref{fig:BaselineExpt}.
        This plot indicates the accuracy of the algorithm when the top $1-5$ ranks of each query are returned to
          the user.
        The $x$-axis indicates the ranks considered for review.
        The $y$-axis shows the percentage of queries with a \groundtrue{} match ranked $x$ or less.
        %, where $x$
        %is a coordinate on the $x$-axis indicating a ranking.
        The most important part of this graph is the percentage of queries with a correct match at rank
          $x\tighteq1$.
        Each colored bar represents a different parameter configuration and is labeled in the legend.

        \BaselineExpt{}

        %In all experiments the accuracy of the algorithm decreased when time
        %  was controlled for.
        % PERCENT DIFFERENCE THAT TIME MAKES
        \begin{comment}
        python -m ibeis -e print --db PZ_Master1 -a ctrl timectrl -t baseline  
        python -m ibeis -e print --db GZ_Master1 -a ctrl timectrl -t baseline  
        python -m ibeis -e print --db GIRM_Master1 -a ctrl timectrl1h -t baseline  
        \end{comment}
        The effect of time is most notable for plains zebras where there is a $~10\percent$ decrease in accuracy
          between \ctrl{} and \timectrl{}.
        This is because a significant number of the correct matches in \ctrl{} were between near-duplicate
          images.
        The drop for Grévy's zebras and Masai giraffes is less pronounced.
        This is likely due to the higher density of distinctive patterns on Grévy's zebras and Masai giraffes as
          well as a lack of near duplicate images in those datasets.
        The difference in accuracy for plains zebras shows that near-duplicate matching has a significant impact
          on identification accuracy.
        Therefore, to avoid this issue, the remainder of our experiments use \timectrl{} as the baseline for
          selecting query and database annotations.

    \subsection{Foregroundness experiment}\label{sub:exptfeatmatchscore}

    \ForegroundExpt{}

        In this experiment we test the effect of foregroundness --- weighting the score of each features
          correspondence with a foregroundness weight --- on identification accuracy.
        Two pipeline configurations are tested in this experiment.
        In the first we score each feature correspondences using only the LNBNN~\cite{mccann_local_2012} score.
        In the second we weight the LNBNN score using a foregroundness measure learned using a deep convolutional
          neural network~\cite{parham_photographic_2015}.
        Currently, we have only trained a foregroundness measure for plains zebras and Grévy's zebras.
        Therefore Masai giraffes are excluded from this test.
        The accuracy of the foregroundness is shown in~\cref{fig:ForegroundExpt}

        For plains zebras, using the foregroundness measure results in a significant $3.79\percent$ increase in
          identification accuracy.
        For Grévy's zebras there is also a significant  $3.3\percent$ increase.
        This experiment clearly shows the importance of eliminating background feature correspondences.
     
    \subsection{Invariance experiment}\label{sub:exptinvar}  
        In this experiment we vary the feature invariance configuration.
        Because adding feature invariance is meant to detect and describe the same features across multiple
          viewing positions, we also evaluate the different invariance settings both
        (1) when the viewpoint between each query and its \groundtrue{} exemplar are the same as well as
        (2) when the viewpoint is varied.
        In the first case we use the same \timectrl{} configuration as the baseline experiment.
        In the second case we constrain each database annotation's viewpoint to be the primary viewpoint of the
          species and each query annotation's viewpoint to be adjacent to the primary viewpoint.
        \Eg{} for plains zebras each query has a frontleft viewpoint, and each database annotation has a left
          viewpoint.
        Unfortunately, there are not many \names{} in our databases that contain both a primary and non-primary
          viewpoint annotations.
        Therefore, to increase the size of the dataset we do not control for time in the viewpoint varied
          experiment.
        The following is a list describing the invariance settings that we investigate in each experiment.

        \begin{itemln}

            \item \NoInvar{} (\pvar{AI=F,QRH=F,RI=F}): % 
            In this configuration the gravity vector is assumed and the shape of each detected feature is not
              adapted.

            \item \AIAlone{} (\pvar{AI=T,QRH=F,RI=F}): % 
            This is the baseline setting that assumes the gravity vector and where each feature's shape is skewed
              from a circle into an ellipse.

            \item \RIAlone{} (\pvar{AI=F,QRH=F,RI=T}): % 
            Here, each feature is assigned one or more dominant gradient orientations (the gravity vector is not
              used) and the shape is not adapted.

            \item Query-side rotation heuristic (\QRHCirc{}) (\pvar{AI=F,QRH=T,RI=F}): %
            This is a novel invariance heuristic where each {database} feature assumes the gravity vector, but
              {query} feature is $3$ orientations:
            the gravity vector and two other orientations at $\pm15\degrees$ from the gravity vector.
            Ideally, this will allow feature correspondences to be established between features seen from
              slightly different orientations.

            \item \QRHEll{} (\pvar{AI=T,QRH=T,RI=F}): %
                This is the combination of \QRHCirc{} and \AIAlone{}.

            \item \AIRI{} (\pvar{AI=T,QRH=F,RI=T}): %
                This is the combination of \RIAlone{} and \AIAlone{}.

        \end{itemln}

        % Invar Conclusions
        The identification accuracy of the viewpoint and time controlled invariance experiment is shown
          in~\cref{fig:InvarExpt}.
        We find that full rotation invariance provides the poorest results for all datasets.
        % Invar Conclusions (same view)
        For plains zebras, \QRHCirc{} scores significantly ahead of all other invariance settings.
        %
        The results for Gravy's zebras show that \AIAlone{} is the most accurate invariance setting, but
          \QRHEll{} performs almost as well.
        %
        There is not enough data to draw definitive conclusions about Masai giraffes, but the small sample that
          we experiment with shows that \AIAlone{} and \QRHEll{} perform similarly to Grévy's zebras.

        \InvarExpt{}

        % Invar Conclusions (diff view)
        The identification accuracy of the viewpoint varied invariance is shown in~\cref{fig:InvarViewExpt}.
        There is only a small amount of data available to run this experiment, specifically there are $53$ query
          annotations available in \pzmasterI{}, $29$ in \gzall{}, and $14$ in \girmmasterI{}.
        For plain zebras \QRHCirc{} and \NoInvar{} are tied for the highest accuracy, but \QRHEll{} is the most
          accurate if ranks greater than $1$ are considered.
        For Grévy's zebras \AIAlone{} provides the highest accuracy.
        The \AIAlone{} and \QRHEll{} configurations are tied for highest accuracy in the Masai giraffes
          experiment.
        Overall, it seems that configurations with affine invariance perform the best on viewpoint varied cases,
          which agrees with intuition.
        However, these results show that the algorithm is overall less accurate when matching across viewpoints
          and that matching between different viewpoints of an animal is significantly more difficult than matching
          between the same views.
        The claim is further supported by the small amount of data available to perform this test.
        Most of the groundtruth used in these experiments was created with assistance from this identification
          algorithm, therefore if the identification algorithm does not reliably match between viewpoints then that
          will be reflected by a lack of viewpoint varied groundtruth.
        However, in creating this groundtruth the focus was on identifying primary views of the animal, so it is
          not clear how significant this effect is.
        The effect of failure cases due to viewpoint is further discussed in~\cref{sub:exptfail}.

        \InvarViewExpt{}

        \kptstype{}

        % MAKE SURE THAT BEST SETTINGS DISCCSSED HERE REFLECTS THE FIGURES

        % General conclusions
        The results in this experiment support the claim that the best choice of invariance settings is data
          dependent.
        The baseline invariance parameter \AIAlone{} produces the most accurate identification of  Grévy's zebras
          and Masai giraffes.
        % FIXME: which actually performs better?
        However, \QRHCirc{} performs better for plains zebras.
        This is likely because affine keypoints tend to describe only one or two coarse stripes on plains zebras.
        In contrast, distinctive details on Grévy's zebras and Masai giraffes are finer and well captured by
          affine keypoints.
        Even though affine keypoints provide more precise localization, the area they describe is often smaller
          than a circular keypoint.
        It makes sense that affine keypoints would not describe coarse features, like those seen on plains
          zebras, as well as a circular keypoint covering a larger area.
        This difference between \AIAlone{} and  \QRHEll{} features for plains and Grévy's zebras is illustrated
          in~\cref{fig:kptstype}.
        For the remainder of our experiments we use \QRHCirc{} as the invariance setting for plains zebras and
          the baseline of \AIAlone{} on Grévy's zebras and Masai giraffes.

    \subsection{Scoring mechanism experiment}\label{sub:exptscoremech}  

        % Database setup for name scoring
        The purpose of the scoring mechanism is to aggregate scores of individual feature correspondences across
          multiple annotations into a single score for each name.
        The experiments in this subsection tests the identification accuracy of two name scoring mechanisms:
        (1) \cscoring{} (\csum{}) and
        (2) \nscoring{} (\nsum{}).
        Because the scoring mechanism is meant to take advantage of multiple \exemplars{} per \name{} we vary the
          number of \exemplars{} per query between $1$, $2$, and $3$, except in the case of the \girmmasterI{}
          dataset we only vary the number of \exemplars{} between $1$ and $2$ (due to dataset size constraints).
        The accuracy of the scoring mechanism experiment is shown in~\cref{fig:NScoreExpt}

        %\GIRMNscore
        \NScoreExpt{}

        % Describe results
        The number of \exemplars{} per \name{} is the most significant factor in this test.
        All results indicate that \nsum{} is either as good or performs slightly better than \csum{}.
        It is interesting that in most of these results \nsum{} is exactly as good as \csum{}.
        We hypothesize that the reason for this is that the nearest neighbors of most query features tend to be
          from a single most visually similar exemplar in the database.
        This would cause a majority of the feature correspondences in \nsum{} to cast their vote for a single
          exemplar, giving results similar to \csum{}.
        %It is unlikely that some part of another exemplar would appear
        %  more similar to the query annotation than the most similar
        %  exemplar.
        The small gain seen by \nsum{} is likely from cases where part of the most similar exemplar is occluded
          or obscured, thus allowing some query features to be matched with features from different \exemplars{}.
        % Conclusions about scoring mechanism
        We continue to use \nsum{} as the scoring mechanism for the remainder of the experiments.
        

    \subsection{K experiment}\label{sub:exptk}  

        % Introduce varied parameters
        In this experiment we investigate the effect $\K$ (the number of nearest neighbors used in establishing
          feature correspondences) on identification accuracy.
        We vary $\K$ between the values $1, 2, 3, 4, 5, 7$, and $10$.
        In all of these experiments we set the number of normalizing neighbors to be $\Knorm=1$.
        We hypothesize that the optimal choice of $\K$ depends on the size of the database.
        This subsection will present two experiments:
        (1) an experiment to test the accuracy at different values of $\K$ on a large database, and
        (2) an experiment to test the accuracy at different values of $\K$ for many database sizes.
        Database size depends on both the number of \names{} in the database and the number of \exemplars{} per
          \name{}.
        Therefore, we vary both of these variables.
        We vary the total number of \exemplars{} in the database between $\frac{1}{4}$, $\frac{1}{2}$, and
          $\frac{3}{4}$ of total number of annotations available.
        The number of \exemplars{} per \name{} is varied between $1$, $2$ and $3$.
        Because of the small size of the \girmmasterI{} dataset we only vary the number of \exemplars{} between
          $1$ and $2$.

        The effect of $\K$ on matching accuracy using a static database size is shown in~\cref{fig:KExpt}.
        The results for plains zebras shows little difference in accuracy between different values of $\K$.
        For Grévy's zebras there appears to be a negative relationship between $\K$ and accuracy.
        This may be because the first match of a highly distinctive pattern (like those seen in Grévy's zebras)
          will not be confused other \names{}.
        % giraffes 
        Setting $\K\tighteq4$ results in the highest accuracy of the Masai giraffe dataset.

        \KExpt{}
        
        Our second test varies the size of the database as well as the value of $\K$.
        The effect of $\K$ and database size on matching accuracy is shown in~\cref{fig:DBSizeExpt}.
        The results for all species show that the number of \exemplars{} per \name{} is the most important factor
          in this experiment.
        Interestingly, the number of annotations in the database is only a minor factor in identification
          accuracy.
        % plains
        The results for plains zebras show a small positive relationship between the number of annotations in the
          database and $\K$.
        This may be because many plains zebra features are not globally distinctive in a large database and a
          feature's correct correspondence may not be the nearest neighbor.
        For smaller database sizes lower values of $\K$ produce more accurate results.
        % grevys
        For Grévy's zebras lower values of $\K$ seem better for all database sizes.
        It also seems that the best values of $\K$ should be set to the number of \exemplars{} per \name{}, which
          is expected when there is little confusion between features.
        % giraffes
        For Masai giraffes, the amount of data again makes it difficult to draw conclusions.

        \DBSizeExpt{}

        Overall the experiments on the setting of $\K$ does not yield definitive choice for this parameter.
        However, it appears that $\K$ only has a small influence on identification accuracy.
        This section does shows that the number of exemplars per annotation has a significant impact on
          identification accuracy.
        %For the remainder of our experiments we will use $\K\tighteq3$
        %  for plains zebras, $\K\tighteq1$ for Grévy's zebras, and
        %  $\K\tighteq3$ for Masai giraffes.

    \subsection{Failure cases}\label{sub:exptfail}  
        
        In this subsection we investigate the primary causes of identification failure by considering individual
          failure cases.
        When investigating the cause of a failure case we consider two matches from the query annotation to a
          \name{}:
        (1) the match from the query annotation to the \groundfalse{} \name{} at rank $1$, and
        (2) the match from the query annotation to the \groundtrue{} \name{}.

        We manually label the \groundtrue{} match and the \groundfalse{} match in each failure case to indicate
          the factors that heuristically appear to cause identification to fail.
        We accumulate the frequency of these cases into a histogram to illustrate the significance of each type
          of failure case.
        The failure case histograms are shown in~\cref{fig:TagExpt}.

        \TagExpt{}

        The following list gives a definition for each type of failure case and an example.
        The first three types of failure cases denote type 2 errors (false negatives) where the \groundtrue{}
          match fails to produce a high score.
        The last three types of failure cases denote type $1$ errors (false positive) where the \groundfalse{}
          match produces a score that is too high.

        \begin{itemln}

            \item Viewpoint:
            This failure case denotes that there is a viewpoint difference between the query and its
              \groundtrue{} match.
            Viewpoint differences are among the most common causes of failure in all of the datasets.
            A viewpoint failure case is caused by an out-of-plane rotation between the query features and the
              \groundtrue{} database features.
            Out-of-plane rotations of a feature can cause significant difference in appearance and exacerbates
              errors in feature localization, causing inconsistency between feature descriptions.
            The differences in descriptors leaves the approximate nearest neighbor algorithm unable to establish
              the correct correspondence, which ultimately causes identification failure.

            Note that these viewpoint failures cases originate from a test that controls for viewpoint.
            We have found that about half of these cases are due to viewpoint mislabelings.
            The pairs in the other half have correct viewpoint labels, yet a mild viewpoint difference still
              causes the initial assignment of feature correspondences to fail.
            An example illustrating a failure case due to viewpoint is shown in~\cref{fig:FailViewpoint}.

            \FailViewpoint{}

            \item Occlusion:
            This label denotes that either the query or the \groundtrue{} annotation is occluded.
            Objects like grass, tree branches, and other animals can obscure a feature causing it to appear
              dissimilar or mask it entirely.
            This is a much more significant problem for species with relatively few distinctive features like
              plains zebras.
            An example illustrating a failure case due to occlusion is shown in~\cref{fig:FailOcclusion}

                \FailOcclusion{}

            \item Quality:
            This label denotes that either the query or the \groundtrue{} annotation is blurred or distorted.
            Some of these cases are due to an annotation having an ``ok'' quality label, when in fact it should
              be labeled ``poor'' or ``junk''.
            An annotation with low quality may generate fewer, larger, less distinct, and distorted features.
            This case occurs due to mislabeling of the image quality.
            An example illustrating a failure case due to quality is shown in~\cref{fig:FailQuality}.

                \FailQuality{}

            \item Lighting:
            This label denotes that either the query or the \groundtrue{} annotation is poorly illumination or
              shadowed.
            Feature extraction produces too few and unreliable features in under exposed images.
            Non-uniform illumination and shadowing produces noisy intensity gradients that interfere with feature
              description.
            Ideally, a classifier could be trained to determine which images are poorly illuminated and select an
              appropriate preprocessing step.
            An example illustrating a failure case due to lighting is shown in~\cref{fig:FailLighting}

                \FailLighting{}

            \item Scenery Match:
            This denotes a case where the algorithm produces correspondences between shared background features
              between a query annotation and \aan{\groundfalse{}} annotation.
            These cases are typically between pairs of annotations with small (less than $10$ minutes) time
              deltas.
            For plains and Grévy's zebras, these cases are mostly eliminated using a foreground weighting
              algorithm.
            However, for new species --- when a foregroundness measure has not been trained --- this will still
              be a problem.
            An example illustrating a failure case due to a scenery match is shown in~\cref{fig:FailScenery}

              \FailScenery{}

            \item \Photobomb{}:
            A case where the \groundtrue{} animal is seen in the foreground or background of \aan{\groundfalse{}}
              annotation.
            This case is not technically a false positive because the algorithm is correctly matching the same
              individual.
            However, this is a problem for identification because failure to detect a \photobomb{} will cause two
              different individuals to be incorrectly marked as the same \name{}.
            This could potentially start a cascading ``snowball'' effect.
            An example illustrating a failure case due to \photobombing{} is shown in~\cref{fig:FailPhotobomb}

              \FailPhotobomb{}

            %\item SimilarPose - The query and the \groundfalse{} annotation are in the
            %    same pose. This causes incorrect but matches along the edge of
            %    the animal that encode the shape of the animal's position.
            %    \FailPose
            %    A pose failure case is shown in~\cref{fig:FailPose}.
        \end{itemln}


        The above failure cases show that the main causes of algorithm failure are due to viewpoint, occlusion,
          and quality.
        This identification algorithm itself does not seek to correctly identify low quality annotations, however
          the larger system should be able to flag and either fix (\eg{} by applying histogram equalization on a
          case-by-case basis to a poorly illuminated annotations) or remove such annotations.
        We approach the problem of viewpoint as a data issue.
        By adding more exemplars to the database we expect to improve matching accuracy between annotations with
          small viewpoint differences.
        Scenery matches and \photobombings{} also cause false positives, but the foregroundness weighting mostly
          eliminates scenery matches.
        To account for \photobombings{}, a classifier could be trained based on the timestamps between
          annotations, and the spatial distribution of feature correspondences within the annotations.
        This would also help to further reduce failures due to scenery matches.
        
        %To further reduce scenery mathces and \photobombings these cases could be flagged 
        %Scenery matches and \photobombings{} also cause a significant number
        %  of false positives, but these cases should be able to be flagged using
        %  a background detector, the timestamps between images, and the location
        %  of the feature correspondences.

        %A big failure case for the Grévy's may actually be size of the image. 
        %The features are not getting detected properly on many chips.
        %This may not be the issue.
        %Hmmm.

    \subsection{Score separability}\label{sub:exptsep}  
        %Show separability of scores under the best algorithm settings for only
        %  success cases and then with both success and failures.
        %Show only this histogram of scores and the ROC curve.
        %Report results for all species.

        In this subsection we investigate identification accuracy in terms of the scores returned along with each
          \name{} in the ranked list.
        This is in contrast to the results presented in previous experiments where accuracy is evaluated only in
          terms of ranking.
        It is important to look at the overall scores of the algorithm because in a deployment setting a query
          annotation may not have corresponding \groundtrue{} database annotation and the top ranked \name{} would
          always be incorrect.
        Ideally, the scores would be used to decide if each name in the ranked list is either \groundtrue{} or
          \groundfalse{}.
        However, if the scores are to be used in a decision mechanism, they must have a high degree of
          separability.

        %This requires the raw scores between \groundtrue{} and \groundfalse{}
        %  cases will be separable.
        %In this experiment investigate the 
        We run experiments to test the degree to which \groundtrue{} and \groundfalse{} scores can be separated
          by a binary classifier.
        We consider two types of scores:
        (1) \groundtrue{} scores --- the scores of between the query and its \groundtrue{} name (even if it is
          not ranked first), and
        (2) \groundfalse{} scores --- the scores between the query and the highest ranked \groundfalse{} name
          (the \groundfalse{} name is ranked $1$\st{} for failure cases and $2$\nd{} for success cases).
        The experiments in this subsection are run using the \timectrl{} annotation configuration and the best
          pipeline configuration for each species.
           
        Results are reported in the form of two plots:
        The first shows a histogram of the scores.
        The second plot shows an ROC curve where the true positive rate (sensitivity / recall) is plotted as a
          function of the false positive rate (fall-out).
        The area under the curve (AUC) is reported above the graph.
        The AUC is a standard measure used to evaluate a binary classifier.
        It represents the probability that a random \groundtrue{} name receives a higher score than a random
          \groundfalse{} name.
        The results of the separability experiment are shown in~\cref{fig:ScoreSep}.

        %The results of the separability experiment for plains zebras are shown
        %  in~\cref{fig:PZScoreAll}; Grévy's zebras are shown in
        % ~\cref{fig:GZScoreAll}; and Masai giraffes are shown in
        % ~\cref{fig:GIRMScoreAll}.

        \ScoreSep{}

        The results show that a threshold could be set to automatically accept high scoring names, if a small
          amount of false positives are acceptable.
        However, there is still a significant intersection between the \groundtrue{} and \groundfalse{} cases for
          plains and Grévy's zebras.
        For Masai giraffes the intersection is smaller, but there is also less data available.
        Ideally, we would like to find a threshold at which no false positives are accepted.

        In all datasets there does not exist any threshold able to automatically reject a true negative without
          causing false negatives (this is because some correct matches receive scores of zero).
        There are thresholds that can be set to automatically accept true positives without causing any false
          negatives.
        Unfortunately, the percentage of automatically accepted true positives is low.
        For plains zebras, a threshold of $6.1$ automatically accepts $\frac{3}{475} = 0.6\percent$ \groundtrue{}
          matches.
        For Grévy's zebras, a threshold of $2.55$ automatically accepts $\frac{15}{300} = 5.0\percent$
          \groundtrue{} matches.
        For Masai giraffes, a threshold of $3.1$ automatically accepts $\frac{15}{35} = 42.8\percent$
          \groundtrue{} matches.

        After manually labeling the failure cases we have found that the reason for this low acceptance rate is
          due to \photobombings{}.
        If failure cases due to \photobombings{} are ignored, there is significant improvement.
        For plains zebras, a threshold of $.86$ automatically accepts $\frac{180}{463} = 38.8\percent$
          \groundtrue{} matches.
        For Grévy's zebras, a threshold of $2.55$ automatically accepts $\frac{35}{299} = 11.7\percent$
          \groundtrue{} matches.
        For Masai giraffes, a threshold of $0.7$ automatically accepts $\frac{26}{35} = 76.4\percent$
          \groundtrue{} matches.
        \Photobombing{} cases tend to produce the highest false positive matching score because they are
          technically correct matches --- from a feature perspective --- and are scored appropriately.
        Furthermore, \photobombings{} tend to occur be between images with small time-deltas which increases
          visual similarity and increases the scores of the feature matches.
        Building a classifier to detect \photobomb{} cases would be a simple way to significantly reduce the
          amount of manual verification needed.

        We must note an important caveat to developing a decision mechanism based on the LNBNN identification
          scores.
        The scores computed by the single image identification algorithm depend on all of the database
          annotations we match against.
        As the images in the database change, normalizing features used to compute the LNBNN scores may change as
          well.
        Furthermore, as the database size grows it is likely that fewer matches will be discovered.
        Therefore, any threshold set on these scores is only valid in the context of a specific static database.
        The problem of developing a dynamic decision mechanism is addressed in the next chapter.

    \subsubsection{Why is individual animal identification hard?}\label{sub:whyhard}
        %Setup paragraph.
        Even when ignoring \photobombings{}, the amount missed true positives and the number of manual
          verifications necessary is still unsatisfactory.
        It seems that there is an underlying difficulty in generating the initial feature matches in many cases.
        To illustrate this difficulty, consider the Liberty Buildings dataset~\cite{brown_discriminative_2011}
          commonly used in descriptor learning.
        This dataset contains a large number of corresponding patches from architectural structures computed
          using stereo matching.
        SIFT descriptors are computed for each patch, and the L2-distance between \groundtrue{} and a set of
          \groundfalse{} patch descriptors is computed.
        Pairs of \groundtrue{} and \groundfalse{} descriptors can be similarly computed for an animal dataset as
          the set of spatially verified foreground features from correct individuals and the set of feature matches
          between incorrect individuals.
        Examples of patches from both the Liberty dataset and the plains dataset are shown
          in~\cref{fig:PzVsLibertyPatches}.
        \Cref{fig:PzVsLiberty} compares the separability \groundtrue{} and \groundfalse{} \emph{patches} based on
          the L2-distance between SIFT descriptors from the Liberty dataset and the plains zebra dataset.

        There is a high degree of separability between the patches from the Liberty dataset (an ROC AUC of
          $0.96$) and a low degree of separability between patches from the plains zebra dataset (an ROC AUC of
          $0.724$).
        Consider the histogram of plains zebra scores in~\cref{fig:PzVsLiberty}.
        The incorrect matches also appear to be much closer in the plains dataset when compared to the buildings
          dataset.
        This is likely because are the most difficult incorrect matches in the dataset and are likely to be
          correspondences between non-distinctive descriptors.
        % Rexecuting this experiment with random false keypoint pairs would be interesting.
        The histogram of correct scores appears to be bimodal.
        This is evidence of two things:
        (1) there are significantly more non-distinctive correct matches than distinctive ones, and
        (2) there are matches incorrectly marked as correct.

        Even if the modes of the correct match histogram were separated, there are significantly fewer ``good''
          correct matches for zebras than there are for buildings.
        In~\cref{subsec:dcnndiscuss} we noted a failed attempt to learn new convolutional descriptors to replace
          SIFT{}.
        This patch separability experiment shows the reason for the failure.
        The problem of learning descriptors to match animals seems to be more difficult than learning descriptors
          to match buildings.

        \PzVsLibertyPatches{}

        \PzVsLiberty{}


    \subsection{SMK as an alternative}\label{sub:exptsum}  
        %Plains zebras:
        %LNBNN: 75.6% @ rank 1
        %ASMK: 69.69% @ rank 1

        %Grevy's zebras: 
        %LNBNN: 84.94% @rank 1
        %ASMK: 70.87% @rank 1
        In our ranking experiments we only did extensive testing of the LNBNN algorithm.
        We have briefly experimented with using the vocabulary based SMK ranking algorithm (using the VLAD variant)
          and found LNBNN to provide superior results.
        In our preliminary experiments we found that the SMK algorithm was able to correct rank $69.69\percent$ of
          plains zebras and $70.87\percent$ of Grevy's zebras correctly at rank $1$.
        Comparable versions of LNBNN solutions achieved $75.6\percent$ and $69.7\percent$.

    \subsection{Experimental conclusions}\label{sub:exptsum}  

        In this section we have evaluated our baseline algorithm under restrictive conditions to control for the
          effects of time, quality, and viewpoint.
        Based on the results of these experiments we are able to make the following observations and conclusions.
        \begin{itemln}
            %% DATASET SAMPLING MATTERS
            %\item The baseline experiment shows that it is important to
            %  control for time, because near-duplicate images 
            %  the influence of near-duplicate images on matching
            %  accuracy.

            \item \textbf{Identification accuracy improves with more exemplars}:
            % NUM EXEMPLARS MATTERS
            The name scoring experiment and the $\K$ experiment show that the number of \exemplars{} per database
              \name{} is the most significant factor that impacts identification accuracy.

            \item \textbf{Foregroundness weighting reduce scenery matches}:
            % FOREGROUNDNESS GOOD
            Identification accuracy significantly improves by $2-4$ percentage points when using foregroundness
              weighting.
            We have found that enabling foregroundness weighting eliminates nearly all failure cases due to
              scenery matches without significantly affecting other results.

            \item \textbf{Viewpoint and occlusion are the most difficult imaging challenges}:
            % VIEWPOINT HARD
            The viewpoint experiment and the failure cases show that there is a significant loss in accuracy when
              matching annotations from different viewpoints.
            Viewpoint seems to be the most difficult challenge across all species, however the failure cases show
              that occlusion is a more significant issue for plains zebras.
            This may be because Grévy's zebras and Masai giraffes have distinctive patterns in many places and
              thus can be matched when only part of the animal is visible.

            \item \textbf{Invariance settings are data dependent}:
                The invariance experiment shows that
                % AFFINE GOOD FOR GZ CIRCLE GOOD FOR PZ
                affine invariance produces better results for Grévy's zebras and Masai giraffes, whereas circular
                  keypoints lead to more accurate results for plains zebras.
                % AQH GOOD FOR PZ
                This experiment also demonstrated that the query-side rotation heuristic improves accuracy by
                  adding a small amount of orientation invariance to feature localization, while using full
                  orientation invariance causes a drop in accuracy.
                %This means that it may not be feasible to use more than a
                %  single canonical viewpoint in the final population
                %estimation.

              \item \textbf{The choice of \K{} has a minor impact}: 
                The $\K$ experiment shows that identification accuracy is not significantly influenced by the
                  choice of $\K$ for plains zebras, but for Grévy's zebras the most accurate results were obtained
                  with $\K\tighteq1$.
                This is likely because the features from plains zebras are less distinguishing than features from
                  Grévy's zebras.
                Therefore, the correct match of a plains zebra feature is less likely to be its closest neighbor.
                    %\item 
                        % CHOISE OF K DOES NOT MATTER TOO MUCH FOR LARGE DBS
                Furthermore, the size of the database does not seem to strongly influence the optimal choice of
                  $\K$.
                However, note that most tests were run on different sizes of large databases, and the choice of
                  $\K$ using small databases was not investigated.
                %\end{itemln}

              \item \textbf{\Nsumprefix{} is slightly better than \csumprefix{} \namescoring{}}:
                % NSUM > CSUM
                The scoring mechanism experiment shows that the \nsumprefix{} scoring mechanism is slightly more
                  accurate than the \csumprefix{} scoring mechanism.
                We hypothesize that this effect may be more significant in conditions with more viewpoint
                  variation.

              \item \textbf{LNBNN scores are not enough for automated decision}:
                % Separability
                The separability experiment shows that is a reasonable separation between the scores of
                  annotations seen from the same viewpoint.
                Furthermore, if the effect of \photobombings{} can be reduced it becomes possible to set an
                  threshold to automatically accept high scoring identification results.
                However, there are still a significant number of correct results with non-separable scores.
                Furthermore, it is still unclear to what degree the scores depend on the database size.
                For these reasons we conclude that the decision mechanism should be independent of the single
                  image identification algorithm.
                %More experimentation is needed before threshold levels can
                %  be confidently set for databases of arbitrary sizes.
        \end{itemln}

% L___ CHAPTER ___
