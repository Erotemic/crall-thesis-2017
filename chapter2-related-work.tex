\begin{comment}
    ./texfix.py --fixcap --dryrun
    ./texfix.py --findcite --unused-important
    ./texfix.py --findcite --close-keys
    fixtex --fpaths chapter2-related-work.tex --outline --asmarkdown --numlines=999 --shortcite -w && ./checklang.py outline_chapter2-related-work.md
\end{comment}

\chapter{Related work}\label{chap:relatedwork} 

    To address the individual animal identification problem we draw upon related research in %
    feature detection~\cite{mikolajczyk_comparison_2005,tuytelaars_local_2007, perdoch_efficient_2009}, %
    feature description~\cite{lowe_distinctive_2004, mikolajczyk_performance_2005,simonyan_learning_2014,
      winder_picking_2009,zagoruyko_learning_2015,han_matchnet_2015}, %
    approximate nearest neighbor search~\cite{silpa_anan_optimised_2008, muja_fast_2009}, %
    instance recognition~\cite{sivic_efficient_2009,nister_scalable_2006,
      philbin_object_2007,jegou_hamming_2008,bo_efficient_2009, jegou_aggregating_2012, tolias_aggregate_2013}, %
    face verification~\cite{chopra_learning_2005,huang_labeled_2007,berg_tom_vs_pete_2012,
      chen_blessing_2013,taigman_deepface_2014} %
    fine-grained recognition~\cite{parkhi_cats_2012,berg_poof_2013, gavves_local_2014}, %
    category recognition~\cite{lazebnik_beyond_2006,zhang_local_2006,
      mccann_local_2012,boiman_defense_2008,sanchez_compressed_2013}, %
    and convolutional neural networks~\cite{krizhevsky_imagenet_2012, razavian_cnn_2014,
      zagoruyko_learning_2015,han_matchnet_2015,arandjelovic_netvlad_2016}.

    At a high level, the main questions addressed by the aforementioned research can be summarized as: How should image
    features be detected? How should detected image features be represented? How much invariance should features
    have? Should image features be quantized? How should image features be matched? How should feature matches be
    aggregated? How should feature matches be scored? How should all of this be done accurately? How should all of
    this be done efficiently? Answers to these questions address many of the challenges to animal identification
    previously introduced in~\cref{sec:challenges}.

    This chapter summarizes literature relevant to addressing these questions in the context of animal
    identification. The outline of this chapter is as follows: \Cref{sec:featuredetect} will discuss feature
    detection. \Cref{sec:featuredescribe} will discuss feature description. \Cref{sec:ann} will discuss approximate
    nearest neighbor algorithms. \Cref{sec:ir,sec:cr,sec:fgr}, will discuss approaches to recognition.
    \Cref{sec:dcnn} will discuss convolutional neural networks.


\section{Image feature detection}\label{sec:featuredetect}

    Before an image can be analyzed, it must be broken down into smaller components. An image's visual appearance
    can be captured using a combination of local image patterns --- \glossterm{patch-based features}. The most
    informative patch-based features are typically centered on simple image structures such as junctions, corners,
    edges, and blobs~\cite{tuytelaars_local_2007}. If these features can be reliably detected, localized, and
    described then image matching can be posed as a problem in matching sets of features. This section describes
    work related to detecting features in an image, and~\cref{sec:featuredescribe} will discuss how detected
    features are then described.

    The region where a feature is detected is called a \glossterm{keypoint}. The simplest definition of a keypoint
    is just an $xy$-location in an image. However, images contain information at multiple scales; therefore a
    keypoint is typically associated with a scale. The scale of a keypoint is a non-negative real number that
    defines the level of detail at which to interpret the underlying image information. A keypoint with a scale can
    be thought of as a circular region with a radius that is the scale multiplied by some constant % 
    (\eg{} $3\sqrt{3}$ is the constant used to determine a keypoint's radius in~\cite{perdoch_efficient_2009}). To
    account for changes in viewpoint and pose, it is also common to augment features with an orientation and shape.
    Adding these properties is said to add invariance to the feature. Invariant features can provide similar
    descriptions of the same semantic image region under different viewing conditions. However, adding invariance
    can cause features to lose distinguishing information~\cite{mikolajczyk_comparison_2005,
    tuytelaars_local_2007, perdoch_efficient_2009, lowe_distinctive_2004}.

    Many detectors have been developed to detect patch-based feature keypoints~\cite{mikolajczyk_comparison_2005,
    tuytelaars_local_2007}. Algorithms such as Harris, SUSAN, and FAST detect corners~\cite{harris_combined_1988,
    mikolajczyk_indexing_2001, smith_susannew_1997, rosten_machine_2006}. Blobs and corners can be detected with
    Hessian~\cite{beaudet_rotationally_1978, lindeberg_shape_adapted_1994} or difference of
    Gaussians~\cite{gaussier_neural_1992, lowe_distinctive_2004} detectors. There are also region-based detectors:
    maximally stable extremal regions~\cite{matas_robust_2004}, saliency based
    methods~\cite{buoncompagni_saliency_based_2015} and superpixel-based methods~\cite{ren_learning_2003,
    mori_recovering_2004}. Some applications choose to skip keypoint detection and use a uniform grid of dense
    features~\cite{liu_sift_2008, revaud_deep_2015, iscen_comparison_2015}. Other applications, such as face
    recognition, use specialized keypoint detectors~\cite{dantone_real_time_2012, berg_tom_vs_pete_2012}. There
    currently exists no principled method for selecting the appropriate feature detector. Different feature
    detectors perform differently given the application~\cite{tuytelaars_local_2007}.

    This section describes the representation of an image over multiple scales, the detection of features to
    sub-pixel and sub-scale accuracy, and the adaption of features to specify orientation and shape. We focus on
    the Hessian-based keypoint because it has been experimentally shown to be a reliable choice for instance
    recognition~\cite{tuytelaars_local_2007}.

   \subsection{Scale-space}
        Scale-space theory describes image features as existing at multiple
        scales~\cite{lindeberg_scale_space_1993}. The same point on an object seen close up appears quite different
        compared to when it is at a distance. For example, a zebra in the distance may appear to have two stripes
        that are connected, but when the animal appears closer it becomes clear that the stripes are actually
        disconnected.
        %This problem is addressed by detecting features at multiple scales.
        Multi-scale detection is formalized by the theory of scale-space, which parameterizes a continuous signal,
        $f$, with a scale, $\scale$. The original signal is said to exist at scale $0$. Convolving the original
        signal with a Gaussian kernel produces coarser scales.

        Let $f$ be a continuous $2$-dimensional signal that defines an image. Let vector $\pt=\ptcolvec$ be a
        location in the image. The function $g(\scale)$
        %= \frac{1}{\TAU\scale^2} \exp{-(\vec{i} \cdot \vec{i})/2\scale^2}$
        is the isotropic 2D Gaussian kernel. The scale-space representation of a continuous image (for any non-zero
        scale) takes the form: $\img(\pt, \scale) = g(\scale) \conv f(\pt)$, where $\conv$ is the convolution
        operator. However, we do not have access to a continuous representation of an image. Therefore, in
        practice, the continuous Gaussian kernel is replaced with the discrete Gaussian kernel. This can be
        efficiently implemented as a discrete convolution with the $1$-dimensional discrete Gaussian kernel in the
        $x$-direction and then in the $y$-direction, because the discrete Gaussian kernel is separable in
        orthogonal directions~\cite{lindeberg_scale_space_1993}. Using the definition of an image at a single scale
        the next step is to represent an image at multiple scales.

       \paragraph{Gaussian pyramid}

           \newcommand{\downsamp}[2]{#1[::\tightpad#2,::\tightpad#2]}
            % See page 39 of Scale-space theory.
            %The continuous image signal is defined to be the zeroth scale $\img(\pt, 0) = \rawimg(\pt)$. 
            The discrete scale-space representation of an image is efficiently implemented using a Gaussian
            pyramid. A scale-space pyramid consists of $L$ levels. Each level covers an octave. Starting from the
            base of each level with scale parameter $\sigma$ the next octave is reached when $\sigma$ doubles.
            There are $s$ intervals represented within each octave. A Gaussian pyramid is illustrated
            in~\cref{fig:ScaleSpaceFigure}.

            \ScaleSpaceFigure{}

            The pyramid's base, %
            $\img(\pt, 1) = g( 1 ) \conv \rawimg(\pt)$ %
            is the $\ell=0\th{}$ level of the pyramid, and is computed by blurring the original image (sometimes
            with small initial blurring) with $\sigma=1$. Subsequent levels of the pyramid are produced by doubling
            sigma, thus the $\ell\th{}$ level of the pyramid is $\img(\pt, 2^\ell)$.

            A property of discrete scale-space is that after appropriate smoothing downsampling the image by half
            is equivalent to doubling sigma. Let
            % ---
            %$\img_\ell(\pt) = \downsamp{\rawimg}{2^{\ell}}({\pt} / {2^{\ell}})$ 
            $\img_\ell(\pt)$ %= \downsamp{\rawimg}{2^{\ell}}({\pt} / {2^{\ell}})$ 
            % ---
            denote the raw image downsampled by a factor of $2^{\ell}$ using Lanczos resampling. Now, each level of
            the pyramid can be written as %
            %$\img(\pt, 2^\ell) = g( 1 ) \conv \rawimg^\ell$. 
            $\img(\pt, 2^\ell) = g( 1 ) \conv \img_\ell(\pt)$. 
            Given the raw image at level $\ell$, the scale
            corresponding to $\sigma$ can be written as a relative scale
            % ---
            $\sigma_\ell = \sigma / 2^\ell$.
            % ---
            Thus, a discrete image at any scale can be efficiently computed as:
            \begin{equation}
                \img(\pt, \sigma) =
                    g(\sigma_\ell) \conv \img_\ell(\pt)
            \end{equation}
            Discrete convolution is applied using a window of size
              $\floor{6\sigma_\ell + 1} + (1 -
              (\modfn{\floor{6\sigma_\ell + 1}}{2}))$.
            Interpolation between discrete values of $\pt$ is used to
              sample intensity at sub-pixel accuracy.

            A scale between two levels of the pyramid is called an interval. Typically, $s$ intervals --- with
            relative scales $2^{0/s}, 2^{1/s}, \ldots 2^{s/s}$ --- are computed to represent the octave between
            level $\ell$ and $\ell + 1$. If differences between scales are needed, then the scales $2^{-1/s}$ and
            $2^{1 + 1/s}$ are also computed~\cite{lowe_distinctive_2004}.

    \subsection{Hessian keypoint detection}

        Hessian-based keypoint detection searches for extrema of the Hessian operator in both space and
        scale~\cite{beaudet_rotationally_1978, lindeberg_shape_adapted_1994}. The Hessian detector can
        qualitatively be viewed as a blob detector, but it also detects corners which may appear as blobs in
        scale-space~\cite{tuytelaars_local_2007}. The Hessian keypoint detector will compute a response value for
        each point in scale space indicating how blob-like each pixel is. The extrema of this response defines a
        set of Hessian keypoints. Post processing removes non-robust keypoints and localizes all other keypoints to
        sub-pixel and sub-scale accuracy.

        \paragraph{Hessian response}
            Let subscripts denote the partial derivatives of the image intensity (\eg{} $\img_{x}$ is the first $x$
            derivative, $\img_{xx}$ is the second $x$ derivative, and $\img_{xy}$ is the first derivative in both
            $x$ and $y$). The Hessian is a matrix of second order partial derivatives and is defined at each point
            in scale space.
            \begin{equation}
                \hessMAT = 
                \BIGMAT{
                \img_{xx}(\pt, \scale) & \img_{xy}(\pt, \scale) \\
                \img_{xy}(\pt, \scale) & \img_{yy}(\pt, \scale) } 
            \end{equation}\label{eqn:hessianmatrix}  
            %(derivatives are computed in scale-space over an integration scale %$\scale_I$). 

            The initial response of the detector at each point is the determinant of the Hessian matrix. This
            response is computed for level and every pixel in the scale-space pyramid. At coarser scales the
            Hessian response weakens, so to ensure that responses between scales are comparable, the initial
            response is scale normalized by multiplying with $\sigma^2$. (See~\cite{lindeberg_feature_1998} for
            more details about the choice of this normalization factor.) The extrema of this space defines a set of
            candidate keypoints, $\kpts'$.
            \begin{equation}
                \kpts' = \argextrema{\pt,\scale} \paren{\scale^2 \detfn{\hessMAT}} 
            \end{equation}
            %\devcomment{Is this 2D or 3D extrema detection.  What would need to change in my math if it was 3D?} 
            A point in this 3D space is a maxima/minima if its scale normalized value is greater/less than the
            scale normalized values of all its neighbors in the pyramid --- \ie{} the $8$ neighbors in its current
            interval, its $9$ neighbors in the next interval, and its $9$ neighbors in the previous interval.

        \paragraph{Edge filtering}
            Edge responses are not robust --- \ie{} the same point can not be localized reliably in two views of the
            same scene --- due to their elongated nature. Because of this, the extrema that appear too edge-like
            are filtered using a threshold $t_{\tt{edge}}$ which is compared to the ratio of the Hessian's squared
            trace and the determinant.
            \begin{equation}
                %\kpts = \{(\pt, \scale) \in \kpts' \where
                %  \frac{\trfn{\hessMAT}^2}{\detfn{\hessMAT}} > t_{\tt{edge}}\}
                \kpts = \left\{(\pt, \scale) \in \kpts' \bigwhere \frac{\trfn{\hessMAT}^2}{\detfn{\hessMAT}} > t_{\tt{edge}}\right\}
            \end{equation}

        \paragraph{Sub-pixel and sub-scale localization}
            To compensate for the discrete nature of pixel images, each keypoint detection is localized to
            sub-pixel and sub-scale accuracy. The importance of feature localization is demonstrated
            in~\cite{ke_pca_sift_2004}, where descriptors were computed on the normalized vectors of patch
            gradients using only principal component analysis (PCA)~\cite{jolliffe_principal_1986}. Despite the
            simplicity of the descriptors the authors were still able to effectively match two images due to the
            robust localization of the features.

            Sub-pixel and sub-scale localization transforms a keypoint $\kp_0$ into $\kp^*$ using an iterative
            process. At each iteration $i$, a $2\nd{}$ order Taylor expansion, centered at %
            $\kp_i = (\pt_i, \scale_i)$, approximates the scale normalized Hessian response: %
            $T_i(\pt, \scale) \approx \scale^2 \detfn{\hessMAT}$. The keypoint is updated to the position of the
            maximum response of the Taylor expansion: $\kp_{i + 1} = \argmax{\kp} T_i(\kp)$. This process iterates
            until convergence. If the process does not converge before a threshold number of iterations, the
            keypoint is deemed not robust and thrown out.

    \subsection{Affine adaptation}
        So far, the keypoints we have described correspond to circular regions where the pixel radius is some
        multiple of the scale. To account for small affine changes seen in non-planar objects (like zebras), the
        shape of each circular keypoint is adapted into an ellipse.

        An affine shape $\vmat=\vMATRIX$ is estimated (as a lower triangular matrix) for each keypoint using an
        iterative technique involving the second moment
        matrix~\cite{lindeberg_shape_adapted_1997,baumberg_reliable_2000,mikolajczyk_comparison_2005}. The affine
        shape matrix transforms an ellipse into a unit circle. Note that because the matrix is lower triangular one
        of its eigenvectors points in the downward direction. Thus, the shape has no influence on the orientation
        of the keypoint. For each point in scale space the second moment matrix is evaluated as:
        \begin{equation}\label{eqn:secondmoment}
                \momentmat 
                \tighteq 
                \MAT{ 
                \img_x^2(\pt, \scale)      & \img_x(\pt, \scale) \img_y(\pt, \scale) \\
                \img_x(\pt, \scale) \img_y(\pt, \scale) & \img_y^2(\pt, \scale) }
        \end{equation}

        The goal is to ``stabilize'' each keypoint shape by searching for the transformation, $\vmat^*$, that
        causes the second moment matrix to have equal eigenvalues. For each keypoint, its elliptical shape is
        initialized as a circle $\vmat_0=\eyetwo$. For each iteration $i$:

        \begin{enumln}

            \item Compute the second moment matrix, $\warpedmomentmat{\vmat_i}$, at the warped image patch.

            \item Check if the keypoint shape is stable. A keypoint shape is stable if the eigenvalue ratio of the
            second moment matrix is close to $1$. If the keypoint has been stable for two consecutive iterations,
            then accept $\vmat^* \leftarrow \vmat_{i}$ and stop iteration. Otherwise, if the number of iterations,
            $i$, is greater than some threshold, then stop and discard the keypoint.

            \item Update the affine shape  using the rule
                % ---
                $\vmat_{i + 1} = \sqrtm{\warpedmomentmat{\vmat_i}} \vmat_i$.
        \end{enumln}

        The matrix $\vmat$ only defines the transformation from an ellipse to a circle. The standard representation
        of an ellipse is a conic of the form $\mat{E} = \vmat^T\vmat$. This means that $\vmat$ is only defined up
        to an arbitrary rotation~\cite{mikolajczyk_comparison_2005,perdoch_efficient_2009}. Thus, we can freely
        rotate $\vmat$ into a lower triangular matrix. This ensures that one of its eigenvectors is pointing
        downwards --- \ie{} in the direction of the ``gravity vector''~\cite{perdoch_efficient_2009}. Making use of
        the gravity vector removes a dimension of invariance. To allow for the specification of keypoint
        orientation, the keypoint representation can be extended with a parameter $\theta$ that defaults to $0$.

    \subsection{Orientation adaptation}

        The keypoint orientation is defined using the parameter $\theta$.
        By default, the orientation of a keypoint can be assumed to be aligned with the ``gravity vector'' ---
          \ie{} $\theta=0$~\cite{perdoch_efficient_2009}.
        Otherwise, an orientation must be computed.
        A common method for determining a keypoint's orientation is to use the dominant gradient orientation.
        In theory adapting the orientation to match the dominant gradient will cause a computed keypoint
          description to be invariant to rotations.

        To compute a keypoint's dominant orientation the pixels around a keypoint vote into a fine-binned
        orientation histogram~\cite{lowe_distinctive_2004}. A pixel's vote is weighted by its gradient magnitude
        multiplied by its Gaussian weighted distance to the keypoint center. The dominant orientation %
        $\ori \in \rangeinex{0,\TAU}$ is chosen as the peak of this histogram. If there is more than a single peak
        it is common to create a copy of the keypoint for each maxima in this histogram. This process is
        illustrated in~\cref{fig:testfindkpdirection}.

        \testfindkpdirection{}

        \FloatBarrier{}

    \subsection{Discussion --- detector and invariance choices}
        To identify individual animals, features must be detected in distinguishing areas of an animal. For a
        feature to be useful, it must be detected in the multiple images of the same individual despite variations
        in viewpoint, pose, lighting, and quality. In our baseline algorithm we choose to use a Hessian based
        detector~\cite{perdoch_efficient_2009, lindeberg_feature_1998} because it generally produces a large number
        of features and has been experimentally shown to be repeatable, accurate, and adaptable to multiple degrees
        of invariance~\cite{tuytelaars_local_2007}.

        Once a keypoint is detected, it is described using a keypoint description algorithm. It is desirable for a
        keypoint description to be invariant to small changes in viewpoint, pose, and lighting. Accurate
        localization of a keypoint in scale and space helps to ensure that similar images contain similar features.
        Sometimes, it is beneficial to further localize a keypoint in shape and orientation, thus adding invariance
        to the feature. However, if too much invariance is used, it may not be possible to distinguish between
        semantically different features.

        It is a challenge to choose the correct level of invariance when computing features. Often an application
        chooses one of two extremes. Consider the computation of keypoint orientation. Standard methods for
        orientation invariance assume patches can freely rotate, when in fact they may be constrained to be
        consistent with the orientation of surrounding patches~\cite{lowe_distinctive_2004}. On the other side
        extreme is the ``gravity vector''~\cite{perdoch_efficient_2009}, which globally enforces all keypoint to
        have a downward orientation. This may be a safe assumption when working with features from images of rigid
        objects taken in an upright position. It may not be correct when dealing with non-rigid objects like
        zebras.
        %Orientation invariance assumes that orientation is a local property
        %  of a patch, but the orientation of a patch is usually not
        %  independent of its surrounding patches on an object.
        In our experiments in~\cref{sub:exptinvar} we test different degrees on invariance. This test includes a
        novel method that achieves a middle ground between full orientation invariance and the gravity vector.


\section{Image feature description}\label{sec:featuredescribe}  

    Once each feature has been localized its visual appearance must be described before it can be matched. The goal
    of feature description is to encode raw image data into a vector --- \ie{} a \glossterm{descriptor}. To
    represent the visual appearance of a keypoint a descriptor vector should have the following properties: (1) two
    visually similar patches produce vectors with a small metric distance and (2) visually dissimilar patches have
    vectors with large distances between them.

    Constructing such a descriptor vector has been a core problem throughout the history of computer vision.
    The first texture descriptor robust to small image transformations was the scale invariant feature
      transform (SIFT) descriptor first published in 1999~\cite{lowe_object_1999, lowe_distinctive_2004}.
    Since then, other hand-crafted algorithms have been proposed.
    However, results have always been at least comparable to the SIFT descriptor, and SIFT is still an effective
      and widely used hand-designed descriptors~\cite{mikolajczyk_performance_2005, calonder_brief_2010,
      bay_surf_2006, leutenegger_brisk_2011, alahi_freak_2012, jegou_triangulation_2014}.
    A promising direction for outperforming the SIFT descriptor is descriptor
      learning~\cite{simonyan_descriptor_2012, simonyan_learning_2014, winder_picking_2009}; specifically
      descriptor learning using deep neural networks~\cite{razavian_cnn_2014, bengio_representation_2013,
      russakovsky_imagenet_2014}.
    This section first describes the basic SIFT algorithm and then provides an overview of alternatives that have
      been proposed to SIFT{}.
    Work related to learning descriptor vectors using deep neural networks is discussed later in~\cref{sec:dcnn}.
      
    \subsection{SIFT}
        The {SIFT descriptor} is a $128$ dimensional vector that summarizes the spatial distribution of the
        gradient orientations in an image patch~\cite{lowe_distinctive_2004}. To describe a keypoint with a SIFT
        descriptor, the keypoint's image data is warped using the affine transform of the scale space gradient
        image into a normalized reference frame (typically $41 \times 41$ pixels). For a descriptor to be useful in
        matching it is important that the keypoint is properly localized before a descriptor is
        computed~\cite{ke_pca_sift_2004}. Because it is not always possible to perfectly localize a keypoint, the
        SIFT descriptor aggregates information into a soft-histogram. Allowing data to contribute to multiple bins
        helps the SIFT descriptor to be robust to small localization errors and viewpoint variations. Distance
        between two SIFT descriptors is typically computed using the Euclidean distance. The SIFT descriptor of a
        patch is visualized in~\cref{fig:vizfeatrow}.

        The structure of a SIFT descriptor is as follows: A $4\times4$ regular grid is superimposing over the
        normalized patch. Each of the $16$ spatial grid cells contains an orientation histogram discretized into
        $8$ bins. The SIFT descriptor is the concatenation of all orientation histograms, resulting in a single %
        $16 \times 8 = 128$ dimensional vector.

        The patch information populates the SIFT descriptor as follows: For every pixel, the patch gradient (the
        derivative in the $x$ and $y$ direction) is computed. Next, each pixel computes its gradient magnitude and
        orientation. Each pixel then casts a weighted vote. The bin that a pixel votes into is computed from its
        $xy$-location and gradient orientation. The weight of a pixel's vote is based on its gradient magnitude and
        Gaussian weighted distance to the patch center. To be robust to small localization errors, a pixel's vote
        is split via trilinear interpolation ($x$-location, $y$-location, and orientation) into the orientation
        histograms of the pixel's nearest grid cells as well as neighboring orientation bins in each grid cell's
        orientation histogram.

        Once voting is completed a SIFT descriptor is normalized to account for lighting differences between
        images. First, the vector is L2-normalized to unit length, which makes the descriptor invariant to linear
        changes in intensity. Then, a heuristic --- that truncates each dimension to a maximum value of $0.2$ ---
        is applied to increase robustness to non-linear changes in illumination. Finally, the vector is
        renormalized.

        For storage considerations the resulting $512$-byte floating-point (float32) descriptor is typically cast
          as an array of unsigned $8$-bit integers (uint8), resulting in a $128$-byte descriptor.
        To reduce the impact of this quantization a trick is to multiply by $512$ instead of $255$ and then
          truncate values to $255$ before converting from a float to a uint8.
        Even though each component is $8$-bits and therefore can only store a maximum value of $255$, value
          overflow is not likely to occur because of truncation, L2-normalization, and properties of natural
          images.
    
       \vizfeatrow{}

    \subsection{Other descriptors and SIFT extensions}
        Even more than a decade after its original publication, SIFT remains a popular descriptor for patch-based
        matching because it is versatile, unsupervised, widely available, and easy to use. The principles used to
        guide the construction of the SIFT descriptor --- particularly the use of aggregated gradients --- have
        inspired many variants, extensions, and new techniques~\cite{mikolajczyk_performance_2005,
        dalal_histograms_2005, bay_surf_2006}. Hand crafted alternatives to SIFT have been developed that are
        faster to compute and more efficient to store, but these alternatives do not significantly outperform
        SIFT's matching accuracy on general data~\cite{lowe_distinctive_2004, mikolajczyk_performance_2005,
        alahi_freak_2012}. This subsection provides a brief overview of these alternatives.

        % ALTERNATIVES FOR DETECTION
        The use of aggregated gradient information in SIFT has been adapted for use in other computer vision
        problems such as detection and scene classification.
        % GIST
        The GIST descriptor is a low dimension descriptor used for scene classification that coarsely summarizes
        rough appearance of an entire image~\cite{oliva_modeling_2001, douze_evaluation_2009}.
        % HOG
        The histogram of oriented gradients (HoG) descriptor is a high dimensional descriptor used in detection.
        The HoG descriptors describes the shapes of objects in an image~\cite{dalal_histograms_2005}. Like the SIFT
        descriptor, the HoG descriptor illustrates the value of gradient-based image descriptions and has inspired
        extensions such as the discriminatively trained parts model~\cite{felzenszwalb_object_2010}.

        As a general single-scale patch-based descriptor, the matching accuracy of SIFT has not been significantly
        outperformed on general datasets.
        % GLOH
        One attempt at an improved general descriptor is the gradient location-orientation histogram (GLOH)
        descriptor~\cite{mikolajczyk_performance_2005}. GLOH uses a similar structure to SIFT but replaces the
        rectangular-bins with log-polar bins. GLOH did achieve higher matching accuracy on some datasets, but it
        was not by a significant margin.
        Despite the lack of generic success, hand-crafted SIFT variants have been successful when applied to
        specific tasks.
        % COLORED SIFT
        Colored SIFT variants such as opponent-SIFT are valuable in category recognition tasks, where a color
        difference could be the distinguishing factor between categories~\cite{van_de_sande_evaluating_2010}
        % SCALE-LESS SIFT
        Combining multiple SIFT descriptors over different scales has also shown moderate improvements. The
        scale-less SIFT descriptor combines SIFT descriptors computed at multiple scales into a single descriptor.
        It has been shown to produce more accurate dense correspondences than representing each scale with an
        individual descriptor~\cite{hassner_sifts_2012}.

        %Despite However, domain specific modifications have shown promising results.
        %The Rotation Invariant Feature Transform (RIFT) descriptor~\cite{lazebnik_sparse_2005} uses concentric
        %  circles to make a similar modification.
        %The RIFT descriptor are used in texture classification~\cite{lazebnik_sparse_2005}.

        % SURF
        Efficiency is one area where SIFT has been significantly outperformed.
        An approximation to SIFT called speeded up robust features (SURF) is a fast approximation to SIFT
          based on integral images that achieves similar accuracy using a smaller $64$ dimensional
          descriptor~\cite{bay_surf_2006}.
        % DAISY
        The DAISY descriptor uses a similar binning structure to GLOH, but uses convolutions with Gaussian
          kernels to quickly aggregate gradient histograms~\cite{tola_fast_2008}.
        % BINARY PATTERNS
        Binary descriptors such as local binary patterns (LBP)~\cite{ojala_comparative_1996, zhang_local_2010},
          local derivative patterns~\cite{heikkila_description_2009}, and their variants such as
          BRIEF~\cite{calonder_brief_2010}, BRISK~\cite{leutenegger_brisk_2011}, and FREAK~\cite{alahi_freak_2012}
          also quickly compute compact distinctive descriptors.
        Binary descriptors are built using multiple pairwise comparisons of average image intensity at
          predetermined locations.
        This results in a small descriptor that effectively represents aggregated gradient information.

        Machine learning is able to outperform the matching accuracy of SIFT, however these techniques require
        training data to adapt to each new problem domain. Learned descriptors make use of the same aggregated
        gradient information used in the construction of SIFT descriptors. The Liberty, Yosemite, and Notre-Dame
        buildings datasets are standard datasets for descriptor learning~\cite{brown_discriminative_2011}. Error on
        these datasets is measured using false positive rate at $95\percent$ recall (FPR95). The baseline SIFT
        error on this dataset is $27.02\percent$. The configuration of a DAISY descriptor is learned
        in~\cite{winder_picking_2009} and achieves an error of $15.16\percent$ on the buildings datasets.
        In~\cite{simonyan_learning_2014}, large scale non-convex optimization is used to learn a spatial pooling
        configuration of log-polar bins, a dimensionality reduction matrix, and a distance metric to further reduce
        the FPR95 error to $10.98\percent$. The current state-of-the-art error of $4.56\percent$ on the buildings
        dataset is achieved using a convolutional neural network~\cite{zagoruyko_learning_2015}.

    \subsection{Discussion --- descriptor choices}
        In our application we use the SIFT~\cite{lowe_distinctive_2004} as our
        baseline descriptor because it is one of the most widely used and well-known descriptors. SIFT describes
        images patches in such a way that small localization errors do not significantly impact the resulting
        representation. Exploration of alternative convolutional descriptors is discussed later in~\cref{sec:dcnn}.


\section{Approximate nearest neighbor search}\label{sec:ann}  

    In computer vision applications it is often necessary to search a database of high dimensional
    vectors~\cite{shakhnarovich_nearest_neighbor_2006, datar_locality_sensitive_2004, muja_fast_2009,
    kulis_kernelized_2012, weiss_spectral_2009}. Patch descriptor vectors like SIFT are constructed such that the
    distance (under some metric) between vectors is small for matching patches and large for non-matching patches.
    Thus, finding matching descriptor vectors is often framed as a nearest neighbor search
    problem~\cite{lowe_distinctive_2004}. It becomes prohibitively expensive to perform exact nearest neighbor
    search as the size of the database increases. Therefore, approximate algorithms --- which can trade off a small
    amount of accuracy for substantial speed-ups --- can be used instead.

    \subsection{Kd-tree}

        A \glossterm{kd-tree} is a data structure used to index high dimensional vectors for fast approximate
        nearest neighbor search~\cite{bentley_multidimensional_1975}. A kd-tree is an extension of a binary tree to
        multiple dimensions. Each non-leaf node of the tree is assigned a dimension and threshold value. The node
        splits data vectors between the left and right children by comparing the value of the data vector at the
        assigned dimension to the assigned threshold.

        \paragraph{Building a kd-tree index}
        Indexing a set of vectors involves first choosing a dimension and threshold to split the data into two
        partitions. Then this procedure is recursively applied to each of the partitions. A common measure for
        choosing the dimension is to choose the dimension with the greatest variance in the data. The threshold is
        then selected as the median value of the chosen dimension.

        \paragraph{Augmenting a kd-tree index}
        It is possible to augment an existing kd-tree by adding and removing vectors. Addition of a vector to a
        kd-tree is performed by appending the point to its assigned leaf. Removal of points from a kd-tree is done
        using lazy deletion --- \ie{} by masking the removed data. To avoid tree imbalance, a kd-tree is re-indexed
        after the number of points added or removed passes a threshold. Any masked point is deleted whenever the
        tree is re-indexed.

        \paragraph{Searching a kd-tree index}
        Searching for a query point's exact nearest neighbor in a kd-tree has been shown to take expected
        logarithmic time for low ($k < 16$) dimensional data~\cite{friedman_algorithm_1977}. However, for higher
        dimensional data this same method takes nearly linear time~\cite{sproull_refinements_1991}. This is because
        a query point and its nearest neighbor might be on opposite sides of a partition. Therefore, searching for
        nearest neighbors is typically done by approximate search using a priority queue~\cite{beis_shape_1997}. A
        priority queue orders nodes to further search based on their distance to the query vector. The search
        returns the best result after a threshold number of checks have been made. % $ ~\cite{beis_shape_1997}.

        Search accuracy is improved by using multiple randomized kd-trees~\cite{silpa_anan_optimised_2008}. If a
        single kd-tree has a probability of failure $p$, then $m$ independently constructed trees have a $p^m$
        probability of failure. For each kd-tree a random Householder matrix is used to efficiently rotate the
        data. Using a random rotation preserves distances between rotated vectors but does not preserve the
        dimension of maximum variance. This means that each of the $m$ kd-trees yields a different partitioning of
        the data, although it is not guaranteed to be independent. When searching multiple random kd-trees, a
        single priority queue keeps track of the next nearest bin boundaries to search over all the trees.

    \subsection{Hierarchical k-means}
        Another tree-based method for approximate nearest neighbor search is the hierarchical k-means. Each level
        in the hierarchical k-means tree partitions the data using the k-means algorithm~\cite{lloyd_least_1982}
        with a small value of $k$ (\eg{} 3). To query a new point it moves down the tree into the bin of the
        closest centroid at each level until it reaches a leaf node. Hierarchical k-means was one of the first
        techniques used to define a visual vocabulary~\cite{nister_scalable_2006} --- a structure used for indexing
        and quantizing large amounts of descriptors.
    
    \subsection{Locality sensitive hashing}
        A hashing-based method for approximate nearest neighbor search is locality-sensitive hashing (LSH). This
        method is able to search a dataset of vectors for approximate nearest neighbors in sub-linear
        time~\cite{charikar_similarity_2002, datar_locality_sensitive_2004, kulis_fast_2009, kulis_kernelized_2012,
        tao_locality_2013}. LSH trades off a small amount of accuracy for a large query speed-up. A database is
        indexed using $M$ hash tables. Each hash table uses its own randomly selected hash function. For each hash
        table, a query vector computes its hash and adds the database vectors it collided with to a shortlist. The
        shortlist is sorted by distance and returned as the approximate nearest neighbors.

    \subsection{FLANN}
        The fast library for approximate nearest neighbors (FLANN) is a software package built to quickly index and
        search high dimensional vectors~\cite{muja_fast_2009}. The FLANN package implements efficient algorithms
        for hierarchical k-means, kd-trees, and LSH{}. It also implements a hybrid between the k-means and kd-tree,
        as well as configuration optimization, to select the combination of algorithms that best reaches the
        desired speed/accuracy trade-off for a given dataset. Configuration optimization is performed using the
        Nelder-Mead downhill simplex method~\cite{nelder_simplex_1965} with cross-validation.

    \subsection{Product quantization}
        Product quantization is a method for speeding up approximate nearest neighbor search of a set of high
        dimensional vectors~\cite{jegou_product_2011,ge_optimized_2013}. Each vector is split up into a set of
        sub-vector components. For each component, the sub-vectors are separately quantized using a
        codebook/dictionary/vocabulary. The pairwise squared distances between centroids in the vocabulary are
        stored in a lookup table. To comparing the distance between two vectors first each vector is split into
        sub-vectors, next the sub-vectors are quantized, and then the squared distances between quantized
        sub-vectors are read from the lookup table. The approximated squared distance between these two vectors is
        the sum of the squared distances between the quantized sub-vectors.

    \subsection{Discussion --- choice of approximate nearest neighbor algorithm}
        In our single annotation identification algorithm a query descriptor searches for its nearest neighbor in a
        database containing all descriptors from all \exemplars{}. Each annotation contains \OnTheOrderOf{4}
        features, which are described with $128$-component SIFT descriptors. Searching exact nearest neighbors
        becomes prohibitive when hundreds or thousands of images are searched. Thus, we turn towards approximate
        nearest neighbor algorithms. In this \thesis{} all of our approximate nearest neighbors are found using the
        multiple kd-tree implementation in the FLANN package~\cite{muja_fast_2009}. Using the configuration
        optimization built into the FLANN package, we have found that multiple kd-trees provide more accurate
        feature matches for our datasets than those computed by hierarchical k-means trees or LSH{}. In addition to
        being fast and accurate, multiple kd-trees support efficient addition and removal of points, which is
        needed in a dynamic setting~\cite{silpa_anan_optimised_2008}.


\section{Instance recognition}\label{sec:ir}
    There are many variations on the problem of visual recognition such as: specific object recognition (\eg{}
    CD-covers)~\cite{lowe_distinctive_2004, sivic_efficient_2009, nister_scalable_2006},
      % --
    location recognition~\cite{jegou_hamming_2008,jegou_aggregating_2012,tolias_aggregate_2013},
      % --
    person re-identification~\cite{shi_embedding_2016,karanam_person_2015,wu_viewpoint_2015},
      % --
    face verification/recognition~\cite{chopra_learning_2005, huang_labeled_2007, berg_tom_vs_pete_2012,
    chen_blessing_2013, taigman_deepface_2014, schroff_facenet_2015},
      % --
    category recognition~\cite{lazebnik_beyond_2006,zhang_local_2006,mccann_local_2012,boiman_defense_2008},
      % --
    and fine-grained recognition~\cite{parkhi_cats_2012,berg_poof_2013, gavves_local_2014}.
      % --
    The different types of recognition problems lie on a spectrum of specificity \wrt{} the objects they attempt to
    recognize. On one end of the spectrum, \glossterm{instance recognition} techniques --- like scene recognition
    or face verification --- search for matches of the same exact object. On the other end of the spectrum category
    recognition algorithms --- like car, bird, dog, and plane detectors --- look for the same type of objects.
    Other problems sit --- like fine-grained recognition where the goal might be to recognize specific subspecies
    of dog (\eg{} German shepherd, golden retriever, boxer, beagle, \ldots{}) --- somewhere in the middle. Animal
    identification is closest to the instance recognition side of the spectrum, but the proposed solution draws
    upon techniques from other forms of recognition.

    The discussion in this section focuses on instance recognition.
    The next two sections will discuss category recognition and fine-grained recognition.

    \subsection{Spatial verification}\label{subsec:sverreview}
        Before discussing specific techniques in instance recognition, we describe work related to spatial
        verification. Most instance recognition techniques initially match local image features without using any
        spatial information~\cite{lowe_distinctive_2004, sivic_efficient_2009, philbin_object_2007,
        tolias_image_2015}. This results in pairs of images with spatially inconsistent feature correspondences.
        Spatially inconsistent matches are illustrated in~\cref{fig:figSVInlier}. Inconsistent features are removed
        using \glossterm{spatial verification}, a process based on the random sample consensus (RANSAC)
        algorithm~\cite{fischler_random_1981}.

        RANSAC has come to refer to a family of iterative techniques to sample inliers from a noisy dataset that
        are consistent with some model~\cite{fischler_random_1981, hartley_multiple_2003, chum_locally_2003,
        raguram_usac_2013}. In the context of spatial verification the model is an affine transformation matrix,
        and the dataset is a set of feature correspondences~\cite{lowe_distinctive_2004, sivic_video_2003,
        philbin_object_2007, chum_total_2011, arandjelovic_three_2012}. At each iteration of RANSAC a small subset
        of points is sampled from the original dataset and used to fit a hypothesis model. All other data points
        are tested for consistency with the hypothesis model. A score is assigned to the hypothesis model based on
        how well the out of sample data fit the model (\eg{} the number of transformed points that are within a
        threshold distance of their corresponding feature). After a certain number of iterations the process stops
        and returns the hypothesized model with the highest score as well as the inliers to that model.

        When RANSAC returns a large enough set of inliers (\wrt{} some threshold), the hypothesis model it is
        generally considered to be a ``good fit''. In this case a more complex model --- that may be more sensitive
        to outliers --- can be fit. In spatial verification, it is common to use the RANSAC-inliers to estimate a
        homography transformation~\cite[311--320]{szeliski_computer_2010}. The homography is then used to estimate
        a new set of refined inliers, and these are returned as the spatially verified feature correspondences.

        \figSVInlier{}

        \FloatBarrier{}

    \subsection{Lowe's object recognition}

        Lowe's introduction of SIFT descriptors includes an algorithm for recognizing objects in a training
        database and serves as an instance recognition baseline~\cite{lowe_distinctive_2004}. A single kd-tree
        indexes all database image descriptors. Approximate nearest neighbor search of the kd-tree is performed
        using the best-bin-first algorithm~\cite{beis_shape_1997}. For a query image, each keypoint is assigned to
        its nearest neighbor as a match. The next nearest neighbor (belonging to a different object) is used as a
        normalizer --- a feature used to measure the distinctiveness of a match. Any match with a ratio of
        distances to the match and the normalizer greater than threshold $t_{\tt{ratio}} \teq 0.8$ is filtered as
        not distinctive. Features likely to belong to the same object are clustered using a Hough Transform, and
        then clusters of features are spatially verified with a RANSAC approach~\cite{fischler_random_1981}.


    \subsection{Bag-of-words instance recognition}\label{subsec:bow}

        One of the most well-known techniques in instance recognition is the \glossterm{bag-of-words} model
        introduced to computer vision by Sivic and Zisserman~\cite{sivic_video_2003, sivic_efficient_2009}. The
        bag-of-words model addresses instance recognition using techniques from text retrieval. An image is cast as
        a text document where the image patches (detected at keypoints and described with SIFT) are the words. The
        concept of a visual word is formalized using a visual vocabulary. A \glossterm{visual vocabulary} is
        defined by clustering feature descriptors traditionally constructed using k-means~\cite{lloyd_least_1982}
        (however more recent methods have learned vocabularies using neural
        networks~\cite{arandjelovic_netvlad_2016}). The centroids of the clusters represent the \glossterm{visual
        words} in the vocabulary. These centroids are used to quantize descriptor space. A feature in an image is
        assigned (quantized) to the visual word that is the feature's approximate nearest neighbor in the
        vocabulary. This means that each descriptor vector can be represented using just a single number --- \ie{}
        its index in the vocabulary. Vocabulary indices are used to construct an inverted index, which allows
        multiple feature correspondences to be made using a single lookup.

        Given a visual vocabulary, the bag-of-words algorithm consists of two high level steps: (1) an offline
        indexing step and (2) an online search step. The offline step indexes a database of images for fast search.
        First each descriptor in each database image is assigned to its nearest visual word. An inverted index is
        constructed to map each visual word to the set of database features assigned to that word. Each database
        feature is assigned a weight based on its term frequency (tf). Finally, each word in the vocabulary is
        assigned a weight based on its inverse document frequency (idf). The online step searches for the images in
        the database that are visually similar to the query image. First, each descriptor in the query image is
        assigned to its visual word, and the term frequency of each visual word in the query image is computed.
        Then, the inverted index is used to build a list of all images that share a visual word with the query. For
        each matching image, the sum of the tf-idf scores of the corresponding features is used as the image score.
        Finally, the ranked list of images is returned. These steps are now described in further detail.

        \paragraph{The inverted index}
            The visual vocabulary allows for a constant length image representation. An image is represented as a
            histogram of visual words called a bag-of-words. A bag-of-words histogram is sparse because each image
            contains only a handful of words from a vocabulary. The sparsity of these vectors allows for efficient
            indexing using an inverted index. An inverted index maps each word to the database images that contain
            the word. Therefore, when a feature in a new query image is quantized the inverted index looks up all
            the database features that it matches to. A new feature correspondence is created for each database
            feature the inverted index maps to. For each correspondence a feature matching score is computed.
            Because all the word assignments and feature correspondences are known, the scores of all matching
            images can be efficiently computed by summing the scores of their respective feature correspondences.

        \paragraph{Vocabulary tf-idf weighting}
            Each word in the database is weighted by its inverse document frequency (idf), and each individual
            descriptor is weighted by its term-frequency (tf)~\cite{sivic_efficient_2009}. The idea behind the idf
            weight is that words appearing infrequently in the database are discriminative and should receive
            higher weight. The idea behind the tf weight is that words occurring more than once in the same image
            are more important.

        \paragraph{Formal bag-of-words scoring}
            Let $\X$ be the set of descriptor vectors in an image. We also use $\X$ to refer to the image in
            general. Descriptor space is quantized using a visual vocabulary where $\C$ is the set of word
            centroids and $w_\c$ is the weight of a specific word. Let $\X_\c \subset \X$ be the set of descriptors
            in an image assigned to visual word $\c$. Let $q(\x)$ be the function that maps a vector to a visual
            word. We overload notation to also let $q(\X)$ map a set of descriptors into a set of visual words.

            The tf-idf weighting of a single word $\c$ in the vocabulary is computed as follows: Let $N$ be the
            number of images in the database. Let $N_\c$ be the number of images in the database that contain word
            $\c$. $\card{\X}$ is the number of descriptors in an image, and $\card{\X_\c}$ is the number of
            descriptors quantized to word $\c$ in that image. The idf weighting of word $\c$ is:
            \begin{equation}
                w_\c = \opname{idf}(\c) = \ln{N/N_\c}
            \end{equation}
            The tf weighting of a word $\c$ in an image $\X$ is:
            \begin{equation}
                \opname{tf}(\X, \c) = \frac{\card{\X_\c}}{\card{\X}}
            \end{equation}

            Similarity between bag-of-words vectors is computed using the weighted cosine similarity. It is only
            necessary to sum the scores of matching features because the weight of a word that is not in both a
            query and database image is $0$. The tf-idf similarity between two images can be written as
            \begin{equation}
                \opname{sim}(\X, \Y) = \sum_{\c \in q(\X) \isect q(\Y)} \opname{tf}(\X, \c) \opname{tf}(\Y, \c) \opname{idf}(\c) 
            \end{equation}
            or equivalently
            \begin{equation}
                \opname{sim}(\X, \Y) = \frac{1}{\card{\X}\card{\Y}}\sum_{\c \in \C} w_\c \sum_{\xdesc \in \X_\c} \sum_{\ydesc \in \Y_\c} 1
            \end{equation}
            The second formulation unifies the bag-of-words model with other vocabulary-based methods in the SMK
            framework, which will be discussed later in~\cref{sec:smk}.

        \paragraph{Extensions to bag-of-words}
            The main strength and the main weakness of vocabulary-based matching is its use of quantization.
            Quantization allows for large databases of images to be searched very
            rapidly~\cite{nister_scalable_2006}. However, quantization causes raw descriptors to lose much of their
            discriminative information~\cite{philbin_lost_2008, boiman_defense_2008}. When a high-dimensional
            feature vector is quantized, it only encodes the presence of a word in a single number. This number is
            as descriptive as the partitioning of descriptor space, which is quite coarse in $128$ dimensions, even
            with a large vocabulary. Several methods have been developed to help reduce errors caused by
            quantization.

            Soft-assignment helps avoid quantization errors by mapping each raw descriptor to multiple
            words~\cite{philbin_lost_2008}. Another way to reduce quantization error is to use a finer partitioning
            of descriptors space~\cite{philbin_object_2007}. Approximate hierarchical clustering and approximate
            k-means have been used to build vocabularies with up to $1.6 \times 10^7$
            words~\cite{nister_scalable_2006, philbin_object_2007, mikulik_learning_2010}. Alternative similarity
            measures for descriptor quantization are also explored in~\cite{mikulik_learning_2010}. A projection
            matrix for SIFT descriptors is learned in~\cite{philbin_descriptor_2010} to preserve information that
            would be lost in quantization.

            Because the tf-idf weighting was originally designed for text recognition, it does not take into
            account challenges that occur in image recognition such as bursty features --- a single feature that
            appears in an image with a higher than term expected frequency (\eg{} bricks on a wall or vertical
            stripes on a zebra). Strategies for accounting for burstiness involve penalizing frequently occurring
            features by removing multiple matches to the same feature, using inter-image normalization, and using
            intra-image normalization.~\cite{jegou_burstiness_2009}.
            
            Query Expansion is a way to increase the recall of retrieval techniques and recover from tf-idf
            failure~\cite{chum_total_2007, chum_total_2011, arandjelovic_three_2012, tolias_visual_2014}. After an
            initial query, all spatially verified feature correspondences are back-projected onto the query image.
            Then the query is re-issued. A model of ``confusing features'' --- features more likely to belong
            to the background --- can be used to filter out matches that should not be back projected onto the
            query image. Query expansion enriches the query with intermediate information that may help retrieve
            other viewpoints of the query image. However, because this technique requires at least one correct
            result in the ranked list, it only improves recall for queries that already have high accuracy.

            One method to improve the performance of bag-of-words search is to remove non-useful features. It is
            found that as few as $4\percent$ of the features can be used in location recognition without loss in
            accuracy~\cite{turcot_better_2009}. This related work defines a useful feature as one that is robust
            enough to be matched with a corresponding feature and stable enough to exist in multiple viewpoints.
            Thus, these useful features are computed as those that produced a spatially verified match to a correct
            image. Any feature that does not produce at least one spatially verified match is removed. Removing
            non-robust features both saves space and improves matching accuracy.

    \subsection{Min hash}
        Min-hashing is the analog of locality-sensitive hashing for sets. Min-hashing has been applied as an
        instance recognition technique for near-duplicate image detection~\cite{chum_near_2008}, logo
        recognition~\cite{romberg_bundle_2013}, large scale image search~\cite{wang_semi_supervised_2012}, scene
        recognition~\cite{zhang_image_2011}, and unsupervised object discovery~\cite{chum_geometric_2009,
        chum_large_scale_2010}.

        The basic idea is to represent an image as a set of hashes based on permutations of a visual vocabulary.
        Recognition is accomplished by performing a lookup for each hash. Collisions are returned as the recognition
        results. Like LSH, the primary advantage of using min hash for instance recognition is its speed.

    \subsection{Hamming embedding}
        Hamming embedding is an extension of the bag-of-words framework that reduces the information lost in
        quantization by assigning each descriptor a small binary vector~\cite{jegou_hamming_2008,
        jegou_burstiness_2009, jegou_improving_2010}. Each visual word $\c$, is assigned a $d_b \times d$ random
        orthogonal projection matrix $\mat{P}_\c$, where $d$ is the number of descriptor dimensions and $d_b$ is
        the length of the binary code. A set of $d_b$ thresholds, $\vec{t}_\c \in \Real^{d_b}$, is pre-computed for
        each word using the descriptors used to form the visual word cluster. These descriptors are projected using
        the word's random orthogonal matrix, and the median value of each dimension is chosen as that dimension's
        threshold.

        When any descriptor, $\desc$, is assigned to a word $\c$ it is also assigned a binary Hamming code,
        $\vec{b}$. To compute the binary Hamming code the descriptor is projected using the word's orthogonal
        matrix, $\vec{b}' = \mat{P}_\c \desc$, and then each dimension is binarized based on a threshold, %
        $b_i = (b'_i > t_{\c{}i})$.

        When a query descriptor, $\desc$, is assigned to a word, $\c$, it is matched to all database descriptors
        belonging to that word. Each match is then assigned a score. First, the Hamming distance, $h_d$, is
        computed between the binary signature of the query and database descriptors. If the Hamming distance of the
        match is not within threshold, $h_t$, the score of the match is $0$ and does not contribute to bag-of-words
        scoring. Otherwise, the score is the word's squared idf weight multiplied by a Gaussian falloff based on
        the Hamming distance. Using the inverted index, each image is scored by summing the scores of the
        descriptors that matched that image. The image scores are used to define a ranked list of results.
        %In~\cite{jegou_hamming_2008} only the idf weight is used 
        %In~\cite{jegou_burstiness_2009} squared idf and the Gaussian falloff
        %In~\cite{jegou_improving_2010} squared idf and the probability having a Hamming distance lower than or equal to a.
        %\begin{equation}
        %w_d(h_d) = -\log_2\paren{2^{-d_b} \sum_{i = 0}^{h_d} \binom{i}{d_b}}
        %\end{equation}

    \subsection{Fisher vector}
        A Fisher vector is an alternative to a bag-of-words~\cite{perronnin_large_scale_2010_1,
        jegou_aggregating_2010}. Like bag-of-words, Fisher vector representations have been used in both instance
        and category recognition~\cite{perronnin_fisher_2007, cinbis_image_2012, sun_large_scale_2013,
        sanchez_image_2013, juneja_blocks_2013, douze_combining_2011, ma_local_2012, murray_generalized_2014,
        gosselin_revisiting_2014}. Instead of training discrete visual vocabulary using the cluster centers of
        k-means, a Fisher vector encoding uses a continuous Gaussian mixture model (GMM). The number of Gaussian
        components in the GMM is  similar to the number of words in a vocabulary. An image is encoded using the GMM
        by computing the likelihood of each feature with respect to the GMM{}. Likelihoods for different components
        of the GMM are aggregated using a soft-max function. Often, each component of this vector $\vec{v}$ is then
        power law normalized with fixed constant $0 \leq \beta < 1$. Power law normalization is a simple post
        processing method written as $v_i = \txt{sign}\paren{v_i}\abs{v_i}^\beta$~\cite{jegou_aggregating_2012}.
        Fisher vectors produce a much richer representation than normal bag-of-words vector because each descriptor
        is assigned to a continuous mixture of words rather than a single word.

        It is noted in~\cite{perronnin_large_scale_2010_1} that using Fisher vectors for instance recognition is
        similar to tf-idf. Normalized Fisher vectors down-weight frequently occurring GMM components --- \ie{}
        words with low idf weights. Furthermore, Fisher vector representations are well suited for compression,
        which allows scaling to large image collections.

    \subsection{VLAD --- vector of locally aggregated descriptors} 

        A vector of locally aggregated descriptors (VLAD) is similar to a Fisher vector descriptor --- in fact it
        is a discrete analog of a Fisher vector~\cite{jegou_aggregating_2010, jegou_aggregating_2012}. Like Fisher
        vectors, VLAD has been used in the context of both instance and category
        recognition~\cite{jegou_negative_2012, delhumeau_revisiting_2013, arandjelovic_all_2013}. VLAD still
        computes a visual vocabulary and assigns each feature to its nearest word, but instead of only recording
        presence or absence of a word, each feature computes the residual vector from the centroid of its assigned
        word. The residual vectors are summed to produce one constant length vector per word. All summed residuals
        are concatenated to produce a constant length image representation. Aggregation of the residual vectors
        allows for an accuracy similar to bag-of-words methods to be obtained, but using a smaller vocabulary
        ($\OnTheOrderOf{1} - \OnTheOrderOf{2}$ words). Like Fisher vectors, VLAD descriptors are also power-law
        normalized~\cite{jegou_aggregating_2012}.

        There have been many extensions of the VLAD descriptor. The value of PCA, whitening, and negative evidence
        was shown in~\cite{jegou_negative_2012}. The MultiVLAD scheme is inspired by~\cite{torii_visual_2011}, and
        allows for retrieval of smaller objects that appear in larger images~\cite{arandjelovic_all_2013}. The
        basic idea is that VLAD descriptors are tiled in $3 \times 3$ grids. An integral
        image~\cite{viola_robust_2004} of unnormalized VLAD descriptors is used to represent many possible tiles.

        A vocabulary adaptation scheme is also introduced in~\cite{arandjelovic_all_2013}. The vocabulary is
        updated when a new image is added to the VLAD inverted index. This is performed by updating any word
        centroid $\c$ to $\c'$, where $\c'$ is the average of all the descriptors currently assigned to that
        word. The residuals of the affected words are recomputed and re-aggregated into updated VLAD descriptors.

        Recently, NetVLAD --- a convolutional variant of the VLAD descriptor --- has been
        introduced~\cite{arandjelovic_netvlad_2016,radenovic_cnn_2016}. NetVLAD uses deep learning with a triplet
        loss function to simultaneously learn both the patch-based descriptors and the vocabulary. This
        convolutional approach shows large improvements (a $19\percent$ improvement on Oxford 5k) over previous
        state-of-the-art image retrieval techniques.

    \subsection{SMK --- the selective match kernel}\label{sec:smk}
        The selective match kernel (SMK) encapsulates the vocabulary-based techniques such as bag-of-words, Hamming
        embedding, VLAD, and Fisher vectors into a unified framework~\cite{bo_efficient_2009,
        tolias_aggregate_2013, tolias_image_2015, jegou_triangulation_2014}. SMK provides a framework that
        ``bridges the gap'' between matching-based (here a match refers to a feature correspondence) approaches and
        aggregation-based approaches. The scores of matching-based approaches such as Hamming embedding and
        bag-of-words are based on establishing individual features correspondences. In contrast, the scores of
        aggregation approaches such as VLAD and Fisher vectors are computed from compressed image representations,
        where the individual features are not considered.

        An advantage of a matching-based approach like Hamming embedding is that it can define a selectivity
        function. A selectivity function down weights individual feature correspondence with low descriptor
        similarity. Aggregation schemes have been shown to have their own advantages. Aggregated approaches like
        VLAD allow for matching applications to scale to a large number of images because each image is indexed
        with a compressed representation. Furthermore, aggregation-based approaches have been shown to provide
        better matching results on many datasets because they implicitly down weight bursty
        features~\cite{tolias_aggregate_2013, tolias_image_2015}.

        In the SMK framework a matching function and selectivity function are chosen. Different selections of these
        functions can implement and blend desirable attributes of the aforementioned frameworks. The matching
        function assigns correspondences between query and database descriptors. The choice of the matching
        function determines whether the resulting kernel is aggregated or non-aggregated. The selectivity function
        weights a correspondence's contribution to image similarity. It also can apply either power-law like
        normalization or hard thresholding in order to down weights correspondences with low visual similarity. One
        advantage of the SMK framework is that the selectivity function can be used in aggregated matching. In this
        case the selectivity function is applied to all correspondences assigned to a particular word.

    \subsection{Face recognition and verification}
        Face recognition is a specific form of instance recognition with the goal of recognizing individual human
        faces~\cite{zhao_face_2003, huang_labeled_2007}. Related to face recognition is the problem of face
        verification. In contrast to face recognition, face verification takes two unlabeled images and decides if
        they show the same face or different faces~\cite{taigman_deepface_2014}. Clearly these techniques are
        complementary because highly ranked results from a face recognition algorithm can be verified as true or
        false by a face verification algorithm.

        Due to the specific nature of this problem specialized features detectors are often used. Facial feature
        detectors localize facial-landmarks such as the eye, mouth, and nose center and corner
        locations~\cite{dantone_real_time_2012, berg_tom_vs_pete_2012}. Local texture-based descriptors such as Gabor
        filters~\cite{liu_gabor_2002, zhang_histogram_2007, shen_review_2006} and local binary patterns
        (LBP)~\cite{ahonen_face_2006, chen_blessing_2013} are extracted at detected facial
        regions~\cite{belhumeur_localizing_2011}. Facial recognition researchers have also developed global
        descriptors --- such as eigenfaces~\cite{turk_eigenfaces_1991},
        Fisherfaces~\cite{belhumeur_eigenfaces_1997}, and neural network based
        descriptions~\cite{lawrence_face_1997, taigman_deepface_2014}. --- that represent the entire face.
        Recently, algorithms using both local and global representations computed using deep convolutional neural
        networks have shown state-of-the-art performance on both machine and human verification and recognition
        benchmarks~\cite{taigman_deepface_2014}.

        In face recognition, each face image is encoded into a single vector. A function is trained to classify an
        unseen test image as an individual from the database of known faces. Many techniques are used in the
        literature to retrieve or classify a face. Examples of these techniques are: neural
        networks~\cite{turk_eigenfaces_1991, taigman_deepface_2014}, sparse coding~\cite{wright_robust_2009,
        jiang_label_2013}, principal component analysis (PCA)~\cite{craw_face_1992}, Fisher linear discriminant
        (FLD)~\cite{liu_robust_2000}, linear discriminant analysis (LDA)~\cite{lu_face_2003}, and support vector
        machines (SVMs)~\cite{phillips_support_1999, levy_svm_minus_2013}.

        Before the neural network revolution~\cite{krizhevsky_imagenet_2012}, sparse coding was one of the most
        popular techniques to retrieve faces~\cite{aharon_k_svd_2006, wright_robust_2009, zhang_sparse_2011,
        jiang_label_2013}. Sparse coding attempts to reconstruct unlabeled test vectors by searching for a linear
        combination of basis vectors from an over-complete labeled training database. Coding-based techniques are
        very similar to vocabulary-based methods. A codebook, dictionary, and vocabulary all are used to build
        image-level vector representations by quantizing raw features.

        Another interesting technique is the Tom-vs-Pete classifier~\cite{berg_tom_vs_pete_2012}. Given a set of
        $N$ individuals (classes), a set of Tom-vs-Pete classifiers are used for both verification and indexing. At
        each facial landmark, $k$ Tom-vs-Pete classifiers are computed. A single Tom-vs-Pete classifier is a linear
        SVM trained on a single corresponding feature for a single pair of classes. \Eg{} all the nose descriptors
        from class $T$ and class $P$ make up the SVM training data, and the learned SVM classifies a new nose
        feature as $T$-ish or $P$-ish. A descriptor vector for a single face is made by selecting $5000$ out of the
        total $k\binom{N}{2}$ classifiers and concatenating the signed distances from all the classifiers'
        separating hyperplanes. This descriptor facilitates both search and verification. A pair of face descriptor
        vectors can be verified as either a correct or incorrect match by constructing a new vector. The new vector
        is constructed by concatenating the element-wise product and difference of the two descriptor vectors. Then
        this new vector is classified using a radial basis function SVM{}.

        One of the most recent advances in face verification and recognition is the DeepFace
          system~\cite{taigman_deepface_2014}.
        The DeepFace system implements face verification using the following pipeline:
        (1) detect,
        (2) align,
        (3) represent, and
        (4) classify.
        Specialized facial point detectors and a 3D face model are used to register a 3D affine camera to an
          RGB-image.
        The image is then warped into a ``frontalized'' view using a piecewise affine transform.
        A face is represented as the $4096$ dimensional output of a deep $7$ layer convolutional neural network
          that exploits the aligned nature input images.
        An $8$\th layer is used in supervised training where each output unit corresponds to a specific
          individual.
        At test time the L2-normalized output of the network is used as the feature representation.
        In a supervised setting, a $\chi^2$-SVM is trained to recognize the individuals in a training dataset
          using the descriptor vectors produced by the network.
        In an unsupervised setting an ensemble of classifiers is used.
        The ensemble is composed of the output of a Siamese network~\cite{chopra_learning_2005} and several
          non-linear SVM classifiers with different inputs.
        The inputs are deep representations --- the activations of a deep neural network's output layer --- of
          the 3D aligned RGB-image, the 2D aligned RGB-image (generated using a simpler model based on similarity
          transforms), and an image comprised of intensity, magnitude, and orientation channels.
        Each input was fed through four deep networks each with different initialization seeds.
        DeepFace achieves an accuracy of $0.9735$ on the Labeled Faces in the Wild
          dataset~\cite{huang_labeled_2007}, which is comparable to the human performance measured at $0.975$.
        When using unaligned faces the ROC score drops to $0.879$, which demonstrates that alignment is very
          important for handling the problem of viewpoint in face verification.

    \subsection{Person re-identification}
        %Radke dictionary learning ICCV~\cite{karanam_person_2015}
        %Radke pose priors TPAMI~\cite{wu_viewpoint_2015}
        %Deep model of person re-id~\cite{shi_embedding_2016}.
        The person re-identification problem is typically posed in the context of locating the same person within
          a few minutes or hours from low-resolution surveillance
          video~\cite{hirzer_relaxed_2012,karanam_person_2015,wu_viewpoint_2015,shi_embedding_2016}.
        Common approaches to person re-identification typically transform images into a fixed length mid-level
          vector representation and a learned distance metric is used to compare representations.
        Mid-level representations can be built from color and texture histograms or extracted using a
          convolutional neural network.
        The distance metric is commonly learned as a Mahalanobis distance using linear discriminant
          analysis~\cite{hirzer_relaxed_2012}.
        However, alternative approaches using dictionary learning ~\cite{karanam_person_2015} have also been
          shown to work well.
        Improvements to baseline can be achieved by conditioning person descriptors on viewpoint and
          pose~\cite{wu_viewpoint_2015}.
        Recently both features and distance metric have been learned using neural
          networks~\cite{shi_embedding_2016}.
        %The data for person re-identification typically is composed of
        %  low-resolution image captured by surveillance cameras.
        %The goal is often to identify images of people taken within minutes or
        %  hours of each other.
        %and
        %  therefore keypoint algorithms have typically proven most successful.
        %Therefore, additional work is needed to generalize to other species.
        %We have performed initial experiments that support this claim.

    \subsection{Discussion --- instance recognition}
        Most instance recognition techniques use an indexing scheme based on a visual
        vocabulary~\cite{tolias_image_2015, jegou_hamming_2008, philbin_object_2007, cao_learning_2012,
        arandjelovic_all_2013, jegou_negative_2012, chum_fast_2012, gong_multi_scale_2014}. However, our baseline
        approach for animal identification does not use a visual vocabulary. This is because a visual vocabulary
        quantizes the raw features in the image and thus removes some of their discriminative
        ability~\cite{philbin_lost_2008, boiman_defense_2008}. We have found this quantization to cause a
        noticeable drop in performance. Many aspects of our baseline algorithm are similar to Lowe's recognition
        algorithm~\cite{lowe_distinctive_2004}, which does not quantize descriptors. The guiding principles of
        matching, filtering based on distinctiveness, filtering based on spatial consistency, and scoring are
        shared with our approach. However, our approach features several improvements to this algorithm.
        Furthermore, animal identification is a dynamic problem with specific domain-based concerns --- such as
        quality and viewpoint in natural images --- and requires innovation beyond Lowe's recognition algorithm.

        % chktex-file 8
        Even though we would prefer to retain the discriminative information contained in raw descriptors,
          quantized image search has the ability to scale beyond our current suites~\cite{chum_fast_2012,
          perronnin_large_scale_2010_1, tolias_image_2015}.
        In the future it may be necessary to investigate a VLAD-based SMK framework as a quantized alternative to
          our matching algorithm.
        Techniques such as soft-assignment~\cite{philbin_lost_2008} and learned
          vocabularies~\cite{mikulik_learning_2010} could be used to reduce quantization errors.
        It is necessary to update the vocabulary as new images are added to the system.
        This issue could be addressed using the vocabulary adaptation technique in~\cite{arandjelovic_all_2013}.
        However, in this research we are more focused on the problem of verifying identifications to reduce
          manual effort.
        As such we leave the scalable search issue for future work.

        Facial recognition is similar to the problem of animal identification.
        %Technically is is a subset of the problem.
        Both problems seek to identify individuals. Some techniques used for face verification such as the Siamese
        network~\cite{chopra_learning_2005, taigman_deepface_2014} can be extended to the scope of animal
        identification. However, there is a much more mature literature on face recognition that has resulted in
        easily accessible and specialized algorithms for face feature detection and --- most importantly --- for
        face alignment. Individual animal identification does not have such a corpus of knowledge. We do not have
        access to highly specialized animal part detectors and alignment algorithms. Furthermore, we would like our
        algorithms to generalize to multiple species, so we would like to avoid over-specialized approaches. These
        are some reasons why convolutional neural networks will not make a prominent appearance in this \thesis{}.
        Other reasons involve the size of our datasets. The recent NetVLAD network was trained using training
        datasets with $10,000$ to $90,000$ images~\cite{arandjelovic_netvlad_2016}. We simply do not have this much
        labeled data. However, one goal of this \thesis{} is to develop techniques that will help bootstrap labeled
        datasets of this size. Future research should investigate these deep learning techniques so they can be
        used after enough data has been collected for a specific species.

        While the problem of animal identification and person re-identification are conceptually similar ---
        sharing challenges such as lighting, pose, and viewpoint variation --- differences in data collection
        creates the need for different solutions in practice. In contrast to the low-resolution image captured by
        surveillance cameras, the images used in animal identification are often manually captured by scientists in
        the field using high resolution DSLR cameras, and the goal is to match individuals over longer periods of
        time (years). Furthermore, re-identification techniques commonly focus on aggregate features that emphasize
        clothing, color, texture, and the presence of objects such as coats and backpacks, while in the animal id
        problem, it is often subtle localized variations in patterns on the skin and fur that distinguish
        individuals.


\section{Category recognition}\label{sec:cr}  

    Different types of image recognition lie at different points on a spectrum of specificity. If instance
    recognition is at one end of the spectrum, then category recognition is at the other. The goal of a
    \glossterm{category recognition} algorithm is to assign a categorical class label to a query
    image~\cite{everingham_pascal_2010, everingham_pascal_2015, russakovsky_imagenet_2014, deng_imagenet_2009,
    fei_fei_one_shot_2006, griffin_caltech_256_2007}. The categories often have visual appearances with a high
    degree of intra-class variance. \Eg{}, a recliner and a bench both belong to the chair category. Image
    representations and similarity measures are constructed to account for this. Despite this, techniques in
    category recognition have many similarities to instance recognition techniques. Until the neural network
    revolution~\cite{krizhevsky_imagenet_2012}, most category recognition techniques have been based on vocabulary
    methods~\cite{csurka_visual_2004, yang_linear_2009, sanchez_compressed_2013, russakovsky_imagenet_2014,
    krizhevsky_imagenet_2012} similar to those discussed in~\cref{subsec:bow}. This section first provides a brief
    overview of this literature. Then, we discuss naive Bayes classification
    techniques~\cite{boiman_defense_2008,mccann_local_2012} that play a large role in our baseline animal
    identification algorithms.

    \subsection{Vocabulary-based methods for category recognition}
        After vocabulary-based techniques demonstrated success in instance recognition, these techniques were
        quickly adapted and applied to category recognition~\cite{csurka_visual_2004}. Thus, there are many
        similarities --- and some differences --- in the techniques used to address these two problems. One
        difference is the size of the visual vocabulary. Instance recognition tends to require huge vocabularies
        ($\OnTheOrderOf{5}$ --- $\OnTheOrderOf{7}$ words) to achieve a fine sampling of descriptor
        space~\cite{nister_scalable_2006, philbin_object_2007}. In contrast, category recognition uses smaller
        vocabulary sizes ($\OnTheOrderOf{4}$ words) to more coarsely sample descriptor
        space~\cite{zhang_local_2006}. However, the vocabularies used in instance recognition have decreased in
        size with the advent of aggregated representations like VLAD and the Fisher
        vector~\cite{arandjelovic_all_2013, sanchez_compressed_2013}.

        A second difference is how similarity between images is computed. In instance recognition the similarity
        between bag-of-word vectors is computed using a weighted cosine similarity. However, in
        category-recognition intra-class variation requires more sophisticated similarity measurements. Here, image
        similarity is computed using SVMs with different either linear or non-linear kernels such as $\chi^2$,
        earth mover's distance, Hellinger, and Jensen-Shannon~\cite{zhang_local_2006, varma_learning_2007,
        vedaldi_efficient_2012}.

        A third difference is the way that spatial information is used. Instead of filtering correspondences using
        spatial verification, spatial information is incorporated into category recognition algorithms using
        spatial pyramids~\cite{grauman_pyramid_2005, lazebnik_beyond_2006}. A spatial pyramid sub-divides an image
        into a hierarchy of grids. Max pooling is often used to select only the strongest features in each spatial
        region~\cite{boureau_theoretical_2010, boureau_learning_2010}. Each section of the image is encoded using
        the vocabulary and images are scored based on matches in each region.

        \paragraph{Enhancements to category recognition}
        There are a wide variety of extensions and enhancements for image classification techniques based on
        bag-of-words, such as soft assignment of visual-words~\cite{liu_defense_2011} and vocabulary
        optimization~\cite{wang_locality_constrained_2010}. Numerous matching kernels --- both linear and
        non-linear --- have been developed such as kernel PCA, histogram intersection, and SVM square root
        bag-of-words vectors~\cite{vedaldi_multiple_2009, maji_classification_2008, perronnin_large_scale_2010}.

        % chktex-file 8
        Generalized coding schemes improve performance over a bag-of-words image encoding. Vocabularies can be seen
        as codebooks or dictionaries in coding-based image classification techniques such as sparse coding and
        locally constrained linear coding~\cite{jurie_creating_2005, yang_linear_2009, yang_supervised_2010,
        yang_efficient_2010, wang_locality_constrained_2010}. Many coding schemes learn both the centroids 
        and the function that quantizes a raw descriptor into a word~\cite{jurie_creating_2005, yang_linear_2009,
        yang_supervised_2010, yang_efficient_2010, wang_locality_constrained_2010, vedaldi_multiple_2009}.
        Techniques other than k-means are used to create vocabularies such as mean
        shift~\cite{jurie_creating_2005}, coordinate descent with the locally constrained linear code
        criterion~\cite{wang_locality_constrained_2010}, and random forests~\cite{perronnin_fisher_2007}. Fisher
        vectors with linear classifiers have been found to outperform non-linear bag-of-words based SVM classifiers
        by using an L1-based distance measure and careful L2 and power-law normalization of
        descriptors~\cite{perronnin_improving_2010, perronnin_large_scale_2010}.

    \subsection{\Naive{} Bayes classification}\label{sec:nbnn}  

        The \naive{} Bayes nearest neighbor (NBNN) classifier is a simple non-parametric algorithm for category
        recognition that does not quantize descriptor vectors~\cite{boiman_defense_2008}. Boiman responds to the
        dominance of complex non-linear category recognition algorithms in the field~\cite{varma_learning_2007,
        marszalek_learning_2007} by showing that simple techniques can compete with complex methods for category
        recognition. Boiman's paper also provides insight into the magnitude of information loss resulting from
        quantization.
          
        Previous to~\cite{boiman_defense_2008}, nearest neighbor classifiers had shown underwhelming accuracy in
        category recognition~\cite{varma_unifying_2004, lazebnik_beyond_2006, marszalek_learning_2007}. This was
        shown to be a result of using image-to-image distance. To remedy this, NBNN aggregates the information from
        multiple images by swapping the image-to-image distance for an image-to-class distance.

        In NBNN, features of each class are indexed for fast nearest neighbor search, typically with a
        kd-tree~\cite{bentley_multidimensional_1975}. For each feature, $\desc_i$, in a query image, the algorithm
        searches for the feature's nearest neighbors in each class, $\opname{NN}_C(\desc_i)$. The result of the
        algorithm is the class, $C$, that minimizes the image-to-class distance. In other words, the class of a
        query image is chosen by searching for the class that minimizes the total distance between each query
        descriptor and the nearest database descriptor in that class. This is expressed in the following equation:
        \begin{equation}
            C = \argmin{C} \sum_{i=1}^n ||\desc_i - \opname{NN}_C(\desc_i)||^2
        \end{equation}

        This formulation where each descriptor is assigned to only the single nearest neighbor has been shown to be
        a good approximation to the minimum image-to-class Kullback-Leibler divergence~\cite{boiman_defense_2008}
        --- a measure of how much information is lost when the query image is used to model the entire class.

    \subsection{Local \naive{} Bayes nearest neighbor}\label{sec:lnbnn}  

        Local \naive{} Bayes nearest neighbor (LNBNN) is an improved version of the NBNN algorithm in both accuracy
        and speed~\cite{mccann_local_2012}. In the original NBNN formulation a search is executed find each query
        descriptor's nearest neighbor in the database for each class separately. In contrast, the LNBNN
        modification searches all database descriptors simultaneously and ignores classes that do not return
        descriptor matches.
        
        Each descriptor $\desc_i$ in the query image searches for its nearest $K+1$ neighbors, %
        $\{\desc_1, \ldots, \desc_K, \desc_{K + 1}\}$ over all classes.
        The first $K$ neighbors are used as matches.
        The last neighbor is used as a normalizing term to weight the query descriptor's distinctiveness.
        Let $(\desc_i, \desc_j)$ be a matching descriptor pair, and let $C$ be the class of $\desc_j$.
        The score of each match is computed as the distance to the match subtracted from the distance to the
          normalizer.
        \begin{equation}
            s_{i, C} = \elltwo{\desc_j - \desc_K} - \elltwo{\desc_i - \desc_j}
        \end{equation}
        The score of a class $C$ is the sum of all the descriptor scores that match to it.

    \subsection{Discussion --- class recognition}

        Progress in category recognition is generally made using techniques that allow classes with high
        intra-class variance to have lower matching scores. This is of little value to an instance recognition
        application, therefore we do not investigate most of the techniques in this section. However, the
        LNBNN~\cite{mccann_local_2012} approach is interesting to us because it is a simple algorithm that does not
        suffer from quantization artifacts. NBNN and LNBNN~\cite{boiman_defense_2008,mccann_local_2012} never
        achieved state-of-the-art performance in image classification, however they have produced competitive
        results using simple techniques.

        The simplicity of the techniques allowed for the authors to gain insight into visual recognition. Due to
        its simplicity and the insight that quantization significantly reduces the descriptive power of SIFT
        features, we adopt LNBNN as the baseline algorithm for animal identification.


\section{Fine-grained recognition}\label{sec:fgr}  

    Fine-grained recognition is a problem more general than instance recognition, but more specific than category
    recognition~\cite{parkhi_cats_2012, berg_poof_2013, gavves_local_2014}. Given an object of a known category,
    such as a bird, the goal of fine-grained recognition is to sub-classify the object into a fine-grained category
    such as a blackbird or a raven~\cite{berg_how_2013}.

    Algorithms for fine-grained recognition typically start by localizing the object and its parts with a detection
    algorithm~\cite{dalal_histograms_2005} and parts-based models. Parts are segmented to remove background noise
    using algorithms like GrabCut~\cite{rother_grabcut_2004}. Classification is performed locally on aligned parts
    as well as globally on the entire body and aggregated to yield a final classification.

    Because fine-grained recognition lies on the same spectrum as instance recognition and category recognition it
    is not surprising that many of the same techniques --- like Fisher vectors --- are
    used~\cite{gosselin_revisiting_2014}. Recently convolutional models have been successfully applied to
    fine-grained recognition~\cite{catherine_wah_similarity_2014, branson_bird_2014, zhang_weakly_2015,
    xiao_application_2015}.

    \subsection{Discussion --- fine-grained recognition}
        The goal of fine-grained recognition is somewhat similar to animal identification. Fine-grained recognition
        localizes subtle information to distinguish between two similar species, whereas animal identification
        localizes subtle information to distinguish between two similar individuals. However, the domains of
        species and individuals are dissimilar enough that off the shelf techniques for fine-grained recognition
        would need to be adapted before identification could be performed. One interesting avenue of research would
        be to use a parts model~\cite{felzenszwalb_object_2010} as in~\cite{gavves_local_2014}, to align
        individuals before they are compared.


\section{Deep convolutional neural networks}\label{sec:dcnn}
    Convolutional networks have been around for over more than two decades~\cite{lecun_gradient_based_1998,
    fukushima_neocognitron_1988}. However, they did not receive major attention from computer vision researchers
    until 2012 when a deep convolutional neural network (DCNN)~\cite{krizhevsky_imagenet_2012} outperformed the
    best support vector machines (SVMs)~\cite{vapnik_statistical_1998} by over $10\percent$ in the ImageNet
    category recognition challenge~\cite{russakovsky_imagenet_2014}. Since then, many successful category
    recognition techniques based on DCNNs have been published~\cite{simonyan_very_2015, chatfield_efficient_2015,
    chatfield_return_2014, oquab_learning_2014, szegedy_going_2015, long_convnets_2014, he_spatial_2014,
    dean_fast_2013}. DCNNs have also been shown to produce excellent results when applied to other computer vision
    problems such as: %
    instance recognition~\cite{razavian_cnn_2014, razavian_baseline_2015, liu_learning_2015,
    held_deep_2015,arandjelovic_netvlad_2016,radenovic_cnn_2016}, %
    fine-grained recognition~\cite{branson_bird_2014, donahue_decaf_2014, catherine_wah_similarity_2014}, %
    detection~\cite{girshick_rich_2014, sermanet_overfeat_2013, li_wan_end_end_2015}, %
    face verification~\cite{huang_learning_2012, taigman_deepface_2014, sun_deep_2013}, %
    and learning similarity between feature patches~\cite{osendorfer_convolutional_2013, han_matchnet_2015,
    ng_exploiting_2015, zagoruyko_learning_2015, han_matchnet_2015}. The sudden success of deep nets has been
    attributed (1) a larger volume of available of training data, and (2) implementations using faster
    GPUs~\cite{krizhevsky_imagenet_2012}.
      
    Several techniques are employed to increase accuracy, reduce over-fitting,  and reduce training time.
    Data augmentation is used to artificially increase the amount of training
      data~\cite{ciresan_multi_column_2012, ciresan_high_performance_2011, simard_best_2003}.
    The dropout technique has been shown to reducing over-fitting~\cite{dahl_improving_2013,
      srivastava_dropout_2014}.
    At training time outputs of hidden units are randomly suppressed which forces the network to learn a more
      robust representation.
    It has been shown that dropout can be viewed as a form of model averaging~\cite{hinton_improving_2012}.
    Rectified linear units (ReLU) have been shown to be a faster alternative to the standard sigmoid activation
      functions~\cite{vinod_rectified_2010, dahl_improving_2013}.
    A ReLU is similar to a hinge function and simply outputs the signal of a unit if it is positive and outputs a
      zero otherwise.
    Leaky rectified linear units (LReLU) further improve network accuracy by including a ``leakiness'' term while
      maintaining the speed of ReLUs~\cite{maas_rectifier_2013}.
    While a ReLU strictly suppresses a feature activation if it is negative a LReLU returns a small negative
      signal (by multiplying by a constant) instead of zero.

    A deep neural network is constructed by stacking several layers of units (neurons) together. Data is used to
    initialize the activations of an input layer, and the information is forward propagated through the network.
    Weights are chosen to optimize a loss function --- \eg{} categorical cross-entropy error or triplet
    loss~\cite{schroff_facenet_2015} --- which is chosen to depend on the application. Optimization of the loss
    function is performed using back-propagation~\cite{rumelhart_learning_1986} --- typically using mini-batches
    and stochastic gradient descent with momentum~\cite{sutskever_importance_2013}. Traditionally each layer in a
    neural network is fully connected --- each pair of units between the previous layer and the current layer has
    its own edge weight ---  to the previous layer. However, in computer vision networks are constructed using
    convolutional layers.

    A DCNN connects the input layer to a stack of convolutional layers~\cite{krizhevsky_imagenet_2012}. A
    convolutional layer differs from a fully connected layer in that it is sparsely connected and that most of the
    edge weights between layers are shared~\cite{lecun_gradient_based_1998, fukushima_neocognitron_1988,
    serre_robust_2007}. Each convolutional layer is broken into several channels. Each channel is given its own
    weight matrix with a fixed width and height. This matrix of weights is convolved with the input layer to
    produce a feature activation map, one for each channel. Convolutional layers often use several pooling layers
    that aggregate information over a small area, reduce the size of the feature map, and increase robustness to
    transformations. Common pooling operations are max-pooling~\cite{serre_robust_2007, krizhevsky_imagenet_2012}
    and maxout~\cite{goodfellow_maxout_2013}. The convolutional layers may also be connected to a stack of fully
    connected layers. In this case, hierarchies of feature maps are built in the low level convolutional layers,
    and then fully connected layers learn decision boundaries between these
    features~\cite{zeiler_visualizing_2014}.

    Because of weight sharing, convolutional networks must learn significantly fewer parameters than fully
      connected networks.
    This allows convolutional networks to be trained much faster.
    Fewer weights also acts as a form of regularization for the network.
    Intuitively learned convolutional filters are similar to Gabor filters~\cite{gabor_theory_1946}, which are a
      naturally suited for extracting features from images.
    Even without learning weights, convolutions can be used to extract powerful features for
      matching~\cite{revaud_deep_2015}.
    The popular SIFT and HoG features~\cite{mahendran_understanding_2015} can even be implemented as
      convolutional networks.
    Despite the lack of hard theoretical insight into the inner workings of these networks, their empirical
      performance cannot be denied.

  \subsection{Discussion --- deep convolutional neural networks}\label{subsec:dcnndiscuss}
        Because of the astounding success of convolutional networks in almost every area in computer vision, we have
        investigated their use in animal identification. Specifically, we have investigated two approaches.

        The first approach used deep convolutional feature descriptors as a replacement for the
        SIFT~\cite{lowe_distinctive_2004} descriptor following the patch-based scheme
        in~\cite{zagoruyko_learning_2015}. The basic idea is to have two patches fed through the same
        (Siamese)~\cite{chopra_learning_2005} architecture and then compare their resulting encodings. This
        comparison can be as simple as Euclidean distance, or as complex as a learned distance measure. Training
        can be performed on pairs of patches, labeled as correct or incorrect, using the discriminative loss
        function~\cite{lecun_loss_2005}. Unfortunately, due to issues with the quality and quantity of our training
        data our convolutional replacements for the SIFT descriptor have not been successful.

        The second approach aimed to use Siamese networks to directly compare two images of an animal to
          determine if they were the same or different, similar to the method used in
          DeepFace~\cite{taigman_deepface_2014} for face verification.
        However, without the large training datasets and specialized alignment procedures used in DeepFace, we
          were unable to produce promising results.
        
        Due to these issues, this \thesis{} does not further pursue techniques based on DCNNs.
        We include this discussion to note the potential of deep learning applied to animal identification and to
          strongly suggest further investigation of these techniques in the future research.
        Of particular interest for future research is the matching technique presented
          in~\cite{rocco_convolutional_2017}.
        This method is particularly interesting because it learns to match and align images by mimicking a
          classic computer vision pipeline while using only synthetic training data.
        This may be able to overcome the issues mentioned above,
        however further investigation is needed.
