\newcommand{\HypothAffMat}[0]{\hat{\mat{A}}}
\newcommand{\HypothSet}[0]{\set{A}}
\newcommand{\MatchesGroup}[0]{\Matches^{G}}
\newcommand{\AffMat}[0]{\mat{A}}
\newcommand{\AffMatij}[0]{\mat{A}_{i, j}}


\chapter{Identification using a single query}
This chapter addresses the core problem of animal identification in a static context.


\section{Annotation representation}
    For each annotation in the database we
    (1) normalize the image geometry and intensity,
    (2) compute features,
    (3) and weight the features.
    \subsection{Chip extraction}
        Each annotation has a bounding box and an orientation specified in a previous detection step.
        
    \subsection{Keypoint detection and description}
        Keypoints are detected within each annotation's chip using a modified implementation of the Hessian detector
          described in~\cite{perdoch_efficient_2009} and reviewed in~\cref{sec:featuredetect}.
        
    \subsection{Feature weighting}
        In animal identification, there will often be many annotations containing the same background.
        
    \subsection{Keypoint structure overview}
        The keypoint of a feature is represented as:
        $\kp\tighteq(\pt, \vmat, \ori)$, %
        The vector $\pt\tighteq\ptcolvec$ is the feature's $xy$-location.

        \paragraph{Encoding keypoint parameters in an affine matrix}
        It will be useful to construct two transformations that encode all keypoint information in a single matrix.

        \paragraph{Extracting keypoint parameters from an affine matrix}
        During the spatial verification step, described in~\cref{sec:sver}, keypoints are warped from one image into
          the space of another.
\section{Matching against a database of individual animals}
    To identify a query annotation, it is matched against a database of known \names{}.
    
    \subsection{Establishing initial feature correspondence}
        \paragraph{Offline Indexing}
        Before feature correspondences can be established, an offline algorithm indexes descriptors from all database
          annotations for fast approximate nearest neighbor search.
        \paragraph{Approximate Nearest Neighbor Search}
        Matching begins by establishing multiple feature correspondences between each query feature and several
          visually similar database features.
        \paragraph{Normalizer selection}
        A single descriptor $\descnorm_{i}$ is selected from the $\Knorm{}$ candidate normalizers and used in
          computing the LNBNN score for all (up to $\K$) of the $i$\th{} query descriptor's correspondences in the
          database.
        
    \subsection{Feature correspondence scoring}
        Each feature correspondence is given a score representing how likely it is to be a correct match.
        
        \paragraph{LNBNN score}
        Using the normalizing feature, $\descnorm_{i}$, LNBNN compares a query feature's similarity to the match and
          query feature's similarity to the normalizer.
        \paragraph{Foregroundness score}
        To reduce the influence of background matches, each feature correspondence is assigned a score based on the
          foregroundness of both the query and database features.
        
        \paragraph{Final feature correspondence score}
        The final score of the correspondence $(i, j)$ captures both the distinctiveness of the match as well as the
          likelihood that the match is part of the foreground.
        
    \subsection{Feature score aggregation}
        So far, each feature in a query annotation has been matched to several features in the database and a score
          has been assigned to each of these correspondences based on its distinctiveness and foregroundness.
        
        \paragraph{The set of all feature correspondences}
        All scoring mechanisms presented in this subsection are based on aggregating scores from features
          correspondences.
        
        \paragraph{Annotation scoring}
        An \annotscore{} is a measure of similarity between two annotations.
        \paragraph{Name scoring (1) --- \csumprefix{}} %
        The \cscoring{} mechanism aggregates \annotscores{} into \namescores{} by taking as the score highest scoring
          annotation for each \name{}.
        
        \paragraph{Name scoring (2) --- \nsumprefix{}} %
        The \cscoring{} mechanism accounts for the fact that animals will be seen multiple times, but it does not take
          advantage of complementary information available when \aan{\name{}} has multiple annotations.
        

\section{Spatial verification}
    The basic matching algorithm treats each annotation as an orderless set of feature descriptors (with a small
      exception in name scoring, which has used a small amount of spatial information).
    
    \subsection{Shortlist selection}
        Standard methods for spatial verification are defined on the feature correspondences between two annotations
          (images).
        
    \subsection{Affine hypothesis estimation}
        Here, we will compute an affine transformation that will remove a majority of the spatially inconsistent
          feature correspondences.
        
        \subsubsection{Enumeration of affine hypotheses}
            Let $\Matches_{\annotII}$ be the set of all correspondences between features from query annotation
              $\annotI$ to database annotation $\annotII$.
            
        \subsubsection{Measuring the affine transformation error}
            The transformation $\AffMatij$ perfectly aligns the corresponding $i$\th{} query feature with the $j$\th{}
              database feature in the space of the database annotation ($\annotII$).
            
        \subsubsection{Selecting affine inliers}
            Any keypoint match $(\idxI, \idxII) \in \Matches_{\annotII}$  is considered an inlier \wrt{} $\AffMat$ if
              its absolute differences in position, scale, and orientation are all within a spatial distance threshold
              $\xythresh$, scale threshold $\scalethresh$, and orientation threshold $\orithresh$.
    \subsection{Homography refinement}
        Matches that are inliers to the best hypothesis affine transformation, $\HypothAffMat$, are used in the least
          squares refinement step.
        
        \subsubsection{Estimation of warped shape parameters}
            Because we cannot warp the shape of an affine keypoint using a projective transformation, we instead
              estimate the warped scale and orientation for the $\idxI$\th{} query feature using a reference point.
            
        \subsubsection{Homography inliers}
            The rest of homography inlier estimation is no different than affine inlier estimation.

\section{Experiments}
    This section presents an experimental evaluation of the identification algorithm using annotated images of plains
      zebras, Grevy's zebras, and Masai giraffes.
    {\large{\textbf{NEW}}}

    \begin{verbatim}
        * Evaluate multiple configurations of our algorithms
        * Compare against state of the art methods in instance retrieval
        * Test on plains zebras, Grevy's zebras, and Masai giraffes.
        * Why is this hard?
        * Photobombs
    \end{verbatim}

    \subsection{Datasets}
        All of the images in the datasets used in these experiments were taken by photographers in the field.
        
        \begin{itemize}
            \item \textbf{\pzmasterI{}} is an aggregated dataset of
                plains zebras.
                
            \item \textbf{\gzall{}} is an aggregated dataset of Grevy's
                zebras.
                
            \item \textbf{\girmmasterI{}} is a dataset of Masai giraffes.
                The images were all taken in Nairobi National Park.
        \end{itemize}

    \subsection{Evaluation of SMK for animal identification}
        {\large{\textbf{NEW}}}

        \begin{verbatim}
            * We first evaluate the performance of the SMK algorithm --- a
            state of the art approach in instance recognition --- applied to
            animal identification.

            * We find that our algorithms perform favorably to VLAD
            representations.  We attribute this to the quantization error
            induced by VLAD representations.

            * Note that the fisher vector like VLAD-vectors used here could be
            used as a fixed-length feature vector for evaluating person
            re-identification techniques. However, this amounts to some form of
            learning on top of the VLAD representation.  While learning will
            improve performance, it will have to work to undo the quantization
            error.  For this reason we opt to pursue a different approach where
            learning is applied to features derived from the results of our
            quantization-free algorithms.
        \end{verbatim}
        
    \subsection{Baseline experiment}
        This first experiment determines the accuracy of the identification algorithm using the baseline pipeline
          configuration.
        
    \subsection{Foregroundness experiment}
        
        In this experiment we test the effect of foregroundness --- weighting the score of each features
          correspondence with a foregroundness weight --- on identification accuracy.
        
    \subsection{Invariance experiment}
        In this experiment we vary the feature invariance configuration.
        
        \begin{itemize}
            \item \NoInvar{} (\pvar{AI=F,QRH=F,RI=F}): % 
                In this configuration the gravity vector is assumed and the shape of each detected feature is not
                  adapted.
                
            \item \AIAlone{} (\pvar{AI=T,QRH=F,RI=F}): % 
                This is the baseline setting that assumes the gravity vector and where each feature's shape is skewed
                  from a circle into an ellipse.
                
            \item \RIAlone{} (\pvar{AI=F,QRH=F,RI=T}): % 
                Here, each feature is assigned one or more dominant gradient orientations (the gravity vector is not
                  used) and the shape is not adapted.
                
            \item Query-side rotation heuristic (\QRHCirc{})
                (\pvar{AI=F,QRH=T,RI=F}): %
                This is a novel invariance heuristic where each {database} feature assumes the gravity vector, but
                  {query} feature is $3$ orientations:
                the gravity vector and two other orientations at $\pm15\degrees$ from the gravity vector.
                
            \item \QRHEll{} (\pvar{AI=T,QRH=T,RI=F}): %
                This is the combination of \QRHCirc{} and \AIAlone{}.
                
            \item \AIRI{} (\pvar{AI=T,QRH=F,RI=T}): %
                This is the combination of \RIAlone{} and \AIAlone{}.
                
        \end{itemize}

    \subsection{Scoring mechanism experiment}
        The purpose of the scoring mechanism is to aggregate scores of individual feature correspondences across
          multiple annotations into a single score for each name.
        
    \subsection{K experiment}
        In this experiment we investigate the effect $\K$ (the number of nearest neighbors used in establishing
          feature correspondences) on identification accuracy.
        
    \subsection{Failure cases}
        In this subsection we investigate the primary causes of identification failure by considering individual
          failure cases.
        
        \begin{itemize}
            \item Viewpoint:
                This failure case denotes that there is a viewpoint difference between the query and its \groundtrue{}
                  match.
                
            \item Occlusion:
                This label denotes that either the query or the \groundtrue{} annotation is occluded.
                
            \item Quality:
                This label denotes that either the query or the \groundtrue{} annotation is blurred or distorted.
                
            \item Lighting:
                This label denotes that either the query or the \groundtrue{} annotation is poorly illumination or
                  shadowed.
                
            \item Scenery Match:
                This denotes a case where the algorithm produces correspondences between shared background features
                  between a query annotation and \aan{\groundfalse{}} annotation.
                
            \item \Photobomb{}:
                A case where the \groundtrue{} animal is seen in the foreground or background of \aan{\groundfalse{}}
                  annotation.
                
        \end{itemize}
        
    \subsection{Score separability}
        In this subsection we investigate identification accuracy in terms of the scores returned along with each
          \name{} in the ranked list.
        
        \subsubsection{Why is individual animal identification hard?}
            Even when ignoring \photobombings{}, the amount missed true positives and the number of manual
              verifications necessary is still unsatisfactory.
            
    \subsection{Experimental conclusions}
        In this section we have evaluated our baseline algorithm under restrictive conditions to control for the
          effects of time, quality, and viewpoint.
        \begin{itemize}
            \item \textbf{Identification accuracy improves with more exemplars}:
                The name scoring experiment and the $\K$ experiment show that the number of \exemplars{} per database
                  \name{} is the most significant factor that impacts identification accuracy.
                
            \item \textbf{Foregroundness weighting reduce scenery matches}:
                Identification accuracy significantly improves by $2-4$ percentage points when using foregroundness
                  weighting.
                
            \item \textbf{Viewpoint and occlusion are the most difficult imaging challenges}:
                The viewpoint experiment and the failure cases show that there is a significant loss in accuracy when
                  matching annotations from different viewpoints.
                
            \item \textbf{Invariance settings are data dependent}:
                The invariance experiment shows
            \item \textbf{The choice of \K{} has a minor impact}: 
                The $\K$ experiment shows that identification accuracy is not significantly influenced by the choice
                  of $\K$ for plains zebras, but for Grevy's zebras the most accurate results were obtained with
                  $\K\tighteq1$.
            \item \textbf{\Nsumprefix{} is slightly better than \csumprefix{} \namescoring{}}:
                The scoring mechanism experiment shows that the \nsumprefix{} scoring mechanism is slightly more
                  accurate than the \csumprefix{} scoring mechanism.
                
            \item \textbf{LNBNN scores are not enough for automated decision}:
                The separability experiment shows that is a reasonable separation between the scores of annotations
                  seen from the same viewpoint.
        \end{itemize}
        
\section{Single query identification summary}
    In this section we have introduced an algorithm that ranks known database \names{} by their similarity to a single
      query annotation.
    
    \begin{enumerate}
        
        
        \item \textbf{Exploit information in multiple images}:
            One of the strongest results in the experiment section was that having more \exemplars{} per name in the
              database improves identification accuracy.
        \item \textbf{Make database independent identification decisions}:
            Another potential avenue is to develop an one-vs-one verification algorithm that can be applied on pairs
              of annotations (or pairs of annotation sets) independent of the current identification algorithm.
    \end{enumerate}
    The next chapter and outlines a proposal for making these improvements and discusses the associated challenges.
    
