
\section{Spatial verification}\label{sec:sver}

    The basic matching algorithm treats each annotation as an orderless set of feature descriptors (with a small
      exception in name scoring, which has used a small amount of spatial information).
    This means that many of the initial feature correspondences will be spatially inconsistent.
    Spatial verification removes these spatially inconsistent feature correspondences.
    Determining which features are inconsistent is done by first estimating an affine transform between the two
      annotations.
    Then a projective transform is estimated using the inliers to the affine transform.
    Finally any correspondences that do not agree with the projective transform transformation are
      removed~\cite{fischler_random_1981, philbin_object_2007}.
    We have reviewed related work in spatial verification in~\cref{subsec:sverreview}.

    %In our problem, the animals are seen in a wide variety of poses,
    %  and projective transforms may not always be sufficient to capture
    %  all correctly corresponding features.
    %Yet, without strong spatial constraints on matching, many
    %  background features will be spatially verified.
    %For now, we proceed with standard techniques for spatial
    %  verification and evaluate if more sophisticated methods are needed.

    \subsection{Shortlist selection}
        Standard methods for spatial verification are defined on the feature correspondences between two
          annotations (images).
        Normally, a shortlist of the top ranked annotations are passed onto spatial verification.
        However, in our application we rank \names{}, which may have multiple annotations.
        In our baseline approach we simply apply spatial verification to the top $N_{\tt{nameSL}}=40$ \names{}
          and the top $N_{\tt{annotSL}}=3$ annotations within those \names{}.

    \subsection{Affine hypothesis estimation}
        Here, we will compute an affine transformation that will remove a majority of the spatially inconsistent
          feature correspondences.
        Instead of using random sampling of the feature correspondences as in the original RANSAC
          algorithm~\cite{hartley_multiple_2003}, we estimate affine hypotheses using a deterministic method
          similar to~\cite{philbin_object_2007, chum_homography_2012}.
        Given a set of matching features between annotation $\annotI$ and $\annotII$, the shape, scale,
          orientation, and position of each pair of matching keypoints are used to estimate a hypothesis affine
          transformation.
        Each hypothesis transformation warps keypoints from annotation $\annotI$ into the space of $\annotII$.
        Inliers are estimated by using the error in position, scale, and orientation between each warped keypoint
          and its correspondence.
        The transformation with the most inliers determines the final affine transform.

        % vmat = V here
        % V maps from ellipse to u-circle
        \newcommand{\AffMat}{\mat{A}}
        \newcommand{\HypothSet}{\set{A}}
        \newcommand{\AffMatij}{\mat{A}_{i, j}}
        \newcommand{\HypothAffMat}{\hat{\mat{A}}}

        \subsubsection{Enumeration of affine hypotheses}
            %The deterministic set of hypothesis transformations mapping from
            %  query annotation $\annotI$ to database annotation $\annotII$ is
            %  computed for each feature correspondence in a match from  to an
            %  annotation.
            Let $\Matches_{\annotII}$ be the set of all correspondences between features from query annotation
              $\annotI$ to database annotation $\annotII$.
            For each feature correspondence $(i, j) \in \Matches_{\annotII}$, we construct a hypothesis
              transformation, $\AffMatij$ using the matrices $\rvmat_{i}$ and $\inv{\rvmat_{j}}$, which where
              defined in~\cref{eqn:RVTConstruct} and~\cref{eqn:invTVRConstruct}.
            The first transformation $\rvmat_{i}$, warps points from $\annotI$-space into a normalized reference
              frame.
            Then the second transformation, $\inv{\rvmat_{j}}$, warps points in the normalized reference frame
              into $\annotII$-space.
            Formally, the hypothesis transformation is defined as $\AffMatij \eqv \inv{\rvmat_{j}}\rvmat_{i}$,
              and the set of hypothesis transformations is:
            \begin{equation}
                \HypothSet \eqv \curly{ \AffMatij \where (i, j) \in \Matches_{\annotII} }
            \end{equation}

        \subsubsection{Measuring the affine transformation error}
            The transformation $\AffMatij$ perfectly aligns the corresponding $i$\th{} query feature with the
              $j$\th{} database feature in the space of the database annotation ($\annotII$).
            If the correspondence is indeed correct, then we can expect that other corresponding features will be
              well aligned by the transformation.
            The next step is to determine how close the other transformed features from the query annotation
              ($\annotI$) are to their corresponding features in database annotation ($\annotII$).
            This can be measured using the error in distance, scale, and orientation for every correspondence.
            The following procedure is repeated for each hypothesis transform %
            $\AffMatij \in \HypothSet$.
            Note that the following description is in the context of the $i$\th{} query feature and the $j$\th{}
              database feature, and the $i,j$ suffix is omitted for clarity.
            In this context, the suffixes $\idxI$ and $\idxII$ will be used to index into features
              correspondences.

            Let $\set{B}_{\idxI} = \curly{\invvrmatI \where (\idxI, \idxII) \in \Matches_{\annotII}}$ be the set
              of keypoint matrices in the query annotation with correspondences to database annotation $\annotII$.
            Given a hypothesis transform $\AffMat$, each query keypoint in the set of matches
            %(mapping from the normalized reference frame to feature space)
            $\invvrmatI \in \set{B}_{\idxI}$, is warped into $\annotII$-space:
            \begin{equation}
                \warp{\invvrmatI} = \AffMat \invvrmatI
            \end{equation}
            %---
            The warped position $\warp{\ptI}$, scale $\warp{\scaleI}$, and orientation $\warp{\oriI}$, can be
              extracted from $\warp{\invvrmatI}$ using~\cref{eqn:affinewarp}.
            The warped query keypoint properties in $\annotII$-space and can now be directly compared to the
              keypoint properties of their database correspondences.
            %Each warped point is checked for consistency with its
            %  correspondence's $\ptII$, scale $\scaleII$, and orientation
            %  $\oriII$, in $\annotII$.
            The absolute distance in position, scale, and orientation between the $\idxI$\th{} query feature and
              the $\idxII$\th{} database feature with respect to hypothesis transformation $\AffMat$ is measured as
              follows:
            \begin{equation}\label{eqn:inlierdelta}
                \begin{aligned}
                    \Delta \pt_{\idxI, \idxII}     & \eqv  \elltwo{\warp{\ptI} - \ptII}\\
                    \Delta \scale_{\idxI, \idxII}  & \eqv  \max(
                        \frac{\warp{\scaleI}}{\scaleII},
                        \frac{\scaleII}{\warp{\scaleI}}) \\
                    \Delta \ori_{\idxI, \idxII}    & \eqv  \min(
                        \modfn{\abs{\warp{\oriI} - \oriII}}{\TAU},\quad 
                        \TAU - \modfn{\abs{\warp{\oriI} - \oriII}}{\TAU})
                \end{aligned}
            \end{equation}

        \subsubsection{Selecting affine inliers}
            %Valid inliers are those matches that have all absolute differences
            %  within a certain spatial distance threshold $\xythresh$, orientation
            %  threshold $\orithresh$, and scale threshold $\scalethresh$.
            %  %$\xythresh$ is specified as a percentage of the matched chip's
            %  %  diagonal length.
            Any keypoint match $(\idxI, \idxII) \in \Matches_{\annotII}$  is considered an inlier \wrt{}
              $\AffMat$ if its absolute differences in position, scale, and orientation are all within a spatial
              distance threshold $\xythresh$, scale threshold $\scalethresh$, and orientation threshold
              $\orithresh$.
            This is expressed using the function $\isinlierop$, which determines if match is an inlier:
             %\begin{equation}
              %\label{eqn:inlierchecks}
              %    \begin{gathered}
              %    %\begin{aligned}
              %        \txt{isinlier}(\kp_1, \kp_2) \rightarrow \elltwo{\pt_1' - \pt_2} < \xythresh \AND \\
              %  %-----
              %        {\frac{\scale_1'}{\scale_2} < \scalethresh \txt{ if }
              %        \paren{\scale_1' > \scale_2} \txt{ else }
              %        \frac{\scale_2}{\scale_1'} < \scalethresh}  \AND\\
              %  %-----
              %        \txt{minimum}(
              %        \modfn{\abs{\ori_1' - \ori_2}}{\TAU},
              %        \TAU - \modfn{\abs{\ori_1' - \ori_2}}{\TAU}) < \orithresh
              %    %\end{aligned}
              %    \end{gathered}
            %\end{equation}
            \begin{equation}\label{eqn:inlierchecks}
                \begin{gathered}
                %\begin{aligned}
                    \isinlierop((\idxI, \idxII), \AffMat)  \eqv  
                        \Delta \pt_{\idxI, \idxII} < \xythresh \AND 
                        \Delta \scale_{\idxI, \idxII} < \scalethresh \AND 
                        \Delta \ori_{\idxI, \idxII} < \orithresh
                %\end{aligned}
                \end{gathered}
            \end{equation}
        The set of inlier matches for a hypothesis transformation $\AffMat$ can then be written as:
        \begin{equation}\label{eqn:affinliers}
            \Matches_{\AffMat} \eqv \curly{m \in \Matches_{\annotII} \where \isinlierop(m, \AffMat)}
        \end{equation}
        The best affine hypothesis transformation, $\HypothAffMat$, maximizes the weighted sum of inlier scores.
        % FIXME
        \begin{equation}
            \HypothAffMat \eqv \argmax{\AffMat \in \HypothSet} 
                \sum_{(\idxI, \idxII) \in \Matches_{\AffMat}} \fs_{\idxI, \idxII}
        \end{equation}

    \subsection{Homography refinement}
        Matches that are inliers to the best hypothesis affine transformation, $\HypothAffMat$, are used in the
          least squares refinement step.
        This step is only executed if there are at least $4$ inliers to $\HypothAffMat$, otherwise all
          correspondences between features in query annotation $\annotI$ to features in database annotation
          $\annotII$ are removed.
        The refinement step estimates a projective transform from $\annotI$ to  $\annotII$.
        To avoid numerical errors the $xy$-locations of the correspondence are normalized to have a mean of $0$
          and a standard deviation of $1$ prior to estimation.
        A more comprehensive explanation of estimating projective transformations using point correspondences can
          be found in~\cite[311--320]{szeliski_computer_2010}.

        Unlike in the affine hypothesis estimation where many transformations are generated, only one homography
          transformation is computed.
        Given a set of inliers to the affine hypothesis transform $\Matches_{\HypothAffMat}$, the least square
          estimation of a projective homography transform is:
        \begin{equation}
            \HmgMatBest \eqv \argmin{\HmgMat} \sum_{(i, j) \in
              \Matches_{\HypothAffMat}} \elltwosqrd{\HmgMat \pt_{i} - \pt_{j}}
        \end{equation}

        Similar to affine error estimation, we will identify the subset of inlier features correspondences
          $\Matches_{\HmgMatBest} \subseteq \Matches_{\annotII}$.
        A correspondence is an inlier if the query feature is transformed to within a certain spatial distance
          threshold $\xythresh$, orientation threshold $\orithresh$, and scale threshold $\scalethresh$ of its
          corresponding database feature.
        For convenience, let $\tohmg{\cdot}$ transform points into homogeneous coordinates.
        For each feature correspondence $(\idxI, \idxII) \in \Matches_{\annotII}$, the query feature position,
          $\ptI$, is warped from $\annotI$-space into $\annotII$-space.
        \begin{equation}
            \warp{\ptI} = \unhmg{\HmgMatBest \tohmg{\ptI}}
        \end{equation}
        However, because projective transformations are not guaranteed
          to preserve the structure of the affine keypoints, warped
          scales and orientations cannot be estimated with the method
          previously shown in~\cref{eqn:affinewarp}.

        \subsubsection{Estimation of warped shape parameters}
        Because we cannot warp the shape of an affine keypoint using a projective transformation, we instead
          estimate the warped scale and orientation for the $\idxI$\th{} query feature using a reference point.
        Given a single feature match $(\idxI, \idxII) \in \Matches_{\annotII}$, we associate a reference point
          $\refptI$ with the query location $\ptI$, scale $\scaleI$ and orientation $\oriI$.
        The reference point is defined to be $\scaleI$ distance away from the feature center at an angle of
          $\oriI$ radians in $\annotI$-space.
          \begin{equation}
            \refptI = \ptI + \scale_1 \BVEC{\sin{\oriI} \\ -\cos{\oriII}}
          \end{equation}

        To estimate the warped scale and orientation, first the reference
          point is warped from $\annotI$-space into $\annotII$-space.
        \begin{equation}
            \warp{\refptI} = \unhmg{\HmgMatBest \tohmg{\refptI}}
        \end{equation}
        %-----------
        Then we compute the residual vector $\ptres$ between the warped point and the warped reference point:
        \begin{equation}
            %\Delta \warp{\refptI} = \BVEC{\Delta \warp{\inI{x}} \\ \Delta \warp{\inI{y}}} = \warp{\ptI}- \warp{\refptI}.
            \ptres = \BVEC{\xres \\ \yres} = \warp{\ptI}- \warp{\refptI}.
        \end{equation}
        The warped scale is estimated using the length of the residual vector, and the warped orientation is
          estimated using the angle of the residual vector.
        In summary, the warped location, scale, and orientation of the $\idxI$\th{} query feature is:
        \begin{equation}\label{eqn:homogwarp}
            \begin{aligned}
                \warp{\ptI}      &\eqv \unhmg{\HmgMatBest \tohmg{\refptI}} \\
                 \warp{\scaleI}  &\eqv \elltwo{\ptres}\\
                %\warp{\oriI}    &= \atantwo{\yres, \xres} - \frac{\TAU}{4}.
                %\warp{\oriI}    &= \atantwo{\yres, \xres} - \frac{\pi}{2}.  % is this adjustment right?
                \warp{\oriI}     &\eqv \atantwo{\yres, \xres}
            \end{aligned}
        \end{equation}

        \subsubsection{Homography inliers}
        The rest of homography inlier estimation is no different than affine inlier estimation.
        \Cref{eqn:inlierdelta} is used to compute the errors $( %
        \Delta \pt_{\idxI, \idxII}, %
        \Delta \scale_{\idxI, \idxII}, %
        \Delta \ori_{\idxI, \idxII})$
        %
        between the warped query location, scale, and orientation, $(\warp{\ptI}, \warp{\scaleI}, \warp{\oriI})$, %
        and the corresponding database location, scale, and orientation, %
        $({\ptII}, {\scaleII}, {\oriII})$.
        The final set of homograph inliers is given as:
        \begin{equation}\label{eqn:homoginliers}
            \Matches_{\HmgMatBest} \eqv \curly{m  \in \Matches_{\annotII} \where \isinlierop(m, \HmgMatBest)}
        \end{equation}

        Spatial verification results in a reduced set of inlier feature correspondences from the query annotation
          to the database annotations.
        The \namescoring{} mechanism from~\cref{subsec:namescore} is then applied to these inlier feature
          correspondences.
        This final per-name score is the output of the identification algorithm and used to form a ranked list
          that is presented to a user for review.

     \sver{}
