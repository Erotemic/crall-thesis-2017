%\edef\pzcode{PZ_PB_RF_TRAIN_2567}
%\edef\gzcode{GZ_Master1_21589}
%%\edef\gzcode{GZ_Master1_222}


%\newcommand{\PairFeatVec}{
%\begin{figure}
%\begin{minted}[gobble=4]{python}
%    OrderedDict([('global(qual_min)',    3),
%                 ('global(qual_max)',    nan),
%                 ('global(qual_delta)',  nan),
%                 ('global(gps_delta)',   5.79),
%                 ('len(matches)',        20),
%                 ('sum(ratio)',          10.05),
%                 ('mean(ratio)',         0.50),
%                 ('std(ratio)',          0.09)])
%\end{minted}
%\caption[\caplbl{PairFeatVec}A pairwise feature vector]{\caplbl{PairFeatVec} %
%Example of a small pairwise feature vector containing local and global information.
%In practice the pairwise feature vectors include many more dimensions.
%}
%\label{fig:PairFeatVec}
%\end{figure}
%}


\newcommand{\MatchStateExample}{
\begin{figure}[h] \centering
\begin{subfigure}[h]{0.26\textwidth}\centering\includegraphics[height=160pt]{figures4/classesC.png}\caption{positive}\label{sub:classesC}\end{subfigure}
\begin{subfigure}[h]{0.31\textwidth}\centering\includegraphics[height=160pt]{figures4/classesA.png}\caption{negative}\label{sub:classesA}\end{subfigure}
\begin{subfigure}[h]{0.31\textwidth}\centering\includegraphics[height=160pt]{figures4/classesB.png}\caption{incomparable}\label{sub:classesB}\end{subfigure}
\caption[\caplbl{MatchStateExample}Match-state example]{\caplbl{MatchStateExample}
% ---
Examples of positive \cref{sub:classesC}, negative  \cref{sub:classesA}, and incomparable \cref{sub:classesB}
pairs of annotations. Local feature correspondences are superimposed over the pairs.
% ---
}
\label{fig:MatchStateExample}
\end{figure}
}


\newcommand{\LeftRightFace}{
\begin{figure}[h]
\centering
\includegraphics[width=\textwidth]{figures4/custom_match_leftrightface_5245_5161.jpg}
\caption[\caplbl{LeftRightFace}A comparable pair with different viewpoints]{\caplbl{LeftRightFace}
% ---
Even though this pair has different viewpoints, it is positive and comparable
because we can establish a distinctive correspondence in the face. 
% ---
}
\label{fig:LeftRightFace}
\end{figure}
}


% -------------------
% --- Experiments ---
% -------------------

\begin{comment}
    python -m ibeis Chap4.measure_all --db PZ_PB_RF_TRAIN
    python -m ibeis Chap4.measure_all --db GZ_Master1
    python -m ibeis Chap4.measure_all --db PZ_Master1

    python -m ibeis Chap4.draw_all --db PZ_MTEST
    python -m ibeis Chap4.draw_all --db PZ_Master1
    python -m ibeis Chap4.draw_all --db GZ_Master1
\end{comment}


\newcommand{\PositiveHist}{
\begin{figure}[h]
\centering
\begin{subfigure}[h]{0.47\textwidth}\centering\includegraphics[width=\textwidth]{figures4/PZ_PB_RF_TRAIN/score_hist_lnbnn.png}\caption{Plains zebras LNBNN}\label{sub:pos_lnbnn_hist_pz}\end{subfigure}
\begin{subfigure}[h]{0.47\textwidth}\centering\includegraphics[width=\textwidth]{figures4/PZ_PB_RF_TRAIN/score_hist_pos_learn(sum,glob).png}\caption{Plains zebras learned}\label{sub:pos_hist_pz}\end{subfigure}
\begin{subfigure}[h]{0.47\textwidth}\centering\includegraphics[width=\textwidth]{figures4/GZ_Master1/score_hist_lnbnn.png}\caption{Grévy's zebras LNBNN}\label{sub:pos_lnbnn_hist_gz}\end{subfigure}
\begin{subfigure}[h]{0.47\textwidth}\centering\includegraphics[width=\textwidth]{figures4/GZ_Master1/score_hist_pos_learn(sum,glob).png}\caption{Grévy's zebras learned}\label{sub:pos_hist_gz} \end{subfigure}
%~
\caption[\caplbl{PositiveHist}Positive score histogram experiment]{\caplbl{PositiveHist}
% ---
Positive scores of LNBNN (left) and pairiwse (right) matching for plains (top)
and Grévy's (bottom) zebras. In addition to being more interpretable than LNBNN
scores the learned probabilities exhibit better separability.
% ---
}
\label{fig:PositiveHist}
\end{figure}
}



\newcommand{\PositiveROC}{
\begin{figure}[h]
\centering
\begin{subfigure}[h]{0.47\textwidth}\centering\includegraphics[width=\textwidth]{figures4/PZ_PB_RF_TRAIN/roc_match_state.png}\caption{Plains zebra}\end{subfigure}
\begin{subfigure}[h]{0.47\textwidth}\centering\includegraphics[width=\textwidth]{figures4/GZ_Master1/roc_match_state.png}\caption{Grévy's zebra}\end{subfigure}
\caption[\caplbl{PositiveROC}Positive match-state ROC experiment]{\caplbl{PositiveROC}
% ---
The positive match-state ROC for scores computed using the pairwise classifier and the LNBNN ranking algorithm.
The scores from the pairwise classifier are better at separating out the positive cases.
% ---
}
\label{fig:PositiveROC}
\end{figure}
}


\newcommand{\ReRank}{
\begin{figure}[h]
\centering
\begin{subfigure}[h]{\textwidth}\centering\includegraphics[width=\textwidth]{figures4/PZ_Master1/rerank.png}\caption{Plains zebra}\end{subfigure}
\begin{subfigure}[h]{\textwidth}\centering\includegraphics[width=\textwidth]{figures4/GZ_Master1/rerank.png}\caption{Grévy's zebra}\end{subfigure}
\caption[\caplbl{ReRank}Re-ranking experiment]{\caplbl{ReRank}
% ---
Re-ranking the top LNBNN results using the positive probabilities from the match-state classifier improves the
  number of correct matches at rank $1$ for both plains and Grévy's zebra.
% ---
}
\label{fig:ReRank}
\end{figure}
}


\begin{comment}
    python -m ibeis.scripts.thesis ExptChapter4.write_metrics --db GZ_Master1 --task-key=match_state
    python -m ibeis.scripts.thesis ExptChapter4.write_metrics --db PZ_PB_RF_TRAIN --task-key=match_state
\end{comment}
\newcommand{\ConfusionMatch}{
\begin{table}[b]
    \centering
    \caption[\caplbl{ConfusionMatch}Match-state experiment confusion matrix]{\caplbl{ConfusionMatch}
    % ---
    Multiclass match-state confusion for plains and Grévy's zebras.
    The rows are the real (groundtruth) state, and the columns are the predicted states.
    The final column indicates the number of examples for each class.
    % ---
    }
    \label{tbl:ConfusionMatch}
    \begin{subtable}[h]{\textwidth}\centering\input{figures4/PZ_PB_RF_TRAIN/confusion_match_state.tex}\caption{Plains zebras match-state confusion}\end{subtable} %
    \begin{subtable}[h]{\textwidth}\centering\input{figures4/GZ_Master1/confusion_match_state.tex}\caption{Grévy zebras match-state confusion}\end{subtable} %
\end{table}
}


\newcommand{\EvalMetricsMatch}{
\begin{table}[b]
    \caption[\caplbl{EvalMetricsMatch}Match-state experiment evaluation metrics]{\caplbl{EvalMetricsMatch}
    % ---
    Multiclass match-state evaluation metrics for plains and Grévy's zebras
    computed from the confusion matrix.
    % ---
    }
    \label{tbl:EvalMetricsMatch}
    \centering
    \begin{subtable}[h]{\textwidth}\centering\input{figures4/PZ_PB_RF_TRAIN/eval_metrics_match_state.tex}\caption{Plains zebras match-state metrics}\end{subtable} %
    \begin{subtable}[h]{\textwidth}\centering\input{figures4/GZ_Master1/eval_metrics_match_state.tex}\caption{Grévy zebras match-state metrics}\end{subtable} %
\end{table}
}

\begin{comment}
    python -m ibeis.scripts.thesis ExptChapter4.write_metrics --db GZ_Master1 --task-key=photobomb_state
    python -m ibeis.scripts.thesis ExptChapter4.write_metrics --db PZ_PB_RF_TRAIN --task-key=photobomb_state
\end{comment}
\newcommand{\ConfusionPhotobomb}{
\begin{table}[h]
    \caption[\caplbl{ConfusionPhotobomb}Photobomb-state experiment confusion matrix]{\caplbl{ConfusionPhotobomb}
    % ---
    The photobomb-state confusion matrix. The columns indicate predicted classes, and the rows indicate real
    (groundtruth) classes.  The final column indicates the number of examples of each class.
    % ---
    }
    \label{tbl:ConfusionPhotobomb}
    \begin{subtable}[h]{\textwidth}\centering\input{figures4/PZ_PB_RF_TRAIN/confusion_photobomb_state.tex}\caption{Plains zebra photobomb confusion}\end{subtable} %
    \begin{subtable}[h]{\textwidth}\centering\input{figures4/GZ_Master1/confusion_photobomb_state.tex}\caption{Grévy's zebra photobomb confusion}\end{subtable} %
\end{table}
}


\newcommand{\EvalMetricsPhotobomb}{
\begin{table}[h]
    \caption[\caplbl{EvalMetricsPhotobomb}Photobomb-state experiment evaluation metrics]{\caplbl{EvalMetricsPhotobomb}
    % ---
    The photobomb-state evaluation metrics computed from the confusion matrix.
    % ---
    }
    \label{tbl:EvalMetricsPhotobomb}
    \centering
    \begin{subtable}[h]{\textwidth}\centering\input{figures4/PZ_PB_RF_TRAIN/eval_metrics_photobomb_state.tex}\caption{Plains zebras photobomb metrics} \end{subtable} %
    \begin{subtable}[h]{\textwidth}\centering\input{figures4/GZ_Master1/eval_metrics_photobomb_state.tex}\caption{Grévy zebras photobomb metrics}\end{subtable} %
\end{table}
}



\begin{comment}
    python -m ibeis.scripts.script_vsone report_classifier_importance --db PZ_PB_RF_TRAIN \
    --dpath ~/latex/crall-thesis-2017/ --save "figures4/wc_pz_clipwhite2.png" \
    --clipwhite --diskshow

    python -m ibeis.scripts.script_vsone report_classifier_importance --db GZ_Master1 \
    --dpath ~/latex/crall-thesis-2017/ --save "figures4/wc_gz_clipwhite2.png" \
    --clipwhite --diskshow
\end{comment}

\newcommand{\MatchWordCloud}{
\begin{figure}[h]
\centering
\begin{subfigure}[h]{0.47\textwidth}\centering\includegraphics[width=\textwidth]{figures4/PZ_PB_RF_TRAIN/wc_match_state.png}\caption{Plains zebra}\label{sub:wc_pz}\end{subfigure}
\begin{subfigure}[h]{0.47\textwidth}\centering\includegraphics[width=\textwidth]{figures4/GZ_Master1/wc_match_state.png}\caption{Grévy's zebra}\label{sub:wc_gz}\end{subfigure}
\caption[\caplbl{MatchWordCloud}Word cloud of important features for match-state prediction]{\caplbl{MatchWordCloud}
% ---
Word cloud of important features for predicting match-state.
For plains zebras, the difference in viewpoint (denoted as \pvar{global(view\_delta)}) has high importance
  because of its role in distinguishing incomparable cases.
% ---
}
\label{fig:MatchWordCloud}
\end{figure}
}



\newcommand{\ImportantMatchFeat}{
\begin{table}[h]
    \centering
    \caption[\caplbl{ImportantMatchFeat}Important features for match-state prediction]{\caplbl{ImportantMatchFeat} 
    % ---
    Top $5$ most important features for predicting the match-state (positive,
    negative, incomparable) for a pair of annotations.
    %The sum of the importance for all $205$ feature dimensions is $1$.
    % ---
    }
    \label{tbl:ImportantMatchFeat}
    \begin{subtable}[h]{0.44\textwidth}
        \centering
        \begin{tabular}{l c}
            \toprule
            dimension & importance\\
            \midrule
            \input{figures4/PZ_PB_RF_TRAIN/feat_importance_match_state.tex}
            \bottomrule
        \end{tabular}
        \caption{Plains zebra}
    \end{subtable} %
    ~~~~~~~~ 
    \begin{subtable}[h]{0.44\textwidth}
        \centering
        \begin{tabular}{l c}
            \toprule
            dimension & importance\\
            \midrule
            \input{figures4/GZ_Master1/feat_importance_match_state.tex}
            \bottomrule
        \end{tabular}
        \caption{Grévy's zebra}
    \end{subtable} %
\end{table}
}


\newcommand{\ImportantPBFeat}{
    \begin{table}[h]
        \centering
        \caption[\caplbl{ImportantPBFeat}Important features for photobomb-state prediction]{\caplbl{ImportantPBFeat} 
        % ---
        Top $5$ most important features for predicting if a pair of annotations has a photobomb.  Notice that these
        features are related to the spatial location of the matches and to global information like speed that might
        preclude a match from occurring. Annotations taken at the same time and place are more likely to contain
        photobombs.
        % ---
        }
        \label{tbl:ImportantPBFeat}
        \begin{subtable}[h]{0.44\textwidth} 
            \centering
            \begin{tabular}{l c}
                \toprule dimension & importance\\
                \midrule
                \input{figures4/PZ_PB_RF_TRAIN/feat_importance_photobomb_state.tex}
                \bottomrule
            \end{tabular}
            \caption{Plains zebra phototomb importance}
        \end{subtable} %
        ~~~~
        \begin{subtable}[h]{0.44\textwidth}
            \centering 
            \begin{tabular}{l c}
                \toprule
                dimension & importance\\ 
                \midrule
                \input{figures4/GZ_Master1/feat_importance_photobomb_state.tex}
                \bottomrule
            \end{tabular}
            \caption{Grévy's zebra phototomb importance}
        \end{subtable} %
    \end{table}
}


% ---------------------
% --- Failure Cases ---
% ---------------------


\newcommand{\PairFailIN}{
\begin{figure}[h]
\centering
\includegraphics[width=\textwidth]{figures4/PZ_PB_RF_TRAIN/cases_match_state/fail_Incomparable_Negative_383_503.jpg}
\caption[\caplbl{PairFailIN}Incomparable pairwise failure case]{\caplbl{PairFailIN}
% ---
The pair is incomparable, but the classifier predicts negative due to the small amount of incomparable training
data available.
% ---
}
\label{fig:PairFailIN}
\end{figure}
}

\newcommand{\PairFailNP}{
\begin{figure}[h]
\centering
\includegraphics[width=\textwidth]{figures4/PZ_PB_RF_TRAIN/cases_match_state/fail_Negative_Positive_424_465.jpg}
\caption[\caplbl{PairFailNP}Negative pairwise failure case]{\caplbl{PairFailNP}
% ---
This pair is negative, but the classifier predicts positive because of
similarity in pose and coarse patterns. The classifier is unable to recognize
that the shoulder is the same region on both animals and contains distinctive
patterns that do not match.
% ---
}
\label{fig:PairFailNP}
\end{figure}
}

\newcommand{\PairFailPN}{
\begin{figure}[h]
\centering
\includegraphics[width=\textwidth]{figures4/PZ_PB_RF_TRAIN/cases_match_state/fail_Positive_Negative_63_72.jpg}
\caption[\caplbl{PairFailPN}Positive pairwise failure case]{\caplbl{PairFailPN}
% ---
The pair is positive, but the classifier predicts negative because of occlusion and viewpoint variations.
% ---
}
\label{fig:PairFailPN}
\end{figure}
}
