\chapter{Graph identification}\label{chap:graphid}
\newcommand{\nT}{N}

%In this chapter we combine our ranking and verification algorithms from \Cref{chap:ranking,chap:pairclf} into a
%  framework that identifies animals by constructing a \glossterm{decision graph}.
In this chapter we frame the problem of animal identification in terms of constructing a %
\glossterm{decision graph}.
In this graph, each vertex is an annotation, and each edge represents a decision made between two annotations.
Edges determine if two annotations are the same (positive) or different (negative) individuals or if they cannot
  be compared (incomparable).
Thus, a correctly constructed decision graph naturally addresses the problem of identifying individual animals
  because each connected component of positive edges will be all the annotations from an individual.
Therefore, stated abstractly, goal of graph identification is to determine a correct, consistent set of edges in
  the decision graph.

To construct the decision graph, we develop a semi-automatic review procedure that combines the ranking and
verification algorithms presented in Chapters~\ref{chap:ranking} and \ref{chap:pairclf}. The ranking algorithm will
be used to suggest candidate edges to be placed in the graph, and the verification algorithm will be used to
automatically review as many edges as possible.  The key reason for combining these algorithms with a decision
graph is to take advantage of its connectivity information.  Connectivity not only identifies the individuals, but
it can also be used to develop graph measures of \emph{redundancy}, \emph{completeness}, \emph{consistency}, and
\emph{convergence}.  By combining these graph measures with the ranking and verification algorithms we can
prioritize edges for review based on both their pairwise probabilities and their ability to affect the consistency
of the graph, which in turn allows us to:
\begin{enumin}
\item increase confidence that the identifications are correct, %
\item reduce the number of manual reviews,  % 
\item detect and recover from review errors, and %
\item determine when identification is complete. %
\end{enumin}

Another important property of the graph identification framework is that it is agnostic to the underlying computer
vision procedures, which are abstracted into into three components:
\begin{enumin}
\item a ranking algorithm used to search for candidate positive edges, %
\item a verification algorithm used to automatically review edges, and %
\item a probability algorithm used assign probabilities to edges (note this is typically a by-product of the
ranking or verification algorithm).
\end{enumin}
In this thesis we use ranking algorithm from \cref{chap:ranking}, and the verification algorithm from
\cref{chap:pairclf} to define these components because these are suitable for identifying textured species.
However, while graph identification benefits from accurate computer vision subroutines, it can stand alone without
them.  This means that existing identification algorithms that only define a subset of these procedures (\eg{}
contour-based rank-only identification of humpback whales and bottlenose dolphins) could be seamlessly incorporated
into our framework and realize the benefits of graph identification (\eg{} a reduced number of manual reviews and
error recovery mechanisms).  Furthermore, because pairwise decisions are gathered and maintained by this framework,
verification algorithms can be retrained and improved, moving closer to a fully-automatic algorithm.


The first section (\Cref{sec:decisiongraph}) of this chapter formalizes the decision graph and summarizes the
  priority based review procedure used to construct it.
This provides a brief overview of each component of the processes, and then details are described in subsequent
  sections.
Then, \Cref{sec:rename} describes how existing database can be ported into this framework without extensive
  re-review.
\Cref{sec:graphexpt} experimentally demonstrates that ability of the graph identification algorithm to reduce the
  number of manual reviews and recover from errors.
\Cref{sec:graphconclusion} concludes and summarizes the chapter.

%This outline of this chapter is as follows: \cref{sec:decisiongraph} formalizes the decision graph and summarizes the
%priority based review procedure used to construct it. \Cref{sec:cand} describes candidate edge generation.
%\Cref{sec:decision} discuss how the verification algorithm is used to automatically review candidate edges.
%\Cref{sec:redun} introduces the redundancy criteria used to eliminate candidate edges. \Cref{sec:incon} describes the
%built-in mechanisms for error recovery. \Cref{sec:coverge} discusses the convergence criteria that determines when
%identification is complete. 


%%%%%%%%%%%%%%%%%%%%%%%%%%%%%%%%%%%%%%%%%%%%%%%%%%%%
\section{The decision graph}\label{sec:decisiongraph}

\decisiongraph{}

The graph identification algorithm is a review procedure formalized around the notion of a \glossterm{decision
graph} $G = (V, E)$ whose nodes are annotations and whose edges are suggested by a ranking algorithm (LNBNN in our
case) and decided upon by a combination of the probabilities output by a verification algorithm and by manual
review.  The edge set $E = E_p \cup E_n \cup E_i$ is composed of three disjoint sets. Each edge in $E_p$ is
\emph{positive}, meaning that it connects two annotations determined to be from the same individual. Each edge in
$E_n$ is \emph{negative}, meaning that it connects annotations determined to be from different individuals.
Finally, each edge in $E_i$ is \emph{incomparable}, meaning that it connects two annotations where it has been
determined that there is not enough information to tell to tell if they are from the same individual (\eg{} when
one annotation shows the left side of an animal and another other shows the right side).  An example of a decision
graph with all three edge types is illustrated in \cref{fig:decisiongraph}. The goal of graph identification is to
construct these edges.

The most important task is to determine the positive edges $E_p$.  This is because each connected component in the
subgraph $G_p = (V, E_p)$ corresponds to a unique individual.  Producing an accurate set of these
\glossterm{positive connected components} (PCCs) addresses solves the larger problem of animal identification.
However, an algorithm that only determines positive edges is not enough.  This is because the algorithm may have
failed to find all positive edges, resulting in two unconnected PCCs that should be \emph{merged} into one.

We can gain confidence that all positive edges have been found by using negative edges $E_n$, which provide direct
evidence that two annotations are different individual.  A negative edge between two PCCs means that no other
unreviewed edge between those PCCs can be positive.  Another important case is when a negative edge is contained
within a PCC. When this happens, the PCC is \emph{inconsistent}, and it implies that it contains at least one
mistake. Whenever an inconsistencies is detected, we resolve it using an algorithm we will define in
\cref{sec:incon} that restores consistency.

Lastly, incomparable edges, $E_i$, play a minor but necessary role by simply signifying that a positive or negative
decision cannot be made.  Incomparable edges can exist internally in a PCCs without causing inconsistencies or
between two PCCS without precluding them from being matched at a later point.

To reduce the number of potential reviews, notice, that once a group of nodes is connected by (a tree of) positive
edges, all those nodes in that PCC can be inferred to belong to the same individual, and it is not necessary to
consider any other edge internal to the PCC for review. Likewise once, a negative edges has been placed between two
PCCs, all edges between those PCCs can be ignored. By ignoring these redundant edges we can reduce the number of
reviews. Furthermore, if a negative edge is placed between every pair of PCCs, then all individuals must have been
discovered and identification has converged.

Unfortunately, there are several issues with these previous observations. These observations depend on the
condition that each edge was correctly reviewed. A small amount of redundant decisions is desirable because it
reduces the probability that errors have been made and signifies when errors have occurred by introducing
inconsistencies. Therefore we will define a redundancy criteria in \cref{sec:redun} which ignores edges within and
between PCCs, but only after they meet a minimum level of redundancy. Additionally this deterministic convergence
criteria would requires that $O(|V|^2)$ edges are reviewed as negative. We address this concern in
\cref{sec:coverge} using a probabilistic convergence criteria.

%Finally, we add a measure of redundancy to both the positive and negative
%  subgraphs to allow the algorithm to detect and recover from mistakes, either
%  in the automatic or manual decision making.

\subsection{The review algorithm}\label{sub:graphalgo}


% Algorithm overview
The review algorithm that produces the edges of a decision graph is outlined in Algorithm~\ref{alg:AlgoOverview}.
Akin to a segmentation algorithm~\cite{fulkerson_class_2009} that starts with an over-segmentation of an image, the
identification graph starts with an empty set of edges, $G = (V, \{ \})$, so in essence each annotation starts by
itself as an individual animal. Throughout the main algorithm, the graph is maintained in a \emph{consistent}
state, which means that each PCC has no internal negative edges.

\begin{algorithm}
    \begin{enumln}
    \item Generate and prioritize candidate edges 
    \item Insert candidate edges into a priority queue 
    \item Repeat until the priority queue is empty
    \begin{enumln}
        \item Pop an edge from the priority queue
        \item Make a decision and add the edge to the graph
        \item If the edge causes an inconsistency drop into inconsistency recovery mode
        \item Update the priority queue based on the new edge
        \item If candidate edges require refresh, goto step 1
    \end{enumln}
    \end{enumln}
\caption[Algorithm Overview]{Overview of the graph identification review procedure}
\label{alg:AlgoOverview}
\end{algorithm}

The first step of the algorithm is to generate candidate edges predict probability measures (positive, negative, or
incomparable) for each candidate edge. In the next step each edge is then entered into a priority queue. Next, the
algorithm enters a loop where the next candidate edge is selected, a decision is made about this edges --- either
automatically (as much as possible) or by the user --- and it is added to the graph. The algorithm proceeds toward
convergence by removing candidate edges from the priority queue, either directly from the top of the queue or
indirectly by eliminating candidate edges that are no longer needed. A candidate edge is no longer needed when
there are sufficient redundancies in the edge set within or between its PCCs. A pair of PCCs is \emph{complete}
when there are enough negative edges between them.

Each new edge addition could trigger two important events:
\begin{enumin}
    \item a \emph{merge} --- addition of a positive edge between different
      PCCs combines them into one PCC, and

    \item an \emph{inconsistency} --- addition of either a negative edge within a PCC or a positive edge between
    PCCs that already have a negative edge between them creates an inconsistent PCC.
\end{enumin}
Handling a merge is largely a matter of bookkeeping. Finding an inconsistency, however, drops the user into
inconsistency recovery mode where a cycle of hypothesizing one or more edges to fix and manually verifying these
with the user until consistency is restored.

Finally, the outer loop of the overall algorithm allows the ranking algorithm to generate additional candidate
edges --- this allows the ranking algorithm to take advantage of more subtle matches as the PCCs begin to form. The
priority queue will gradually be emptied as each PCC obtains a sufficiently redundant set of positive edges and
enough negative edges to be complete.
%Ensuring completeness requires examining $O(|V|^2)$ edges, so in practice we
%  develop a learned probabilistic completeness measure.
%If sufficient training data is not available simple heuristics can be used to
%  terminate.

Details of each component in the review algorithm are described in the following sections.  First we describe the
redundancy criteria in \cref{sec:redun} and inconsistency recovery in \cref{sec:incon}. These will serve to inform
the sections that define candidate edge generation, \cref{sec:cand}, decision making \cref{sec:decision},  and
convergence \cref{sec:coverge}.

% will be emptied when each PCC is sufficiently re
% Obtaining sufficient redundancy and completeness in order to empty the priority queue can,
% Ensuring all PCCs are complete leads to the need to ,
%  so to prevent this we develop a probabilistic measure (\cref{sec:coverge})
%  that triggers much earlier convergence when positive edges are no longer
%  likely to be found.

\section{Positive and negative redundancy}\label{sec:redun}
%One paragraph on notion.
%One paragraph on algorithm.
%One paragraph on book-keeping and elimination from priority queue.

Before describing the removal of edges from the priority queue and the detection and correction of inconsistencies,
we formalize the notion of redundancy that is the basis for both of these.
% Before we remove edges from the priority queue, we enforce a minimum level of redundancy.
%This $k$-redundancy criteria is tied to the number of mistakes that must be
%  made in order for part of the graph to incorrectly appear consistent or
%  complete.
%For a PCC (or pair of PCCs) to simultaneously contain a mistake and be
%  $k$-redundant then at least $k$ consistent mistakes must be made.
We define both a positive and a negative redundancy criteria: \begin{enumln}

    \item positive-redundancy --- % 
        A PCC is $k$-positive-redundant if its positive subgraph is $k$-edge-connected (contains no cut-sets
        involving fewer than $k$ positive edges\cite{eswaran_augmentation_1976}), or if the PCC has $k$ or fewer
        nodes and the union of positive and incomparable edges is forms a complete graph.

    \item negative-redundancy --- % 
        A pair of PCCs $C$ and $D$ is $k$-negative-redundant if there are $k$ negative edges between $C$ and $D$,
        or if either PCC has fewer than $k$ nodes and there are at least $\mathop{max}(|C|, |D|)$ negative edges
        between them.
        
    %which can be determined in $O(n_1 n_2)$ time.
    %(by looping over adjacency sets of nodes
    %in $C$ and performing set intersection with nodes in $D$ to get the edges
    %between $C$ and $D$).

    %$k$-negative-redundant if there are $k$ negative edges between them.
\end{enumln}
To understand these criteria better, consider what it means for an incorrect PCC that has been determined to be
$k$-positive-redundant to have an undiscovered error. The error means that the PCC really should be split into (at
least) two separate PCCs. Suppose these PCCs correspond to animals $C$ and $D$. If the combined PCC is
$k$-positive-redundant then are $k$ separate undiscovered mistakes connecting $C$ and $D$, and no negative edges.
This may be plausible if $C$ were identical twins, but these tend not to occur for species where the distinguishing
markings (\eg{} hip and shoulder of zebras) are mostly random. Note that $k$ can be different for positive and
negative redundancy, but in our current implementation we use $k=2$ for both positive and negative redundancy.

For positive-redundancy, determining that the minimal size cut-set in a PCC with $n$ nodes and $m$ edges is at
least $k$ is can be determined using edge-connectivity in $O(n)$ if $k \leq
3$~\cite{eswaran_augmentation_1976,wang_simple_2015}, and $O(mn)$ if $k > 3$ \cite{esfahanian_connectivity_2017}.
Determining if two components $C$ and $D$ with sizes $n_1$ and $n_2$ are $k$-negative-redundant can be done in
$O(n_1 n_2)$ time using adjacency lists and set intersections.

  \kredun{}

When a positive edge is added within a single PCC, we check for positive-redundancy. If this passes, all remaining
internal edges for that PCC may be  removed from the priority queue. When a negative edge is added between a pair
of PCCs, we run the negative-redundancy check on the pair, and if this passes, all remaining edges between the PCCs
may be removed from the priority queue. When a positive edge is added between a pair of PCCs, the two PCCs are
merged into a single new PCC $C'$, and the above negative-redundancy check must be run between $C'$ and all other
PCCs having a negative edge connecting to $C'$. It can be shown that if the graph is in a consistent state, that
these are the only updates required.

\section{Recovering from inconsistencies}\label{sec:incon}
Whenever a decision is made that either adds a negative edge within a PCC or adds a positive between two PCCs with
at least one negative edge between them, the graph becomes inconsistent. In both of these cases we add the edge and
create the result that there is a single PCC $C$ with internal negative edges. The goal of inconsistency recovery
mode is to change the labels of edges in order to make $C$ consistent. An inconsistency implies that a mistake was
made, but does not necessarily determine which edge contains the mistake. Therefore, we develop an algorithm to
hypothesize the edge(s) most likely to contain the mistake(s) using a minimum cut. An example of an inconsistent
PCC with hypothesis edges is illustrated in \cref{fig:inconpcc}.

\inconpcc{}

We describe the case where $C$ only contains one negative edge, but the general case replaces minimum $s$-$t$-cut
with multicut~\cite{vazirani_approximation_2013}. The procedure alternates between steps of generating ``mistake
hypothesis'' edges, and presenting these to the user for review. The ``hypothesis generation algorithm'' returns a
set of negative edges or a set of positive edges, which if the re-labeled as positive or negative respectively
would cause $C$ to become consistent. The algorithm starts by creating an instance of the min-cut using the
subgraph of $C$ containing only positive edges and the endpoints of the negative edge as the terminal nodes. The
weight for each edge is its initial priority plus the number of times that edge was manually reviewed. The minimum
cut returns a set of edges that disconnects the terminal nodes. We compare the total weight of cut positive edges
with the weight of the terminal negative edge (weighted using the same scheme). If the positive weight is smaller
the algorithm suggests that the cut positive edges should be relabeled as negative. Otherwise, it suggests that the
negative edge should become positive.

The user reviews each edge and the algorithm changes the label of the edge until the reviewer disagrees with the
algorithm's suggestion or the review set is empty. If consistency has not been restored, the algorithm must be
repeated. When this happens, the weights of the newly reviewed edges are increased by $1$ in order to force the
algorithm to look elsewhere for a cut. This repeats until all inconsistencies are eliminated.

%If this results in splitting one PCC into two or more, then the
%positive-redundancy and negative-redundancy tests must be repeated, potentially
%re-adding edges to the priority queue.

Fixing inconsistencies can result in splitting $C$ into multiple PCCs and invaliding implicit reviews inferred from
redundancy either within or incident to this subgraph. Therefore we recompute positive-redundancy within each new
PCC, implicitly reviewing edges where the criteria is satisfied and re-adding edges where it is no longer valid. A
similar process happens for negative-redundancy between each pair of new PCCs as well as between each new PCC and
all other PCCs previously negative-redundant with $C$.


%The algorithm starts by creating an instance of the multicut~\cite{vazirani_approximation_2013} using the
  %subgraph of $C$ containing only positive edges.
%Each edge is weighted by its assigned positive pairwise probability plus the number of times that edge was
  %manually reviewed.
%The terminal pairs are the negative edges.
%A feasible multicut returns a subset of  that disconnects all terminal pairs.
%Multicut is NP-hard, but it can be approximated by taking the union of min-cuts between each terminal pair.
%To transform the multicut into a mistake hypothesis, we compare the total weight of cut positive edges with the
  %total weight of the terminal negative edges (weighted using the same scheme).
%If positive weight is smaller we suggest that the cut positive edges should be relabeled as negative.
%Otherwise, we suggest the negative edges should be relabeled as positive.

%Inconsistency recovery proceeds as follows.
%Generate a mistake hypothesis, and order the edges by positive probability.
%Present each edge hypothesis to the user in order.
%If the user agrees with the hypothesis, then change the edge label, increment its review count, and continue.
%If the user disagrees, then generate a new hypothesis (using new weights and labels) and restart.
%It can be shown that this process is guaranteed to converge on a consistent graph state.
%Once $C$ is consistent, re-add it to $G$ and return to the main loop.


\section{Candidate edge generation and priorities}\label{sec:cand}

To generate candidate positive edges the we issue each annotation query to the ranking algorithm (the LNBNN from
\cref{chap:ranking}), forming edges from the resulting ranked lists. For each candidate edge we use the pairwise
algorithm (see \cref{chap:pairclf}) to estimate the positive, negative, and incomparable probabilities. Any edge
whose maximum probability is above the threshold for automatic decision making is ranked according to this
probability. All other edges are ordered by their positive probability. This ensures automatic decision making is
first, followed by an ordering of the edges needed for manual review that are most likely to be positive and
therefore add the most to the graph.

It is desirable to add positive decisions to the graph first because (1) they are the most important edges with
respect to determining the animal identities, and (2) larger PCCs increase the number of edges that can be
invalided by positive and negative reviews using the redundancy criteria.

In the case where the positive-redundancy criteria specifies $k=1$, the order positive edges are added in does not
matter because all PCCs will be trees and all trees with $n$ vertices have $n-1$ edges. Therefore, the optimal
priority scheme simply orders manually reviewed edges by positive probability. However, if $k>1$ the priority
scheme that minimizes the number of reviews is unclear and is an open question. This is partially because
edge-augmentation where some edges may not be feasible is NP-hard \cite{khuller_approximation_1993}. Even so, if
the PCCs were known ahead of time approximation algorithms could be used, but the fact that they are not makes this
problem difficult and an interesting topic for future research. One possible solution might involve prioritizing
edges based on a combination of their probability and current positive degree. Another might attempt to review
edges to achieve $1$-positive-redundancy and establish the PCCs while preferring to add positive edges to existing
chains, then once $1$-positive-redundancy was achieved the algorithm could increase $k$ and find candidate edges
using augmentation approximation algorithms.

\section{Making decisions}\label{sec:decision}

For each proposed new edge, its positive, negative, and incomparable state probabilities are produced by the
pairwise algorithm. Each of these three states have an associated hyperparameter threshold, and if a single
probability is greater than its threshold that edge is automatically added to the edge set corresponding to the
predicted state. If the candidate edge is within a consistent PCC, we can also use local edge connectivity to
determine if the new edge is within a $k$-connected component. In this case adding this edge to the graph would be
redundant and we can simply discard it. In all other cases, the edge is sent for manual labeling and then added to
the appropriate edge set. After each decision we determine the effects of the new review and update candidate edge
priorities as we will discuss in\cref{sec:redun}. However, if the edge causes an inconsistency we drop into
inconsistency mode. Inconsistency recover will be discussed in \cref{sec:incon}.

Whenever a decision is made we record a user-id to identify the reviewer or algorithm making the decision. We also
follow the approach of~\cite{branson_visual_2010} and store a user-specified categorical confidence value of
guessing, not-sure, pretty-sure, and absolutely-sure. Currently the user-id is unused but the user confidence is
used in error detection and recovery.


\section{Refreshing candidate edges}\label{sec:refresh}

As the review process executes we want to continue to review positive matches as long as we are discovering them.
However, at some point the candidate edges may no longer contain positive results, but undiscovered positive matches may
still exist. This is because LNBNN, working initially with each annotation having a separate label, can miss more subtle
but correct matches, especially when there are several annotations for an animal and subtle viewpoints. As the labeling
improves, so does the reliability of the LNBNN. Therefore, we define a refresh criteria to determine when we should
recompute candidate edges.

The goal is to refresh if there has been a significant number of positive reviews, but new results are consistently
negative. If we have not found any positive edges then we do not want to refresh. We keep track of the fraction of
positive review decisions as a moving average of manual decisions. We also maintain the total number of positive reviews
made since the last candidate edge generation. Thus the candidate edges are refreshed whenever the number of positive
reviews is above a threshold and the positive review fraction is below a threshold

As the last outer iteration of the overall algorithm before convergence, triggered when the LNBNN ranking algorithm
fails to produce positive edges, candidate edges between untested pairs of annotations are added within PCCs that are
not positive-redundant and between PCCs that are not negative-redundant. This is because the ranking algorithm itself is
imperfect and the missed matches tend to affect small PCCs disproportionately, which are the last to satisfy redundancy
tests.


\section{Convergence}\label{sec:coverge}

To be completely certain that the current set of PCCs is optimal, each pair of
PCCs must contain $k$ negative edges or all edges between them must be labeled
as incomparable. Achieving this is infeasible for large database because it
requires $O(|V|^2)$ reviews. Therefore we develop two probabilistic measures 


\subsection{Probabilistic convergence}
The goal of probabilistic convergence is to determine if a PCC $C$ is negative-redundant with all other components with
high probability. When all components are positive-redundant and satisfy this, then all edges will be removed from the
priority queue and the algorithm will converge. We consider the probability $\Pr{E_c \given \nT_C}$ that an undiscovered
positive edge exists ($E_C$) given $C$'s existing set of outgoing negative edges ($\nT_C$). Under mild conditions (if we
assume that $\Pr{E_c \given \nT_C} < .5$), we can show that the probability $\Pr{\nT_C \given E_C}$ of observing the
negative edges bounds this p given that an undiscovered match exists can be used as a surrogate. We can learn this
probability offline by measuring the frequency that correct results are at a given rank in a PCC's ranked list
(constructed by aggregating the ranked lists of all annotations in the PCC).

To predict $\Pr{\nT_C \given E_C}$ we issue all queries as a single LNBNN query to obtain a single ranked list for the
entire PCC. This can be done by treating all query descriptors as if they were the from the same annotation except
during the spatial verification stage. Let $R_C$ denote the ranks of every PCC marked as negative with $C$. In an
offline step we learn a probability mass function $\phi$ that predicts the probability that a correct match appears at a
given rank for the PCC $C$. The predicted probability is %
$\Pr{\nT_C \given E_C} = 1 - \sum_{r \in R_C} \phi(r)$.

To learn $\phi$, we measure the probability that a correct match appears at a given rank, given a correct match exists.
To do this initialize an histogram. For each $C$ in the training set, divide it into a query $C_q$ and target $C_t$. The
target and the rest of the PCCs in the training set become database PCCs. Use LNBNN to score each annotation in $C_q$
against the database PCCs. Determine the best rank that $C_t$ appears in each ranked list, and increment the
corresponding index in the histogram. Repeat this process for all PCCs in the training set and for multiple partitions
of each PCC. Normalizing the histogram array results in the PMF $\phi$. In order to prevent marginalization across
important attributes (such as the number of exemplars in a PCC), construct multiple PMFs for different numbers of
exemplars in a query.


\subsection{Convergence as a Poisson process}

We model both the refresh and convergence criteria by considering the question ``What is the likelihood that
  there will be a positive match anytime soon''.
We can obtain an upper bound on this probability using a Poisson process.


An alternative approach for determining when to stop is to consider the question: ``How many unreviewed edges in the
graph are likely to be positive?''. We can obtain an upper bound on this number by modeling the probability that a new
edge will be positive as a Poisson processes.

We consider a window of previous reviews between annotations where (at the time) the annotations belonged to different
names. A fraction of these reviews will be labeled as positive, resulting in a merge. This fraction $\mu$ is the mean of
a Poisson distribution. Multiplying $\mu$ by a positive integer $k$ estimates the expected number of positive matches we
will observe in next $k$ reviews between different PCCs. If the current number of PCCs is $N$, and we know $M$ pairs of
PCCs are negative-redundant, then the number of edges to complete the negative labeling is $k=\binom{N}{2} - M$. Thus,
at any point we can estimate the number of undiscovered positive reviews as $k\mu$.

In practice, this number will be much greater than the actual number of remaining merges because our event is not a
strict Poisson process. This is because the probability of observing a positive edge is not constant over time, it
depends on the previously observed positive edges. However, because each positive review removes other positive reviews
remaining in the graph, and because we prioritize by positive probability, the probability of observing a positive edge
will decrease with each new review. This is what allows us to use the Poisson process as an upper bound.

%Re-estimating the $k\mu$ at each time-step should


\section{Converting existing datasets}\label{sec:rename}
To apply graph identification to a previously existing dataset where annotations have been assigned name labels and
connectivity between the annotations is unknown we follow the following process. First we compute the pairwise
probabilities between each pair of annotations labeled with the same name. We automatically review any edge above a
threshold as positive, negative, or incomparable (these may potentially result in inconsistencies). For any unconnected
PCC we compute an edge augmentation to connect the PCC \cite{eswaran_augmentation_1976,khuller_approximation_1993}, and
insert these edges into the graph labeling them as positive but assigning them the confidence of guessing.

It will be common for such datasets to contain errors, we resolve any inconsistent PCCs as normal, but then we search
for additional split cases using the pairwise classifier. In the case that a pairwise classifier does not exist, the
temporary edges defined by the maximum spanning trees can be used to construct one. The main idea is to re-review all
edges where the pairwise classifier prediction disagrees with its assigned match state. Edges are sorted by the
magnitude of the disagreement, but any edge with a confidence of absolutely-sure is ignored. This will present edges
labeled as guessing for the user to re-review. At this point the dataset is in a legal state, where the name labels
correspond to PCCs. The final step is to execute normal graph review in order to find any merge cases and explicitly
label hard negative edges.

It is sometimes desirable new PCCs to keep the old name labels from the original database (\eg{} sometimes ecologists
encoded information in these names). This is a simple matter when the original database contained no mistakes, but when
the original database contains errors care must be taken. We address this problem by seeking to minimize the number of
annotations that have their name label changed from the original dataset. This can be computing by finding a maximum
linear sum assignment using the Munkres algorithm implemented in SciPy~\cite{eric_jones_scipy_2001}. We create a matrix
where each rows represents a group of annotations in the same PCC and each column represents an original name. If there
are more PCCs than original names the columns are padded with extra values. The matrix is first initialized to be
negative infinity representing impossible assignments. Then for each column representing a padded name, we set we its
value to $1$ indicating that each new name could be assigned to a padded name for some small profit. Finally, we encode
both the profit of assigning a new name with an original name and the extra one ensures that these original names are
always preferred over padded names. Let $f_{rc}$ be the number of annotations in row $r$ with an original name of $c$,
and set matrix value $(r, c)$ to $f_{rc} + 1$ if $f_{rc} > 0$. The maximum linear sum assignment of this matrix results
in the optimal consistent assignment of PCCs to original name labels.
  
  
\section{Experiments}\label{sec:graphexpt}

    TODO: make figures for PZ and GZ

    In this section we design an experiment to measure the impact of graph identification on the number of manual
      reviews required to complete identification as well as the accuracy of those identifications.
    Our experiments simulates the semi-automatic identification process from a user's perspective.
    We consider three algorithms:
    (1) our graph identification algorithm,
    (2) a baseline ranking-based protocol, and
    (3) an intermediate ranking protocol that combines the ranking algorithm with pairwise classifier.
    To simulate our algorithms, we model noisy user response using ground-truth data.
    The simulated user returns the ground-truth classification $98\percent$ of the time, making errors
      $2\percent$ of the time uniformly at random.

    The protocol for the graph identification algorithm is defined in \cref{sub:graphalgo}.
    The pairwise classifier, corresponding thresholds, and termination criteria are learned on a disjoint
      training set.
    Because our algorithm is able to handle errors, we set our automatic classification thresholds to achieve a
      false positive rate of $.1\percent$ on a validation set.
    We disable automatic review for incomparable cases due to the small number of labeled training examples.

    The baseline algorithm captures the effect of using a purely rank-based approach for animal identification.
    We first run the ranking algorithm and add the top $5$ results from each query to a list sorted by the LNBNN
      score.
    The user reviews each result in the list sequentially, without regard to the underlying graph structure.
    Once, new matches are consistently negative (using the approach from~\cref{sec:refresh}), we regenerate a new
      ranked list and iterate.
    We terminate after two rounds of ranking and review.

    The intermediate protocol will demonstrate the effect of augmenting the baseline algorithm with our pairwise
      classifier.
    The procedure is the exactly the same as the baseline algorithm, except that any item with a predicted
      probability above a threshold is automatically reviewed.
    Because this approach has no mechanism for error recovery, we choose conservative classification thresholds
      that achieve a $0\percent$ false positive rate on a validation dataset.

    In all tests we record two measurements after each review pertaining to accuracy and error.
    (1) The accuracy measurement is the number of merges remaining before all individuals have been identified.
    This is the number of edges in a spanning forest of the ground-truth positive subgraph minus same measurement
      but applied to the subgraph of all correctly predicted positive edges.
    (2) The error measurement is the total number of edges with a predicted state that differs from its
      ground-truth state.

    Results of this test for the graph, intermediate and baseline algorithms are illustrated in \cref{fig:ETE}
      for each dataset.
    These results demonstrate that the algorithms using the pairwise classifier significantly improves the rate
      at which annotations from the same individual are merged.
    However, the graph algorithm better suited to take advantage of the pairwise classifier because it can afford
      to a make initial mistakes and expect to recover from them later.
    The baseline and intermediate algorithm have no mechanism for error recovery, and thus their error steadily
      increases over time.
    In these cases a significant number of individuals will be misidentified.
    These results also demonstrate that the graph algorithm is able to recover from many of these errors.
    We note that the remaining errors might be resolved if more images of individuals involved in those errors
      are added to the system.

 
\section{Conclusion and summary of graph identification}\label{sec:graphconclusion}

TODO

%A nice and important property of the graph identification framework is that it is agnostic to the underlying procedures
%used to generate and review candidate edges to be placed in the decision graph. Throughout this chapter we will have
%used the ranking algorithm from \cref{chap:ranking}, and the semi-automatic verification algorithm from
%\cref{chap:pairclf}, which work well for distinctively textured species like zebras, giraffes, jaguars, and lionfish.
%However, the benefits of graph identification algorithm could be realized for less textured species such as humpbacks
%whales or bottlenose dolphins by simply using existing contour-based ranking algorithms and a manual reviewer.

%The framework requires two external components:
%\begin{enumin}
%    \item a ranking algorithm to search for candidate edges, and
%    \item a verification algorithm (either manual or automatic) that can
%      decide if a pair of annotation is positive, negative, or incomparable.
%  \end{enumin}

%The verification algorithm (which can be a manual reviewer) is what determines if the graph identification
%  algorithm is manual, semi-automatic, or fully-automatic.
%In our experiments we use the ranking algorithm from \cref{chap:ranking}, and the verification algorithm from
%  \cref{chap:pairclf} to construct these components, which work well for
  
%These algorithms could easily be swapped out for other algorithms tuned
%  towards specific species (\eg{} contour-based ranking algorithms for marine
%  species).
%(or any object that can be visually identified).

Furthermore, as more pairwise training data is gathered and maintained using this framework, more sophisticated
  pairwise classifiers trained using deep learning could be applied, perhaps removing the manual component
  completely and resulting in a fully automatic identification algorithm.

Because of this flexibility the graph identification framework generalizes beyond animal identification and could
  be applied to any instance recognition problem using algorithms tuned for those tasks under the condition that
one annotation corresponds to one individual object (note the photobomb classifier or a segmentation mask can be
  used to relax this
  constraint).
