\begin{comment}
    ./texfix.py --outline --fpaths chapter1-intro.tex
    ./texfix.py --fpaths chapter1-intro.tex --outline --asmarkdown --numlines=999  -w
    ./texfix.py --grep "\\\\[A-Za-z]*[^{a-zA-Z]"
    ./texfix.py --reformat --fpaths figdef1.tex
\end{comment}


\chapter{Introduction}\label{chap:intro}

\section{Image based identification applied to population ecology}

    Population ecology relies on estimating the number of individual animals that inhabit an
    area~\cite{krebs_ecological_1999}. Estimating a population size is done in two phases: data collection and analysis.
    Data are collected as a set of \glossterm{sighting} and \glossterm{resighting} observations. A sighting is the first
    observation of an individual, and a resighting is a subsequent observation of a previously sighted individual. The
    observed data are then analyzed using software such as program MARK~\cite{white_program_1999,
    schwarz_jollyseber_2006} or Wildbook that applies statistical models such as the Lincoln-Petersen
    index~\cite{seber_estimation_1982}, Jolly-Seber model,~\cite{jolly_explicit_1965, seber_note_1965}, or other related
    models~\cite{cormack_estimates_1964, chao_estimating_1987,kenneth._h._pollock_statistical_1990}. For an ecologist
    recording that an individual has been observed is simple, but determining if that observation is a sighting or a
    resighting can be challenging. This requires the ecologist to identify the individual against all other observations
    in the data set.

    Current methods to estimate a population size are limited by the data collection
    phase~\cite{sundaresan_network_2007, rubenstein_ecology_2010}. The statistical population models require an
    observation sample size that grows with the size of the population being studied~\cite{seber_estimation_1982}. As
    the number of observations increases so does the difficulty of determining identity. Thus the scope of a population
    study is limited by the number of raw observations that can be made, and by the rate of determining the individual
    identity within a set of observations. Overcoming these limitations is of particular importance to wildlife
    preservation because population statistics are necessary to guide conservation
    decisions~\cite{rubenstein_behavioral_1998}.

    Consider images as a source of sight-resight observations. There are numerous advantages. Many observations can be
    made rapidly and simultaneously, due to the simplicity and availability of cameras. Recording an observation is as
    cheap and simple as taking a picture. Camera traps can be employed for autonomous data collection. In a wildlife
    conservancy or national park, observations can be crowd-sourced by gathering images from safari tourists and citizen
    scientists. Images can be accumulated and stored in a large dynamic dataset of observations that grows by thousands
    of images each day. However, the challenge of identifying the individuals in the images remains. Manual methods are
    infeasible due to the rapid rate at which images can be collected. Therefore, we must turn towards computer vision
    based methods.

    This \thesis{} develops the foundation of the image analysis component of \IBEIS{} --- the Image Based Ecological
    Information System. The purpose of this system is to gain ecological insight from images using computer vision. We
    focus on estimating the size of a population of animals as just on example of ecological insight that might be
    gained from images. Thus we come to the core problem addressed in this \thesis{}: image based identification of
    individual animals.

\section{Challenges of animal identification}\label{sec:challenges}

    In animal identification we are given a database of images. This database may initially be empty. Each image is
    cropped to a bounding box around an animal of interest and labeled with that animal's identity. For a new query
    image, the goal is to determine whether or not any other images of the individual are in the database. If the query
    is matched, it is added to the database as a resighting of that individual. If the query is not matched, then it is
    added as a new individual.

    In this work we focus on identifying individuals of species with distinguishing textures, such as zebras, giraffes,
    lionfish, nautiluses, hyenas, whale sharks, wildebeest, wild dogs, jaguars, cheetahs, leopards, frogs, toads,
    snails, and seals. The three species this \thesis{} will focus on are plains zebras, Grevy's zebras, and Masai
    giraffes. The difficulty of animal identification depends on the distinctiveness of the visual patterns that
    distinguish an individual from others of its species. In addition, the images we identify are collected ``in the
    wild'' and therefore contain occlusion, distracting features, variations in viewpoint and quality.

    This section will present several examples to illustrate the challenges faced in animal identification. The
    discussion will begin with the challenges posed by the three primary species. Then the problems common to all
    species will be described. These will be illustrated using plains zebras because they are the most challenging
    species considered in this \thesis{}.

    \subsection{Distinguishing features of each species}
        The plains zebra --- shown in~\cref{fig:PlainsFigure} --- is challenging to visually identify because
        individuals have relatively few distinguishing features. For most plains zebras, the majority of distinctive
        information lies in a small area on the front shoulder. \Cref{fig:HardCaseFigure} illustrates that the patterns
        that distinguish two individuals can be subtle, even when the features are clearly visible. The matching
        difficulty greatly increases when features are partially occluded, the viewpoint changes, or the image quality
        is poor.

        In contrast, Masai giraffes and Grevy's zebras, shown in~\cref{fig:GirMasaiFigure} and~\cref{fig:GrevysFigure}
        respectively, have an abundance of distinctive features. Distinctive textures that are unique to each individual
        are spread across the entire body of a Masai giraffe. For a Grevy's zebra there is a high density of
        distinguishing information above both front and back legs, as well as a moderate density of distinctive textures
        along the side of the body. The high density of distinctive textures in Masai giraffes and Grevy's zebras make
        it possible to distinguish them from one another even in the presence of occlusion and viewpoint changes.
        However, these factors still increase the difficulty of the problem.

        \PlainsFigure{}

        \HardCaseFigure{}

        \GirMasaiFigure{}

        \GrevysFigure{}

    \subsection{Viewpoint and pose}
        One of the most difficult challenges faced in the animal identification problem is viewpoint. Animals are seen
        in a variety of poses and viewpoints, which can cause distinctive features to appear distorted. The patterns on
        the left and right sides of animals are almost always asymmetric. Therefore, matches can only be established
        using overlapping viewpoints and only if the viewpoints are distinctive. Some viewpoints, such as the backs of
        plains zebras, lack distinguishing information as shown in~\cref{fig:BacksFigure}. The effect of pose and
        viewpoint variation can be seen in~\cref{fig:ThreeSixtyFigure} and~\cref{fig:PoseFigure}.

        \BacksFigure{}

        \ThreeSixtyFigure{}

        \PoseFigure{}

    \subsection{Occluders and distractors}
        Because images of animals are often taken ``in the wild'', other objects in the image can act as
        \glossterm{occluders} or \glossterm{distractors}. Objects such as grass, bushes, trees or other animals, can act
        as occluders by partially obscuring the features that distinguish one individual from another. The appearance of
        the other animals nearby can be distracting because features from these animals will match different animals in
        the database. These \glossterm{distractors} may also be from non-animal features when multiple pictures are
        taken against the same background as animals move through the same field of view. Several examples of occlusions
        and distractors are illustrated in~\cref{fig:OccludeFigure}

        \OccludeFigure{}

    \subsection{Image quality}
        Image quality is influenced by lighting, shadows, the camera used, image resolution, and the size of the animal
        in the image. Outdoor images will naturally have large variations in illumination. Different cameras can produce
        visual differences between images of an object. Images taken out of focus, from far away, or with a non-steady
        camera can cause animals to appear blurred. The effects of outdoor shadow and illumination are illustrated
        in~\cref{fig:IlluminationFigure}. \Cref{fig:QualityFigure} illustrates five categories of image quality that
        will be described later in~\cref{sub:viewqual}.

        \IlluminationFigure{}

        \QualityFigure{}

    \subsection{Aging and injuries}
        The appearance of an individual changes over time due to aging and other factors including injuries. An example
        of the difference between a juvenile and adult zebra is shown in~\cref{fig:AgeFigure}. An example of how
        injuries can both remove distinctive features and add new ones is shown in~\cref{fig:GashFigure}.

        \AgeFigure{}

        \GashFigure{}

\section{The \GZCFull{}}\label{sec:introgzc}

    To further illustrate the problems addressed in this \thesis{}, we consider the \GZCFull{} (\GZC{}), held at Nairobi
    National Park on March 1\st{} and 2\nd{}, 2015. This event was designed with two purposes in mind: (1) to involve
    citizens in the scientific data collection effort, thereby increasing their interest in conservation, and (2) to
    determine the number of plains zebras and Masai giraffes in the park.

    \subsection{Data collection}
        Volunteer participants --- each with his or her own camera --- arrived by car at the park. Some cars had more
        than one photographer. Each car was assigned a route to drive through the park. We attached a GPS dongle to each
        car to record time and location throughout the drive. Correlating this with the time stamp on each image (after
        adding a correction offset for each camera) allowed us to determine the geolocation of each image. Each
        photographer was given instructions guiding them toward taking quality images of the left sides of the animals
        they saw. When the cars returned --- some after just an hour or two, others after the whole day --- the images
        were copied from the cameras, a small sample of each photographer's images was immediately processed to
        illustrate what we would do with the data, and the entire set of images was stored for further processing. The
        result of this crowd-sourced collection event was a 48GB dataset consisting of $9406$ images.

    \subsection{Data processing}\label{subsec:introdataprocess}

        After the event, the entire collection of images was processed using a preliminary version of the system in
        order to generate the final count. The preliminary system followed the workflow of: %
        \begin{enumin}
            \item ingest images  %
            \item \occurrence{} grouping  %
            \item animal detection %
            \item viewpoint and quality labeling  %
            \item \intraoccurrence{} matching %
            \item \vsexemplar{} identification %
            \item consistency checks  %
            \item population estimation.  %
        \end{enumin}
        %\Cref{chap:application} discusses this workflow
        %in greater detail. 
        Here, we provide a brief overview of each step involved in the processing of the \GZC{} image data, and then we
        will describe the challenges that arose.

        \subsubsection{Occurrence grouping}
            The images were first divided into \glossterm{\occurrences{}} --- a standard term defined by the Darwin
            Core~\cite{wieczorek_darwin_2012} to denote a collection of evidence (\eg{} images) that an organism exists
            within defined location and time-frame. In the scope of this application, an \occurrence{} is a cluster of
            images taken within a small window of time and space. Images are grouped into \occurrences{} using the GPS
            and time data. Details are provided in~\cref{sec:occurgroup}.

            There are several benefits to first grouping images into an \occurrence{}. One benefit is that an
            \occurrence{} can be used as a semantic processing unit to distribute manageable chunks of work to users of
            the system. Another is that \occurrences{} can be used to improve the results of identification. Typically
            there will be only a small number of individuals within an \occurrence{}, and it is not uncommon for each
            individual to photographed multiple times and from multiple viewpoints. This redundancy in images will be
            exploited in \Cref{chap:graphid}.

        \subsubsection{Animal Detection}
            Before matching begins each image is cropped to focus on a particular animal and remove background
            distractors. A detection algorithm localizes animals within the images. Each verified detection generates an
            \glossterm{\annot{}} --- a bounding box around a single animal in an image. An example illustrating
            detection of plains zebras is shown in~\cref{fig:DetectFigure}. In the \GZC{} each detection was manually
            verified before becoming an \annot{}, but recent work introduces an automatic verification mechanism and
            reduces the need for complete manual review. The details of the detection algorithm are beyond the scope of
            this \thesis{}, and will be described in the work of Jason Parham~\cite{parham_photographic_2015}.

            \DetectFigure{}

        \subsubsection{Viewpoint and quality labeling}\label{sub:viewqual}
            When determining the number of animals in a population it is important to account for factors that can lead
            to over-counting. If two \annots{} of the same individual are not matched, then that individual will be
            counted twice. This could happen due to factors such as viewpoint and quality. For example, one \annot{}
            showing the only left side of an animal and another \annot{} showing only the right side the same animal
            cannot be matched because they are \glossterm{incomparable}. The two \annots{} are comparable when they
            share regions with distinguishing patterns that can be put in correspondence. Viewpoint is the primary
            reason that two \annots{} are not comparable. However, other factors like image quality and heavy occlusion
            can corrupt distinguishing patterns rendering the \annot{} unidentifiable --- not comparable with any other
            \annot{}. We must define what it means for two \annots{} to be comparable before we can estimate a
            population size.

            Determining if an individual can be identified is analogous to the
            notion of a marked-individual~\cite{seber_estimation_1982}. For an
            \annot{} to be identifiable the patterns that can distinguish it
            from the rest of the population must be clear and visible, otherwise
            the \annot{} may not be able find or be compared to potential
            matches. This means an \annot{} is only identifiable if
            \begin{enumin}
                \item the image quality is high enough, and %
                \item it has a viewpoint that is comparable to all potential
                matches. %
            \end{enumin}
            
            To address this challenge we label each \annot{} with $5$ discrete quality labels and $8$ discrete viewpoint
            labels. The quality labels we define are: \qualJunk{}, \qualPoor{}, \qualOk{}, \qualGood{}, and
            \qualExcellent{}. The \qualJunk{} label is given to \annots{} that almost certainly will not be able to be
            identified, and \qualPoor{} labels are given to \annots{} that will likely be unidentifiable for a computer
            vision algorithm. The $\qualGood{}$ and \qualExcellent{} labels are given to clear well illuminated
            \annots{} with little to no occlusion with \qualExcellent{} being reserved for the best of the best. All
            other \annots{} are labeled as $\qualOk$. The viewpoint labels we define are: \vpFront{}, \vpFrontLeft{},
            \vpLeft{}, \vpBackLeft{}, \vpBack{}, \vpBackRight{}, \vpBack{}, and \vpFrontRight{}. Note, that additional
            viewpoint labels like $\vpUp{}$ and $\vpDown{}$ may be necessary for animals such as lionfish or turtles.
            However, the $8$ labels we use are sufficient for animals like zebras and giraffes because they are most
            commonly seen in upright positions.

            In an effort to ensure that all \annots{} used in the \GZC{} were comparable, we did not include any
            \annot{} that had junk or poor qualities. We also did not include \annots{} not labeled with a left or
            frontleft viewpoint to account for limitations in the initial ranking algorithm. All labelings of viewpoint
            and quality were generated manually. Since then we have trained viewpoint and quality classifiers using this
            manual data. Automatic detection of quality and viewpoint is discussed in Jason Parham's
            work~\cite{parham_photographic_2015}.

        \subsubsection{Matching within each \occurrence{}} %
            Animals often have multiple redundant views within an \occurrence{}, which can be the same, better, or
            complementary to other views. The images in~\cref{fig:OccurrenceComplementFigure} illustrate redundant and
            complementary views of an individual in an \occurrence{}. Merging all of an individual's views is a
            challenge, but also potentially an advantage as we can exploit redundancy to better handle missing features,
            subtle viewpoint changes, and occlusions.

            We exploit this redundancy to gain the benefit of complimentary views by matching all \annots{} within an
            \occurrence{} in a process called \glossterm{\intraoccurrence{} matching}. In the \GZC{}, each \annot{} was
            queried against all other \annots{} in its \occurrence{}, returning a ranked list of candidate matches. The
            person running the software made the final decisions about which \annots{} match. Details about the ranking
            algorithm are given in~\cref{chap:ranking}.

            The result of \intraoccurrence{} matching is a set of \glossterm{\encounters{}}. \Aan{\encounter{}} is a
            group of \annots{} that were matched within an \occurrence{}. Each \encounter{} is either (1) the first
            sighting an individual or a (2) resighting. The task now becomes to determine which of these is the case by
            identifying each \encounter{} against a \masterdatabase{}.

            \OccurrenceComplementFigure{}
 
        \subsubsection{Matching against the \masterdatabase{}} %
            To determine if \aan{\encounter{}} is a new sighting or a resighting of an individual, it is matched against
            the \masterdatabase{} in a process called \glossterm{\vsexemplar{} matching}. Before matching begins the
            \masterdatabase{} is prepared for search. For each \name{} in the \masterdatabase{} a subset of
            \glossterm{\exemplar{}} \annots{} are chosen to represent the appearance of that individual. The
            \exemplars{} are indexed using a search data structure.

            After the \masterdatabase{} has been prepared, the ranking algorithm is able to issue a subset of the
            \encounter{}'s \annots{} as a query. The result is a ranked list of \exemplars{} that are visually similar
            to the \encounter{}. The top \exemplars{} in the ranked list are used as candidate matches. The candidate
            matches are reviewed, and the \encounter{} is either merged into an existing \mastername{} or added to the
            \masterdatabase{} as a new \mastername{}.

        \subsubsection{Consistency checks}
            When merging \encounters{} into the \masterdatabase{} it is possible that mistakes were made.  There are two
            error cases that commonly occur.
            %%%
            \begin{enumln}
            \item  A \glossterm{split case} occurs when a set of \annots{} (from two or more different animals) is
            incorrectly labeled with the same \name{}.  The main cause of this error is when distracting features are
            matched causing the \annots{} to appear visually similar.
            %%%
            \item A \glossterm{merge case} occurs when two sets of \annots{} (from the same animal) are incorrectly
            labeled with different \names{}.  This is caused by an algorithm or human error where a query \encounter{}
            was not correctly matched to the database \exemplars{}.
            \end{enumln}
            %%%
            This is usually because the query and database \annots{} have a low degree of \emph{comparability} (\eg{}
            differences in viewpoint or low quality).  Of course, if no visual overlap exists between the two sets ---
            such as one set exclusively from the left side and another exclusively from the right --- nothing can be
            done.  This is why the animal must be seen from a predetermined view in order to be counted.  In the \GZC{}
            this is the left side.

            In the \GZC{} suspect individuals were flagged for split checks using various heuristics such as the number
            of \annots{} in the \name{} or speed of the animal estimated using GPS and time data. To check a flagged
            individual we used the ranking algorithm to search for pairs of \annots{} with low matching scores that
            belong to the flagged \name{}. Low similarity between two \annots{} within a \name{} suggested that an error
            had occurred. These low scoring results were then manually reviewed. When breaking apart split cases, care
            was taken to account for the fact that right and left images should not match. Likewise, care was taken to
            ensure that an intermediate \annot{} linking two disjoint \annots{} has enough information to establish the
            link. Merge checks issue the all \exemplars{} as queries against all other \exemplars{}. High similarity
            between two different \names{} suggested that a match that was missed. These high scoring results were
            manually reviewed. More sophisticated error detection and recovery will be discussed in \Cref{sec:incon}.

        \subsubsection{Population estimation}
            The final step for the \GZC{} workflow was to estimate the number of animals in the park.  Using the
            identification algorithm we defined which \annots{} were sightings and which were resightings. Because we
            were using a preliminary version of the system we were conservative in defining when an animal was sighted
            by only using the left and frontleft \annots{} with quality labels of ok, good, or excellent.  Each
            individual that met this criteria was counted as a sighting.  If a sighted individual had an \annot{} from
            both days, then we counted that individual as resighted.

    \subsection{Processing challenges}
        Our experience with the \GZCFull{} has highlighted a number of challenges that must be addressed if this system
        is to be applied in future events. These challenges include the number of manual reviews required, the detection
        of and recovery from manual errors, and the overall lack of a systematic identification framework.

        Perhaps the greatest challenge faced during the \GZC{} was the considerable amount time that was required to
        manually verify identification results. It can take several seconds to manually verify if a pair of \annots{} is
        a correct match even if the results are presented in a ranked list. This task is illustrated
        in~\cref{fig:RankFigure}. Requiring the manual verification of each result is untenable for a system that
        accepts thousands of new images a day. The lack of a systematic approach for identification meant that whenever
        two \annots{} were matched, the name labels of all annotations of those names were changed. This made it
        difficult to tease apart errors when they occurred. Furthermore, manual errors (likely caused by fatigue from
        the large number of manual reviews) resulted in numerous split cases that were not able to be detected and
        resolved until the end of the process. Reviews of results were also done in order of matching scores regardless
        of previous decisions, causing the manual reviewer to inefficiently review redundant results between the same
        individual. Additionally no stopping criteria for reviews was defined resulting in an ad-hoc approach to
        determining when all matches were found.

        Motivated by these observations we seek to develop a semi-automatic approach to animal identification. This
        approach will should be governed by a system that reduces the number of manual reviews and is able to detect and
        recover from errors, and determine when to stop searching for new matches.

        %Furthermore, as new \exemplars{} are added to the system the search
        %  data structure must be updated before additional queries can be made.
        %Rebuilding this data structure is another source of delays.
        %We consider addressing this problem as two separate challenges.
        %The first challenge is algorithmic, and the second challenge is system
        %  based.
        %We will use these challenges to motivate the development a system that
        %is able to dynamically detect and identify individual animals in large
        %volumes of images.
        %The algorithmic challenge is to develop a confidence based decision
        %  mechanism.

        %We will use these challenges to motivate a verification mechanism that
        %automatically accepts or dismisses candidate matches. 
        %Only a subset of the most difficult identification results should be
        %  manually reviewed, the rest should be handled automatically.
        %This motivates developing a 

        %On the system side, the challenge is to dynamically update the search
        %  data structure.
        %This involves intelligent bookkeeping because the image analysis
        %  system is designed as a stateless API{}.
        %Statelessness is essential if multiple users are to access the same
        %  instance of image analysis and makes the system compatible with web
        %  technologies.
        %A stateless API is allowed to cache results, but it cannot maintain a
        %  single canonical object such as an indexer.
        %Instead the API{} works by accepting and responding to requests.
        %This has the effect of enforcing that objects are immutable, but also
        %  eliminates bugs due to race conditions, gives the program a large
        %  degree of thread safety, and encourages extensible and testable coding
        %  practices.
        %Updating search structures dynamically is a challenging problem in a
        %  stateless framework, but it can be addressed with careful system
        %  design.
              
        \RankFigure{}

\section{Approach}
    The problem addressed in this \thesis{} is to identify individual animals
    ``in the wild'' and to count the individuals in a population. To do this we
    develop both a suite of algorithms and a software system. The algorithms
    will allow us to infer properties about images and \annots{}. The system
    will allow us to maintain the images, \annots{}, algorithms, and inferred
    properties in a controlled and reproducible manner.

    We are given a set images containing \annots{} of the same species. The
    images are collected in an uncontrolled environment and likely contain
    imaging challenges such as occlusion, distracting features, viewpoint
    variations, pose variations, and quality variations. Furthermore, the images
    may be collected either over many years or over just a few days as in the
    \GZC{}. Each \annot{} is labeled with time, GPS, quality, and viewpoint. We
    may also be given an initial \masterdatabase{} --- \ie{} some \annots{} may
    labeled as \exemplars{} --- but this need not be the case. Our task is label
    each \annot{} with a \glossterm{\name{}} that uniquely identifies the
    individual. Once this is complete it forms the data needed to estimate the
    size of the population using techniques from sight-resight statistics.

    The first step of the identification process is a ranking algorithm. The
    inputs to the algorithm are a single query \annot{} and a set of database
    \annots{}. Sparse patch-based features are localized in all \annots{}, and a
    descriptor vector is extracted for each feature. The descriptors of the
    database \annots{} are indexed for fast nearest neighbor search. We then
    find a set of matches in the database for each descriptor in the query
    \annot{}. The matches are scored based on visual similarity, distinctiveness
    within the database, and likelihood of belonging to the foreground. Matches
    are combined across multiple \exemplar{} \annots{} to produce a matching
    score for each \name{} in the database, resulting in a ranked list of
    results for each query.

    We evaluate the ranking algorithm by performing experiments on databases of
    three species: plains zebras, Grevy's zebras, and Masai giraffes. The
    experiments test the algorithm's ability to find potential matches of an
    individual animal over large periods of time, different viewpoints,
    different sized databases, and different numbers of \exemplars{}. We
    determine the configuration of the algorithm pipeline that works best for
    identifying each species.

    We then extend ranking algorithm by developing a classifier able to
      automatically review its results.
    Then, we place the problem of animal identification in a graph framework
      able to systematically guide the identification process.
    The graph framework will be able to detect and recover from errors by
      taking advantage of multiple images seen of each individual.
    
    %We build a workflow on top of the matching algorithm.
    %This workflow accepts new \annots{} in groups defined by \occurrences{}.
    %The matching algorithm groups \annots{} within the \occurrence{}, and
    %  then leverages redundant and multiple viewpoints to perform identification
    %  against the database.
    %As the database grows we handle multiple views of each \exemplar{} by
    %  maintaining a set of \exemplars{} for each \name{}.
    %We develop methods for recovering from any errors in identification when
    %  multiple individuals are grouped into the same \exemplar{} as well as when
    %  multiple \exemplars{} actually represent the same individual.

    %To address the challenges introduced by this workflow we extend the core
    %  matching algorithm using a probabilistic graph based inference algorithm.
    %We will learn the probability of matching given two \annots{} as well as a
    %  confidence in that estimate.
    %We will use this information build a weighted graph of potential matches.
    %To perform inference on this graph we propose to to develop a decision
    %  mechanism that will make probabilistic decisions about \intraoccurrence{}
    %  matching, \vsexemplar{} matching, and consistency checks.

    %To support continuous and dynamic use of the system we develop a caching
    %scheme that supports seamless invalidation of outdated data, computes
    %requested data on the fly, and disallows duplicate data. We use this scheme
    %to dynamically update the underlying data structures as more data is added
    %to the system. This is all accomplished in a stateless framework which
    %allows for the image analysis software to be used concurrently by web-based
    %frameworks.

\section{\Thesis{} organization} %
    This \thesis{} is organized as follows:
    %
    \Cref{chap:relatedwork} describes related work with a focus the details of
      techniques used in the system as well as an overview of those which are
      indirectly related.
    %
    \Cref{chap:ranking} describes a ranking algorithm for identifying
      individual animals, one \annot{} at a time, against a database of
      \exemplars{}.
    It includes an experimental evaluation of the ranking algorithm.
    This is the algorithm that was used in the \GZC{}.
    \Cref{chap:pairclf} addresses the problem of semi-automatic verification
      of results from the ranking algorithm.
    \Cref{chap:graphid} combines the ranking and verification algorithm into a
      semi-automatic framework that detects and corrects errors while reducing
      the number of manual reviews .
    \Cref{chap:system} describes the system architecture used throughout this
    \thesis{} to efficiently accomplish the on-line computer vision tasks.
      \Cref{chap:conclusion} concludes this \thesis{} and summarizes its contributions.
