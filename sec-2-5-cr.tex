\section{Category recognition}\label{sec:cr}  

    Different types of image recognition lie at different points on a
      spectrum of specificity.
    If instance recognition is at one end of the spectrum, then
      category recognition is at the other.
    %Category recognition is at the other end of the spectrum of
    %  specificity from instance recognition.
    The goal of a \glossterm{category recognition} algorithm is to
      assign a categorical class label to a query
      image~\cite{everingham_pascal_2010, everingham_pascal_2015,
      russakovsky_imagenet_2014, deng_imagenet_2009, feifei_oneshot_2006,
      griffin_caltech256_2007}.
    The categories often have visual appearances with a high degree of
      intra-class variance.
    %\Eg, a sedan and a school both belong to the car category.
    \Eg{}, a recliner and a bench both belong to the chair category.
    Image representations and similarity measures are constructed to
      account for this.
    Despite this, techniques in category recognition have many
      similarities to instance recognition techniques.
    Until the neural network
      revolution~\cite{krizhevsky_imagenet_2012}, most category
      recognition techniques have been based on vocabulary
      methods~\cite{csurka_visual_2004, yang_linear_2009,
      sanchez_compressed_2013, russakovsky_imagenet_2014,
      krizhevsky_imagenet_2012} similar to those discussed
      in~\cref{subsec:bow}.
    This section first provides a brief overview of this literature.
    Then, we discuss Naive Bayes classification
      techniques~\cite{boiman_defense_2008,mccann_local_2012} that play a
      large role in our baseline animal identification algorithms.

    \subsection{Vocabulary based methods for category recognition}
        After vocabulary based techniques demonstrated success in
          instance recognition, these techniques were quickly adapted and
          applied to category recognition~\cite{csurka_visual_2004}.
        Thus there are many similarities --- and some differences ---
          in the techniques used to address these two problems.
        One of the differences between instance recognition and
          category recognition is the size of the visual vocabulary.
        Instance recognition tends to require huge vocabularies
          ($\OnTheOrderOf{5}$ --- $\OnTheOrderOf{7}$ words) to achieve a
          fine sampling of descriptor space~\cite{nister_scalable_2006,
          philbin_object_2007}.
        In contrast, category recognition uses smaller vocabulary sizes
          %- \OnTheOrderOf{5}$ 
        ($\OnTheOrderOf{4}$ words) to more coarsely sample descriptor
          space~\cite{zhang_local_2006}.
        However, the vocabularies used in instance recognition have
          decreased in size with the advent of aggregated representations
          like VLAD and the Fisher vector~\cite{arandjelovic_all_2013,
          sanchez_compressed_2013}.

        A second difference between category and instance recognition
          is how similarity between images is computed.
        In instance recognition the similarity between bag-of-word
          vectors is computed using a weighted cosine similarity.
        However, in category-recognition intra-class variation requires
          more sophisticated similarity measurements.
        Here, image similarity is computed using SVMs with different
          either linear or non-linear kernels such as $\chi^2$, earth
          mover's distance, Hellinger, and
          Jensen-Shannon~\cite{zhang_local_2006, varma_learning_2007,
          vedaldi_efficient_2012}.

        A third difference is the way that spatial information is used.
        Instead of filtering correspondences using spatial
          verification, spatial information is incorporated into category
          recognition algorithms using spatial
          pyramids~\cite{grauman_pyramid_2005, lazebnik_beyond_2006}.
        A spatial pyramid sub-divides an image into a hierarchy of
          grids.
        Max pooling is often used to select only the strongest features
          in each spatial region~\cite{boureau_theoretical_2010,
          boureau_learning_2010}.
        Each section of the image is encoded using the vocabulary and
          images are scored based on matches in each region.

        \paragraph{Enhancements to category recognition}
        There are a wide variety of extensions and enhancements for
          bag-of-words based image classification, such as soft
          assignment of visual-words~\cite{liu_defense_2011} and
          vocabulary optimization~\cite{wang_localityconstrained_2010}.
        Numerous matching kernels --- both linear and non-linear ---
          have been developed such as kernel PCA, histogram intersection,
          and SVM square root bag-of-words
          vectors~\cite{vedaldi_multiple_2009, maji_classification_2008,
          perronnin_largescale_2010}.

        % chktex-file 8
        Generalized coding schemes improve performance over a
          bag-of-words image encoding.
        Vocabularies can be seen as codebooks or dictionaries in coding
          based image classification techniques such as sparse coding and
          locally constrained linear coding~\cite{jurie_creating_2005,
          yang_linear_2009, yang_supervised_2010, yang_efficient_2010,
          wang_localityconstrained_2010}.
        Many coding schemes learn both the centroids as well as the
          function that quantizes a raw descriptor into a
          word~\cite{jurie_creating_2005, yang_linear_2009,
          yang_supervised_2010, yang_efficient_2010,
          wang_localityconstrained_2010, vedaldi_multiple_2009}.
        Techniques other than k-means are used to create vocabularies
          such as mean shift~\cite{jurie_creating_2005}, coordinate
          descent with the locally constrained linear code
          criteria~\cite{wang_localityconstrained_2010}, and random
          forests~\cite{perronnin_fisher_2007}.
        Fisher Vectors with linear classifiers have been found to
          outperform non-linear bag-of-words based SVM classifiers by
          using an L1-based distance measure and careful L2 and power-law
          normalization of descriptors~\cite{perronnin_improving_2010,
          perronnin_largescale_2010-1}.

        %Currently, deep convolutional neural nets nets achieve the
        %  highest performance on object classification datasets
        %  \cite{russakovsky_imagenet_2014}.

        %The high speed, high accuracy, and low memory usage made
        %  bag-of-words the most widely used technique in category
        %  retrieval.

        %  \paragraph{Relation to this work}
        %Spatial pyramids would not work for Animals Identification because
        %  of the large amount of pose variation. Regular grid cells  may be too coarse to 
        %If grid cells were selected based on part detections then aggregating 

        %\devcomment{Bag-of-Words became outdated for classification in
        %  2010, when The Fisher Vector took over as the new standard.
        %Then convnets came in} \cite{perronnin_improving_2010}

        %Extensions of bag-of-words originally 
        %was in \cite{csurka_visual_2004}.

        %Went to soft-assignment \cite{liu_defense_2011}

        %Added in spatial pyramid

        %Nonlinear SVMs were state of the art \cite{perronnin_largescale_2010, vedaldi_efficient_2012}

    \subsection{\Naive{} Bayes classification}\label{sec:nbnn}  

        The \Naive{} Bayes nearest neighbor (NBNN) classifier is a
          simple non-parametric algorithm for category recognition that
          does not quantize descriptor
          vectors~\cite{boiman_defense_2008}.
        Boiman responds to the dominance of complex non-linear category
          recognition algorithms in the field~\cite{varma_learning_2007,
          marszalek_learning_2007} by showing that simple techniques can
          compete with complex methods for category recognition.
        Boiman's paper also provides insight into the magnitude of
          information loss resulting from quantization.
          
        Previous to~\cite{boiman_defense_2008}, nearest neighbor
          classifiers had shown underwhelming accuracy in category
          recognition~\cite{varma_unifying_2004, lazebnik_beyond_2006,
          marszalek_learning_2007}.
        This was shown to be a result of using image-to-image distance.
        To remedy this, NBNN aggregates the information from multiple
          images by swapping the image-to-image distance for an
          image-to-class distance.

        In NBNN, features of each class are indexed for fast nearest
          neighbor search, typically with a
          kd-tree~\cite{bentley_multidimensional_1975}.
        For each feature, $\desc_i$, in a query image, the algorithm
          searches for the feature's nearest neighbors in each class,
          $\opname{NN}_C(\desc_i)$.
        The result of the algorithm is the class, $C$, that minimizes
          the image-to-class distance.
        In other words, the class of a query image is chosen by
          searching for the class that minimizes the total distance
          between each query descriptor and the nearest database
          descriptor in that class.
        This is expressed in the following equation:
        \begin{equation}
            C = \argmin{C} \sum_{i=1}^n ||\desc_i - \opname{NN}_C(\desc_i)||^2
        \end{equation}

        This formulation where each descriptor is assigned to only a
          single nearest neighbor has been shown to be a good
          approximation to the minimum image-to-class Kullback-Leibler
          divergence~\cite{boiman_defense_2008} --- a measure of how much
          information is lost when the query image is used to model the
          entire class.

    \subsection{Local \Naive{} Bayes Nearest Neighbor}\label{sec:lnbnn}  

        Local \Naive{} Bayes nearest neighbor (LNBNN) is an improved
          version of the NBNN algorithm in both accuracy and
          speed~\cite{mccann_local_2012}.
        In the original NBNN formulation a search is executed find each
          query descriptor's nearest neighbor in the database for each
          class separately.
        In contrast, the LNBNN modification searches all database
          descriptors simultaneously and ignores classes that do not
          return descriptor matches.
        
        Each descriptor $\desc_i$ in the query image searches for its
          $K+1$ nearest neighbors, $\{\desc_1, \ldots, \desc_K, \desc_{K
          + 1}\}$ over all classes.
        The first $K$ neighbors are used as matches.
        The last nearest neighbor is used as a normalizing term to
          weight the query descriptor's distinctiveness.
        Let $(\desc_i, \desc_j)$ be a matching descriptor pair, and let
          $C$ be the class of $\desc_j$.
        The score of each match is computed as the distance to the
          match subtracted from the distance to the normalizer.
        The score of a class $C$ is the sum of all the descriptor
          scores that match to it.
        \begin{equation}
            s_{i, C} = \elltwo{\desc_j - \desc_K} - \elltwo{\desc_i - \desc_j}
        \end{equation}

    \subsection{Discussion --- class recognition}
        %Should image features be quantized?
        %How should feature matches be scored?

        Progress in category recognition is generally made using
          techniques that allow classes with high intra-class variance to
          have lower matching scores.
        This is of little value to an instance recognition application,
          therefore we do not investigate most of the techniques in this
          section.
        However, the LNBNN~\cite{mccann_local_2012} approach is
          interesting to us because it is a simple algorithm that does
          not suffer from quantization artifacts.

        NBNN and LNBNN~\cite{boiman_defense_2008,mccann_local_2012}
          never achieved state of the art performance in image
          classification, however they have produced competitive results
          using simple techniques.
        The simplicity of the techniques allowed for the authors to
          gain insight into visual recognition.
        This simplicity is the main reason we adopt LNBNN as our
          baseline algorithm for animal identification.

