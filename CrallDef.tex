%%%%%%%%%%%%%%%
% Cross platform stuff

%https://en.wikipedia.org/wiki/Wikipedia:LaTeX_symbols
\ifwindows{}
\newcommand{\wincommand}[2]{\renewcommand{#1}{#2}}
\newcommand{\lincommand}[2]{\newcommand{#1}{#2}}
\newcommand{\linonlycommand}[2]{}
\else
\newcommand{\wincommand}[2]{\newcommand{#1}{#2}}
\newcommand{\lincommand}[2]{\renewcommand{#1}{#2}}
\newcommand{\linonlycommand}[2]{\newcommand{#1}{#2}}
\fi
%%%%%%%%%%%%%%%
% MACROS
%
\newcommand{\bb}{\mathbb}

%\newcommand{\zspace}{\xspace}
\newcommand{\zspace}{}

\newcommand{\Naive}{Na\"{\i}ve\zspace}
\newcommand{\Naively}{Na\"{\i}vely\zspace}
\newcommand{\naive}{na\"{\i}ve\zspace}
\newcommand{\naively}{na\"{\i}vely\zspace}
\newcommand{\iid}{i.i.d.\zspace}
% WACV 
% Add a period to the end of an abbreviation unless there's one already
\makeatletter
\DeclareRobustCommand\onedot{\futurelet\@let@token\@onedot}
\def\@onedot{\ifx\@let@token.\else.\null\fi\xspace}

\def\eg{\emph{e.g}\onedot} \def\Eg{\emph{E.g}\onedot}
\def\ie{\emph{i.e}\onedot} \def\Ie{\emph{I.e}\onedot}
\def\cf{\emph{c.f}\onedot} \def\Cf{\emph{C.f}\onedot}
\def\etc{\emph{etc}\onedot} \def\vs{\emph{vs}\onedot}
%\def\wrt{w.r.t\onedot}
\def\dof{d.o.f\onedot}
\def\etal{\emph{et al}\onedot}
\makeatother

\newcommand{\wrt}{with respect to\xspace{}}


%\newcommand{\one}{1\zspace{}}
%\newcommand{\two}{2\zspace{}}
%\newcommand{\three}{3\zspace{}}
%\newcommand{\four}{4\zspace{}}
%\newcommand{\five}{5\zspace{}}

\newcommand{\one}{one\zspace{}}
\newcommand{\two}{two\zspace{}}
\newcommand{\three}{three\zspace{}}
\newcommand{\four}{four\zspace{}}
\newcommand{\five}{five\zspace{}}


% Definition of context sensitive articles
% http://stackoverflow.com/questions/4233707/a-an-substitution-in-latex
% http://tex.stackexchange.com/questions/43200/extract-first-last-characters-of-macro-argument
%http://tex.stackexchange.com/questions/132248/test-if-the-first-character-of-a-string-is-a
%\makeatletter
%\newcommand\aan[1]{%
%  a%
%  \@for\@vowel:=a,e,i,o,u,y,A,E,I,O,U,Y\do{%
%    \expandafter\ifx\@vowel#1%
%      n%
%    \fi
%  } % keep this space
%  #1%
%}
%\newcommand\Aan[1]{%
%  A%
%  \@for\@vowel:=a,e,i,o,u,y,A,E,I,O,U,Y\do{%
%    \expandafter\ifx\@vowel#1%
%      n%
%    \fi
%  } % keep this space
%  #1%
%}
%\makeatother


\makeatletter
\newcommand\aan[1]{%
\StrLeft{#1}{1}[\firstchar]%
a\IfStrEq{\firstchar}{a}{n}{%
\IfStrEq{\firstchar}{e}{n}{%
\IfStrEq{\firstchar}{i}{n}{%
\IfStrEq{\firstchar}{o}{n}{%
\IfStrEq{\firstchar}{u}{n}{%
\IfStrEq{\firstchar}{y}{n}{%
\IfStrEq{\firstchar}{A}{n}{%
\IfStrEq{\firstchar}{E}{n}{%
\IfStrEq{\firstchar}{I}{n}{%
\IfStrEq{\firstchar}{O}{n}{%
\IfStrEq{\firstchar}{U}{n}{%
\IfStrEq{\firstchar}{Y}{n}{}}}}}}}}}}}} #1}
\makeatother

\makeatletter
\newcommand\Aan[1]{%
\StrLeft{#1}{1}[\firstchar]%
A\IfStrEq{\firstchar}{a}{n}{%
\IfStrEq{\firstchar}{e}{n}{%
\IfStrEq{\firstchar}{i}{n}{%
\IfStrEq{\firstchar}{o}{n}{%
\IfStrEq{\firstchar}{u}{n}{%
\IfStrEq{\firstchar}{y}{n}{%
\IfStrEq{\firstchar}{A}{n}{%
\IfStrEq{\firstchar}{E}{n}{%
\IfStrEq{\firstchar}{I}{n}{%
\IfStrEq{\firstchar}{O}{n}{%
\IfStrEq{\firstchar}{U}{n}{%
\IfStrEq{\firstchar}{Y}{n}{}}}}}}}}}}}} #1}
\makeatother

\newcommand{\DIM}{D\zspace}
\newcommand{\see}[1]{see \cref{#1}}
\newcommand{\See}[1]{See \cref{#1}}
\newcommand{\Refin}[1]{Referenced in \cref{#1}}
\newcommand{\refin}[1]{referenced in \cref{#1}}
\newcommand{\Imgcite}[1]{Image from~\cite{#1}}

\newcommand{\topic}[1]{\subsubsection{#1}}

%\newcommand{\todo}[1]{\textbf{TODO: #1} \cite{crall2013hotspotterWACV}} 

\renewcommand{\cal}[1]{\mathcal{#1}}
\newcommand{\superscript}[1]{\ensuremath{^{\textrm{#1}}}}
\newcommand{\subscript}[1]{\ensuremath{_{\textrm{#1}}}}
\newcommand{\ds}{\displaystyle}
\newcommand{\dsi}{\displaystyle{} \hspace*{5mm}}
\newcommand{\ind}{\ensuremath{\hspace*{5mm}}}
\newcommand{\txt}[1]{\textrm{#1}}
\newcommand{\txtbf}[1]{\textbf{#1}}
\newcommand{\ttxt}[1]{{\tt #1}}
\renewcommand{\th}{\superscript{th}\zspace}
\newcommand{\rd}{\superscript{rd}\zspace}
\newcommand{\nd}{\superscript{nd}\zspace}
% st is such a weird command. Doesnt seem to want to work unless it is in 
% the document on windows. it is pdfcomment's fault
%\linonlycommand{\st}{\superscript{st}\zspace}
\newcommand{\st}{\superscript{st}\zspace}
\newcommand{\bincase}[1]{
    \ensuremath{
    \begin{cases}
    1 & \txt{if } #1 \\
    0 & \txt{otherwise}
    \end{cases}
    }}
\newcommand{\bincases}[2]{
    \ensuremath{
    \begin{cases}
    #1 & \txt{if } #2 \\
    0 & \txt{otherwise}
    \end{cases}
    }}

\newcommand{\fullbincases}[3]{
    \ensuremath{
    \begin{cases}
    #1 & \txt{if } #2 \\
    #3 & \txt{otherwise}
    \end{cases}
    }}

\newcommand{\tightpad}{\mkern-6mu}
\newcommand{\tight}[1]{\mkern-6mu#1\mkern-6mu }
\newcommand{\ttimes}{\tight{\times}}
\newcommand{\tighteq}{\tight{=}}
\newcommand{\teq}{\tight{=}}

%\newcommand{\slice}[2]{{[#1:#2]}}
\newcommand{\slice}[2]{{[#1\mkern-3mu:\mkern-3mu#2]}}

%%%%%%%%%%%%%%%
% Constants
\newcommand{\PTIME}{{\tt PTIME}}
\newcommand{\TAU}{2\pi}

%%%%%%%%%%%
% MATH MACROS
\newcommand{\paren}[1]{\left(#1\right)}
\newcommand{\brak}[1]{\left[#1\right]}
\newcommand{\curly}[1]{\left\{#1\right\}}
\newcommand{\func}[2]{\mathbin{#1}\paren{#2}}

\newcommand{\crly}[1]{\curly{#1}}
\newcommand{\brk}[1]{\brak{#1}}

%%%%%%%%%%%%%%%
% Operators
% 
% Logic
\newcommand{\assign}{:=}
\newcommand{\eq}{=}
%\newcommand{\where}{\;\big\vert\;}
\newcommand{\whereI}{\;\big\vert\;}
\newcommand{\whereII}{\mid}
\newcommand{\where}{\whereII}
%\newcommand{\where}{\whereI}
%\newcommand{\given}{\;\big\vert\;}
\newcommand{\given}{\mid}
\newcommand*\AND{\txtbf{ and } }
\newcommand*\OR{\txtbf{ or } }
\newcommand*\xor{\mathbin{\oplus}}
\newcommand{\union}{\cup}
\newcommand{\symdiff}{\Delta}
\newcommand{\setdiff}{-}
\newcommand{\Union}{\bigcup}
\newcommand{\intersect}{\cap}
\newcommand{\isect}{\cap}
\renewcommand{\mod}{\txt{mod}}
\newcommand{\Normal}{\cal{N}}
\newcommand{\Real}{\mathbb{R}}
\newcommand{\Int}{\mathbb{Z}}
\newcommand{\eye}{\mat{I}}

\newcommand{\opname}[1]{\operatorname{#1}}

% Probability Operator
%\newcommand{\probop}{\mathbin{\opname{\mathbb{P}}}}
%\newcommand{\probop}{\mathbin{\opname{Pr}}}
\newcommand{\probop}{\mathbin{\opname{P}}}
\newcommand{\oddsop}{\mathbin{\opname{O}}}
%\newcommand{\oddsop}{\mathbin{\opname{odds}}}
%\newcommand{\oddsop}{\mathbin{\mathcal{O}}}
\newcommand{\expectop}{\mathbb{E}}
\newcommand{\parzenop}{\mathbin{\hat{\probop}}}
\newcommand{\Parz}[1]{\parzenop\paren{#1}}
\newcommand{\logop}{\mathbin{\opname{log}}}
\newcommand{\lnop}{\mathbin{\opname{ln}}}
\newcommand{\expop}{\mathbin{\opname{exp}}}
\newcommand{\distri}{\sim}


\RenewDocumentCommand\ln{g}{\IfValueTF{#1}{\logop\paren{#1}}{\logop}}
\NewDocumentCommand\asin{g}{\IfValueTF{#1}{\opname{asin}\paren{#1}}{\opname{asin}}}


% Convolution
\newcommand{\conv}{\mathop{\scalebox{1.5}{\raisebox{-0.2ex}{$\ast$}}}}
\newcommand{\laplace}{\nabla^2} 
\newcommand{\del}{\Delta}
% Probability
%\renewcommand{\Pr}[1]{\ensuremath{\txt{Pr} \left(#1\right)}}
%\renewcommand{\Pr}{\mathbb{P}}
\renewcommand{\Pr}[1]{\func{\probop}{#1}}
\newcommand{\Odds}[1]{\func{\oddsop}{#1}}  % this is logit
\newcommand{\Prs}[2]{\func{\probop_{#1}}{#2}}
\newcommand{\Ex}[1]{\ensuremath{\expectop \left[#1\right]}}
\newcommand{\Err}[1]{\func{\opname{Err}}{#1}}
\newcommand{\ExUnder}[2]{\ensuremath{\underset{#1}{\expectop \left[#2\right]}}}
\newcommand{\ExSub}[2]{\ensuremath{\expectop_{#1} \left[#2\right]}}

\newcommand{\logit}[1]{\func{{\mathbin{\opname{logit}}}}{#1}}
\newcommand{\logitI}[1]{\ensuremath{-\ln{\frac{1}{#1} - 1}}}
\newcommand{\logitII}[1]{\ensuremath{\ln{\frac{#1}{1 - #1}}}}
\newcommand{\expit}[1]{\func{{\mathbin{\opname{expit}}}}{#1}}
\newcommand{\expitI}[1]{\ensuremath{\ln{\frac{1}{1 + \exp{#1}}}}}

% Optimization
\newcommand{\argmax}[1]{\underset{#1}{\opname{argmax}}\;}
\newcommand{\argmin}[1]{\underset{#1}{\opname{argmin}}\;}
\newcommand{\argsort}[1]{\underset{#1}{\opname{argsort}}\;}
\newcommand{\localmax}[1]{\underset{#1}{\opname{localmax}}}
\newcommand{\arglocalmax}[1]{\underset{#1}{\opname{localmax}}}
\newcommand{\arglocalextrema}[1]{\underset{#1}{\opname{argextrema}}}
\newcommand{\argextrema}[1]{\underset{#1}{\opname{argextrema}}}
% Calculation
\newcommand{\expI}[1]{\txt{exp}\paren{#1}}
\newcommand{\expII}[1]{\ensuremath{e^{#1}}}
\renewcommand{\exp}[1]{\expI{#1}}
%\renewcommand{\ln}[1]{\txt{ln}\paren{#1}}

\newcommand{\overbar}[1]{\mkern1.5mu\overline{\mkern-1.5mu#1\mkern-1.5mu}\mkern1.5mu}
%\newcommand{\complement}[1]{\overbar{#1}}
\newcommand{\setcomp}[1]{\overbar{#1}}
\newcommand{\indicator}{\mathbbm{1}}


% Linear Algebra
%\newcommand{\sqrtm}[1]{\opname{sqrtm}(#1)}
\newcommand{\sqrtm}[1]{#1^{\frac{1}{2}}}
\newcommand{\inv}[1]{#1^{-1}}

\newcommand{\tr}{\opname{Tr}}
\renewcommand{\det}{\opname{Det}}
\newcommand{\modop}{\opname{mod}}
\newcommand{\modfn}[2]{#1\; \modop{} \;\,#2}

\newcommand{\trop}{\txt{Tr}}
\newcommand{\detop}{\txt{Det}}

\newcommand{\trfn}[1]{\func{\trop}{#1}}
\newcommand{\detfn}[1]{\func{\detop}{#1}}

\newcommand{\arctantwo}{\opname{arctan2}}
\newcommand{\arctanII}{\opname{arctan2}}
\newcommand{\atantwo}[1]{\func{\arctantwo}{#1}}
\newcommand{\atan}[1]{\func{\arctan}{#1}}

\newcommand{\cov}{\Sigma}
\newcommand{\ltwonormvec}[1]{\frac{#1}{\elltwo{#1}}} 
\newcommand{\card}[1]{|#1|} 
\newcommand{\braket}[2]{\left<#1|#2\right>} 
\newcommand{\bra}[1]{\left<#1|} 
\newcommand{\ket}[2]{|#1\right>} 
%overbar\newcommand{\card}[1]{\txt{card}(#1)} 
%\newcommand{\sqrtm}[1]{#1^(.5)}


%%%%%%%%%%%%%%%
% Variables
% 

% Better ensure math with a space  Actually no. just use \cmd\
%\newcommand{\enmath}[1]{\ensuremath{#1}\zspace}

\newcommand{\mat}[1]{\ensuremath{\mathbf{#1}}} %Should always be capital
\newcommand{\rand}[1]{\ensuremath{\mathbf{#1}}}  %Should always be capital
\newcommand{\set}[1]{\ensuremath{\mathcal{#1}}} 
%\newcommand{\set}[1]{\ensuremath{\mathpzc{#1}}} 
\newcommand{\multiset}[1]{\ensuremath{\mathscr{#1}}} 
%\newcommand{\multiset}[1]{\ensuremath{\mathfrak{#1}}} 
%\newcommand{\multiset}[1]{\ensuremath{\mathrsfs{#1}}} 

%\let\arrowvec\vec % Keep old arrowvec functionality
\renewcommand{\vec}[1]{\ensuremath{\mathbf{#1}}} %Should always be lowercase
%\newcommand{\greekvec}[1]{\ensuremath{\boldsymbol{#1}}} %Should always be lowercase
%\newcommand{\greekvec}[1]{#1}
\newcommand{\greekvec}[1]{\mbox{\boldmath\ensuremath{#1}}}
%
%boldsymbol
% An explicit vector 
%  e.g. \VEC{ x \\ y \\ z}
\newcommand{\VEC}[1]{\ensuremath{
    \Bigl[\negthinspace \begin{smallmatrix} #1
    \end{smallmatrix} \negthinspace \Bigr] }}
\newcommand{\BVEC}[1]{
    \begin{bmatrix}
        #1
    \end{bmatrix}}
\newcommand{\BIGVEC}[1]{\BVEC{#1}}
% An explicit matrix 
% e.g. \MAT{ a & b \\ c & d}
\newcommand{\MAT}[1]{\ensuremath{
    \Bigl[ \begin{smallmatrix} #1
    \end{smallmatrix} \Bigr] }}

\newcommand{\BMAT}[1]{\ensuremath{
    \left[ \begin{matrix} #1
    \end{matrix} \right] }}
\newcommand{\BIGMAT}[1]{\BMAT{#1}}


\newcommand{\REAL}{\mathbb{R}}
%\renewcommand{\Pr}{\color{red} \txtbf{{Pr}} \color{black}}

\newcommand{\ellp}[2]{||#1||_{#2}}
%\newcommand{\elltwo}[1]{\ellp{#1}{2}}
\newcommand{\elltwo}[1]{\ellp{#1}{}}
%\newcommand{\elltwosqrd}[1]{\ellp{#1}{2}^2}
\newcommand{\elltwosqrd}[1]{||#1||^2}
\newcommand{\ellone}[1]{\ellp{#1}{1}}
\newcommand{\ltwo}[1]{||#1||_2}
\newcommand{\lone}[1]{||#1||_1}
\newcommand{\len}[1]{|#1|}
\newcommand{\abs}[1]{|#1|}

%\newcommand{\tau}[1]{|#1|}


%%%%%%%%%%%%%%%%%%
% Common notation
\newcommand{\boldgreek}[1]{\mbox{\boldmath{$ #1 $}}}

\newcommand{\eps}{\epsilon}
\newcommand{\prefers}{\succ}
\newcommand{\preferseq}{\succeq}

%\newcommand{\binom}[2]{{#1 \choose #2}}


\newcommand{\floor}[1]{\left\lfloor{} #1 \right\rfloor}
\newcommand{\bigoh}[1]{\func{\mathcal{O}}{#1}}

%\newcommand{\rotMATRIXtwo}[1]{\MAT{-\sin{#1} & \cos{#1}\\\;\;\;\cos{#1} & \sin{#1}}}
\newcommand{\mspc}{\;\;\;}
\newcommand{\rotMatII}[1]{\MAT{-\sin{#1} & \cos{#1}\\\mspc\cos{#1} & \sin{#1}}}
\newcommand{\rotMatIII}[1]{
    \MAT{
             \cos{#1} &     -\sin{#1} & 0\\
             \sin{#1} & \mspc\cos{#1} & 0\\ 
                    0 &             0 & 1
        }
}
\newcommand{\rotBigMatIII}[1]{
    \BIGMAT{
             \cos{#1} &     -\sin{#1} & 0\\
             \sin{#1} &      \cos{#1} & 0\\ 
                    0 &             0 & 1
        }
}
\newcommand{\invrotMatIII}[1]{
    \MAT{
        \mspc\cos{#1} &      \sin{#1} & 0\\
            -\sin{#1} &      \cos{#1} & 0\\ 
                    0 &             0 & 1
        }
}
%\newcommand{\invrotMatIII}[1]{\MAT{-\sin{#1} & -\cos{#1}  & 0\\-\cos{#1} & \mspc\sin{#1} & 0\\  0 & 0 &  1}}

\newcommand{\transMatIII}[2]{
    \MAT{
        1 & 0 & #1\\
        0 & 1 & #2\\
        0 & 0 & 1
        }
}

\newcommand{\transBigMatIII}[2]{
    \BIGMAT{
        1 & 0 & #1\\
        0 & 1 & #2\\
        0 & 0 & 1
        }
}


\newcommand{\sciE}[1]{\ensuremath{\times 10^{#1}}}
\newcommand{\E}[1]{\sciE{#1}}

\newcommand{\OnTheOrderOf}[1]{\ensuremath{\sim\mkern-6mu 10^{#1}}}

\newcommand{\suchthat}{\quad\txt{s.t. }}
\newcommand{\subjectto}{\suchthat}
%\newcommand{\suchthat}{\txt{ such that }}
%\newcommand{\subjectto}{\txt{ subject to }}

% Elementwise product / Hadamard Product
\newcommand{\elemprod}{\circ}

\newcommand{\concat}[2]{\brak{\paren{#1^T}, \paren{#2^T}}^T}

\newcommand{\percent}{\%}

\newcommand{\degrees}{^{\circ}}
%\renewcommand{\degree}{\degrees}
%\newcommand{\degree}{^{\circ}}


%\newcommand{\baseIdx}{0}
%\newcommand{\lastIdx}[1]{(#1 - 1)}
\newcommand{\baseIdx}{1}
\newcommand{\nextIdx}{2}
\newcommand{\lastIdx}[1]{#1}

\newcommand{\xdotseqIII}[4]{#1_{#2}, \ldots{}, #1_{#3}, \ldots, #1_{#4}}
\newcommand{\xdotseqII}[3]{#1_{#2}, \ldots, #1_{#3}}


\newcommand{\dotsubseq}[3]{#1_{#2}, \ldots{}, #1_{\lastIdx{#3}}}
\newcommand{\dotseqxII}[2]{\xdotseqII{#1}{\baseIdx{}}{\lastIdx{#2}}}
\newcommand{\dotseqIII}[3]{#1_\baseIdx{}, \ldots{}, #1_{#2}, \ldots, #1_{\lastIdx{#3}}}
\newcommand{\dotseqIV}[3]{#1_\baseIdx{}, #1_\nextIdx{} \ldots, #1_{#2}, #1_{#2 + 1}, \ldots, #1_{\lastIdx{#3}}}


\newcommand{\xdotseqIIITwoD}[5]{\xdotseqIII{#1}{\baseIdx,\baseIdx}{#2,#3}{\lastIdx{#4},\lastIdx{#5}}}


\newcommand{\dotseqII}[2]{#1_\baseIdx{}, \ldots{}, #1_{\lastIdx{#2}}}

\newcommand{\dotsubarr}[3]{\brak{\dotsubseq{#1}{#2}{#3}}}
\newcommand{\dotarrII}[2]{\brak{\dotseqxII{#1}{#2}}}
\newcommand{\dotarrIII}[3]{\brak{\dotseqIII{#1}{#2}{#3}}}
\newcommand{\dotarrIV}[3]{\brak{\dotseqIV{#1}{#2}{#3}}}

\newcommand{\dotsubset}[3]{\curly{\dotsubseq{#1}{#2}{#3}}}
\newcommand{\dotsetII}[2]{\curly{\dotseqxII{#1}{#2}}}
\newcommand{\dotsetIII}[3]{\curly{\dotseqIII{#1}{#2}{#3}}}
\newcommand{\dotsetIV}[3]{\curly{\dotseqIV{#1}{#2}{#3}}}

\renewcommand{\emptyset}{\varnothing}


% https://en.wikipedia.org/wiki/Normal_distribution#Definition
%\newcommand{\NormalEqnI}{
%    \frac{1}{\sqrt{\sigma^2 \TAU}}
%    \exp{-\frac{(x - \mu)^2}{2\sigma^2}}
%}

\newcommand{\NormalVarEqnI}[2]{
    \ensuremath{
    \frac{1}{\sqrt{#2^2 \TAU}}
    \exp{\frac{-{(#1)}^2}{2#2^2}}
}
}

\newcommand{\NormalVarConst}[1]{
    \ensuremath{
    \frac{1}{\sqrt{#1^2 \TAU}}
}}

\newcommand{\NormalVarExp}[2]{
    \ensuremath{
    \exp{\frac{-{#1}^2}{2#2^2}}
}}

\newcommand{\NormalVarEqnII}[2]{
    \ensuremath{ \NormalVarConst{#2} \NormalVarExp{#1}{#2} }
}
\newcommand{\NormEqnII}[2]{\NormalVarEqnII{#1}{#2}}

\newcommand{\NormalEqnI}{\NormalVarEqnI{\ensuremath{x - \mu}}{\sigma}}

\newcommand{\StandardNormalI}{
    %\NormalVarEqnI{0}{1}
    %\frac{\expII{-\frac{x^2}{2}}}{\sqrt{\TAU}}
    \ensuremath{\frac{\expI{-(x^2) / 2}}{\sqrt{\TAU}}}
}

\newcommand{\StandardNormalKernelI}[1]{
    %\NormalVarEqnI{0}{1}
    %\frac{\expII{-\frac{x^2}{2}}}{\sqrt{\TAU}}
    \ensuremath{\frac{\expI{-(#1^2) / 2}}{\sqrt{\TAU}}}
}

\newcommand{\ParzenEqn}{
    \ensuremath{\frac{1}{nh} \sum_{i=1}^n \StandardNormalKernelI{(\frac{x - x_i}{h})}}
}

% https://en.wikipedia.org/wiki/Multivariate_normal_distribution
\newcommand{\NormalEqnII}{
    \ensuremath{
        \frac{1}{\sqrt{\detfn{\Sigma} {(\TAU)}^k}}
        \exp{-\frac{{(\vec{x} - \greekvec{\mu})}^T \Sigma^{-1} (\vec{x} - \greekvec{\mu})}{2}}
    }}


%\newcommand{\fboxII}[1]{\fbox{#1}}
\newcommand{\fboxII}[1]{#1}


% ------------


\newcommand{\ImageFigureDraft}[4]{ 
    \begin{figure}[ht!]
    \centering 
    \fboxII{\includegraphics[draft,width=#2]{#1}}
    \caption{\caplbl{#4}#3}\label{fig:#4} 
    \end{figure} 
} 


\makeatletter
\newcommand{\ImageCommandDraft}[4]{%
    \expandafter\newcommand\csname #4\endcsname{\ImageFigureDraft{#1}{#2}{#3}{#4}}%
}
\makeatother



\newcommand{\ImageFigure}[4]{ 
    \begin{figure}[ht!]
    \centering 
    \fboxII{\includegraphics[width=#2]{#1}}
    \caption{\caplbl{#4}#3}\label{fig:#4} 
    \end{figure} 
} 

\makeatletter
\newcommand{\ImageCommand}[4]{%
    \expandafter\newcommand\csname #4\endcsname{\ImageFigure{#1}{#2}{#3}{#4}}%
}
\makeatother


\newcommand{\CaptionedImageFigure}[5]{ 
    \begin{figure}[ht!]
    \centering 
    \fboxII{\includegraphics[width=#2]{#5}}
    \caption[\caplbl{#1}#3]{\caplbl{#1}#4}\label{fig:#1} 
    \end{figure} 
} 

\makeatletter
% {label}{textwidth_percent}{shortcap}{caption_str}{fpath}
\NewDocumentCommand\SingleImageCommand{mmmmm}{%
    \expandafter\newcommand\csname #1\endcsname{\CaptionedImageFigure{#1}{#2\textwidth}{#3}{#4}{#5}}
}
\makeatother




% ----------
% Super hacky def for multiple fig stuffs

%http://tex.stackexchange.com/questions/132956/reduce-space-between-subfigure-and-the-subfigure-captions

% TODO: change to use keyval arguments for multifigure
% http://tex.stackexchange.com/questions/180147/newcommand-for-tabular-entries-using-keyval

% command to specify up to 4 subfigures
% FIXME: if one of the IfValueTF goes to its false case it leaves an ungly space between the figure and the caption
% Not sure how to fix this robustly. Adding \vspace{-1\baselineskip} in the last else seems to help some.
% chktex-file 39
\NewDocumentCommand\MultiImageFigure{mmmgggg}{ 
    \begin{figure}[ht!]
        \centering 
        \IfValueTF{#4}{
            \begin{subfigure}[h]{#2\textwidth}
                \centering
                \fboxII{\includegraphics[width=\textwidth]{#4}}\caption{}\label{sub:#1A}\end{subfigure}
           ~~%
        }{}
        \IfValueTF{#5}{
            \begin{subfigure}[h]{#2\textwidth}
                \centering
                \fboxII{\includegraphics[width=\textwidth]{#5}}\caption{}\label{sub:#1B}\end{subfigure}
            ~~%
        }{}
        \IfValueTF{#6}{
            \begin{subfigure}[h]{#2\textwidth}
                \centering
                \fboxII{\includegraphics[width=\textwidth]{#6}}\caption{}\label{sub:#1C}\end{subfigure}
            ~~%
        }{}
        \IfValueTF{#7}{
            \begin{subfigure}[h]{#2\textwidth}
                \centering
                \fboxII{\includegraphics[width=\textwidth]{#7}}\caption{}\label{sub:#1D}\end{subfigure}
            ~~%
        }{
            \vspace{-.5\baselineskip}
        }
        %\IfValueTF{#8}{
        %    \begin{subfigure}[h]{#2\textwidth}
        %        \centering
        %        \fboxII{\includegraphics[width=\textwidth]{#8}}\caption{}\label{sub:#1E}\end{subfigure}
        %    ~~%
        %}{}
        %\IfValueTF{#9}{
        %    \begin{subfigure}[h]{#2\textwidth}
        %        \centering
        %        \fboxII{\includegraphics[width=\textwidth]{#9}}\caption{}\label{sub:#1F}\end{subfigure}
        %    ~~%
        %}{}
        %\IfValueTF{#10}{
        %    \begin{subfigure}[h]{#2\textwidth}
        %        \centering
        %        \fboxII{\includegraphics[width=\textwidth]{#10}}\caption{}\label{sub:#1G}\end{subfigure}
        %    ~~%
        %}{}
        %\IfValueTF{#11}{
        %    \begin{subfigure}[h]{#2\textwidth}
        %        \centering
        %        \fboxII{\includegraphics[width=\textwidth]{#11}}\caption{}\label{sub:#1H}\end{subfigure}
        %    ~~%
        %}{}
        %\IfValueTF{#12}{
        %    \begin{subfigure}[h]{#2\textwidth}
        %        \centering
        %        \fboxII{\includegraphics[width=\textwidth]{#12}}\caption{}\label{sub:#1I}\end{subfigure}
        %    ~~%
        %}{}
%\caption[#3]{\caplbl{#1}#3} 
\caption{\caplbl{#1}#3}  % TODO: use first sentence as the short title.
\label{fig:#1} 
    \end{figure} 
} 

% Set up the keys.  Only the ones directly under /myparbox
% can be accepted as options to the \myparbox macro.
%http://tex.stackexchange.com/questions/34312/how-to-create-a-command-with-key-values
%\pgfkeys{
% /MultiImageFigure/.is family, /MultiImageFigure,
% % Here are the options that a user can pass
% default/.style = 
%  {width = \textwidth, height = \baselineskip},
% width/.estore in = \myparboxWidth,
% height/.estore in = \myparboxHeight,
%}


\NewDocumentCommand\MultiImageFigureII{mmmmgggg}{ 
    \begin{figure}[ht!]
        \centering 
        \IfValueTF{#5}{
            \begin{subfigure}[h]{#2\textwidth}
                \centering
                \fboxII{\includegraphics[width=\textwidth]{#5}}\caption{}\label{sub:#1A}\end{subfigure}
        }{}
        \IfValueTF{#6}{
            \begin{subfigure}[h]{#2\textwidth}
                \centering
                \fboxII{\includegraphics[width=\textwidth]{#6}}\caption{}\label{sub:#1B}\end{subfigure}
        }{}
        \IfValueTF{#7}{
            \begin{subfigure}[h]{#2\textwidth}
                \centering
                \fboxII{\includegraphics[width=\textwidth]{#7}}\caption{}\label{sub:#1C}\end{subfigure}
        }{
            \vspace{-.5\baselineskip}
        }
        \IfValueTF{#8}{
            \begin{subfigure}[h]{#2\textwidth}
                \centering
                \fboxII{\includegraphics[width=\textwidth]{#8}}\caption{}\label{sub:#1D}\end{subfigure}
        }{}
\caption[\caplbl{#1}#3]{\caplbl{#1}#4}  % TODO: use first sentence as the short title.
\label{fig:#1} 
    \end{figure} 
} 


\NewDocumentCommand\MultiImageFigureDraft{mmmgggg}{ 
    \begin{figure}[ht!]
        \centering 
        \IfValueTF{#4}{
            \begin{subfigure}[h]{#2\textwidth}
                \centering
                \fboxII{\includegraphics[draft,width=\textwidth]{#4}}\caption{}\label{sub:#1A}\end{subfigure}
           ~~%
        }{}
        \IfValueTF{#5}{
            \begin{subfigure}[h]{#2\textwidth}
                \centering
                \fboxII{\includegraphics[draft,width=\textwidth]{#5}}\caption{}\label{sub:#1B}\end{subfigure}
            ~~%
        }{}
        \IfValueTF{#6}{
            \begin{subfigure}[h]{#2\textwidth}
                \centering
                \fboxII{\includegraphics[draft,width=\textwidth]{#6}}\caption{}\label{sub:#1C}\end{subfigure}
            ~~%
        }{}
        \IfValueTF{#7}{
            \begin{subfigure}[h]{#2\textwidth}
                \centering
                \fboxII{\includegraphics[draft,width=\textwidth]{#7}}\caption{}\label{sub:#1D}\end{subfigure}
            ~~%
        }{
            \vspace{-.5\baselineskip}
        }
        %\IfValueTF{#8}{
        %    \begin{subfigure}[h]{#2\textwidth}
        %        \centering
        %        \fboxII{\includegraphics[width=\textwidth]{#8}}\caption{}\label{sub:#1E}\end{subfigure}
        %    ~~%
        %}{}
        %\IfValueTF{#9}{
        %    \begin{subfigure}[h]{#2\textwidth}
        %        \centering
        %        \fboxII{\includegraphics[width=\textwidth]{#9}}\caption{}\label{sub:#1F}\end{subfigure}
        %    ~~%
        %}{}
        %\IfValueTF{#10}{
        %    \begin{subfigure}[h]{#2\textwidth}
        %        \centering
        %        \fboxII{\includegraphics[width=\textwidth]{#10}}\caption{}\label{sub:#1G}\end{subfigure}
        %    ~~%
        %}{}
        %\IfValueTF{#11}{
        %    \begin{subfigure}[h]{#2\textwidth}
        %        \centering
        %        \fboxII{\includegraphics[width=\textwidth]{#11}}\caption{}\label{sub:#1H}\end{subfigure}
        %    ~~%
        %}{}
        %\IfValueTF{#12}{
        %    \begin{subfigure}[h]{#2\textwidth}
        %        \centering
        %        \fboxII{\includegraphics[width=\textwidth]{#12}}\caption{}\label{sub:#1I}\end{subfigure}
        %    ~~%
        %}{}
%\caption[#3]{\caplbl{#1}#3} 
\caption{\caplbl{#1}#3}  % TODO: use first sentence as the short title.
\label{fig:#1} 
    \end{figure} 
} 

\makeatletter
\NewDocumentCommand\MultiImageCommand{mmmgggg}{%
    \expandafter\newcommand\csname #1\endcsname{\MultiImageFigure{#1}{#2}{#3}{#4}{#5}{#6}{#7}}
    %\expandafter\newcommand\csname #1\endcsname{\MultiImageFigure{#1}{.1}{#3}{#4}{#5}{#6}{#7}}
}
\makeatother


\makeatletter
\NewDocumentCommand\MultiImageCommandII{mmmmgggg}{%
    \expandafter\newcommand\csname #1\endcsname{\MultiImageFigureII{#1}{#2}{#3}{#4}{#5}{#6}{#7}{#8}}
    %\expandafter\newcommand\csname #1\endcsname{\MultiImageFigure{#1}{.1}{#3}{#4}{#5}{#6}{#7}}
}
\makeatother

%\expandafter\newcommand\csname #1\endcsname{\MultiImageFigure{#1}{#2}{#3}{#4}{#5}{#6}{#7}{#8}{#9}{#10}{#11}{#12}}


% chktex-file 9
\def\lbrak{[}
\def\rbrak{]}
\newcommand{\rangeexex}[1]{\lparen#1\rparen}
\newcommand{\rangeinex}[1]{\lbrak#1\rparen} 
\newcommand{\rangeexin}[1]{\lparen#1\rbrak}
\newcommand{\rangeinin}[1]{\lbrak#1\rbrak}

%\newcommand{\rangeexex}[1]{(#1)}
%\newcommand{\rangeinex}[1]{[#1)} 
%\newcommand{\rangeexin}[1]{(#1]}
%\newcommand{\rangeinin}[1]{[#1]}

%\newcommand{\getitem}[1]{[#1]}
\newcommand{\getitem}[1]{\ensuremath{_{#1}}}


\newcommand{\gammav}{\greekvec{\gamma}}
\newcommand{\deltav}{\greekvec{\delta}}
\newcommand{\onevec}{\vec{1}}
\newcommand{\partfrac}[2]{\frac{\partial{} #1}{\partial{} #2}}

\newcommand{\rarrow}{\ensuremath{\rightarrow}}

%\newcommand{\rmultiarrow}{\ensuremath{*\mkern-10mu\rightarrow}}
%\newcommand{\rmultiarrow}{*\rightarrow}
%\newcommand{\rmultiarrow}{*\mskip-10mu\rightarrow}
\newcommand{\rmultiarrow}{\ensuremath{\overset{*}{\rarrow}}}
