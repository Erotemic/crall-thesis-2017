\chapter{Pairwise classification}\label{chap:pairclf}

In this chapter we consider the problem of verifying whether or not two
  annotation are the same or different.
By addressing this problem we improve upon the ranking algorithm from
  \cref{chap:ranking} --- that ranks the \names{} in a database based on
  similarity to a query --- by making semi-automatic decisions about results
  returned in the ranked lists.
The algorithms introduced in this chapter will assign a confidence to results
  in the ranked list, and any pair above a confidence threshold can be
  automatically reviewed.
We will demonstrate that our decision algorithms can significantly reduce the
  number of manual interactions required to identify all individuals in an
  unlabeled set of annotations.

To make semi-automatic decisions up to a specified confidence we develop a
  \emph{pairwise probabilistic classifier} that predicts a probability
  distribution over a set of events given two annotations (typically a query
  annotation and one of its top results in a ranked list).
Given only the information in two annotations,  there are three possible
  decisions that can be made.
A pair of annotations is either:
\begin{enumln}
    \item incomparable --- the annotations are not visually comparable,

    \item positive --- the annotations are visually comparable and the
      same individual, or

    \item negative --- the annotations are visually comparable and different
      individuals.
\end{enumln}
%Two annotations are incomparable if no distinguishing parts of the annotations
%  can be put in correspondence.
%Otherwise, we can determine if the distinguishing parts are the same or
%  different.
Two annotations can be incomparable if the annotations show different parts or
  sides of an animal, or if the distinguishing information on an animal is
  obscured or occluded.
The positive and negative states each require distinguishing information to be
  present.
Each of these mutually exclusive ``match-states'' are illustrated in
  \cref{fig:MatchStateExample}.
The multi-label classifier then predicts the probability of each of the three
  states, with the probabilities necessarily summing to $1$.

\MatchStateExample{}

To construct a pairwise probabilistic classifier we turn towards supervised
  machine learning, which requires that we:
%Probabilistic machine learning problems can be generally stated as
%  constructing a function using a labeled dataset of of fixed length feature
%  vectors and corresponding labels indicating the desired output class.
%The resulting function maps a novel feature vector to a probability
%  distribution over each possible output class.
\begin{enumin}

    \item determine a set of labeled annotation pairs for training, 

    \item construct a fixed-length feature vector to represent a pair of
        annotations,  and 

    \item choose a probabilistic learning algorithm.
\end{enumin}
The first requirement can be satisfied by carefully selecting representative
  annotations pairs, and the last requirement is satisfied by many pre-existing
  algorithms (\eg{} random forests and neural networks).
It is the second requirement --- constructing an appropriate fixed-length
  feature vector --- that is the most challenging.
If given enough training data and a technique to align annotations, using
  image data with a Siamese or triplicate network might appropriate, but without
  both of these pre-conditions we must turn towards more traditional methods.
Recall from \cref{sec:annotrepr} that our annotation representation is an
  unordered bag-of-features, which cannot be directly fed to most learning
  algorithms.
Therefore, we develop a method for constructing a \glossterm{pairwise feature
  vector} for a pair of annotations.
This novel feature vector will take into account local matching information as
  well as more global information such as GPS and viewpoint.
These feature are used to fit a random forest~\cite{breiman_random_2001} which
  implements our pairwise classifier.
We choose to use a random forest classifier, in part because they are fast to
  train, robust to overfitting, and naturally output probabilities in a
  multiclass setting, and in part because they can handle (and potentially take
  advantage of) missing data using the ``separate class''
  method~\cite{ding_investigation_2010}.
  

A final concern that is addressed in this chapter is the issue of image
  challenges that may confound the match-state pairwise classifier.
Photobombs --- pairs of annotations where matches are caused by a secondary
  animal --- are the most notable cause of such a challenge.
Several examples of pairs of annotations with photobombs are illustrated in
  \cref{fig:photobomb}.
By most accounts photobombs appear very similar to matches, and it may be
  reasonable to expand the match-state classifier to learn this subtlety.
However, because photobombs are inherently a pairwise property between
  annotations, it should be possible to learn a separate classifier explicitly
  tasked with the challenge.
Therefore, we also learn a photobomb classifier using the same sort of
  pairwise feature vector and random forest classifier.
This supporting classifier will allow us to increase the accuracy of our
  identification, by restricting automatic classification to pairs where where
  the decision is straightforward.


This outline of this chapter is as follows.
\Cref{sec:feature} details the construction of the feature vector that we use
  as input to the pairwise classifier.
\Cref{sec:learning} describes the process of collecting training data and
  learning the match-state pairwise classifier.
\Cref{sec:photobomb} .
\Cref{sec:expt} .
\Cref{sec:summary} .


\section{Constructing the pairwise feature vector}\label{sec:pairfeat}

In order to use the random forest learning algorithm to address the problem of
  pairwise verification, we must construct a fixed-length feature vector that
  contains information able to differentiate between each class.
This feature vector will contain both global and local information.
Global information is higher level and serves to augment visual information.
The local information aggregates statistics about feature correspondences
  between the two annotations.
The local and global vectors are constructed separately and then concatenated
  to form the final pairwise feature vector.
The remainder of this section discusses the construction of these vectors.

\subsection{The global feature vector}

The global feature vector contains information to allow the classifier to take
  advantage of semantic labels and non-visual attributes of our data to solve
  the matching problem.
Semantic labels such as quality and viewpoint are derived from visual
  information and can provide abstract knowledge to help the classifier make a
  matching decision.
Non-visual attributes such as GPS and timestamp can be extracted from EXIF
  metadata and may help determine facts not discernible from visual data alone.
  %(\eg{} it is impossible to match two animals if they are close in time but far
  %in space).
In both cases these base global attributes are fixed-length values that are
  associated with the annotation as a whole.
The global feature vector is a concatenation of these original values and a
  vector of comparisons we make between these values.

\newcommand{\nan}{\text{NaN}}

To construct the global feature vector we use the following annotation attributes:
\begin{enumln}

    \item Timestamp, represented in POSIX format as a float.

    \item GPS latitude and longitude, represented in radians as two floats. 

    \item Viewpoint classification label, represented as a categorical integer in $[1,8]$.

    \item Quality classification label, represented as a categorical integer
      in $[1,5]$.
\end{enumln}
The GPS and timestamp attributes are gathered from image EXIF data, and the
  viewpoint and quality labels are outputs of the deep classifiers discussed in
  \cref{subsec:introdataprocess}.
The GPS and timestamp attribute inform the classifier of when its not possible
  for two annotations to match (\eg{} when a pair of annotations is close in
  time but far in space).
The viewpoint and quality attributes should help the classifier predict when
  pairs of annotations are not comparable --- forcing there to be stronger
  evidence to form a match, such as strong correspondences on a face turned
  toward the camera in both a left and right side view.


%Of these four ``unary'' attributes only the qualities are used directly.
%By themselves the other unary attributes like time should not influence an
%  annotations ability to match, so to prevent potential overfitting they are not
%  included in the global vector.
%Instead we

These four ``unary'' attributes are gathered for each annotation, and can be
  stacked at the front of the global vector.
This will allow a classifier to consider each of the properties of each
  annotation separately.
However, we note that some unary attributes like GPS and time should be
  independent of an annotations ability to match.
To prevent biases in data collection from causing overfitting we typically do
  not include unary feature by themselves, but instead encode their
  relationships.
The exceptions are quality and viewpoint which might be meaningful
  independently.
In these cases, because the ordering of the annotations is arbitrary we record
  the minimum of the two values in a feature dimension and the maximum in
  another.

To allow the classifier to account for relationships between the unary
  attributes that could influence a pair of annotation's ability to match we
  encode distance measurements between the unary attributes.
In the case of GPS coordinates we use the haversine distance (as detailed in
  \Cref{sec:occurgroup}), but for all other measures we use the absolute
  difference of their values.
Lastly, we include the ``speed'' of the annotations, which is the GPS-distance
  divided by the time-delta.

In the event that an attribute is not provided or not known we replace the
  measurement with a \nan{} value.
If data is not missing completely at random, then a random forest will be able
  to take advantage of \nan{} values by treating them as a separate
  class~\cite{ding_investigation_2010}.

%New:
%We no longer include the unary attributes except for quality, but we remove
%  the asymetry by recording the dimensions as minqual and maxqual.

\subsection{The local feature vector}
The local feature vector distills two orderless bag-of-features
  representations into a fixed length vector indicating the nature of a pairwise
  match.
There are three steps needed to construct the local feature vector.
First we determine feature correspondences between the two annotations.
Then for each correspondence we make several measurements (\eg{} descriptor
  distance and spatial position).
Then we aggregate these measurements over all correspondences using summary
  statistics (\eg{} mean, sum, std).
Thus the total length of the feature vector is the number of measurements
  times the number of summary statistics used.

To determine feature correspondences between two annotations, $\qaid$ and $\daid$, we
  use a one-vs-one matching algorithm.
Each annotation's descriptors are indexed for fast nearest neighbor
  search~\cite{muja_fast_2009}.
Keypoint correspondences are formed by searching for the reciprocal nearest
  neighbors between annotation descriptors~\cite{qin_hello_2011}.
For each correspondence the next nearest neighbor from annotation $\qaid$ to
  annotation $\daid$ is used as a normalizer for a ratio
  test~\cite{lowe_distinctive_2004} (note this introduces slight asymmetry).
If the descriptor distance between correspondences is divided by the distance
  to the normalizer is above a threshold ($.625$), the correspondence is
  regarded as non-distinct and removed.
Spatial verification~\cite{philbin_object_2007} is applied to further refine
  the correspondences.
This results in a richer set of correspondences between annotations $\qaid$ and
  $\daid$ that would be found using the ranking algorithm.

After the one-vs-one matching stage, measurements are made at each feature
  correspondence.
For a feature correspondence between two features $i$ and $j$ with descriptors
  $\desc_i$ and $\desc_j$, normalizing descriptor $\descnorm$, the measurements
  we consider are:
\begin{itemln}

    \item Correspondence distance:
        This is the Euclidean distance between the feature correspondence,
        $\elltwo{\desc_i - \desc_j} / Z$. This serves as a measure of visual
        similarity between the features. (Recall $Z\tighteq\sqrt{2}$ for SIFT
        descriptors).

    \item Normalizer distance:
        This distance between the first descriptor and the normalizing descriptor,
        $\elltwo{\desc_i - \descnorm} / Z$.
        This serves as a measure of visual distinctivness of the features.

    \item Ratio score:
        This is the ratio between the distance between the correspondence, and 
        the distance to the normalizer, %
        $\elltwo{\desc_i - \desc_j} / \elltwo{\desc_i - \descnorm}$. This is the
        same measure used in the ratio test and weights the similarity against
        the distinctivness.

    \item Foregroundness score, the geometric product of the
        features' foregroundness measures, $\sqrt{w_i w_j}$ along with the
        individual foregroundness weights $w_i$ and $w_j$.
        
    \item Spatial verification error in location, scale, and orientation,
        $( %
        \Delta \pt_{i, j}, %
        \Delta \scale_{i, j}, %
        \Delta \ori_{i, j})$, as measured using~\cref{eqn:inlierdelta}.

    \item Keypoint scales, indicating the size of each keypoint in its
        annotation $\sigma_i$ and $\sigma_j$.

    \item Keypoint relative locations, the xy-locations of the keypoints
        divided by the width and height of the annotation 
         $x_i / w_{\qaid}, y_i / h_{\qaid}$ and
         $x_i / w_{\daid}, y_j / h_{\daid}$.
\end{itemln}

Once these measurements have been made for each keypoint correspondence we
  summarize them using summary statistics.
We consider the sum, mean, and standard deviation over all correspondences.
We also consider taking values from individual correspondences based on
  rankings and percentiles with respect to a particular attribute (\eg{} ratio
  score), however in practice these provided little information.
The resulting measurements are stacked to form the local feature vector.
A final step we have found useful is to append an extra dimension simply
  indicating the total number of feature correspondences.

Because setting the threshold for the ratio test produces a binary decision as
  to whether or not a correspondence is used in generating local statistics, we
  generate these statitics for multiple values of the ratio test threshold.
This helps account for viewpoint and pose variations that may cause
  correspondences to appear slightly less distinctive.

\subsection{The final pairwise feature vector}

\begin{figure}
\begin{minted}[gobble=4]{python}
    OrderedDict([('global(qual_1)',      nan),
                 ('global(qual_2)',      nan),
                 ('global(qual_delta)',  nan),
                 ('global(gps_1[0])',    -1.37),
                 ('global(gps_2[0])',    -1.41),
                 ('global(gps_1[1])',    36.81),
                 ('global(gps_2[1])',    36.77),
                 ('global(gps_delta)',   5.79),
                 ('len(matches)',        20),
                 ('sum(ratio)',          10.05),
                 ('mean(ratio)',         0.50),
                 ('std(ratio)',          0.09)])
\end{minted}
\caption[\caplbl{PairFeatVec}A pairwise feature vector]{\caplbl{PairFeatVec}
Example of a small pairwise feature vector containing local and global
information. In practice the pairwise feature vectors include many more
dimensions.
}
\label{fig:PairFeatVec}
\end{figure}

The pairwise feature vector is constructed by concatenating the local and the
  global vector.
This results in a final descriptor vector that can be quite large (thousands
  of dimensions) and some of the dimensions might not be useful.
To this end we prune only the feature vector using only the most useful of the
  proposed dimensions.
This is done by considering the importance of each feature dimension as
  reported by the random forest classifier.
More details on this will be given in the experiment section.
An example of what a final feature vector is illustrated
  in~\cref{fig:PairFeatVec}.


\section{Learning the match-state classifier}
\label{sec:learnclf}

Having defined the pairwise feature vector the only remaining steps to
  constructing the pairwise classifier are to select a sample of labeled
  training data and choose a probabilistic learning algorithm.
We have previously stated that we will use the random forest learning
  algorithm, so the last remaining step to explain is the selection of training
  data and generation of labels.

Recall that our the purpose of our classifier is to output a probability
  distribution over three labels:
positive, negative and incomparable.
Given a pair of annotations we need to assign one of these three labels using
  ground-truth data.
In the most recent version of our system, this ground-truth label is stored
  along with each unordered pair of annotations that has been manually reviewed,
  but because this is a new feature there does not exist many pairs with the
  explicit three state label.
Therefore we must make use of data contained in previous versions of the
  system that simply assign a name label to each annotation.
This allows us to determine if a pair of annotation is the same or different,
  but it does not provide information about whether or not two annotations are
  comparable.
To account for this we approximate the incomparable label using the assigned
  viewpoints, and if either annotation is not assigned a viewpoint it is assumed
  that they are comparable.
Thus, training labels are assigned to a pair as follows:
use the explicit labels if they exist, otherwise guess if the pair is
  comparable based on viewpoint information, if not then return incomparable,
  otherwise return positive if the annotations share a name label and negative
  if they do not.

We sample representative pairs of annotations guided by the principal of
  selecting examples that exhibit a range of
  difficulties~\cite{shi_embedding_2016} (\eg hard-negatives and moderate
  positives).
We use the ranking algorithm to estimate how easy or difficult it might be to
  predict the match-state of a pair.
Pairs with higher scores will be easier to classify for positive examples and
  will be more difficult for negative examples, and lower scores will have the
  reverse effect.
We included all pairs explicitly labeled as incomparable because there is only
  a small number of such examples.

The training data for the pairwise classifier is a set of annotation pairs and
  a set of corresponding labels.
To train the classifier we follow the methodology of selecting hard negatives
  and moderate positives~\cite{shi_embedding_2016}.

Specifically, to sample a dataset for learning, we first rank the database for
  each query image using the ranking algorithm.
We partition the ranked lists into two parts:
a list of correct result matches and a list of incorrect matches.
We select annotations randomly and from the top, middle, bottom of each list.
For positive examples we select $4$ from the top, $2$ from the middle, $2$
  from the bottom, and $2$ randomly.
For negative examples we select $3$ from the top, $2$ from the middle, $1$
  from the bottom, and $2$ randomly.
Because our current datasets only contain a small number of explicitly labeled
  incomparable cases, we include all such pairs.
%The same is true for pairs labeled as photobombs.

\section{Secondary classifier to address photobombs}\label{sec:learnpb}
    It is useful to augment the primary match-state pairwise classifier with a
      secondary classifier able to determine if a pair of annotations contains
      information that might confuse the main classifier.
    These confusing annotation pairs should not be considered for automatic
      review.
    In annotation one of the most challenging of these secondary states is one
      that we refer to as a \glossterm{photobomb}.
    A pair of annotations is a photobomb if a secondary animal in one
      annotation matches an animal in another annotation (\eg see
      Figure~\ref{fig:PhotobombExample}).
    Only the primary animal in each annotation should be used to determine
      identity, but photobombs provide strong evidence of matching that can
      confuse a matching algorithm.

    \PhotobombExample{}

    During events like the \GZC{} we labeled any pair of annotations where
      matching information was between secondary annotations as photobombs.
    Using these labels we are able to construct a classifier similarly to the
      method used to construct the primary match-state classifier.
    Here we can select all pairs of annotations labeled as photobombs for
      positive training data.
    For negative training data we must be careful only to choose pairs that
      have been explicitly reviewed as not a photobomb.
    We attempt to balance positive, negative, and incomparable state in the
      remainder of the negative training data.

\section{Pairwise classification experiments}

    We evaluate the pairwise classifiers on two datasets, one of plains zebras
      and the other of Grévy's zebras.
    For each dataset we choose a sample of annotation pairs as detailed in
      \cref{sec:learnclf}.
    Each pair is associated with a ground-truth matching state label of either
      positive, negative, or incomparable.
    Additionally each pair is also labeled as either a photobomb or not a
      photobomb.
    For each pair we construct a pairwise feature vector as described in
      \cref{sec:pairfeat}.
    We have found that, for plains zebras, it is important to use orientation
      augmentation when computing one-vs-one matches.
    In this case we should not use the second nearest neighbor as the
      normalizer for the ratio test, because the augmented keypoints will have
      similar descriptors.
    We account for this by using the 3rd nearest neighbor as the normalizer
      instead.

    We split this dataset of labeled annotation pairs into into multiple
      disjoint training and testing sets using grouped stratified $k$-fold cross
      validation (we use $3$ folds).
    We group the annotation pairs such that each sample (a pair of
      annotations) within the same name or between the same two names are in the
      same group.
    The cross validation is constrained such all samples in a group are either
      in the training or testing set for each split.
    This means that a specific individual cannot appear in both the training
      and testing set, which helps ensures that our results will generalize to
      new individuals.

    For each cross validation split, we train the matching state and photobomb
      state classifier on the training set and then predict probabilities on
      each sample in the testing set.
    Because the cross validation is $k$-fold and the splits are disjoint, we
      make a single prediction for each sample for each classifier.

    We will compare these predictions with the scores generated by LNBNN, so
      we must additionally generate an LNBNN score for each pair.
    This is done by issuing each annotation as a query.
    Any (undirected) pair in our dataset that appears as a query / database
      pair in the ranked list is assigned that LNBNN score.
    If a pair appears multiple times in the ranked list, then the maximum of
      the two scores is used.
    Any pair that does not appear in the ranked list (because LNBNN failed to
      return it) is implicitly given a score of zero.

    We have now predicted match-state probabilities, photobomb-state
      probabilities, and one-vs-many LNBNN scores for each pair in our dataset.
    We now analyze the results for each classifier in the following
      subsections.
    Our tests will directly compare the match state classifier with the direct
      use of LNBNN as a verification mechanism.
    Additionally, for each classifier we will measure the number of
      classification errors and success in a confusion matrix.
    We also inspect the importance of each feature dimension of our pairwise
      feature vector.
    Finally, we will present the several examples of failure cases to
      illustrate where improvements are necessary.

    %To train a classifier to be applied to new data, all collected data becomes
    %  the training set.
    %We measure binary classification performance in a one-vs-rest manner
    %  focusing on positive-vs-rest.
    %Results for binary classification experiments are reported as ROC curves.


    %NOTES:

    \subsection{Matching State}

        The primary classifier predicts the matching state (positive,
          negative, incomparable) of a pair of annotations.
        To demonstrate the effectiveness of our multiclass classifier we
          report the confusion matrix of class predictions for plains and
          Grévy's zebras in \cref{fig:ConfusionMatch}.

        \ConfusionMatch{}

        \EvalMetricsMatch{}

        Using this confusion matrix we compute the average precision, recall,
          and Matthews correlation coefficient (MCC) for each class.
        These number are reported in \cref{EvalMetricsMatch}.
        The MCC provides a measurement of overall multiclass classification
          performance unbiased by class distribution.
        An MCC ranges from $+1$, indicating perfect predictions, to $-1$,
          indicates pathological inverse predictions, with $0$ being random
          uninformed predictions \cite{powers_evaluation_2011}.
        % FIXME: ensure these numbers match the tables
        The MCC of $0.85$ for Grévy's zebras and $0.86$ for plains zebras,
          indicating that our classifiers have strong predictive power.
        

        \paragraph{Binary Positive Classification}
        Although the accuracy of multiclass predictions is important, we are
          most concerned with distinguishing positive pairs from the other
          classes.
        Ideally our learned classifiers will result in superior separation
          when compared to using just the LNBNN scores as done in the previous
          chapter.
        To test this we plot histograms of scores (LNBNN vs positive
          probability) in \cref{fig:PositiveHist}.
        It is immediately noticeable that the pairwise scores seem to have a
          much better separating in addition to being in the interpretable range
          of zero to one.

        %These results demonstrate the advantage of using the pairwise classifier
        %of the pairwise classifier
        %that the pairwise classifier results in
        %interpretable scores

        \PositiveHist{}

        We can make a more direct and precise comparison by augmenting LNBNN
          scores and pairwise probabilities with an adjustable threshold we can
          construct two comparable binary classifiers.
        The separability of each method is indicated by the area under the ROC
          curve.
        The ROC curves for plains and Grévy's zebras are illustrated in
          \cref{fig:PositiveROC}.
        For Grévy's zebras the pairwise AUC of $.95$ is convincingly better
          than the LNBNN AUC of $.86$.
        For plains zebras, there is probably a dataset issue, but the pairwise
          AUC of $.95$ is better than the LNBNN AUC of $.94$.

        \PositiveROC{}


        \paragraph{Feature Importance}
        %The most important features for the zebra datasets are illustrated in
        %\Cref{fig:MatchWordCloud} and the top $5$ are given in \cref{tbl:ImportantMatchFeat}.
        To see what the classifier is using to make its predictions we measure the
        importance of each dimension in our feature vector using <insert name of RF
        importance method>. To gain an intuition for which features are most important
        we create a word cloud where the size of the text is proportional to the
        importance of the feature. The word cloud is illustrated in
        \cref{fig:MatchWordCloud}. The numeric importance of the top features is
        recorded in \cref{tbl:ImportantMatchFeat}.

        \MatchWordCloud{}
        \ImportantMatchFeat{}

        \paragraph{Failure Cases}
        Lastly we investigate several failure cases.

        %illustrate failure cases: TODO Match State Failure Cases
        \PairFailIN{} 

        \PairFailNP{}

        \PairFailPN{}


    %---------
    \subsection{Photobomb State}
        We perform a subset of similar experiments to demonstrate the effectiveness of
        our photobomb classifier.
        The overall performance is given in the classification confusion matrix
        illustrated in \cref{tbl:ConfusionPhotobomb}.

        \ConfusionPhotobomb{}

        \EvalMetricsPhotobomb{}

        \paragraph{Feature Importance}
        The most features most important to photobomb classification are shown in

        \ImportantPBFeat{}


\section{Summary of pairwise classification}

    In this chapter we have shown how to predict the matching state of a pair of annotations. 
    We have shown that our classifiers produce probabilities that are more
    interpretable and better able to separate positive from negative cases than
    LNBNN scores. 
