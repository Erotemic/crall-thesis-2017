\begin{comment}
# Summarize Commands
python -m ibeis.scripts.gen_cand_expts --exec-parse_latex_comments_for_commmands --fname figdefindiv.tex
gvim ~/code/ibeis/regen_figdefindiv.sh
~/code/ibeis/regen_figdefindiv.sh --overwrite
\end{comment}


% -------------------------------------------
% Individual Results


\begin{comment}
#
NonDistinct 16644 PZ
Pose 4709 PZ
\end{comment}

\newcommand{\textwidthII}{1\textwidth}



% --- VIEWPOINT
\begin{comment}
python -m ibeis.dev -e draw_cases -a timectrl -t best --filt :fail=True,with_tag=Viewpoint,sortdsc=gtscore --db GZ_Master1  --qaid 2787 \
    --hargv=match --render  --cmdaug="FailViewpoint" \
    --cappref="Failure case due to different viewpoints." --overwrite
%--qaid 2660 \
\end{comment}
\SingleImageCommand{FailViewpoint}{1}{
Unaligned failure case
}{
% ---
Due to pose and viewpoint variations, the correctly matching pair of annotations (on the right) is returned at
  rank $2$ while the incorrect pair of annotations (on the left) is returned at rank $1$.
In the correct pair, the features on the front leg are not aligned and failed to match.
In the incorrect pair, the heads of the animals are in a similar pose and thus creating several correspondences
  that are distinctive by coincidence.
% ---
}{figuresC/case_FailViewpoint.png}


%\begin{comment}
%python -m ibeis.dev -e draw_cases -a timectrl -t best --filt :fail=True,with_tag=Viewpoint,sortdsc=gfscore --db PZ_Master1 --qaid 2814 \
%    --hargv=match --render  --cmdaug="FailViewpoint" \
%    --cappref="Failure case due to different viewpoints. The viewpoint of the query image is mislabeled. "
%\end{comment}
%\newcommand{\PZMasterICaseFailViewpoint}{
%\begin{figure*}
%\centering
%\fboxII{\includegraphics[width=\textwidthII]{cases_PZ_Master1/individual_results/qaid=2814/best:K=3,AI=F,QRH=T+timectrl.png}}
%\caption{
%    Failure case due to different viewpoints. The viewpoint of the query image is mislabeldThis figure depicts correct and incorrect matches from configuration: 
%    {\tt best:K=3,\hspace{0pt}AI=F,\hspace{0pt}QRH=T\hspace{0pt}+\hspace{0pt}timectrl}.}
%\label{fig:PZMasterICaseFailViewpoint}
%\end{figure*}
%}


% --- OCCLUSION
\begin{comment}
python -m ibeis.dev -e draw_cases -a timectrl -t best --filt :fail=True,with_tag=Occlusion,sortdsc=gfscore --db PZ_Master1 --qaid 3812 \
    --hargv=match --render  --cmdaug="Occlusion" \
    --cappref="Failure case due to occlusion. "
\end{comment}
\SingleImageCommand{FailOcclusion}{1}{
Occlusion failure case
}{
% ---
The plants occluding both the query and database annotations inhibit the creation of feature correspondences,
  causing the correct pair of annotations (on the right) to be returned at rank $2$.
This is exacerbated by pose and viewpoint variations.
The incorrect pair of annotation (on the left) at rank $1$ are relatively distinctive by coincidence.
% ---
}{figuresC/case_Occlusion.png}

% --- QUALITY

\begin{comment}
python -m ibeis.dev -e draw_cases -a timectrl -t best --filt :fail=True,with_tag=Quality,sortdsc=gfscore --db GIRM_Master1 --qaid 639 \
    --hargv=match --render  --cmdaug="FailQuality" \
    --cappref="Failure case due poor database image quality. The database image was marked as ok, when it is closer to a poor or junk quality." --vert=False
\end{comment}
\SingleImageCommand{FailQuality}{1}{
Quality failure case
}{
% ---
The low resolution of the query annotation and the overall viewpoint difference causes the correct pair of
  annotations (on the right) to be returned at rank $75$.
The incorrect pair of annotation (on the left) did not recieve a particularly high score, but it was returned at
  rank $1$ because there were no feature correspondences established to the correct match.
% ---
}{figuresC/case_FailQuality.png}


% --- ORIENTATION
\begin{comment} 
python -m ibeis.dev -e draw_cases -a timectrl -t best --filt :fail=True,with_tag=Orientation,sortdsc=gfscore  --db GZ_Master1 --qaid 1291 \
    --hargv=match --render  --cmdaug="FailOrientation" \
    --cappref="The orientation of the query image is specified incorrectly." 
    %--qaid 509 \
\end{comment}
\SingleImageCommand{FailOrientation}{1}{
Orientation failure case
}{
% ---
The orientation of the query image is specified incorrectly.
% ---
}{figuresC/case_FailOrientation.png}


% --- Lighting
\begin{comment}
python -m ibeis.dev -e draw_cases -a timectrl -t best --filt :fail=True,with_tag=Lighting,sortdsc=gfscore --db GIRM_Master1 --qaid 274 \
    --hargv=match --render  --cmdaug="FailLighting" \
    --cappref="Failure case due to poor illumination of the query image." --vert
\end{comment}
\SingleImageCommand{FailLighting}{1}{
Lighting failure case
}{
% ---
Failure case due to poor illumination of the query image.
Feature extraction is unreliable in underexposed images.
% ---
}{figuresC/case_FailLighting.png}

% --- Pose

\begin{comment}
python -m ibeis.dev -e draw_cases -a timectrl -t best --filt :fail=True,with_tag=Pose,sortdsc=gfscore --db PZ_Master1 \
--qaid=1680 \
--cappref="Failure Case: Pose. " --hargv=match \
--capsuf="" \
--cmdaug=FailPose --render --overwrite
#
%781
#python -m ibeis.dev -e draw_cases -a timectrl -t best --filt :fail=True,with_tag=Pose,sortdsc=gfscore --db PZ_Master1 --qaid=4709 \
\end{comment}
\SingleImageCommand{FailPose}{1}{
Pose failure case
}{
% ---
 A failure case due to pose.
% ---
}{figuresC/case_FailPose.png}

% --- Non Distinct
\begin{comment}
python -m ibeis.dev -e draw_cases -a timectrl -t best --filt :fail=True,min_gf_timedelta=1h,with_tag=NonDistinct,sortdsc=gfscore --db PZ_Master1 \
--qaid=9038 --cmdaug="FailNonDistinct" --hargv=match \
--cappref="This failure case is caused by non distinct matches receving high scores." \
--capsuf="" \
--render
\end{comment}
\SingleImageCommand{FailNonDistinct}{1}{
Non-distinct failure case
}{
% ---
This failure case is caused by non distinct matches receving high scores.
In most cases the LNBNN~\cite{mccann_local_2012} scoring mechanism downweights
  these incorrect feature correspondences appropriately.
% ---
}{figuresC/case_FailNonDistinct.png}

% --- Photobomb

\begin{comment}
python -m ibeis.dev -e draw_cases -a timectrl -t best --filt :fail=True,with_tag=Photobomb,sortdsc=gfscore --db PZ_Master1 \
--qaid=3727 --cmdaug="FailPhotobomb" --hargv=match \
--cappref="Failure case caused two animals depicted in the annotation. The match to the secondary animal has a small timedelta." \
--capsuf="" \
--render
\end{comment}
\SingleImageCommand{FailPhotobomb}{1}{
    Photobomb failure case
}{
% ---
A photobombing animal in the background of the query annotation cause LNBNN to return the incorrect result (on
  the left) at rank $1$.
The correct match (on the right), has a significant number of matches, but there is a difference of $1$ day
  between the pair.
On the other hand, the annotations in the photobomb pair were taken within minutes of each other and therefore
  have much higher visual similarity.
% ---
}{figuresC/case_FailPhotobomb.png}

% --- Scenery Match

\begin{comment}
python -m ibeis.dev -e draw_cases -a timectrl -t best --filt :fail=True,with_tag=SceneryMatch,sortdsc=gfscore --db GZ_Master1 \
    --qaid 1988 \
    --hargv=match --render  --cmdaug="FailScenery" \
    --cappref="Failure case due to a scenery match. Most scenery match cases have small timedeltas between the images."

 # 2811

python -m ibeis.dev -e draw_cases -a timectrl -t best --filt :fail=True,with_tag=SceneryMatch,sortdsc=gfscore --db GZ_Master1 --qaid 1988 --show

python -m ibeis.dev -e draw_cases -a timectrl -t best --filt :fail=True,with_tag=SceneryMatch,sortdsc=gfscore --db GZ_Master1 --qaid 1988 --show
python -m ibeis.dev -e draw_cases -a timectrl -t best:sv_on=False --filt :fail=True,with_tag=SceneryMatch,sortdsc=gfscore --db GZ_Master1 --qaid 1988 --show
python -m ibeis.dev -e draw_cases -a timectrl -t best:sv_on=False,AI=False --filt :fail=True,with_tag=SceneryMatch,sortdsc=gfscore --db GZ_Master1 --qaid 1988 --show

\end{comment}
\SingleImageCommand{FailScenery}{1}{
    Scenery failure case
}{
% ---
The incorrect pair of annotations (on the left) was returned at rank $1$ because of strong matches in the
  background scenery.
The correct pair was returned at rank $2$ and did not produce matches in the front leg due to pose variations.
The annotations in the scenery match pair were taken $8$ seconds appart in the same location causing their
  backgrounds to be near duplicates.
The foregroundness measure was disabled to produce this example, enabling it addresses nearly all scenery match
  cases.
% ---
}{figuresC/case_FailScenery.png}



%\begin{comment}
%    --hargv=match --render  --cmdaug="FailScenery" \
%    --cappref="Failure case due to a scenery match despite heavy background downweighting. Most scenery match cases have small timedeltas between the images. The correct result did not establish many matches due to viewpoint differences."
%\end{comment}
%\newcommand{\PZMasterICaseFailScenery}{
%\begin{figure*}
%\centering
%\fboxII{\includegraphics[width=\textwidthII]{cases_PZ_Master1/individual_results/qaid=5053/best:K=3,AI=F,QRH=T+timectrl.png}}
%\caption{
%    Failure case due to a scenery match despite heavy background downweighting. Most scenery match cases have small timedeltas between the images. The correct result did not establish many matches due to viewpoint differences.This figure depicts correct and incorrect matches from configuration: 
%    {\tt best:K=3,\hspace{0pt}AI=F,\hspace{0pt}QRH=T\hspace{0pt}+\hspace{0pt}timectrl}.}
%\label{fig:PZMasterICaseFailScenery}
%\end{figure*}
%}



% success cases
\begin{comment}
python -m ibeis.dev -e cases --db PZ_Master1 -a timectrl \
    -t invarbest --filt :sortasc=gtscore,success=True,index=200:201 --show --save successpz1.png

python -m ibeis.dev -e cases --db GZ_ALL -a timectrl \
    -t best --filt :sortdsc=gtscore,success=True --show --save successgz1.png
\end{comment}
