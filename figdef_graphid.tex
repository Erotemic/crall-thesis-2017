
\begin{comment}
python -m ibeis.scripts.specialdraw draw_graph_id \
    --dpath ~/latex/crall-thesis-2017/ --save "figures_graph/decisiongraph.jpg" \
    --figsize=12,8 --clipwhite --dpi=280 --diskshow
\end{comment}
\ImageCommand{figures_graph/decisiongraph.jpg}{\textwidth}{
    An example of a consistent synthetic decision graph with positive,
      negative, and incomparable edges.
    The color of each node represents the positive connected compoment (PCC)
      it belongs to.
}{decisiongraph}




\begin{comment}
python -m ibeis.scripts.specialdraw draw_inconsistent_pcc --show
python -m ibeis.scripts.specialdraw draw_inconsistent_pcc \
    --dpath ~/latex/crall-thesis-2017/ --save "figures_graph/inconpcc.jpg" \
    --figsize=10,5 --clipwhite --dpi=180 --diskshow
\end{comment}
\ImageCommand{figures_graph/inconpcc.jpg}{\textwidth}{
    Any PCC containing at least one negative edge is inconsistent.
}{inconpcc}


  


\begin{comment}
python -m ibeis.scripts.specialdraw redun_demo2 --show
python -m ibeis.scripts.specialdraw redun_demo2 \
    --dpath ~/latex/crall-thesis-2017/ --save "figures_graph/kredun.jpg" \
    --figsize=10,5 --clipwhite --dpi=180 --saveparts
\end{comment}
\newcommand{\kredun}{
\begin{figure}[ht!]
\centering
\begin{subfigure}[h]{0.31\textwidth}
\centering
\includegraphics[width=\textwidth]{figures_graph/kredunA.jpg}\caption{}\label{sub:kredunA}
\end{subfigure}
~~% --
\begin{subfigure}[h]{0.31\textwidth}
\centering
\includegraphics[width=\textwidth]{figures_graph/kredunB.jpg}\caption{}\label{sub:kredunB}
\end{subfigure}
~~% --
\begin{subfigure}[h]{0.31\textwidth}
\centering
\includegraphics[width=\textwidth]{figures_graph/kredunC.jpg}\caption{}\label{sub:kredunC}
\end{subfigure}
~~% --
\begin{subfigure}[h]{0.31\textwidth}
\centering
\includegraphics[width=\textwidth]{figures_graph/kredunD.jpg}\caption{}\label{sub:kredunD}
\end{subfigure}
~~% --
\begin{subfigure}[h]{0.31\textwidth}
\centering
\includegraphics[width=\textwidth]{figures_graph/kredunE.jpg}\caption{}\label{sub:kredunE}
\end{subfigure}
~~% --
\begin{subfigure}[h]{0.31\textwidth}
\centering
\includegraphics[width=\textwidth]{figures_graph/kredunF.jpg}\caption{}\label{sub:kredunF}
\end{subfigure}
\caption[kredun]{\caplbl{kredun}
Examples of positive (top) and negative (bottom) redundancy.
Choosing the level of redundancy is a tradeoff between the number of required
  reviews and the confidence that the reviews are correct. In our current
  implementation we use $2$-redundancy.
}
\label{fig:kredun}
\end{figure}
}

