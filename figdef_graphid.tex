
\begin{comment}
python -m ibeis.scripts.specialdraw draw_graph_id \
    --dpath ~/latex/crall-thesis-2017/ --save "figures_graph/decisiongraph.jpg" \
    --figsize=12,8 --clipwhite --dpi=280 --diskshow
\end{comment}
\ImageCommand{figures_graph/decisiongraph.jpg}{\textwidth}{
    An example of a consistent synthetic decision graph with positive,
      negative, and incomparable edges.
    The color of each node represents the positive connected compoment (PCC)
      it belongs to.
}{decisiongraph}




\begin{comment}
python -m ibeis.scripts.specialdraw draw_inconsistent_pcc --show
python -m ibeis.scripts.specialdraw draw_inconsistent_pcc \
    --dpath ~/latex/crall-thesis-2017/ --save "figures_graph/inconpcc.jpg" \
    --figsize=15,10 --clipwhite --dpi=180 --diskshow --saveparts
\end{comment}
\MultiImageCommandII{inconpcc}{.4}{inconpcc}{
    Any PCC containing at least one negative edge is inconsistent.
    Subfigure \Cref{sub:inconpccA} shows an inconsistent PCC, and
    \Cref{sub:inconpccB} shows the same PCC where the edges hypothesized to be
    errors are highlighted.
}{figures_graph/inconpccA.jpg}{figures_graph/inconpccB.jpg}


  


\begin{comment}
python -m ibeis.scripts.specialdraw redun_demo2 --show
python -m ibeis.scripts.specialdraw redun_demo2 \
    --dpath ~/latex/crall-thesis-2017/ --save "figures_graph/kredun.jpg" \
    --figsize=10,5 --clipwhite --dpi=180 --saveparts
\end{comment}
\newcommand{\kredun}{
\begin{figure}[ht!]
\centering
\begin{subfigure}[h]{0.31\textwidth}
\centering
\includegraphics[width=\textwidth]{figures_graph/kredunA.jpg}\caption{}\label{sub:kredunA}
\end{subfigure}
~~% --
\begin{subfigure}[h]{0.31\textwidth}
\centering
\includegraphics[width=\textwidth]{figures_graph/kredunB.jpg}\caption{}\label{sub:kredunB}
\end{subfigure}
~~% --
\begin{subfigure}[h]{0.31\textwidth}
\centering
\includegraphics[width=\textwidth]{figures_graph/kredunC.jpg}\caption{}\label{sub:kredunC}
\end{subfigure}
~~% --
\begin{subfigure}[h]{0.31\textwidth}
\centering
\includegraphics[width=\textwidth]{figures_graph/kredunD.jpg}\caption{}\label{sub:kredunD}
\end{subfigure}
~~% --
\begin{subfigure}[h]{0.31\textwidth}
\centering
\includegraphics[width=\textwidth]{figures_graph/kredunE.jpg}\caption{}\label{sub:kredunE}
\end{subfigure}
~~% --
\begin{subfigure}[h]{0.31\textwidth}
\centering
\includegraphics[width=\textwidth]{figures_graph/kredunF.jpg}\caption{}\label{sub:kredunF}
\end{subfigure}
\caption[kredun]{\caplbl{kredun}
%--%
Examples of positive (top) and negative (bottom) redundancy.
The positive edges are colored blue and the negative edges are colored red.
Choosing the level of redundancy is a tradeoff between the number of required
  reviews and the confidence that the reviews are correct.
%In our current implementation we use $2$-redundancy.
%
} \label{fig:kredun}
\end{figure}
}



\begin{comment}
python -m ibeis.viz.viz_chip HARDCODE_SHOW_PB_PAIR --db PZ_Master1 --has_any=photobomb --index=1 --match \
    --dpath ~/latex/crall-thesis-2017/ --save "figures_graph/PhotobombExampleC.jpg" \
    --figsize=9,4 --clipwhite --dpi=180 --save

python -m ibeis.viz.viz_chip HARDCODE_SHOW_PB_PAIR --db PZ_Master1 --has_any=photobomb --index=1 \
    --dpath ~/latex/crall-thesis-2017/ --save "figures_graph/PhotobombExample.jpg" \
    --figsize=9,4 --clipwhite --dpi=180 --saveparts

python -m ibeis.core_annots --test-compute_one_vs_one --show
    
\end{comment}

\newcommand{\PhotobombExample}{
\begin{figure}[ht!]
\centering
\begin{subfigure}[h]{0.4\textwidth}
\centering
\includegraphics[height=100pt]{figures_graph/PhotobombExampleA.jpg}\caption{}\label{sub:PhotobombExampleA}
\end{subfigure}
\begin{subfigure}[h]{0.4\textwidth}
\centering
\includegraphics[height=100pt]{figures_graph/PhotobombExampleB.jpg}\caption{}\label{sub:PhotobombExampleB}
\end{subfigure}
%\begin{subfigure}[h]{0.5\textwidth}
%\centering
%\includegraphics[height=60pt]{figures/PhotobombExampleC.jpg}\caption{}\label{sub:PhotobombExampleC}
%\end{subfigure}
\caption[\caplbl{PhotobombExample}Photobomb example]{\caplbl{PhotobombExample}
% ---
A secondary animal in an annotation can cause a ``photobomb''.  Notice the
primary animal in~\cref{sub:PhotobombExampleA} appears in the background
of~\cref{sub:PhotobombExampleB}. 
%The matching regions are displayed in~\cref{sub:PhotobombExampleC}.
% ---
}
\label{fig:PhotobombExample}
\end{figure}
}



\begin{comment}
python -m ibeis.algo.graph.mixin_loops prob_any_remain --num_pccs=40 --size=2 --patience=20 --window=20 --dpi=300 --figsize=7.4375,3.0 '--dpath=~/latex/crall-thesis-2017' --save=figures5/poisson.png --diskshow
\end{comment}
\newcommand{\poisson}{
\begin{figure}[H]
    %
    \centering
    \includegraphics[width=\textwidth]{figures5/poisson.png}
    \caption[Convergence criteria]{\caplbl{poisson}
    %--
    The convergence criteria applied to a synthetic dataset with $40$ names and $2$ annotations per name.
    The red line indicates the fraction of meaningful reviews that remain undiscovered.
    The blue line is the probability that at least one of the next $20$ reviews will be meaningful.
    Notice that the blue line dips when the red line flattens.
    The process terminates once this probability drops below $0.1$.
    %--
} \label{fig:poisson}
\end{figure}
}
