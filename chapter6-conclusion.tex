% +--- CHAPTER --- 
\begin{comment}
    ./texfix.py --fpaths chapter6-conclusion.tex --outline --asmarkdown --numlines=99 -w
\end{comment}

\chapter{Conclusion}\label{chap:conclusion} %

    Outline:
    \begin{itemln}
        \item Our graph-id framework is independent and complementary to DCNN matching approaches.
    \end{itemln}

    In this \thesis{} we have addressed the problem of identifying individual animals from images. We have considered
    two main problems. First given a single bounding box annotation around an animal and given a database of labeled
    annotations, determine the identity of that individual or say that it is new. Second, we are given a set of labeled
    and unlabeled annotations, and we determine the identities of all annotations by clustering them. We have also
    considered the problem of designing a system that to address the main problems in a dynamic context.

    In~\cref{chap:ranking} we introduced the single image identification algorithm that addresses the first problem and
    evaluated this algorithm on three different species. This algorithm produces a ranking of database individuals that
    are similar to a single query annotation.
    %The algorithm is integrated into a system designed to address the
    %  challenges of image analysis in a dynamic context.
    We have used this algorithm in combination with the effort of citizen scientists to provide an estimate of the
    number of plains zebras and Masai giraffes in Nairobi National Park.
    
    Then in~\cref{sec:contributions} we enumerate our current and intended
      contributions, summarizing the work needed to complete this thesis.

    \section{Discussion}\label{sec:discussresult} 

        Overall our experiments have shown that we are able to accurately rank image of plains zebras, Grévy's zebras,
        and Masai giraffes. Using this system we have successfully estimated the population of plains zebras and Masai
        giraffes in Nairobi National Park. We have demonstrated that using information from multiple images improves
        identification accuracy. We have also shown the importance and the difficulty of addressing the problem of
        decision/verification in the context of animal identification. Despite the successes of our system, it is not
        without its limitations. The following list enumerates several limitations of the current system.

        \begin{itemln}

            %\item The current limitations of the system are the
            %  requirement of manual verification of results and the
            %  inability to run robust consistency checks.

            \item \textbf{Scalability of the identification algorithm}:
            There may be a limitation in the scalability our nearest-neighbor based algorithms. However, this may
            be alleviated with the proper selection of exemplars. In the near future we do not see any reason to
            move away from nearest neighbor based techniques. However, farther down the line it may be necessary to
            replace or preprocess the current indexing algorithm using a vocabulary based approach.

            \item \textbf{Identification of animals without distinguishing textures}:
            The SIFT-based matching algorithms that we employ will only be able to match species containing
            distinctive gradient-based markings. New algorithms will need to be developed for animals with
            distinguishing shape based features. However, our system design is able to incorporate new algorithms
            --- such as the whale fluke matching algorithm --- under the same image analysis API{}.

            \item \textbf{Identification of highly deformable animals}:
            Our algorithms use projective transformations to enforce spatial constraints between matches. While
            this works well for the species we consider, it will be more difficult to match other species. Future
            work should seek to more robustly align annotations before comparing them. This might be done directly
            by warping annotations using non-linear alignment
              %(like the scale-cascaded alignment in SLOOP~\cite{duyck_sloop_2015})
            or via localizing and comparing smaller parts. Given a proper alignment, it should be possible to apply
            state-of-the-art neural network techniques to produce reliable image descriptions and similarities.
            %Our techniques will not reliable 
            %Current measures of pairwise probability are not based on
            %  state-of-the-art neural network techniques involving Siamese
            %  networks.
            %Our evaluation of these techniques yielded negative results due
            %  to insufficient training data and the lack of a robust
            %  algorithm to align the parts of animals.

            %It is my hope that the work done in this \thesis{} will allow
            %  for enough diverse training data to be bootstrapped to
            %  accommodate training data-hungry neural networks.
            %Future work should seriously consider using a parts-based model
            %  and computing matching similarity between parts using a Siamese
            %  network.
            %Techniques such as VLADNet %
            %%\ucite{vladnet}
            %can be used to index these parts to reduce the pairwise search
            %  space.
      \end{itemln}
        
    
    \section{Summary of contributions}\label{sec:contributions}
        This section enumerates the current and intended contributions of this work. The intended contributions
        serve as a contract for the work needed to complete this thesis.
      
        \begin{itemln}

            \item \textbf{The single-image animal identification algorithm}:

                \begin{itemln}
                    \item We have applied LNBNN~\cite{mccann_local_2012} to the problem of individual animal
                    identification. We have performed experiments that demonstrate the effect of the $\K$ parameter
                    at multiple database sizes. This is the main focus of our publication
                    in~\cite{crall_hotspotter_2013}.

                    \item We have accounted for the influence of background features using a learned a
                    foregroundness measure to weight the LNBNN scores of feature correspondences. We have
                    empirically shown the effectiveness of this procedure.

                    \item We have introduced a \name{} scoring mechanism to take advantage of information in
                    multiple database exemplars. Our experiments have demonstrated the significance of multiple
                    exemplars for individual identification.

                    \item 
                        We have introduced a heuristic to account for pose issues that augments the orientation of
                        query keypoints. We have evaluated the effect of various levels of feature invariance in
                        our experiments.
                \end{itemln}

           \item \textbf{The annotation-vs-annotation match probability}:
                We have proposed an algorithm to learn a pairwise match probability --- the probability that two
                annotations match, do not match, or are not comparable. Learning if two annotations are not
                comparable provides a measure of confidence about the probability that the annotations are either
                the same or different individuals. To learn this match probability we have proposed to use several
                local and global measures of similarity.
                %Learning this probability will enable our algorithm for
                %  choosing exemplars and our new graph-based identification
                %  algorithm.

                %This enables identification in databases of all sizes
                %  (\occurrences{}), the identification of potential merge and
                %  split cases, and an optimized choice of exemplars.
                %Work in discussion of match, no match, and not comparable.

           \item \textbf{The graph-based identification algorithm}:
                We have proposed an algorithm that will use the learned pairwise match probabilities to perform
                joint identification on a set of multiple annotations. In this way we will take advantage of
                information from multiple images. This algorithm will facilitate addressing the problems of
                \intraoccurrence{} matching, \vsexemplar{} matching, \exemplar{} selection, and split and merge
                case detection.

      \end{itemln}

    %\section{Future work}\label{sec:futurework}
    %    %
    %    This thesis is currently in the candidacy stage.
    %    %The work that will be done in the figure is outlined in
    %    %  \cref{chap:proposal}, which will be eventually be updated with a
    %    %  description and evaluation of the advanced techniques.
    %    In the future this section will be updated.

    %    It may be necessary to normalize the scores with respect to timedelta.

    %    Need to evaluate the scores between annotations of different viewpoints
    %    (show the bi-modality).

% L___ CHAPTER ___

