% +--- CHAPTER --- 
\begin{comment}
    ./texfix.py --fpaths chapter6-conclusion.tex --outline --asmarkdown --numlines=99 -w
\end{comment}

\chapter{Conclusion}\label{chap:conclusion} %


    \paragraph{Open Problems} A list of problems that if solved, have the
      potential to improve the results of this thesis.
    \begin{itemize}
        \item Redundancy with a PCC with ``articulation nodes''.
        These are nodes that bridge between two viewpoints.
        Consider the example where there are 5 left side annotations and 5
          right side annotations.
        There are also 3 intermediate annotations $A$, $B$ and $C$.
        $A$ is comparable to all left side annotations as well as $B$.
        $C$ is comparable to all right side annotations as well as $B$.
        $B$ is only comparable to $A$ and $C$.
        Therefore it is impossible to achieve $k$-positive-redundancy within
          this PCC.
        The current state of the system handles this case by assuming the
          pairwise classifier will be able to mark all $25$ edges between the
          left and right sides as incomparable automatically.
        However, this is inelegant.
        A better solution would be able to note that the left component is
          redundant by itself and the right component is redundant by itself,
          and not work too hard to match all edges between left and right if
          most are turning up as incomparable.
        However, in this case the system should also be extra sure of the
          matches between $A$, $B$, and $C$.



    \end{itemize}

    In this \thesis{} we have addressed the problem of identifying individual
      animals from images.
    We have considered two main problems.
    First given a single bounding box annotation around an animal and given a
      database of labeled annotations, determine the identity of that individual
      or say that it is new.
    Second, we are given a set of labeled and unlabeled annotations, and we
      determine the identities of all annotations by clustering them.
    We have also considered the problem of designing a system that to address
      the main problems in a dynamic context.

    In~\cref{chap:ranking} we introduced the single image identification
      algorithm that addresses the first problem and evaluated this algorithm on
      three different species.
    This algorithm produces a ranking of database individuals that are similar
      to a single query annotation.
    %The algorithm is integrated into a system designed to address the
    %  challenges of image analysis in a dynamic context.
    We have used this algorithm in combination with the effort of citizen
      scientists to provide an estimate of the number of plains zebras and Masai
      giraffes in Nairobi National Park.
    In~\cref{chap:application} we proposed a new algorithm to address the
      second problem.
    This algorithm is designed to improve upon the first algorithm by
      exploiting information in multiple images, naturally resolving errors,
      requiring less manual interaction, and explicitly handling the case of new
      animals.
    The implementation and evaluation will be the main focus of completing
      this thesis.
    In~\cref{chap:system} we described the system architecture and outlined
      improvements to system stability and extensibility.
    This system is designed to address the challenges of image analysis in a
      dynamic context.
    It provides a stateless interface to image analysis algorithms suitable
      for web-based technologies.
    Declarative registration of algorithm dependencies allows for the new
      algorithms to be integrated into the image analysis system.

    To conclude this \thesis{}, we start with discussion of our results as
      well as limitations of our current system.
    %We provide an overview of our improvements over existing animal
    %  identification techniques as well as the current limitations of our
    %  techniques.
    Then in~\cref{sec:contributions} we enumerate our current and intended
      contributions, summarizing the work needed to complete this thesis.

    \section{Discussion}\label{sec:discussresult} 

        Overall our experiments have shown that we are able to accurately rank
          image of plains zebras, Grevy's zebras, and Masai giraffes.
        Using this system we have successfully estimated the population of
          plains zebras and Masai giraffes in Nairobi National Park.
        We have demonstrated that using information from multiple images
          improves identification accuracy.
        We have also shown the importance and the difficulty of addressing the
          problem of decision/verification in the context of animal
          identification.
        Despite the successes of our system, it is not without its
          limitations.
        The following list enumerates several limitations of the current
          system.

        %Our system provides over previous systems such as Stripe Spotter~\cite{lahiri_biometric_2011}
        %Other animal identification systems like
        %  SLOOP~\cite{duyck_sloop_2015} 
        %Using SIFT~\cite{lowe_distinctive_2004} based techniques we
        %  have been able to outperform previous identification algorithms
        %  for zebras such as Stripe Spotter~\cite{lahiri_biometric_2011}.
        %Our system has been designed to be applicable to many different
        %  species and is more general than other SIFT based systems used
        %  to identify species such as manta rays~\cite{town_manta_2013},
        %  leatherback sea turtles~\cite{de_zeeuw_computer_2010}, and
        %  wildebeests~\cite{bolger_wildid_2011,bolger_computerassisted_2012}.
        %Of the other animal identification projects SLOOP is the most
        %  similar to ours in terms of design goals --- such as the
        %  separation of computer vision components from the system
        %  architecture~\cite{duyck_sloop_2015,martineztrinidad_relevance_2014}.
        %Like our system, SLOOP is able to integrate new computer vision
        %  algorithms to aid in the task of individual identification.
        %However, the algorithms implemented in SLOOP are biased towards
        %  identifying small deformable amphibian species such as skinks
        %  and salamanders, similar to the way that our algorithms are
        %  biased towards large textured mammals such as zebras and
        %  giraffes (however both systems are designed with many species
        %  in mind).
        %Both systems require more computer vision tools to address the
        %  challenges presented in identifying any particular species, and
        %  as such both systems have been set up to incorporate new
        %  computer vision algorithms as quickly and cleanly as possible.
        %Recent development of SLOOP is targeted towards relevance
        %  feedback for improving results while we are focused on
        %  verification and error correction.
        %To the best of our knowledge there are no other general purpose
        %  systems for individual animal identification in development.
        %Due to the success of both systems, future collaboration may be
        %  in order.
        
        % In this section we discuss the results of our current system as well
        % as the limitations.
        %\subsection{Improvements over existing techniques}
        %%To the best of our knowledge, there has not been an automated
        %%  identification algorithm developed on this scale.
        %We have demonstrated the ability to identify animals in
        %  databases with thousands of annotations.
        %Our system has integrated animal localization and
        %  foregroundness algorithms.
        %SIFT based systems such as
        %  WILD-ID~\cite{bolger_wildid_2011,bolger_computerassisted_2012},

        %Other algorithms and systems 

        %Other systems such as Stripe
        %  Spotter~\cite{lahiri_biometric_2011}, Manta
        %  Matcher~\cite{town_manta_2013} and
        %  Wild-ID~\cite{bolger_wildid_2011,bolger_computerassisted_2012},
        %  either do not perform with the same level of automation,
        %  accuracy, or at the same scale that we do.

         % leatherbacks sea turtles SIFT id \cite{de_zeeuw_computer_2010}
         % Sloop is a comparable system using SIFT and relavence feedback and deformable matching \cite{duyck_sloop_2015}
        
        %\subsection{Limitations}

        \begin{itemize}

            %\item The current limitations of the system are the
            %  requirement of manual verification of results and the
            %  inability to run robust consistency checks.

            \item \textbf{Scalability of the identification algorithm}:
            There may be a limitation in the scalability our nearest-neighbor
              based algorithms.
            However, this may be alleviated with the proper selection of
              exemplars.
            In the near future we do not see any reason to move away from
              nearest neighbor based techniques.
            However, farther down the line it may be necessary to replace or
              preprocess the current indexing algorithm using a vocabulary based
              approach.

            \item \textbf{Identification of animals without distinguishing textures}:
            The SIFT-based matching algorithms that we employ will only be
              able to match species containing distinctive gradient-based
              markings.
            New algorithms will need to be developed for animals with
              distinguishing shape based features.
            However, our system design is able to incorporate new algorithms
              --- such as the whale fluke matching algorithm --- under the same
              image analysis API{}.

            \item \textbf{Identification of highly deformable animals}:
            Our algorithms use projective transformations to enforce spatial
              constraints between matches.
            While this works well for the species we consider, it will be more
              difficult to match other species.
            Future work should seek to more robustly align annotations before
              comparing them.
            This might be done directly by warping annotations using
              non-linear alignment
              %(like the scale-cascaded alignment in SLOOP~\cite{duyck_sloop_2015})
            or via localizing and comparing smaller parts.
            Given a proper alignment, it should be possible to apply
              state-of-the-art neural network techniques to produce reliable
              image descriptions and similarities.
            %Our techniques will not reliable 
            %Current measures of pairwise probability are not based on
            %  state-of-the-art neural network techniques involving Siamese
            %  networks.
            %Our evaluation of these techniques yielded negative results due
            %  to insufficient training data and the lack of a robust
            %  algorithm to align the parts of animals.

            %It is my hope that the work done in this \thesis{} will allow
            %  for enough diverse training data to be bootstrapped to
            %  accommodate training data-hungry neural networks.
            %Future work should seriously consider using a parts-based model
            %  and computing matching similarity between parts using a Siamese
            %  network.
            %Techniques such as VLADNet %
            %%\ucite{vladnet}
            %can be used to index these parts to reduce the pairwise search
            %  space.
      \end{itemize}
        
    
    \section{Summary of contributions}\label{sec:contributions}
        This section enumerates the current and intended contributions of this
          work.
        The intended contributions serve as a contract for the work needed to
          complete this thesis.
      
        \begin{itemize}

            \item \textbf{The single-image animal identification algorithm}:

                \begin{itemize}
                    \item We have applied LNBNN~\cite{mccann_local_2012} to
                      the problem of individual animal identification.
                    We have performed experiments that demonstrate the effect
                      of the $\K$ parameter at multiple database sizes.
                    This is the main focus of our publication
                      in~\cref{crall_hotspotter_2013}.

                    \item We have accounted for the influence of background
                      features using a learned a foregroundness measure to
                      weight the LNBNN scores of feature correspondences.
                    We have empirically shown the effectiveness of this
                      procedure.

                    \item We have introduced a \name{} scoring mechanism to
                      take advantage of information in multiple database
                      exemplars.
                    Our experiments have demonstrated the significance of
                      multiple exemplars for individual identification.

                    \item 
                        We have introduced a heuristic to account for pose
                          issues that augments the orientation of query
                          keypoints.
                        We have evaluated the effect of various levels of
                          feature invariance in our experiments.
                \end{itemize}

           \item \textbf{The annotation-vs-annotation match probability}:
                We have proposed an algorithm to learn a pairwise match
                  probability --- the probability that two annotations match, do
                  not match, or are not comparable.
                Learning if two annotations are not comparable provides a
                  measure of confidence about the probability that the
                  annotations are either the same or different individuals.
                To learn this match probability we have proposed to use
                  several local and global measures of similarity.
                %Learning this probability will enable our algorithm for
                %  choosing exemplars and our new graph-based identification
                %  algorithm.

                %This enables identification in databases of all sizes
                %  (\occurrences{}), the identification of potential merge and
                %  split cases, and an optimized choice of exemplars.
                %Work in discussion of match, no match, and not comparable.

           \item \textbf{The graph-based identification algorithm}:
                We have proposed an algorithm that will use the learned
                  pairwise match probabilities to perform joint identification
                  on a set of multiple annotations.
                In this way we will take advantage of information from
                  multiple images.
                This algorithm will facilitate addressing the problems of
                  \intraoccurrence{} matching, \vsexemplar{} matching,
                  \exemplar{} selection, and split and merge case detection.

            \item \textbf{The image analysis system architecture}:
                We have developed a framework --- the \depcache{} --- for
                  maintaining the results of computer vision algorithms used by
                  the IBEIS{} system.
                This declarative framework maintains properties statelessly
                  derived from a set of dynamic primary objects (\eg{}
                  annotations{}) and separates the concerns of algorithm
                  development and system maintenance.
                The \depcache{} stores the results of expensive computer
                  vision algorithms and serves as an interactive testing harness
                  using IPython notebooks.
                To complete this contribution we will extend the \depcache{}
                  to fully support for computing and augmenting (\eg{} nearest
                  neighbor indexers and neural network detectors) that depend on
                  multiple inputs.


      \end{itemize}

    % Not written things in TODO
    % System --- model UUIDS
    % System --- primary object properties?
    % System --- primary object UUIDS
    \subsection{Proposal timeline}
    The expected timeline for to complete the proposed work is outlined as
      follows:
    \begin{itemize}

        \item July 2016:
        Implement support for one-to-many edges in the \depcache{}.
        Implement support for model augmentation in the \depcache{}.

        \item August 2016:
        Implement one-vs-one refinement of one-vs-many feature matching.
        Implement the procedure to construct pairwise similarity vectors.

        \item September 2016:
        Implement the learning procedure for the pairwise similarity model.
        Perform experiments to test the separability of the learned pairwise
          similarities.
        Implement the exemplar selection procedure using learned pairwise
          similarities.

        \item November 2016:
        Implement the graph-based identification algorithm.
        Perform experiments that test the precision and recall of the
          graph-based inference algorithm.

        \item February 2017:
        Implement the split, and merge procedures using the graph-based
          algorithm.
        Ensure that all new algorithms are properly integrated into the
          system.

        \item March 2017:
        Begin final write-up of thesis contributions.

        \item May 2017:
        Present and defend thesis.

    \end{itemize}

          %\item February 2017 --- 
          %\item April 2017 --- %Complete write-up of thesis contributions.


            %\item An analysis of patch based convolutional descriptors for
            %    animal identification and discussion of the difficulties.
            %\item Scoring Mechanism --- extension of LNBNN to instance recognition
            %\item Parts based matching algorithm based convolutional
            %    descriptors for instance recognition.
            %\item Caching scheme for a dynamic individual identification
            %\item An image-based method for estimating the size of a population.
            %\item Data augmentation to increase robustness to shadowing.
            %\item An iterative bootstrapping method for training database curation.
            %\item Dynamic updates of kd-forests.
            %\item A visual vocabulary built with convolutional descriptors.
    %\section{Future work}\label{sec:futurework}
    %    %
    %    This thesis is currently in the candidacy stage.
    %    %The work that will be done in the figure is outlined in
    %    %  \cref{chap:proposal}, which will be eventually be updated with a
    %    %  description and evaluation of the advanced techniques.
    %    In the future this section will be updated.

    %    It may be necessary to normalize the scores with respect to timedelta.

    %    Need to evaluate the scores between annotations of different viewpoints
    %    (show the bi-modality).

% L___ CHAPTER ___

