% +--- CHAPTER --- 
\begin{comment}
    ./texfix.py --fpaths chapter6-conclusion.tex --outline --asmarkdown --numlines=99 -w
\end{comment}

\chapter{Conclusion}\label{chap:conclusion} %
    %Outline:
    %\begin{itemln}
    %    \item Our graph-id framework is independent and complementary to DCNN matching approaches.
    %\end{itemln}

    In this \thesis{} we have addressed the problem of identifying individual animals from images.
    We have demonstrated that our approach is effective for identifying plains zebras, Grévy's zebras, Masai
      giraffes, and humpback whales.
    Our approach consists of three main components:
    (1) the ranking algorithm from \cref{chap:ranking} that uses a bounding box annotation around an animal to
      search a labeled database of annotations for likely matches.
    (2) the classification algorithm from \cref{chap:pairclf} that probabilistically verifies if a pair of
      annotation is positive, negative, or incomparable, and
    (3) the graph framework from \cref{chap:graphid} that harnesses the previous algorithms in a principled way
      to dynamically determine the identity of all animals in a dataset.
    Each of these algorithms was designed to improve upon the previous, and in \cref{sec:graphexpt} we
      demonstrated that this was indeed the case.

    By combining these algorithms we have made a several meaningful contributions to the problem of animal
      identification.
    In \cref{sec:introgzc} we discussed the Great Zebra Count (\GZC{}), where the ranking algorithm in
      combination with the effort of citizen scientists to provide an estimate of the number of plains zebras and
      Masai giraffes in Nairobi National Park.
    In \cref{sec:rankexpt} we investigated several parameters and factors that can impact the performance of the
      ranking algorithm.
    We discovered that having multiple photos of each individual significantly improves the accuracy of the
      ranking algorithm and we designed a novel name scoring mechanism with this in mind.
    In \cref{sec:pairexpt} we demonstrated that a classification algorithm can be used to improve the separation
      of positive from negative and incomparable ranked results.
    In \cref{sec:graphexpt} we simulated the \GZC{} and demonstrated that our improvements to the ranking
      algorithm --- made by the classification and graph algorithm --- enable us to perform identification in less
      than $25\percent$ of the time required by the original event.

    The research that resulted in \thesis{} began in $2010$ and was completed in $2017$.
    In that time many significant developments have been made in the field of computer vision and machine
      learning, namely the explosion of deep learning where technological and theoretical acheivments have made it
      possible to efficiently train deep convolutional neural networks (DCNNs)~\cite{lecun_deep_2015}.
      %(DCNNs)~\cite{krizhevsky_imagenet_2012,taigman_deepface_2014,razavian_baseline_2015,lecun_deep_2015,arandjelovic_netvlad_2016,rocco_convolutional_2017}.
    Our approach does contain a small amount of deep learning (\eg{} the foregroundness weights), but it not
      based in it.
    In some sense this is an advantage because the algorithms can be applied to many different species without
      any need for pre-training, but it also does not achieve the incredible accuracy acheivable by these networks.
    Yet, the contributions of this \thesis{} are still relevant and complementary to DCNNs.
    This is trivially true in the case of the ranking and classification algorithms, in part due to the
      aformentioned reasons.
    However, the contribution of the graph algorithm is still very relevant, even in the era of deep learning.
    
    \section{Discussion of the graph algorithm}
    The graph identification algorithm models the abstract constraints of the identification problem and provides
      a framework that can efficiently harness the power of any ranking or verification algorithm, whether it be
      deep or shallow.
    The framework dynamically manages the relationships between annotations.
    In most cases this means deciding if two annotations are the same (positive) or different (negative), but
      this also means handling corner cases like when the annotations are incomparable or when there is some other
      interesting connection between two annotations like scenery matches and photobombs.
    As new relationships are added, errors are discovered and corrected, and the identifications are updated.

    The framework also provides a means of prioritizing which edges need to be reviewed based on the underlying
      computer vision algorithms, the edge-augmentation needed to ensure minimum redundancy, and the minimum cut
      needed to correct an error and split an inconsistent individual.
    Edge prioritization works in conjunction with a convergence criteria that determines when identification has
      been completed.
    When combined this means that the graph algorithm can simply be given a set of annotations and executed.
    A signal is emited whenever manual interaction is needed, and once the responce is received the algorithm
      continues.
    This means that the graph algorithm requires little expertise to use.
    Once it is running it effectively provides an identification wizard and simply guides the user through a set
      of simple questions.
    An obvious choice would be to run the graph algorithm on a web server and send questions to remote users that
      can be quickly done in a web browser.
    The algorithm stops once there is a high probability that the vast majority of identifications have been made
      correctly and consistently.

    \section{Discussion of the ranking and verification algorithm}
    pass

    \section{Contributions}\label{sec:contributions}

    The summary of the contributions made in this \thesis{} is as follows:

    \begin{itemln}
    \item \textbf{The ranking algorithm}:
        \begin{itemln}
        \item We have adapated LNBNN~\cite{mccann_local_2012} to the problem of individual animal
            identification.
            We have performed experiments that demonstrate the effect of the $\K$ parameter at multiple
            database sizes.

        \item We have accounted for the influence of background features using a learned a foregroundness
            measure to weight the LNBNN scores of feature correspondences.
            We have empirically shown the effectiveness of this procedure.

        \item We have introduced a \name{} scoring mechanism to take advantage of information in multiple
            database exemplars.
            Our experiments have demonstrated the significance of multiple exemplars for individual
            identification.

        \item We have evaluated the effect of various levels of feature invariance in our experiments.
            We have introduced a heuristic that augments the orientation of query keypoints to account for
            pose variations.
        \end{itemln}

    \item \textbf{The pairwise classification algorithm}:
        \begin{itemln}

        \item We have developed a novel feature vector that represents the matching information between two
            annotations.

        \item We have used these feature vectors to learn a random forest that can predict the probability
            that two annotations are either positive, negative, or incmparable.

        \item We have compared the learned probabilities to LNBNN scores and demonstrated that the separation of
          positive cases is significantly improved.
        \end{itemln}

            %We have proposed an algorithm to learn a pairwise match probability --- the probability that two
            %annotations match, do not match, or are not comparable. Learning if two annotations are not
            %comparable provides a measure of confidence about the probability that the annotations are either
            %the same or different individuals. To learn this match probability we have proposed to use several
            %local and global measures of similarity.
            %Learning this probability will enable our algorithm for
            %  choosing exemplars and our new graph-based identification
            %  algorithm.

            %This enables identification in databases of all sizes
            %  (\occurrences{}), the identification of potential merge and
            %  split cases, and an optimized choice of exemplars.
            %Work in discussion of match, no match, and not comparable.

    \item \textbf{The graph identification algorithm}:
        \begin{enumln}
        \item We have introduced the graph identification algorithm for semi-automatic animal identification.

        \item We have demonstrated that the algorithm can use the ranking and verification algorithms to improve
          the accuracy and efficiency of identification.

        \item The framework is agnostic to the specific ranking and verification algorithms.
        The algorithms discussed in this paper could be replaced with algorithms based on DCNNs.
        The framework can even be used without any computer vision algorithms in order to facilitate a more
          efficient brute force search for small databases.

        \item We have developed a probabilistic termination criteria that determines when to stop identification.

        \item We have developed a measure of redundancy based on edge-connectivity used to increase accuracy and
          reduce the number of reviews needed.
        \end{enumln}
            %We have proposed an algorithm that will use the learned pairwise match probabilities to perform
            %joint identification on a set of multiple annotations. In this way we will take advantage of
            %information from multiple images. This algorithm will facilitate addressing the problems of
            %\intraoccurrence{} matching, \vsexemplar{} matching, \exemplar{} selection, and split and merge
            %case detection.

    \end{itemln}

    \section{Future work}\label{sec:futurework}
    The clearest direction to continue this research is to develop and train 
    Clearly the most significant area that
    To extend the work 



    %However, each one of these algorithms stands in its own right, and while they do not depend on each other
    %  they can all benifit from one another.


    %\section{Discussion}\label{sec:discussresult} 

    %    Overall our experiments have shown that we are able to accurately rank image of plains zebras, Grévy's
    %      zebras, and Masai giraffes.
    %    Using this system we have successfully estimated the population of plains zebras and Masai giraffes in
    %      Nairobi National Park.
    %    We have demonstrated that using information from multiple images improves identification accuracy.
    %    We have also shown the importance and the difficulty of addressing the problem of decision/verification
    %      in the context of animal identification.
    %    Despite the successes of our system, it is not without its limitations.
    %    The following list enumerates several limitations of the current system.

    %    \begin{itemln}

    %        %\item The current limitations of the system are the
    %        %  requirement of manual verification of results and the
    %        %  inability to run robust consistency checks.

    %        \item \textbf{Scalability of the identification algorithm}:
    %        There may be a limitation in the scalability our nearest-neighbor based algorithms. However, this may
    %        be alleviated with the proper selection of exemplars. In the near future we do not see any reason to
    %        move away from nearest neighbor based techniques. However, farther down the line it may be necessary to
    %        replace or preprocess the current indexing algorithm using a vocabulary based approach.

    %        \item \textbf{Identification of animals without distinguishing textures}:
    %        The SIFT-based matching algorithms that we employ will only be able to match species containing
    %        distinctive gradient-based markings. New algorithms will need to be developed for animals with
    %        distinguishing shape based features. However, our system design is able to incorporate new algorithms
    %        --- such as the whale fluke matching algorithm --- under the same image analysis API{}.

    %        \item \textbf{Identification of highly deformable animals}:
    %        Our algorithms use projective transformations to enforce spatial constraints between matches. While
    %        this works well for the species we consider, it will be more difficult to match other species. Future
    %        work should seek to more robustly align annotations before comparing them. This might be done directly
    %        by warping annotations using non-linear alignment
    %          %(like the scale-cascaded alignment in SLOOP~\cite{duyck_sloop_2015})
    %        or via localizing and comparing smaller parts. Given a proper alignment, it should be possible to apply
    %        state-of-the-art neural network techniques to produce reliable image descriptions and similarities.
    %        %Our techniques will not reliable 
    %        %Current measures of pairwise probability are not based on
    %        %  state-of-the-art neural network techniques involving Siamese
    %        %  networks.
    %        %Our evaluation of these techniques yielded negative results due
    %        %  to insufficient training data and the lack of a robust
    %        %  algorithm to align the parts of animals.

    %        %It is my hope that the work done in this \thesis{} will allow
    %        %  for enough diverse training data to be bootstrapped to
    %        %  accommodate training data-hungry neural networks.
    %        %Future work should seriously consider using a parts-based model
    %        %  and computing matching similarity between parts using a Siamese
    %        %  network.
    %        %Techniques such as VLADNet %
    %        %%\ucite{vladnet}
    %        %can be used to index these parts to reduce the pairwise search
    %        %  space.
    %  \end{itemln}
        
    
    %\section{Summary of contributions}\label{sec:contributions}
    %    This section enumerates the current and intended contributions of this work. The intended contributions
    %    serve as a contract for the work needed to complete this thesis.
      
    %    \begin{itemln}

    %        \item \textbf{The single-image animal identification algorithm}:

    %            \begin{itemln}
    %                \item We have applied LNBNN~\cite{mccann_local_2012} to the problem of individual animal
    %                identification. We have performed experiments that demonstrate the effect of the $\K$ parameter
    %                at multiple database sizes. This is the main focus of our publication
    %                in~\cite{crall_hotspotter_2013}.

    %                \item We have accounted for the influence of background features using a learned a
    %                foregroundness measure to weight the LNBNN scores of feature correspondences. We have
    %                empirically shown the effectiveness of this procedure.

    %                \item We have introduced a \name{} scoring mechanism to take advantage of information in
    %                multiple database exemplars. Our experiments have demonstrated the significance of multiple
    %                exemplars for individual identification.

    %                \item 
    %                    We have introduced a heuristic to account for pose issues that augments the orientation of
    %                    query keypoints. We have evaluated the effect of various levels of feature invariance in
    %                    our experiments.
    %            \end{itemln}

    %       \item \textbf{The annotation-vs-annotation match probability}:
    %            We have proposed an algorithm to learn a pairwise match probability --- the probability that two
    %            annotations match, do not match, or are not comparable. Learning if two annotations are not
    %            comparable provides a measure of confidence about the probability that the annotations are either
    %            the same or different individuals. To learn this match probability we have proposed to use several
    %            local and global measures of similarity.
    %            %Learning this probability will enable our algorithm for
    %            %  choosing exemplars and our new graph-based identification
    %            %  algorithm.

    %            %This enables identification in databases of all sizes
    %            %  (\occurrences{}), the identification of potential merge and
    %            %  split cases, and an optimized choice of exemplars.
    %            %Work in discussion of match, no match, and not comparable.

    %       \item \textbf{The graph-based identification algorithm}:
    %            We have proposed an algorithm that will use the learned pairwise match probabilities to perform
    %            joint identification on a set of multiple annotations. In this way we will take advantage of
    %            information from multiple images. This algorithm will facilitate addressing the problems of
    %            \intraoccurrence{} matching, \vsexemplar{} matching, \exemplar{} selection, and split and merge
    %            case detection.

    %  \end{itemln}

    %\section{Future work}\label{sec:futurework}
    %    %
    %    This thesis is currently in the candidacy stage.
    %    %The work that will be done in the figure is outlined in
    %    %  \cref{chap:proposal}, which will be eventually be updated with a
    %    %  description and evaluation of the advanced techniques.
    %    In the future this section will be updated.

    %    It may be necessary to normalize the scores with respect to timedelta.

    %    Need to evaluate the scores between annotations of different viewpoints
    %    (show the bi-modality).

% L___ CHAPTER ___

