% +--- CHAPTER --- 
\begin{comment}
    ./texfix.py --fpaths chapter6-conclusion.tex --outline --asmarkdown --numlines=99 -w
\end{comment}

\chapter{Conclusion}\label{chap:conclusion} %
    %Outline:
    %\begin{itemln}
    %    \item Our graph-id framework is independent and complementary to DCNN matching approaches.
    %\end{itemln}

    In this \thesis{} we have addressed the problem of identifying individual animals from images.
    We have demonstrated that our approach is effective for identifying plains zebras, Grévy's zebras, Masai
      giraffes, and humpback whales.
    Our approach consists of three main components:
    (1) the ranking algorithm from \cref{chap:ranking} that uses a bounding box annotation around an animal to
      search a labeled database of annotations for likely matches.
    (2) the classification algorithm from \cref{chap:pairclf} that probabilistically verifies if a pair of
      annotation is positive, negative, or incomparable, and
    (3) the graph framework from \cref{chap:graphid} that harnesses the previous algorithms in a principled way
      to dynamically determine the identity of all animals in a dataset.
    Each of these algorithms was designed to improve upon the previous, and in \cref{sec:graphexpt} we
      demonstrated that this was indeed the case.

    By combining these algorithms we have made a several meaningful contributions to the problem of animal
      identification.
    In \cref{sec:introgzc} we discussed the Great Zebra Count (\GZC{}), where the ranking algorithm in
      combination with the effort of citizen scientists to provide an estimate of the number of plains zebras and
      Masai giraffes in Nairobi National Park.
    In \cref{sec:rankexpt} we investigated several parameters and factors that can impact the performance of the
      ranking algorithm.
    We discovered that having multiple photos of each individual significantly improves the accuracy of the
      ranking algorithm and we designed a novel name scoring mechanism with this in mind.
    In \cref{sec:pairexpt} we demonstrated that a classification algorithm can be used to improve the separation
      of positive from negative and incomparable ranked results.
    In \cref{sec:graphexpt} we simulated the \GZC{} and demonstrated that our improvements to the ranking
      algorithm --- made by the classification and graph algorithm --- enable us to perform identification in less
      than $25\percent$ of the time required by the original event.

    The research that resulted in \thesis{} began in $2010$ and was completed in $2017$.
    In that time many significant developments have been made in the field of computer vision and machine
      learning, namely the explosion of deep learning where technological and theoretical achievements have made it
      possible to efficiently train deep convolutional neural networks (DCNNs)~\cite{lecun_deep_2015}.
      %(DCNNs)~\cite{krizhevsky_imagenet_2012,taigman_deepface_2014,razavian_baseline_2015,lecun_deep_2015,arandjelovic_netvlad_2016,rocco_convolutional_2017}.
    Our approach does contain a small amount of deep learning (\eg{} the foregroundness weights), but it not
      based in it.
    In some sense this is an advantage because the algorithms can be applied to many different species without
      any need for pre-training, but it also does not achieve the incredible accuracy achievable by these networks.
    Yet, the contributions of this \thesis{} are still relevant and complementary to DCNNs.
    This is trivially true in the case of the ranking and classification algorithms, in part due to the
      aforementioned reasons.
    However, the contribution of the graph algorithm is still very relevant, even in the era of deep learning.
    
    %\section{Discussion of the graph algorithm}
    The graph identification algorithm models the abstract constraints of the identification problem and provides
      a framework that can efficiently harness the power of any ranking or verification algorithm, whether it be
      deep or shallow.
    The framework dynamically manages the relationships between annotations.
    In most cases this means deciding if two annotations are the same (positive) or different (negative), but
      this also means handling corner cases like when the annotations are incomparable or when there is some other
      interesting connection between two annotations like scenery matches and photobombs.
    As new relationships are added, errors are discovered and corrected, and the identifications are updated.

    The framework also provides a means of prioritizing which edges need to be reviewed based on the underlying
      computer vision algorithms, the edge-augmentation needed to ensure minimum redundancy, and the minimum cut
      needed to correct an error and split an inconsistent individual.
    Edge prioritization works in conjunction with a convergence criteria that determines when identification has
      been completed.
    When combined this means that the graph algorithm can simply be given a set of annotations and executed.
    A signal is emitted whenever manual interaction is needed, and once the response is received the algorithm
      continues.
    This means that the graph algorithm requires little expertise to use.
    Once it is running it effectively provides an identification wizard and simply guides the user through a set
      of simple questions.
    An obvious choice would be to run the graph algorithm on a web server and send questions to remote users that
      can be quickly done in a web browser.
    The algorithm stops once there is a high probability that the vast majority of identifications have been made
      correctly and consistently.

    %\section{Discussion of the ranking and verification algorithm}
    %pass
    %\section{Contributions}\label{sec:contributions}

    A summary of the contributions made in this \thesis{} are as follows:

    \begin{enumln}
    \item {The ranking algorithm}:
        \begin{enumln}
        \item We have adapted LNBNN~\cite{mccann_local_2012} to the problem of individual animal identification.
        We have performed experiments that demonstrate the effect of the $\K$ parameter at multiple database
          sizes.

        \item We have accounted for the influence of background features using a learned a foregroundness measure
          to weight the LNBNN scores of feature correspondences.
        We have empirically shown the effectiveness of this procedure.

        \item We have introduced a \name{} scoring mechanism to take advantage of information in multiple
          database exemplars.
        Our experiments have demonstrated the significance of multiple exemplars for individual identification.

        \item We have evaluated the effect of various levels of feature invariance in our experiments.
        We have introduced a heuristic that augments the orientation of query keypoints to account for pose
          variations.
        \end{enumln}

    \item {The pairwise classification algorithm}:
        \begin{enumln}

        \item We have developed a novel feature vector that represents the matching information between two
          annotations.

        \item We have used these feature vectors to learn a random forest that can predict the probability that
          two annotations are either positive, negative, or incomparable.

        \item We have compared the learned probabilities to LNBNN scores and demonstrated that the separation of
          positive cases is significantly improved.
        \end{enumln}

            %We have proposed an algorithm to learn a pairwise match probability --- the probability that two
            %annotations match, do not match, or are not comparable. Learning if two annotations are not
            %comparable provides a measure of confidence about the probability that the annotations are either
            %the same or different individuals. To learn this match probability we have proposed to use several
            %local and global measures of similarity.
            %Learning this probability will enable our algorithm for
            %  choosing exemplars and our new graph-based identification
            %  algorithm.

            %This enables identification in databases of all sizes
            %  (\occurrences{}), the identification of potential merge and
            %  split cases, and an optimized choice of exemplars.
            %Work in discussion of match, no match, and not comparable.

    \item {The graph identification algorithm}:
        \begin{enumln}
        \item We have introduced the graph identification algorithm for semi-automatic animal identification.

        \item We have demonstrated that the algorithm can use the ranking and verification algorithms to improve
          the accuracy and efficiency of identification.

        \item The framework is agnostic to the specific ranking and verification algorithms.
        The algorithms discussed in this paper could be replaced with algorithms based on DCNNs.
        The framework can even be used without any computer vision algorithms in order to facilitate a more
          efficient brute force search for small databases.

        \item We have developed a probabilistic termination criteria that determines when to stop identification.

        \item We have developed a measure of redundancy based on edge-connectivity used to increase accuracy and
          reduce the number of reviews needed.

        \item We have developed an algorithm for fixing errors whenever inconsistencies in the graph are been
          discovered.
        \end{enumln}
    \end{enumln}

    \section{Future work}\label{sec:futurework}

    We have shown that our ranking and match-state classification algorithms are both accurate and work well for
      identifying animals.
    However, the clearest direction for future research is to replace these algorithms with ones based on DCNNs.
    We believe that the approach in is a good starting point for~\cite{arandjelovic_netvlad_2016} replacing the
      ranking algorithm.
    %This algorithm is similar to the VLAD approach we tested here, the deep features and vocabularies .
    We had briefly investigated the techniques in~\cite{taigman_deepface_2014} to the pairwise classification
      problem, but because we did not have an alignment procedure or as much training data, the results were poor.
    Research into the geometric matching technique described in~\cite{rocco_convolutional_2017} may help address
      both of these these issues.

    There are also improvements that can be made to the graph algorithm.
    First it would be useful to parallelize the algorithm so reviews could be distributed across multiple users.
    This can be obtained by popping multiple edges from the queue at a time, but this could add extraneous
      redundancy if the subsequent edges would have been filtered.
    Second, the current prioritization of edges is based completely on the output of the pairwise classifier.
    In the best case, the ordering would first construct each PCC as a chain and then only $1$ redundant review
      would be needed.
    In the worst case, this order would connect one annotation of an individual to all others causing a star
      shaped PCC.
    Then to make the PCC $2$-positive-redundant, it would take $n - 2$ reviews where $n$ is the number of
      annotations in the PCC.
    Determining the best order in which to review edges depending on the specified level of redundancy is an
      interesting question, which is perhaps made more challenging if considered in a distributed setting.

% L___ CHAPTER ___

