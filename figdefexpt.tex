%%% Experiment results

%
\begin{comment}
# Summarize Commands
python -m ibeis.scripts.gen_cand_expts --exec-parse_latex_comments_for_commmands --fname figdefexpt.tex
./regen_figdefexpt.sh --noshow
gvim ~/code/ibeis/regen_figdefexpt.sh
\end{comment}


% -------------------------------
% --- Baseline Experiments ---
% -------------------------------


\begin{comment}                                                                                                                                       
\end{comment}
                                                                                                                                                      
\begin{comment}                                                                                                                                       
python -m ibeis.scripts.thesis Chap3.draw_agg_baseline --diskshow

python -m ibeis.scripts.thesis Chap3.draw --dbs=GZ_Master1,PZ_Master1,GIRM_Master1
python -m ibeis.scripts.thesis Chap3.draw --db GZ_Master1
python -m ibeis.scripts.thesis Chap3.draw --db PZ_Master1
python -m ibeis.scripts.thesis Chap3.draw --db GIRM_Master1
\end{comment}

\newcommand{\BaselineExpt}{
    \begin{figure}[ht!]\centering
        \begin{subfigure}[h]{\textwidth}\centering\includegraphics[width=\textwidth]{figuresY/agg-baseline.png}\end{subfigure}
        \caption[\caplbl{BaselineExpt}Baseline experiment]{\caplbl{BaselineExpt}
    % ---
    The baseline experiment is a high-level indicator of the ranking accuracy of each species.
    We measure ranking accuracy using a single query and database annotation --- selected from different
      encounters --- per individual.
    The number of query annotations and size of the database are given for each species in the legend.
    % ---
        }
        \label{fig:BaselineExpt}
    \end{figure}
}

\begin{comment}                                                                                                                                       
python -m ibeis.scripts.thesis Chap3.draw --dbs=GZ_Master1,PZ_Master1
\end{comment}                                                                                                                                       

\begin{comment}
python -m ibeis.scripts.thesis Chap3.measure_single --expt=foregroundness --dbs=GZ_Master1,PZ_Master1
python -m ibeis.scripts.thesis Chap3.draw_single --expt=foregroundness --dbs=GZ_Master1,PZ_Master1 --diskshow
\end{comment}


\newcommand{\SMKExpt}{
    \begin{figure}[ht!]\centering
        \begin{subfigure}[h]{\textwidth}\centering\includegraphics[width=\textwidth]{figuresY/PZ_Master1/smk.png}\caption{plains zebras}\label{sub:SMKExptA}\end{subfigure}
        \begin{subfigure}[h]{\textwidth}\centering\includegraphics[width=\textwidth]{figuresY/GZ_Master1/smk.png}\caption{Grévy's zebras}\label{sub:SMKExptB}\end{subfigure}
    \caption[\caplbl{SMKExpt}SMK experiment]{\caplbl{SMKExpt}
    % ---
    The (VLAD based) SMK algorithm compared to our LNBNN ranking algorithm.
    The results demonstrate that LNBNN outperforms the ranking accuracy of SMK.
    % ---
        }
        \label{fig:SMKExpt}
    \end{figure}
}


\newcommand{\ForegroundExpt}{
    \begin{figure}[ht!]\centering
        \begin{subfigure}[h]{\textwidth}\centering\includegraphics[width=\textwidth]{figuresY/PZ_Master1/foregroundness.png}\caption{plains zebras}\label{sub:ForegroundExptA}\end{subfigure}
        \begin{subfigure}[h]{\textwidth}\centering\includegraphics[width=\textwidth]{figuresY/GZ_Master1/foregroundness.png}\caption{Grévy's zebras}\label{sub:ForegroundExptB}\end{subfigure}
        \caption[\caplbl{ForegroundExpt}Foregroundness experiment]{\caplbl{ForegroundExpt}
            % ---
            Weighting the score of the feature correspondences using foregroundness results in more accurate
              identifications.
            Results for plains zebras are shown in~\cref{sub:ForegroundExptA}, Grévy's zebras are shown
              in~\cref{sub:ForegroundExptB}, and Masai giraffes are excluded from this test.
            % ---
        }
        \label{fig:ForegroundExpt}
    \end{figure}
}

% -------------------------------
% --- Invariance Experiments ----
% -------------------------------


\newcommand{\InvarExpt}{
    \begin{figure}[ht!]\centering
        \begin{subfigure}[h]{.8\textwidth}\centering\includegraphics[width=\textwidth]{figuresY/PZ_Master1/invar.png}\caption{plains zebras}\label{sub:InvarExptA}\end{subfigure}
        \begin{subfigure}[h]{.8\textwidth}\centering\includegraphics[width=\textwidth]{figuresY/GZ_Master1/invar.png}\caption{Grévy's zebras}\label{sub:InvarExptB}\end{subfigure}
        \caption[\caplbl{InvarViewExpt}Feature invariance experiment]{\caplbl{InvarExpt}
            % ---
            Results of the feature invariance experiment, testing the effect of affine invariance (AI) and the query-side rotation heuristic (QRH).
            For plains zebras circular keypoints with the QRH are the most accurate. 
            For Grévy's zebras enabling affine invariance works the best.
            % ---
        }
        \label{fig:InvarExpt}
    \end{figure}
}

% -------------------------------
% --- K Experiments
% -------------------------------

% --- Explicit K 


%\MultiImageCommandII{KExpt}{1}{Results of the $\K$ experiment}{
%    % ---
%    Identification accuracy using different values of $\K$ (the number of
%      nearest neighbors assigned to each query feature).
%    \Cref{sub:KExptA} illustrates that the most accurate setting for the less
%      distinctive plains zebras is $\K\tighteq7$ when considering the \names{}
%      ranked first.
%    If the top two ranked \names{} are considered, then $\K\tighteq4$ provides
%      the most accurate results.
%    \Cref{sub:KExptB} illustrates that the most accurate setting for the more
%      distinctive Grévy's zebras is $\K\tighteq1$ when considering the \names{}
%      ranked first, and $\K\tighteq2$ when considering names ranked first or
%      second.
%    \Cref{sub:KExptC} illustrates that both $\K\tighteq1$ and $\K\tighteq2$
%      provide the most accurate results for Masai giraffes.
%%    % ---
%}{figuresX/expt_PZKTime.png}{figuresX/expt_GZKTime.png}{figuresX/expt_GIRMKTime.png}


% --- Database Size Experiments (with time)


\newcommand{\KExpt}{
    \begin{figure}[ht!]\centering
        \begin{subfigure}[h]{\textwidth}\centering\includegraphics[width=\textwidth]{figuresY/PZ_Master1/kexpt.png}\caption{plains zebras}\end{subfigure}
        \begin{subfigure}[h]{\textwidth}\centering\includegraphics[width=\textwidth]{figuresY/GZ_Master1/kexpt.png}\caption{Grévy's zebras}\end{subfigure}
        \caption[\caplbl{KExpt}The $K$ experiment]{\caplbl{KExpt}
            % ---
            Identification accuracy using different values of $\K$ (the number of
              nearest neighbors assigned to each query feature).
            %Note that the scores reported here are higher than the baseline for
            %  the same reasons as explained in~\cref{fig:DBSizeExpt}.
            % ---
        }
        \label{fig:KExpt}
    \end{figure}
}

%\MultiImageCommandII{DBSizeExpt}{1}{Results of the database size experiment}{
%    % ---
%    Accuracy of rank $1$ identifications as a function of $\K$ with
%      different database sizes.
%    The value of $\K$ with the maximum accuracy is denoted in the
%      legend.
%    An asterisk ``*'' denotes that multiple values have this score.
%    While $\K$ does have an impact on matching accuracy, the
%      \emph{number of annotations per name} is the more important factor.
%    \Cref{sub:DBSizeExptA} shows the results for plains zebras.
%    \Cref{sub:DBSizeExptB} shows the results for Grévy's zebras.
%    \Cref{sub:DBSizeExptC} shows the results for Masai giraffes.
%    Note that the overall accuracy of this test is higher than the
%      baseline because this test requires multiple annotations per
%      \name{}.
%    This results in higher scores because of bias due to a smaller test
%      size and the groundtruth bias (\ie{} the identification algorithm
%      has already worked well on these \names{} with multiple
%      annotations).
%    % ---
%}{figuresX/expt_PZDBSizeTime.png}{figuresX/expt_GZDBSizeTime.png}{figuresX/expt_GIRMDBSizeTime.png}


% -------------------------------
% --- Namescore Experiments ----
% -------------------------------

\begin{comment}
python -m ibeis.scripts.thesis Chap3.measure_single --expt=nsum --dbs=GZ_Master1
python -m ibeis.scripts.thesis Chap3.measure_single --expt=nsum --dbs=GZ_Master1,PZ_Master1
python -m ibeis.scripts.thesis Chap3.draw_single --expt=nsum --dbs=GZ_Master1,PZ_Master1 --diskshow
\end{comment}

\newcommand{\NScoreExpt}{
    \begin{figure}[ht!]\centering
        \begin{subfigure}[h]{\textwidth}\centering\includegraphics[width=\textwidth]{figuresY/PZ_Master1/nsum.png}\caption{plains zebras}\label{sub:NScoreExptA}\end{subfigure}
        \begin{subfigure}[h]{\textwidth}\centering\includegraphics[width=\textwidth]{figuresY/GZ_Master1/nsum.png}\caption{Grévy's zebras}\label{sub:NScoreExptB}\end{subfigure}
        \caption[\caplbl{InvarViewExpt}Name scoring experiment]{\caplbl{NScoreExpt}
            % ---
            Results of the name scoring mechanism experiment.
            There is a clear separation between identification accuracy when the number of exemplars per name is
              $1$ compared to when it is $3$.
            Feature based name scoring (\nsum{}) is slightly more accurate than scoring using the annotation
              based name scoring (\csum{}).
            %Note that the scores reported here are higher than the baseline for
            %  the same reasons as explained in~\cref{fig:DBSizeExpt}.
            % ---
        }
        \label{fig:NScoreExpt}
    \end{figure}
}


% -------------------------------
% --- Score Separability Experiments ----
% -------------------------------


% --- all cases
\begin{comment}
python -m ibeis -e scores --db PZ_Master1   -a timectrl -t best --filt : --hargv=scores  --prefix "Separability " --label PZScoreAll  
python -m ibeis -e scores --db PZ_Master1   -a timectrl -t best --filt :without_tag=photobomb --hargv=scores  --prefix "Separability " --label PZScoreAll  
\end{comment}

\begin{comment}
python -m ibeis -e scores --db GZ_Master1   -a timectrl -t best     --filt : --hargv=scores --prefix "Separability "  --label GZScoreAll
python -m ibeis -e scores --db GZ_Master1   -a timectrl -t best     --filt :without_tag=photobomb --hargv=scores --prefix "Separability "  --label GZScoreAll
python -m ibeis -e rank_cmc --db GZ_Master1   -a timectrl -t best best:chip_sqrt_area=700 --filt : --show
\end{comment}

\begin{comment}
python -m ibeis -e scores --db GIRM_Master1 -a timectrl1h -t best   --filt : --hargv=scores --prefix "Separability "  --label GIRMScoreAll
\end{comment}
\MultiImageCommandII{ScoreSep}{1}{
    The score separability for each species 
}{
    % ---
    The score separability for the best time controlled configuration of each
      species.
    \Cref{sub:ScoreSepA} shows the results for plains zebras.
    \Cref{sub:ScoreSepB} shows the results for Grévy's zebras.
    % ---
}{figuresX/expt_PZScoreAll.png}{figuresX/expt_GZScoreAll.png}

% -------------------------------
% --- Tag Histograms  ---
% -------------------------------

\begin{comment}
python -m ibeis -e taghist --db PZ_Master1   -a timectrl -t best --filt :fail=True --no-wordcloud --hargv=tags  --prefix "Failure " --label PZTags  --figsize=10,3  --left=.2
\end{comment}

\begin{comment}
python -m ibeis -e taghist --db GZ_Master1   -a timectrl -t best     --filt :fail=True --no-wordcloud --hargv=tags --prefix "Failure "  --label GZTags  --figsize=10,3   --left=.2
\end{comment}

\begin{comment}
python -m ibeis -e taghist --db GIRM_Master1 -a timectrl1h -t best   --filt :fail=True --no-wordcloud --hargv=tags --prefix "Failure "  --label GIRMTags  --figsize=10,3   --left=.2
\end{comment}

\MultiImageCommandII{TagExpt}{1}{
    Primary causes of identification failure
}{
    % ---
    A histogram indicating the primary causes of identification failures.
    Occlusion and viewpoint seem to be the primary causes of failure.
    \Cref{sub:TagExptA} shows the failure case histogram for plains zebras.
    \Cref{sub:TagExptB} shows the failure case histogram for Grévy's zebras.
    % ---
}{figuresX/expt_PZTags.png}{figuresX/expt_GZTags.png}


\begin{comment}

# Categorize and tag errors

python -m ibeis -e cases --db PZ_Master1   -a timectrl   -t best --filt :sortdsc=gfscore,fail=None,with_tag=BadTail --show
python -m ibeis -e cases --db PZ_Master1   -a timectrl   -t best --filt :sortdsc=gfscore,fail=None --show
python -m ibeis -e cases --db GZ_Master1   -a timectrl   -t best --filt :sortdsc=gfscore,fail=True --show
python -m ibeis -e cases --db GIRM_Master1 -a timectrl1h -t best --filt :sortdsc=gfscore,fail=None --show

python -m ibeis -e cases --db GIRM_Master1 -a timectrl    -t best --show  --filt :sortdsc=gfscore,fail=True
python -m ibeis -e cases --db GIRM_Master1 -a timectrl1h  -t best --show  --filt :sortdsc=gfscore,fail=True
python -m ibeis -e cases --db GIRM_Master1 -a timectrl    -t best --show  --filt :sortdsc=gfscore,fail=True

python -m ibeis -e cases --db GIRM_Master1 -a viewdiff  -t Ell --show  --filt :orderby=gfscore,reverse=1,fail=True

# Find untagged cases
python -m ibeis.dev -e cases --db PZ_Master1  -a timectrl   -t best --filt :sortdsc=gtscore,fail=True,max_tags=0 --show
python -m ibeis.dev -e cases --db GZ_Master1  -a timectrl   -t best --filt :sortdsc=gtscore,fail=True,max_tags=0 --show
python -m ibeis.dev -e cases --db GIRM_Master1  -a timectrl   -t best --filt :sortdsc=gtscore,fail=True,max_tags=0 --show

# Specialized untagged cases
python -m ibeis.dev -e cases --db PZ_Master1  -a timectrl   -t best --filt :sortdsc=gfscore,fail=True,min_gtscore=.0001 --show
python -m ibeis.dev -e cases --db PZ_Master1  -a timectrl   -t best --filt :sortdsc=gfscore,fail=True,max_gf_tags=0,max_gt_tags=0 --show
python -m ibeis.dev -e cases --db PZ_Master1  -a timectrl   -t best --filt :sortdsc=gfscore,fail=True,min_gtscore=.0001,max_gf_tags=0 --show

python -m ibeis.dev -e cases --db GZ_Master1  -a timectrl   -t best --filt :sortdsc=gfscore,fail=None,max_gf_tags=0,max_gt_tags=0 --show
python -m ibeis.dev -e cases --db GZ_Master1  -a timectrl   -t best --filt :sortasc=gtscore,fail=None,max_gf_tags=0,max_gt_tags=0 --show
python -m ibeis.dev -e cases --db GZ_Master1  -a timectrl   -t best --filt :sortdsc=gfscore,fail=True,max_gf_tags=0,max_gt_tags=0 --show
python -m ibeis.dev -e cases --db GZ_Master1  -a timectrl   -t best --filt :sortdsc=gfscore,fail=True,max_gt_tags=0 --show


# Find Pair Cases without Corresponding Single Tag
python -m ibeis.dev -e cases --db PZ_Master1  -a timectrl   -t best --filt :sortdsc=gfscore,fail=True,with_tag=Viewpoint,max_gtq_tags=0 --show
python -m ibeis.dev -e cases --db GZ_Master1  -a timectrl   -t best --filt :sortdsc=gfscore,fail=True,with_tag=Viewpoint,max_gtq_tags=0 --show
python -m ibeis.dev -e cases --db GIRM_Master1  -a timectrl   -t best --filt :sortdsc=gfscore,fail=True,with_tag=Viewpoint,max_gtq_tags=0 --show

python -m ibeis.dev -e cases --db PZ_Master1  -a timectrl   -t best --filt :sortdsc=gfscore,fail=True,with_tag=Quality,max_gtq_tags=0 --show
python -m ibeis.dev -e cases --db GZ_Master1  -a timectrl   -t best --filt :sortdsc=gfscore,fail=True,with_tag=Quality,max_gtq_tags=0 --show
python -m ibeis.dev -e cases --db GIRM_Master1  -a timectrl   -t best --filt :sortdsc=gfscore,fail=True,with_tag=Quality,max_gtq_tags=0 --show
\end{comment}
