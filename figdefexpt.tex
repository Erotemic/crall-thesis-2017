%%% Experiment results

%
\begin{comment}
# Summarize Commands
python -m ibeis.scripts.gen_cand_expts --exec-parse_latex_comments_for_commmands --fname figdefexpt.tex
./regen_figdefexpt.sh --noshow
gvim ~/code/ibeis/regen_figdefexpt.sh
\end{comment}


% -------------------------------
% --- Baseline Experiments ---
% -------------------------------

\begin{comment}
python -m ibeis -e rank_cmc --db PZ_Master1   -a ctrl timectrl   -t baseline --prefix "Baseline" --label PZBaseline --cmap-hack=autumn    --hargv=expt
\end{comment}                                                                                                                                         
                                                                                                                                                      
\begin{comment}                                                                                                                                       
python -m ibeis -e rank_cmc --db GZ_Master1   -a ctrl timectrl   -t baseline --prefix "Baseline" --label GZBaseline --cmap-hack=autumn    --hargv=expt
\end{comment}                                                                                                                                         
                                                                                                                                                      
\begin{comment}                                                                                                                                       
python -m ibeis -e rank_cmc --db GIRM_Master1 -a ctrl timectrl1h -t baseline --prefix "Baseline" --label GIRMBaseline --cmap-hack=autumn  --hargv=expt
\end{comment}


\begin{comment}                                                                                                                                       
python -m ibeis -e rank_cmc --db GIRM_Master1 -a timectrl1h -t baseline --prefix "Baseline" --label GIRMBaseline --cmap-hack=autumn  --save "{label}.png"
python -m ibeis -e rank_cmc --db humpbacks_fb -a timectrl1h -t baseline --prefix "Baseline" --label HBBaseline --cmap-hack=autumn --save "{label}.png"
python -m ibeis -e rank_cmc --db Seals -a timectrl1h -t baseline --prefix "Baseline" --label HBBaseline --cmap-hack=autumn --save "{label}.png"
python -m ibeis -e rank_cmc --db snails_drop1 -a timectrl1h -t baseline --prefix "Baseline" --label HBBaseline --cmap-hack=autumn --save "{label}.png"
python -m ibeis -e rank_cmc --db LF_Bajo_bonito -a timectrl1h -t baseline --prefix "Baseline" --label HBBaseline --cmap-hack=autumn --save "{label}.png"
python -m ibeis -e rank_cmc --db GIRM_Master1 -a ctrl -t baseline --prefix "Baseline" --label GIRMBaseline --cmap-hack=autumn  --save "{label}.png"
python -m ibeis -e rank_cmc --db humpbacks_fb -a ctrl -t baseline --prefix "Baseline" --label HBBaseline --cmap-hack=autumn --save "{label}.png"
python -m ibeis -e rank_cmc --db Seals -a ctrl -t baseline --prefix "Baseline" --label HBBaseline --cmap-hack=autumn --save "{label}.png"
python -m ibeis -e rank_cmc --db snails_drop1 -a ctrl -t baseline --prefix "Baseline" --label HBBaseline --cmap-hack=autumn --save "{label}.png"
python -m ibeis -e rank_cmc --db LF_Bajo_bonito -a ctrl -t baseline --prefix "Baseline" --label HBBaseline --cmap-hack=autumn --save "{label}.png"
\end{comment}

\MultiImageCommandII{BaselineExpt}{1}{Baseline experiment identification accuracy}{
    % ---
    Baseline experiment identification accuracy.
    The results without controlling for time are deceptively better
      than the time controlled experiment.
    This is because the algorithm is able to count near duplicate
      matches as correct.
    Therefore, the time controlled experiment more accurately reflects
      the accuracy of the identification algorithm.
    % ---
    Results for plains zebras are shown in~\cref{sub:BaselineExptA}, Grevy's
      zebras are shown in~\cref{sub:BaselineExptB}, and Masai giraffes are
      shown in~\cref{sub:BaselineExptC}.
    % ---
}{figuresX/expt_PZBaseline.png}{figuresX/expt_GZBaseline.png}{figuresX/expt_GIRMBaseline.png}


% -----------------------------------
% --- Foregroundness Experiments ----
% -----------------------------------

\begin{comment}
python -m ibeis -e rank_cmc --db PZ_Master1   -a timectrl   -t baseline:fg_on=[True,False]  --hargv=expt --prefix "Foregroundness" --label PZForeground  --cmap-hack=spring
\end{comment}

\begin{comment}
python -m ibeis -e rank_cmc --db GZ_Master1   -a timectrl   -t baseline:fg_on=[True,False]  --hargv=expt --prefix "Foregroundness" --label GZForeground --cmap-hack=spring

python -m ibeis -e rank_cmc --db GZ_Master1   -a timectrl   -t baseline:fg_on=[True],fgw_thresh=[None,.8] --show
\end{comment}

%\begin{comment}
%python -m ibeis -e rank_cmc --db GIRM_Master1 -a timectrl1h -t baseline:fg_on=[True,False]  --hargv=expt --prefix "Foreground" --label GIRMForeground
%\end{comment}

\MultiImageCommandII{ForegroundExpt}{1}{
The effect of the foregroundness weight on identification accuracy}{
    % ---
    The effect of the foregroundness weight on identification accuracy.
    Weighting the score of the feature correspondences using foregroundness
      results in more accurate identifications.
    % ---
    Results for plains zebras are shown in~\cref{sub:ForegroundExptA},
      Grevy's zebras are shown in~\cref{sub:ForegroundExptB}, and Masai
      giraffes are excluded from this test.
    % ---
}{figuresX/expt_PZForeground.png}{figuresX/expt_GZForeground.png}


% -----------------------------------
% --- Feature Scoring Mechanism Experiments ----
% -----------------------------------

%\begin{comment}
%python -m ibeis -e rank_cmc --db PZ_Master1   -a timectrl   -t featscoremetch  --hargv=expt --prefix "LNBNN" --label PZLNBNN
%\end{comment}

%\begin{comment}
%python -m ibeis -e rank_cmc --db GZ_Master1   -a timectrl   -t best:fg_on=[True,False]  --hargv=expt --prefix "Foreground" --label GZForeground
%\end{comment}

%\begin{comment}
%python -m ibeis -e rank_cmc --db GIRM_Master1 -a timectrl1h -t best:fg_on=[True,False]  --hargv=expt --prefix "Foreground" --label GIRMForeground
%\end{comment}

% -------------------------------
% --- Invariance Experiments ----
% -------------------------------

% --- Same Viewpoint

\begin{comment}
python -m ibeis -e rank_cmc --db PZ_Master1   -a timectrl   -t invar  --hargv=expt --prefix "Invariance" --label PZInvar --cmap-hack=Set1
\end{comment}

\begin{comment}
python -m ibeis -e rank_cmc --db GZ_Master1   -a timectrl   -t invar  --hargv=expt --prefix "Invariance" --label GZInvar --cmap-hack=Set1
\end{comment}

\begin{comment}
python -m ibeis -e rank_cmc --db GIRM_Master1 -a timectrl1h -t invar  --hargv=expt --prefix "Invariance" --label GIRMInvar --cmap-hack=Set1
\end{comment}

\MultiImageCommandII{InvarExpt}{1}{
Results of the same-viewpoint invariance experiment}{
    % ---
    Results of the invariance experiment when the viewpoints are the
      same.
    \Cref{sub:InvarExptA} illustrates that the most accurate feature
      invariance setting for plains zebras is circular keypoints with the
      query-side rotation heuristic.
    \Cref{sub:InvarExptB} illustrates that the most accurate feature
      invariance setting for Grevy's zebras is affine invariance without
      rotation invariance (with the gravity vector).
    \Cref{sub:InvarExptC} illustrates the same result for Masai giraffes.
    %that most accurate feature
    %  invariance setting for Masai giraffes is affine invariance with the
    %  query-side rotation heuristic.
    % ---
}{figuresX/expt_PZInvar.png}{figuresX/expt_GZInvar.png}{figuresX/expt_GIRMInvar.png}


% --- Across Viewpoints
\begin{comment}
python -m ibeis -e rank_cmc --db PZ_Master1   -a viewdiff   -t invar  --hargv=expt --prefix "Invariance+View"  --label PZInvarView  --cmap-hack=Set2
\end{comment}

\begin{comment}
python -m ibeis -e rank_cmc --db GZ_Master1   -a viewdiff   -t invar  --hargv=expt --prefix "Invariance+View"  --label GZInvarView --cmap-hack=Set2
\end{comment}

\begin{comment}
python -m ibeis -e rank_cmc --db GIRM_Master1 -a viewdiff   -t invar  --hargv=expt --prefix "Invariance+View"  --label GIRMInvarView --cmap-hack=Set2
\end{comment}
\MultiImageCommandII{InvarViewExpt}{1}{
Results of the different-viewpoint invariance experiment}{
    % ---
    Results of the invariance experiment when the viewpoints are different.
    Matching between viewpoints is overall less accurate than matching
      annotations of the same viewpoint.
    This test does not control for time.
    \Cref{sub:InvarViewExptA} illustrates that affine invariant keypoints provide
      the most accurate identifications of plains zebras when viewpoint are
      different.
    This is different than the most accurate feature invariance setting when
      matching between the same viewpoints.
    \Cref{sub:InvarViewExptB} illustrates that affine invariant keypoints provide
      the most accurate identifications of Grevy's zebras when viewpoint are
      different.
    This is the same as the most accurate feature invariance setting when
      matching between the same viewpoints.
    \Cref{sub:InvarViewExptC} illustrates that all feature settings score equally
      for Masai giraffes.
    This is likely due to the small number of query annotations (14) and
      database annotations (27).
    % ---
}{figuresX/expt_PZInvarView.png}{figuresX/expt_GZInvarView.png}{figuresX/expt_GIRMInvarView.png}

% -------------------------------
% --- K Experiments
% -------------------------------

% --- Explicit K 
\begin{comment}
python -m ibeis -e rank_cmc --db PZ_Master1   -a timectrl   -t CircQRH_K --hargv=expt --prefix "K " --label PZKTime  --cmap-hack=gnuplot
\end{comment}
\begin{comment}

python -m ibeis -e rank_cmc --db GZ_Master1   -a timectrl   -t Ell_K     --hargv=expt --prefix "K " --label GZKTime --cmap-hack=gnuplot
\end{comment}

\begin{comment}
python -m ibeis -e rank_cmc --db GIRM_Master1 -a timectrl1h -t Ell_K     --hargv=expt --prefix "K " --label GIRMKTime --cmap-hack=gnuplot
\end{comment}


\MultiImageCommandII{KExpt}{1}{Results of the $\K$ experiment}{
    % ---
    Identification accuracy using different values of $\K$ (the number of
      nearest neighbors assigned to each query feature).
    \Cref{sub:KExptA} illustrates that the most accurate setting for the less
      distinctive plains zebras is $\K\tighteq7$ when considering the \names{}
      ranked first.
    If the top two ranked \names{} are considered, then $\K\tighteq4$ provides
      the most accurate results.
    \Cref{sub:KExptB} illustrates that the most accurate setting for the more
      distinctive Grevy's zebras is $\K\tighteq1$ when considering the \names{}
      ranked first, and $\K\tighteq2$ when considering names ranked first or
      second.
    \Cref{sub:KExptC} illustrates that both $\K\tighteq1$ and $\K\tighteq2$
      provide the most accurate results for Masai giraffes.
    % ---
}{figuresX/expt_PZKTime.png}{figuresX/expt_GZKTime.png}{figuresX/expt_GIRMKTime.png}


% --- Database Size Experiments (with time)

\begin{comment}
ibeis -e rank_surface --db PZ_Master1   -a varysize_td   -t CircQRH_K --hargv=surf --prefix "Database Size + Time" --label PZDBSizeTime
\end{comment}

\begin{comment}
ibeis -e rank_surface --db GZ_Master1   -a varysize_td   -t Ell_K     --hargv=surf --prefix "Database Size + Time" --label GZDBSizeTime
\end{comment}

\begin{comment}
ibeis -e rank_surface --db GIRM_Master1 -a varysize_td1h:qmin_pername=3,dpername=[1,2] -t Ell_K   --hargv=surf --prefix "Database Size + Time" --label GIRMDBSizeTime
\end{comment}

\MultiImageCommandII{DBSizeExpt}{1}{Results of the database size experiment}{
    % ---
    Accuracy of rank $1$ identifications as a function of $\K$ with
      different database sizes.
    The value of $\K$ with the maximum accuracy is denoted in the
      legend.
    An asterisk ``*'' denotes that multiple values have this score.
    While $\K$ does have an impact on matching accuracy, the
      \emph{number of annotations per name} is the more important factor.
    \Cref{sub:DBSizeExptA} shows the results for plains zebras.
    \Cref{sub:DBSizeExptB} shows the results for Grevy's zebras.
    \Cref{sub:DBSizeExptC} shows the results for Masai giraffes.
    Note that the overall accuracy of this test is higher than the
      baseline because this test requires multiple annotations per
      \name{}.
    This results in higher scores because of bias due to a smaller test
      size and the groundtruth bias (\ie{} the identification algorithm
      has already worked well on these \names{} with multiple
      annotations).
    % ---
}{figuresX/expt_PZDBSizeTime.png}{figuresX/expt_GZDBSizeTime.png}{figuresX/expt_GIRMDBSizeTime.png}


% -------------------------------
% --- Namescore Experiments ----
% -------------------------------

\begin{comment}
python -m ibeis -e rank_cmc --db PZ_Master1   -a varypername_td   -t CircQRH_ScoreMech:K=3   --hargv=mech --prefix "Score Method" --label PZNscore  --ncolor-hack=6 --cmap-hack=jet
\end{comment}

\begin{comment}
python -m ibeis -e rank_cmc --db GZ_Master1   -a varypername_td   -t Ell_ScoreMech:K=1     --hargv=mech --prefix "Score Method" --label GZNscore --ncolor-hack=6 --cmap-hack=jet
\end{comment}

\begin{comment}
python -m ibeis -e rank_cmc --db GIRM_Master1 -a varypername_td1h:qmin_pername=3,dpername=[1,2] -t Ell_ScoreMech:K=2     --hargv=mech --prefix "Score Method" --label GIRMNscore --ncolor-hack=6 --cmap-hack=jet
\end{comment}


\MultiImageCommandII{NScoreExpt}{1}{
    Results of the \namescoring{} experiment
}{
    % ---
    Results of the \namescoring{} mechanism experiment.
    There is a clear separation between identification accuracy when
      the number of exemplars per name is $1$ compared to when it is $3$.
    Robust \namescoring{} (\nsum{}) is slightly more accurate than
      baseline \namescoring{} (\csum{}).
    \Cref{sub:NScoreExptA} shows the results for plains zebras.
    \Cref{sub:NScoreExptB} shows the results for Grevy's zebras.
    \Cref{sub:NScoreExptC} shows the results for Masai giraffes.
    Note that the scores reported here are higher than the baseline for
      the same reasons as explained in~\cref{fig:DBSizeExpt}.
    % ---
}{figuresX/expt_PZNscore.png}{figuresX/expt_GZNscore.png}{figuresX/expt_GIRMNscore.png}


% -------------------------------
% --- Score Separability Experiments ----
% -------------------------------


% --- all cases
\begin{comment}
python -m ibeis -e scores --db PZ_Master1   -a timectrl -t best --filt : --hargv=scores  --prefix "Separability " --label PZScoreAll  
python -m ibeis -e scores --db PZ_Master1   -a timectrl -t best --filt :without_tag=photobomb --hargv=scores  --prefix "Separability " --label PZScoreAll  
\end{comment}

\begin{comment}
python -m ibeis -e scores --db GZ_Master1   -a timectrl -t best     --filt : --hargv=scores --prefix "Separability "  --label GZScoreAll
python -m ibeis -e scores --db GZ_Master1   -a timectrl -t best     --filt :without_tag=photobomb --hargv=scores --prefix "Separability "  --label GZScoreAll
python -m ibeis -e rank_cmc --db GZ_Master1   -a timectrl -t best best:chip_sqrt_area=700 --filt : --show
\end{comment}

\begin{comment}
python -m ibeis -e scores --db GIRM_Master1 -a timectrl1h -t best   --filt : --hargv=scores --prefix "Separability "  --label GIRMScoreAll
\end{comment}
\MultiImageCommandII{ScoreSep}{1}{
    The score separability for each species 
}{
    % ---
    The score separability for the best time controlled configuration of each
      species.
    \Cref{sub:ScoreSepA} shows the results for plains zebras.
    \Cref{sub:ScoreSepB} shows the results for Grevy's zebras.
    \Cref{sub:ScoreSepC} shows the results for Masai giraffes.
    % ---
}{figuresX/expt_PZScoreAll.png}{figuresX/expt_GZScoreAll.png}{figuresX/expt_GIRMScoreAll.png}

% -------------------------------
% --- Tag Histograms  ---
% -------------------------------

\begin{comment}
python -m ibeis -e taghist --db PZ_Master1   -a timectrl -t best --filt :fail=True --no-wordcloud --hargv=tags  --prefix "Failure " --label PZTags  --figsize=10,3  --left=.2
\end{comment}

\begin{comment}
python -m ibeis -e taghist --db GZ_Master1   -a timectrl -t best     --filt :fail=True --no-wordcloud --hargv=tags --prefix "Failure "  --label GZTags  --figsize=10,3   --left=.2
\end{comment}

\begin{comment}
python -m ibeis -e taghist --db GIRM_Master1 -a timectrl1h -t best   --filt :fail=True --no-wordcloud --hargv=tags --prefix "Failure "  --label GIRMTags  --figsize=10,3   --left=.2
\end{comment}

\MultiImageCommandII{TagExpt}{1}{
    Primary causes of identification failure
}{
    % ---
    A histogram indicating the primary causes of identification failures.
    Occlusion and viewpoint seem to be the primary causes of failure.
    \Cref{sub:TagExptA} shows the failure case histogram for plains zebras.
    \Cref{sub:TagExptB} shows the failure case histogram for Grevy's zebras.
    \Cref{sub:TagExptC} shows the failure case histogram for Masai giraffes.
    % ---
}{figuresX/expt_PZTags.png}{figuresX/expt_GZTags.png}{figuresX/expt_GIRMTags.png}


\begin{comment}

# Categorize and tag errors

python -m ibeis -e cases --db PZ_Master1   -a timectrl   -t best --filt :sortdsc=gfscore,fail=None,with_tag=BadTail --show
python -m ibeis -e cases --db PZ_Master1   -a timectrl   -t best --filt :sortdsc=gfscore,fail=None --show
python -m ibeis -e cases --db GZ_Master1   -a timectrl   -t best --filt :sortdsc=gfscore,fail=True --show
python -m ibeis -e cases --db GIRM_Master1 -a timectrl1h -t best --filt :sortdsc=gfscore,fail=None --show

python -m ibeis -e cases --db GIRM_Master1 -a timectrl    -t best --show  --filt :sortdsc=gfscore,fail=True
python -m ibeis -e cases --db GIRM_Master1 -a timectrl1h  -t best --show  --filt :sortdsc=gfscore,fail=True
python -m ibeis -e cases --db GIRM_Master1 -a timectrl    -t best --show  --filt :sortdsc=gfscore,fail=True

python -m ibeis -e cases --db GIRM_Master1 -a viewdiff  -t Ell --show  --filt :orderby=gfscore,reverse=1,fail=True

# Find untagged cases
python -m ibeis.dev -e cases --db PZ_Master1  -a timectrl   -t best --filt :sortdsc=gtscore,fail=True,max_tags=0 --show
python -m ibeis.dev -e cases --db GZ_Master1  -a timectrl   -t best --filt :sortdsc=gtscore,fail=True,max_tags=0 --show
python -m ibeis.dev -e cases --db GIRM_Master1  -a timectrl   -t best --filt :sortdsc=gtscore,fail=True,max_tags=0 --show

# Specialized untagged cases
python -m ibeis.dev -e cases --db PZ_Master1  -a timectrl   -t best --filt :sortdsc=gfscore,fail=True,min_gtscore=.0001 --show
python -m ibeis.dev -e cases --db PZ_Master1  -a timectrl   -t best --filt :sortdsc=gfscore,fail=True,max_gf_tags=0,max_gt_tags=0 --show
python -m ibeis.dev -e cases --db PZ_Master1  -a timectrl   -t best --filt :sortdsc=gfscore,fail=True,min_gtscore=.0001,max_gf_tags=0 --show

python -m ibeis.dev -e cases --db GZ_Master1  -a timectrl   -t best --filt :sortdsc=gfscore,fail=None,max_gf_tags=0,max_gt_tags=0 --show
python -m ibeis.dev -e cases --db GZ_Master1  -a timectrl   -t best --filt :sortasc=gtscore,fail=None,max_gf_tags=0,max_gt_tags=0 --show
python -m ibeis.dev -e cases --db GZ_Master1  -a timectrl   -t best --filt :sortdsc=gfscore,fail=True,max_gf_tags=0,max_gt_tags=0 --show
python -m ibeis.dev -e cases --db GZ_Master1  -a timectrl   -t best --filt :sortdsc=gfscore,fail=True,max_gt_tags=0 --show


# Find Pair Cases without Corresponding Single Tag
python -m ibeis.dev -e cases --db PZ_Master1  -a timectrl   -t best --filt :sortdsc=gfscore,fail=True,with_tag=Viewpoint,max_gtq_tags=0 --show
python -m ibeis.dev -e cases --db GZ_Master1  -a timectrl   -t best --filt :sortdsc=gfscore,fail=True,with_tag=Viewpoint,max_gtq_tags=0 --show
python -m ibeis.dev -e cases --db GIRM_Master1  -a timectrl   -t best --filt :sortdsc=gfscore,fail=True,with_tag=Viewpoint,max_gtq_tags=0 --show

python -m ibeis.dev -e cases --db PZ_Master1  -a timectrl   -t best --filt :sortdsc=gfscore,fail=True,with_tag=Quality,max_gtq_tags=0 --show
python -m ibeis.dev -e cases --db GZ_Master1  -a timectrl   -t best --filt :sortdsc=gfscore,fail=True,with_tag=Quality,max_gtq_tags=0 --show
python -m ibeis.dev -e cases --db GIRM_Master1  -a timectrl   -t best --filt :sortdsc=gfscore,fail=True,with_tag=Quality,max_gtq_tags=0 --show
\end{comment}
