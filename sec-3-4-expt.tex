
\section{Experiments}\label{sec:experiments}

    This section presents an experimental evaluation of the
      identification algorithm using annotated images of plains zebras,
      Grevy's zebras, and Masai giraffes.
    %Results of each experiment are reported separately for each species.
    The input to each experiment is
    (1) a dataset,
    (2) a subset of query and database annotations from the database,
    (3) a pipeline configuration.
    % Datasets
    The datasets are described in~\cref{sub:datasets}.
    % Annot Config
    %All query and database annotations are labeled with groundtruth \names,
    %  which is used to measure the accuracy of the identification algorithm.
    The subsets of query and database annotations are carefully chosen
      to measure the accuracy of the algorithm under different conditions
      and to control for time, quality, and viewpoint.
    %Most annotation are also labeled with a viewpoint, quality, and timestamp.
    %Dataset groundtruth is used to evaluate the accuracy of the ranked lists
    %  and then the results of the experiment are presented.
    % Pipe Config
    The pipeline configuration is a set of parameters --- such as the
      level of feature invariance, the number of nearest neighbors, and
      the \namescoring{} mechanism --- given to the identification
      algorithm.
    In this section we will run experiments that vary several of these
      pipeline parameters in order to measure their effect on
      identification accuracy.

    % ALGORITHM ACCURACY PRIMARY MEASURE 
    For each query annotation, the identification algorithm returns a
      ranked list of \names{} with a score for each name.
    The accuracy of identification is measured in two ways:
    database ranking and score separability.
    The first and primary measure of identification accuracy is the
      percentage of queries in the returned list with the \groundtrue{}
      name ranked $1\st$.
    % ALGORITHM ACCURACY SECONDARY MEASURE 
    In some experiments the percentage of queries with their
      \groundtrue{} result ranked $2\nd$, $3\rd$, $4\th$, and $5\th$ is
      also presented.
    These auxiliary ranking measures are also valuable because users
      will often review more than just the top ranked results, but a user
      will rarely review past the $5\th$ rank.
    The second measure of identification accuracy is the returned
      score's separability --- \ie{} how well separated the scores of
      \groundtrue{} and \groundfalse{} matches are.
    This will help to determine when it is possible to choose correct
      matches without a manual decision, further automating the
      identification process.
    %apply a binary classifier to results to
    %  further automate the identification process.
    The separability of scores is measured by first recording the score
      of the \groundtrue{} \name{} and the highest scoring \groundfalse{}
      \name{} for each query and then plotting the receiver operator
      characteristic (ROC) curve and measuring the area under the curve
      (AUC).
    %Other measures that are reported include the percentage of true positive
    %  and false positive results that fall within certain categories (\eg
    %  temporal windows).

    The outline of this section is as follows.
    First, \Cref{sub:datasets} describes the datasets and how they were
      generated.
    Then, \Cref{sub:exptbase} establishes the baseline performance of
      the algorithm using the default pipeline configuration and a subset
      of the data that controls for time, viewpoint, number of exemplars,
      and quality.
    \Cref{sub:exptfeatmatchscore} tests the effect of the
      foregroundness weight on identification accuracy.
    \Cref{sub:exptinvar} investigates the effect of the level of
      feature invariance and viewpoint.
    \Cref{sub:exptscoremech} compares the \csumprefix{} and the
      \nsumprefix{} \namescoring{} mechanism.
    \Cref{sub:exptk} varies the $\K$ parameter (the number of nearest
      neighbors used in establishing feature correspondences) and
      investigates the relationship between $\K$ and database size in
      terms of both the number of annotations and the number of exemplars
      per name.
    \Cref{sub:exptfail} discusses the failure cases of the algorithm.
    \Cref{sub:exptsep} presents an evaluation of the score separability
      for the pipeline configuration with the highest accuracy determined
      for each species.
    Finally,~\cref{sub:exptsum} summarizes this section.


    \subsection{Datasets}\label{sub:datasets}

        All of the images in the datasets used in these experiments
          were taken by photographers in the field.
        Each dataset is labeled with groundtruth in the form of
          annotations with name labels.
        Annotations (bounding boxes) have been drawn to localize
          animals within the image.
        A unique \name{} label has been assigned to all annotations
          with the same identity.
        Some of this groundtruth labeling was generated independently.
        However, large portions of the datasets were labeled with
          assistance from the matching algorithm.
        While this may introduce some bias in the results, there was no
          alternative because the amount of time needed to independently
          label a large dataset is prohibitive.
        %Note that this may introduce bias in the results.

        Note that this section describes the complete dataset of
          annotations that are used throughout the experiments.
        To control for challenging factors in the images such as
          quality and viewpoint many experiments use only a subset of
          this complete dataset.
        Also note that there do exist labeling errors in some datasets.

        \DatabaseInfo{}

        \timedist{}

        The number of names, annotations, and their distribution within
          each database are summarized in the following tables.
        In these tables the term \glossterm{singleton} refers to
          \names{} that only contain one annotation, and
          \glossterm{multiton} refers to \names{} that have more than one
          annotation.
        We make this distinction between singletons and multitons
          because multitons are names that have known groundtruth
          matches.
        \Cref{tbl:DatabaseStatistics} summarizes the number of
          annotations per database.
        \Cref{tbl:AnnotationsPerNameMultiton} summarizes the number of
          annotations for multiton \names{}.
        \Cref{tbl:AnnotationsPerQuality} summarizes the quality labels
          of the annotations.
        \Cref{tbl:AnnotationsPerViewpoint} summarizes the viewpoint
          labels of the annotations.
        The name and a short description of each dataset is given in
          the following list.

        \begin{itemize}
            \item \textbf{\pzmasterI{}} is an aggregated dataset of
              plains zebras.
            There is variation in how the data was collected and
              preprocessed.
            Some of the images are cropped to the flank of the animal,
              while others are cropped to encompass the entire body.
            The datasets contributing to \pzmasterI{} were collected in
              Kenya at several locations including Nairobi National Park,
              Sweetwaters, and Ol Pejeta.
            More than $90\percent$ of the groundtruth generated for
              this dataset was assisted using the matching algorithm.
            This dataset contains many imaging challenges including
              occlusion, viewpoint, pose, quality, and time variation.
            There are some annotations in this dataset without quality
              or viewpoint labelings and some images contain undetected
              animals.
            This data was collected between 2006 and 2015, but the
              majority of the data was collected in 2012--2015.
            The distribution of collection times is shown
              in~\cref{sub:timedistA}.

            \item \textbf{\gzall{}} is an aggregated dataset of Grevy's
              zebras.
            The original groundtruth for this dataset was generated
              independently of the matching algorithm, however the
              matching algorithm revealed several groundtruth errors that
              have since been corrected.
            The Grevy's dataset was collected in Mpala, Kenya.
            Most of the annotations in this database have been cropped
              to the animal's flank.
            This dataset contains a moderate degree of pose and
              viewpoint variation as well as occlusion.
            This data was collected between 2003 and 2012, but the
              majority was collected in 2011 and 2012.
            The distribution of collection times is shown
              in~\cref{sub:timedistB}.

            \item \textbf{\girmmasterI{}} is a dataset of Masai giraffes.
            The images were all taken in Nairobi National Park.
            All groundtruth was established using the matching
              algorithm followed by manual verification.
            This dataset contains a high degree of pose and viewpoint
              variation, as well as occlusion.
            There are also many \glossterm{photobombs} --- instances
              where there is more than one giraffe in an annotation.
            This data was collected in the \GZC{} between February 20,
              2015 and March 2, 2015.
            The distribution of collection times is shown
              in~\cref{sub:timedistC}.
        \end{itemize}

    \subsection{Baseline experiment}\label{sub:exptbase}

        This first experiment determines the accuracy of the
          identification algorithm using the baseline pipeline
          configuration.
        The baseline pipeline configuration uses affine invariant
          features that assume the gravity vector, $\K\tighteq4$ as the
          number of feature correspondences assigned to each query
          feature, and \nscoring{} (\nsum{}).
        In this test we control for several biases that may be
          introduced by carefully selecting a subset of our datasets.
        We only use annotations that
        (1) are known (\ie{} have been assigned a name),
        (2) are assigned the species primary viewpoint (left for plains
          zebras and Masai giraffes and right for Grevy's zebras),
        (3) have not been assigned a quality of ``junk'' or ``poor''.
        Furthermore, to account for the fact that some \names{} contain
          more annotations than others, we constrain our data selection
          such that there is only one \groundtrue{} \exemplar{} in the
          database for each query annotation.

        We test two configurations of databases.
        The first is denoted as \ctrl{} and uses only the
          aforementioned constraints.
        The second is denoted as \timectrl{} and we further restrict
          the annotation selection by enforcing that each query and its
          corresponding \groundtrue{} \exemplar{} are separated in time
          by at least $6$ hours.
        This allows us control for correct results that may be due to
          incidentally identifying a near duplicate image.
        For the smaller Masai giraffes database we relax this
          constraint to $1$ hour.
        %\devcomment{Reword}
        %We choose these constraints to account for cases that may cause
        %  us to overestimate a population.
        %We are not guaranteed that there is be more than one picture of
        %  an animal in a database.
        %Therefore it is important to analyze the identification
        %  accuracy and failure cases when there is only one image of an
        %  animal in the database.
        %Furthermore, it is important for the identification algorithm
        %  to be able to identify an animal across large time deltas.

        The results are presented in the form of a cumulative rank
          histogram in~\cref{fig:BaselineExpt}.
        This plot indicates the accuracy of the algorithm when the top
          $1-5$ ranks of each query are returned to the user.
        The $x$-axis indicates the ranks considered for review.
        The $y$-axis shows the percentage of queries with a
          \groundtrue{} match ranked $x$ or less.
        %, where $x$
        %is a coordinate on the $x$-axis indicating a ranking.
        The most important part of this graph is the percentage of
          queries with a correct match at rank $x\tighteq1$.
        Each colored bar represents a different parameter configuration
          and is labeled in the legend.

        \BaselineExpt{}

        %In all experiments the accuracy of the algorithm decreased when time
        %  was controlled for.
        % PERCENT DIFFERENCE THAT TIME MAKES
        \begin{comment}
        python -m ibeis -e print --db PZ_Master1 -a ctrl timectrl -t baseline  
        python -m ibeis -e print --db GZ_Master1 -a ctrl timectrl -t baseline  
        python -m ibeis -e print --db GIRM_Master1 -a ctrl timectrl1h -t baseline  
        \end{comment}
        The effect of time is most notable for plains zebras where
          there is a $~10\percent$ decrease in accuracy between \ctrl{}
          and \timectrl{}.
        This is because a significant number of the correct matches in
          \ctrl{} were between near-duplicate images.
        The drop for Grevy's zebras and Masai giraffes is less
          pronounced.
        This is likely due to the higher density of distinctive
          patterns on Grevy's zebras and Masai giraffes as well as a lack
          of near duplicate images in those datasets.
        The difference in accuracy for plains zebras shows that
          near-duplicate matching has a significant impact on
          identification accuracy.
        Therefore, to avoid this issue, the remainder of our
          experiments use \timectrl{} as the baseline for selecting query
          and database annotations.

    \subsection{Foregroundness experiment}\label{sub:exptfeatmatchscore}

    \ForegroundExpt{}

        In this experiment we test the effect of foregroundness ---
          weighting the score of each features correspondence with a
          foregroundness weight --- on identification accuracy.
        Two pipeline configurations are tested in this experiment.
        In the first we score each feature correspondences using only
          the LNBNN~\cite{mccann_local_2012} score.
        In the second we weight the LNBNN score using a foregroundness
          measure learned using a deep convolutional neural
          network~\cite{parham_photographic_2015}.
        Currently, we have only trained a foregroundness measure for
          plains zebras and Grevy's zebras.
        Therefore Masai giraffes are excluded from this test.
        The accuracy of the foregroundness is shown
          in~\cref{fig:ForegroundExpt}

        For plains zebras, using the foregroundness measure results in
          a significant $3.79\percent$ increase in identification
          accuracy.
        For Grevy's zebras there is also a significant  $3.3\percent$
          increase.
        This experiment clearly shows the importance of eliminating
          background feature correspondences.
     
    \subsection{Invariance experiment}\label{sub:exptinvar}  
        In this experiment we vary the feature invariance
          configuration.
        Because adding feature invariance is meant to detect and
          describe the same features across multiple viewing positions,
          we also evaluate the different invariance settings both
        (1) when the viewpoint between each query and its \groundtrue{}
          exemplar are the same as well as
        (2) when the viewpoint is varied.
        In the first case we use the same \timectrl{} configuration as
          the baseline experiment.
        In the second case we constrain each database annotation's
          viewpoint to be the primary viewpoint of the species and each
          query annotation's viewpoint to be adjacent to the primary
          viewpoint.
        \Eg{} for plains zebras each query has a frontleft viewpoint,
          and each database annotation has a left viewpoint.
        Unfortunately, there are not many \names{} in our databases
          that contain both a primary and non-primary viewpoint
          annotations.
        Therefore, to increase the size of the dataset we do not
          control for time in the viewpoint varied experiment.
        The following is a list describing the invariance settings that
          we investigate in each experiment.

        \begin{itemize}

            \item \NoInvar{} (\pvar{AI=F,QRH=F,RI=F}): % 
            In this configuration the gravity vector is assumed and the
              shape of each detected feature is not adapted.

            \item \AIAlone{} (\pvar{AI=T,QRH=F,RI=F}): % 
            This is the baseline setting that assumes the gravity
              vector and where each feature's shape is skewed from a
              circle into an ellipse.

            \item \RIAlone{} (\pvar{AI=F,QRH=F,RI=T}): % 
            Here, each feature is assigned one or more dominant
              gradient orientations (the gravity vector is not used) and
              the shape is not adapted.

            \item Query-side rotation heuristic (\QRHCirc{})
              (\pvar{AI=F,QRH=T,RI=F}): %
            This is a novel invariance heuristic where each {database}
              feature assumes the gravity vector, but {query} feature is
              $3$ orientations:
            the gravity vector and two other orientations at
              $\pm15\degrees$ from the gravity vector.
            Ideally, this will allow feature correspondences to be
              established between features seen from slightly different
              orientations.

            \item \QRHEll{} (\pvar{AI=T,QRH=T,RI=F}): %
                This is the combination of \QRHCirc{} and \AIAlone{}.

            \item \AIRI{} (\pvar{AI=T,QRH=F,RI=T}): %
                This is the combination of \RIAlone{} and \AIAlone{}.

        \end{itemize}

        % Invar Conclusions
        The identification accuracy of the viewpoint and time
          controlled invariance experiment is shown
          in~\cref{fig:InvarExpt}.
        We find that full rotation invariance provides the poorest
          results for all datasets.
        % Invar Conclusions (same view)
        For plains zebras, \QRHCirc{} scores significantly ahead of all
          other invariance settings.
        %
        The results for Gravy's zebras show that \AIAlone{} is the most
          accurate invariance setting, but \QRHEll{} performs almost as
          well.
        %
        There is not enough data to draw definitive conclusions about
          Masai giraffes, but the small sample that we experiment with
          shows that \AIAlone{} and \QRHEll{} perform similarly to
          Grevy's zebras.

        \InvarExpt{}

        % Invar Conclusions (diff view)
        The identification accuracy of the viewpoint varied invariance
          is shown in~\cref{fig:InvarViewExpt}.
        There is only a small amount of data available to run this
          experiment, specifically there are $53$ query annotations
          available in \pzmasterI{}, $29$ in \gzall{}, and $14$ in
          \girmmasterI{}.
        For plain zebras \QRHCirc{} and \NoInvar{} are tied for the
          highest accuracy, but \QRHEll{} is the most accurate if ranks
          greater than $1$ are considered.
        For Grevy's zebras \AIAlone{} provides the highest accuracy.
        The \AIAlone{} and \QRHEll{} configurations are tied for
          highest accuracy in the Masai giraffes experiment.
        Overall, it seems that configurations with affine invariance
          perform the best on viewpoint varied cases, which agrees with
          intuition.
        However, these results show that the algorithm is overall less
          accurate when matching across viewpoints and that matching
          between different viewpoints of an animal is significantly more
          difficult than matching between the same views.
        The claim is further supported by the small amount of data
          available to perform this test.
        Most of the groundtruth used in these experiments was created
          with assistance from this identification algorithm, therefore
          if the identification algorithm does not reliably match between
          viewpoints then that will be reflected by a lack of viewpoint
          varied groundtruth.
        However, in creating this groundtruth the focus was on
          identifying primary views of the animal, so it is not clear how
          significant this effect is.
        The effect of failure cases due to viewpoint is further
          discussed in~\cref{sub:exptfail}.

        \InvarViewExpt{}

        \kptstype{}

        % MAKE SURE THAT BEST SETTINGS DISCCSSED HERE REFLECTS THE FIGURES

        % General conclusions
        The results in this experiment support the claim that the best
          choice of invariance settings is data dependent.
        The baseline invariance parameter \AIAlone{} produces the most
          accurate identification of  Grevy's zebras and Masai giraffes.
        % FIXME: which actually performs better?
        However, \QRHCirc{} performs better for plains zebras.
        This is likely because affine keypoints tend to describe only
          one or two coarse stripes on plains zebras.
        In contrast, distinctive details on Grevy's zebras and Masai
          giraffes are finer and well captured by affine keypoints.
        Even though affine keypoints provide more precise localization,
          the area they describe is often smaller than a circular
          keypoint.
        It makes sense that affine keypoints would not describe coarse
          features, like those seen on plains zebras, as well as a
          circular keypoint covering a larger area.
        This difference between \AIAlone{} and  \QRHEll{} features for
          plains and Grevy's zebras is illustrated
          in~\cref{fig:kptstype}.
        %Our experiments also show that providing a small amount of
          %orientation invariance is beneficial for matching coarse plains
          %zebra patterns.
        %% WORDING
        %This may be because species that exhibit a large number of
          %small and distinctive patterns are less sensitive to pose
          %variations.
        %For example, if an Grevy's zebra moves its front leg, the
          %orientation of the features in its leg will change, but the
          %change will also diffuse to the surround area like the
          %shoulder.
        %But because the features on a Grevy's zebra are small and
          %numerous, it is likely that there will be distinctive shoulder
          %features that do not intersect with the area affected by this
          %change.
        %In contrast, consider if a species with larger and less
          %numerous features, like a plains zebra, moves its front leg.
        %Because the distinctive feature are larger, it is more likely
          %that some part of a distinctive shoulder feature will intersect
          %with the area of change.
        %A small amount of rotation invariance allows for some of these
          %patterns to be matched.
        For the remainder of our experiments we use \QRHCirc{} as the
          invariance setting for plains zebras and the baseline of
          \AIAlone{} on Grevy's zebras and Masai giraffes.

    \subsection{Scoring mechanism experiment}\label{sub:exptscoremech}  

        % Database setup for name scoring
        The purpose of the scoring mechanism is to aggregate scores of
          individual feature correspondences across multiple annotations
          into a single score for each name.
        The experiments in this subsection tests the identification
          accuracy of two name scoring mechanisms:
        (1) \cscoring{} (\csum{}) and
        (2) \nscoring{} (\nsum{}).
        Because the scoring mechanism is meant to take advantage of
          multiple \exemplars{} per \name{} we vary the number of
          \exemplars{} per query between $1$, $2$, and $3$, except in the
          case of the \girmmasterI{} dataset we only vary the number of
          \exemplars{} between $1$ and $2$ (due to dataset size
          constraints).
        The accuracy of the scoring mechanism experiment is shown
          in~\cref{fig:NScoreExpt}

        %\GIRMNscore
        \NScoreExpt{}

        % Describe results
        The number of \exemplars{} per \name{} is the most significant
          factor in this test.
        All results indicate that \nsum{} is either as good or performs
          slightly better than \csum{}.
        It is interesting that in most of these results \nsum{} is
          exactly as good as \csum{}.
        We hypothesize that the reason for this is that the nearest
          neighbors of most query features tend to be from a single most
          visually similar exemplar in the database.
        This would cause a majority of the feature correspondences in
          \nsum{} to cast their vote for a single exemplar, giving
          results similar to \csum{}.
        %It is unlikely that some part of another exemplar would appear
        %  more similar to the query annotation than the most similar
        %  exemplar.
        The small gain seen by \nsum{} is likely from cases where part
          of the most similar exemplar is occluded or obscured, thus
          allowing some query features to be matched with features from
          different \exemplars{}.
        % Conclusions about scoring mechanism
        We continue to use \nsum{} as the scoring mechanism for the
          remainder of the experiments.
        

    \subsection{K experiment}\label{sub:exptk}  

        % Introduce varied parameters
        In this experiment we investigate the effect $\K$ (the number
          of nearest neighbors used in establishing feature
          correspondences) on identification accuracy.
        We vary $\K$ between the values $1, 2, 3, 4, 5, 7$, and $10$.
        In all of these experiments we set the number of normalizing
          neighbors to be $\Knorm=1$.
        We hypothesize that the optimal choice of $\K$ depends on the
          size of the database.
        This subsection will present two experiments:
        (1) an experiment to test the accuracy at different values of
          $\K$ on a large database, and
        (2) an experiment to test the accuracy at different values of
          $\K$ for many database sizes.
        Database size depends on both the number of \names{} in the
          database and the number of \exemplars{} per \name{}.
        Therefore, we vary both of these variables.
        We vary the total number of \exemplars{} in the database
          between $\frac{1}{4}$, $\frac{1}{2}$, and $\frac{3}{4}$ of
          total number of annotations available.
        The number of \exemplars{} per \name{} is varied between $1$,
          $2$ and $3$.
        Because of the small size of the \girmmasterI{} dataset we only
          vary the number of \exemplars{} between $1$ and $2$.

        The effect of $\K$ on matching accuracy using a static database
          size is shown in~\cref{fig:KExpt}.
        The results for plains zebras shows little difference in
          accuracy between different values of $\K$.
        For Grevy's zebras there appears to be a negative relationship
          between $\K$ and accuracy.
        This may be because the first match of a highly distinctive
          pattern (like those seen in Grevy's zebras) will not be
          confused other \names{}.
        % giraffes 
        Setting $\K\tighteq4$ results in the highest accuracy of the
          Masai giraffe dataset.

        \KExpt{}
        
        Our second test varies the size of the database as well as the
          value of $\K$.
        The effect of $\K$ and database size on matching accuracy is
          shown in~\cref{fig:DBSizeExpt}.
        The results for all species show that the number of
          \exemplars{} per \name{} is the most important factor in this
          experiment.
        Interestingly, the number of annotations in the database is
          only a minor factor in identification accuracy.
        % plains
        The results for plains zebras show a small positive
          relationship between the number of annotations in the database
          and $\K$.
        This may be because many plains zebra features are not globally
          distinctive in a large database and a feature's correct
          correspondence may not be the nearest neighbor.
        For smaller database sizes lower values of $\K$ produce more
          accurate results.
        % grevys
        For Grevy's zebras lower values of $\K$ seem better for all
          database sizes.
        It also seems that the best values of $\K$ should be set to the
          number of \exemplars{} per \name{}, which is expected when
          there is little confusion between features.
        % giraffes
        For Masai giraffes, the amount of data again makes it difficult
          to draw conclusions.

        \DBSizeExpt{}

        Overall the experiments on the setting of $\K$ does not yield
          definitive choice for this parameter.
        However, it appears that $\K$ only has a small influence on
          identification accuracy.
        This section does shows that the number of exemplars per
          annotation has a significant impact on identification accuracy.
        %For the remainder of our experiments we will use $\K\tighteq3$
        %  for plains zebras, $\K\tighteq1$ for Grevy's zebras, and
        %  $\K\tighteq3$ for Masai giraffes.

    \subsection{Failure cases}\label{sub:exptfail}  
        
        In this subsection we investigate the primary causes of
          identification failure by considering individual failure cases.
        When investigating the cause of a failure case we consider two
          matches from the query annotation to a \name{}:
        (1) the match from the query annotation to the \groundfalse{}
          \name{} at rank $1$, and
        (2) the match from the query annotation to the \groundtrue{}
          \name{}.

        We manually label the \groundtrue{} match and the
          \groundfalse{} match in each failure case to indicate the
          factors that heuristically appear to cause identification to
          fail.
        We accumulate the frequency of these cases into a histogram to
          illustrate the significance of each type of failure case.
        The failure case histograms are shown in~\cref{fig:TagExpt}.

        \TagExpt{}

        The following list gives a definition for each type of failure
          case and an example.
        The first three types of failure cases denote type 2 errors
          (false negatives) where the \groundtrue{} match fails to
          produce a high score.
        The last three types of failure cases denote type $1$ errors
          (false positive) where the \groundfalse{} match produces a
          score that is too high.

        \begin{itemize}

            \item Viewpoint:
            This failure case denotes that there is a viewpoint
              difference between the query and its \groundtrue{} match.
            Viewpoint differences are among the most common causes of
              failure in all of the datasets.
            A viewpoint failure case is caused by an out-of-plane
              rotation between the query features and the \groundtrue{}
              database features.
            Out-of-plane rotations of a feature can cause significant
              difference in appearance and exacerbates errors in feature
              localization, causing inconsistency between feature
              descriptions.
            The differences in descriptors leaves the approximate
              nearest neighbor algorithm unable to establish the correct
              correspondence, which ultimately causes identification
              failure.

            Note that these viewpoint failures cases originate from a
              test that controls for viewpoint.
            We have found that about half of these cases are due to
              viewpoint mislabelings.
            The pairs in the other half have correct viewpoint labels,
              yet a mild viewpoint difference still causes the initial
              assignment of feature correspondences to fail.
            An example illustrating a failure case due to viewpoint is
              shown in~\cref{fig:FailViewpoint}.

            \FailViewpoint{}

            \item Occlusion:
            This label denotes that either the query or the \groundtrue{}
              annotation is occluded.
            Objects like grass, tree branches, and other animals can
              obscure a feature causing it to appear dissimilar or mask
              it entirely.
            This is a much more significant problem for species with
              relatively few distinctive features like plains zebras.
            An example illustrating a failure case due to occlusion is
              shown in~\cref{fig:FailOcclusion}

                \FailOcclusion{}

            \item Quality:
            This label denotes that either the query or the
              \groundtrue{} annotation is blurred or distorted.
            Some of these cases are due to an annotation having an
              ``ok'' quality label, when in fact it should be labeled
              ``poor'' or ``junk''.
            An annotation with low quality may generate fewer, larger,
              less distinct, and distorted features.
            This case occurs due to mislabeling of the image quality.
            An example illustrating a failure case due to quality is
              shown in~\cref{fig:FailQuality}.

                \FailQuality{}

            \item Lighting:
            This label denotes that either the query or the
              \groundtrue{} annotation is poorly illumination or
              shadowed.
            Feature extraction produces too few and unreliable features
              in under exposed images.
            Non-uniform illumination and shadowing produces noisy
              intensity gradients that interfere with feature
              description.
            Ideally, a classifier could be trained to determine which
              images are poorly illuminated and select an appropriate
              preprocessing step.
            An example illustrating a failure case due to lighting is
              shown in~\cref{fig:FailLighting}

                \FailLighting{}

            %\item Orientation - Either the query or the \groundtrue{}
            %    annotation has its orientation localized incorrectly.
            %    \FailOrientation

            %\item NonDistinct:
            %This label denotes that there are either numerous or high scoring
            %  feature correspondences between non-distinctive regions of the
            %  query and the \groundfalse{} annotation.
            %Features with scales larger or smaller than the typical scale of
            %  distinctive patterns on a species can cause incorrect
            %  correspondences.
            %An example illustrating a failure case due to non-distinct feature
            %  correspondences is shown in~\cref{fig:FailNonDistinct}.
            %However, the low frequency of failures cases due to
            %  non-distinctive matches shows that the LNBNN mechanism is properly
            %  downweighting these smaller features.
            %%Intuitively, these smaller features (between single stripes)
            %%  should appear frequently in the database be downweighted by the
            %%  LNBNN scoring mechanism.
            %%However, correspondences at larger scales may not appear
            %%  frequently enough to be properly downweighted.
            %%Another factor that can cause non-distinctive regions to receive
            %%  significant scores is when features are localized on a boundary
            %%  between two objects.
            %%The edge of the boundary becomes the primary gradient used for
            %%  feature description.
            %%This can cause incorrect, but distinctive correspondences between
            %%  two animals that appear in similar poses.

            %  \FailNonDistinct

            \item Scenery Match:
            This denotes a case where the algorithm produces
              correspondences between shared background features between
              a query annotation and \aan{\groundfalse{}} annotation.
            These cases are typically between pairs of annotations with
              small (less than $10$ minutes) time deltas.
            For plains and Grevy's zebras, these cases are mostly
              eliminated using a foreground weighting algorithm.
            However, for new species --- when a foregroundness measure
              has not been trained --- this will still be a problem.
            An example illustrating a failure case due to a scenery
              match is shown in~\cref{fig:FailScenery}

              \FailScenery{}

            \item \Photobomb{}:
            A case where the \groundtrue{} animal is seen in the
              foreground or background of \aan{\groundfalse{}}
              annotation.
            This case is not technically a false positive because the
              algorithm is correctly matching the same individual.
            However, this is a problem for identification because
              failure to detect a \photobomb{} will cause two different
              individuals to be incorrectly marked as the same \name{}.
            This could potentially start a cascading ``snowball''
              effect.
            An example illustrating a failure case due to
              \photobombing{} is shown in~\cref{fig:FailPhotobomb}

              \FailPhotobomb{}

            %\item SimilarPose - The query and the \groundfalse{} annotation are in the
            %    same pose. This causes incorrect but matches along the edge of
            %    the animal that encode the shape of the animal's position.
            %    \FailPose
            %    A pose failure case is shown in~\cref{fig:FailPose}.
        \end{itemize}


        The above failure cases show that the main causes of algorithm
          failure are due to viewpoint, occlusion, and quality.
        This identification algorithm itself does not seek to correctly
          identify low quality annotations, however the larger system
          should be able to flag and either fix (\eg{} by applying
          histogram equalization on a case-by-case basis to a poorly
          illuminated annotations) or remove such annotations.
        We approach the problem of viewpoint as a data issue.
        By adding more exemplars to the database we expect to improve
          matching accuracy between annotations with small viewpoint
          differences.
        Scenery matches and \photobombings{} also cause false
          positives, but the foregroundness weighting mostly eliminates
          scenery matches.
        To account for \photobombings{}, a classifier could be trained
          based on the timestamps between annotations, and the spatial
          distribution of feature correspondences within the annotations.
        This would also help to further reduce failures due to scenery
          matches.
        
        %To further reduce scenery mathces and \photobombings these cases could be flagged 
        %Scenery matches and \photobombings{} also cause a significant number
        %  of false positives, but these cases should be able to be flagged using
        %  a background detector, the timestamps between images, and the location
        %  of the feature correspondences.

        %A big failure case for the Grevy's may actually be size of the image. 
        %The features are not getting detected properly on many chips.
        %This may not be the issue.
        %Hmmm.

    \subsection{Score separability}\label{sub:exptsep}  
        %Show separability of scores under the best algorithm settings for only
        %  success cases and then with both success and failures.
        %Show only this histogram of scores and the ROC curve.
        %Report results for all species.

        In this subsection we investigate identification accuracy in
          terms of the scores returned along with each \name{} in the
          ranked list.
        This is in contrast to the results presented in previous
          experiments where accuracy is evaluated only in terms of
          ranking.
        It is important to look at the overall scores of the algorithm
          because in a deployment setting a query annotation may not have
          corresponding \groundtrue{} database annotation and the top
          ranked \name{} would always be incorrect.
        Ideally, the scores would be used to decide if each name in the
          ranked list is either \groundtrue{} or \groundfalse{}.
        However, if the scores are to be used in a decision mechanism,
          they must have a high degree of separability.

        %This requires the raw scores between \groundtrue{} and \groundfalse{}
        %  cases will be separable.
        %In this experiment investigate the 
        We run experiments to test the degree to which \groundtrue{}
          and \groundfalse{} scores can be separated by a binary
          classifier.
        We consider two types of scores:
        (1) \groundtrue{} scores --- the scores of between the query
          and its \groundtrue{} name (even if it is not ranked first),
          and
        (2) \groundfalse{} scores --- the scores between the query and
          the highest ranked \groundfalse{} name (the \groundfalse{} name
          is ranked $1$\st{} for failure cases and $2$\nd{} for success
          cases).
        The experiments in this subsection are run using the
          \timectrl{} annotation configuration and the best pipeline
          configuration for each species.
           
        Results are reported in the form of two plots:
        The first shows a histogram of the scores.
        The second plot shows an ROC curve where the true positive rate
          (sensitivity / recall) is plotted as a function of the false
          positive rate (fall-out).
        The area under the curve (AUC) is reported above the graph.
        The AUC is a standard measure used to evaluate a binary
          classifier.
        It represents the probability that a random \groundtrue{} name
          receives a higher score than a random \groundfalse{} name.
        The results of the separability experiment are shown
          in~\cref{fig:ScoreSep}.

        %The results of the separability experiment for plains zebras are shown
        %  in~\cref{fig:PZScoreAll}; Grevy's zebras are shown in
        % ~\cref{fig:GZScoreAll}; and Masai giraffes are shown in
        % ~\cref{fig:GIRMScoreAll}.

        \ScoreSep{}

        The results show that a threshold could be set to automatically
          accept high scoring names, if a small amount of false positives
          are acceptable.
        However, there is still a significant intersection between the
          \groundtrue{} and \groundfalse{} cases for plains and Grevy's
          zebras.
        For Masai giraffes the intersection is smaller, but there is
          also less data available.
        Ideally, we would like to find a threshold at which no false
          positives are accepted.

        In all datasets there does not exist any threshold able to
          automatically reject a true negative without causing false
          negatives (this is because some correct matches receive scores
          of zero).
        There are thresholds that can be set to automatically accept
          true positives without causing any false negatives.
        Unfortunately, the percentage of automatically accepted true
          positives is low.
        For plains zebras, a threshold of $6.1$ automatically accepts
          $\frac{3}{475} = 0.6\percent$ \groundtrue{} matches.
        For Grevy's zebras, a threshold of $2.55$ automatically accepts
          $\frac{15}{300} = 5.0\percent$ \groundtrue{} matches.
        For Masai giraffes, a threshold of $3.1$ automatically accepts
          $\frac{15}{35} = 42.8\percent$ \groundtrue{} matches.

        After manually labeling the failure cases we have found that
          the reason for this low acceptance rate is due to
          \photobombings{}.
        If failure cases due to \photobombings{} are ignored, there is
          significant improvement.
        For plains zebras, a threshold of $.86$ automatically accepts
          $\frac{180}{463} = 38.8\percent$ \groundtrue{} matches.
        For Grevy's zebras, a threshold of $2.55$ automatically accepts
          $\frac{35}{299} = 11.7\percent$ \groundtrue{} matches.
        For Masai giraffes, a threshold of $0.7$ automatically accepts
          $\frac{26}{35} = 76.4\percent$ \groundtrue{} matches.
        \Photobombing{} cases tend to produce the highest false
          positive matching score because they are technically correct
          matches --- from a feature perspective --- and are scored
          appropriately.
        Furthermore, \photobombings{} tend to occur be between images
          with small time-deltas which increases visual similarity and
          increases the scores of the feature matches.
        Building a classifier to detect \photobomb{} cases would be a
          simple way to significantly reduce the amount of manual
          verification needed.

        We must note an important caveat to developing a decision
          mechanism based on the LNBNN identification scores.
        The scores computed by the single image identification
          algorithm depend on all of the database annotations we match
          against.
        As the images in the database change, normalizing features used
          to compute the LNBNN scores may change as well.
        Furthermore, as the database size grows it is likely that fewer
          matches will be discovered.
        Therefore, any threshold set on these scores is only valid in
          the context of a specific static database.
        The problem of developing a dynamic decision mechanism is
          addressed in the next chapter.

    \subsubsection{Why is individual animal identification hard?}\label{sub:whyhard}
        %Setup paragraph.
        Even when ignoring \photobombings{}, the amount missed true
          positives and the number of manual verifications necessary is
          still unsatisfactory.
        It seems that there is an underlying difficulty in generating
          the initial feature matches in many cases.
        To illustrate this difficulty, consider the Liberty Buildings
          dataset~\cite{brown_discriminative_2011} commonly used in
          descriptor learning.
        This dataset contains a large number of corresponding patches
          from architectural structures computed using stereo matching.
        SIFT descriptors are computed for each patch, and the
          L2-distance between \groundtrue{} and a set of \groundfalse{}
          patch descriptors is computed.
        Pairs of \groundtrue{} and \groundfalse{} descriptors can be
          similarly computed for an animal dataset as the set of
          spatially verified foreground features from correct individuals
          and the set of feature matches between incorrect individuals.
        Examples of patches from both the Liberty dataset and the
          plains dataset are shown in~\cref{fig:PzVsLibertyPatches}.
        \Cref{fig:PzVsLiberty} compares the separability \groundtrue{}
          and \groundfalse{} \emph{patches} based on the L2-distance
          between SIFT descriptors from the Liberty dataset and the
          plains zebra dataset.

        There is a high degree of separability between the patches from
          the Liberty dataset (an ROC AUC of $0.96$) and a low degree of
          separability between patches from the plains zebra dataset (an
          ROC AUC of $0.724$).
        Consider the histogram of plains zebra scores
          in~\cref{fig:PzVsLiberty}.
        The incorrect matches also appear to be much closer in the
          plains dataset when compared to the buildings dataset.
        This is likely because are the most difficult incorrect matches
          in the dataset and are likely to be correspondences between
          non-distinctive descriptors.
        % Rexecuting this experiment with random false keypoint pairs would be interesting.
        The histogram of correct scores appears to be bimodal.
        This is evidence of two things:
        (1) there are significantly more non-distinctive correct
          matches than distinctive ones, and
        (2) there are matches incorrectly marked as correct.

        Even if the modes of the correct match histogram were
          separated, there are significantly fewer ``good'' correct
          matches for zebras than there are for buildings.
        In~\cref{subsec:dcnndiscuss} we noted a failed attempt to learn
          new convolutional descriptors to replace SIFT{}.
        This patch separability experiment shows the reason for the
          failure.
        The problem of learning descriptors to match animals seems to
          be more difficult than learning descriptors to match buildings.

        \PzVsLibertyPatches{}

        \PzVsLiberty{}

    \subsection{Experimental conclusions}\label{sub:exptsum}  

        In this section we have evaluated our baseline algorithm under
          restrictive conditions to control for the effects of time,
          quality, and viewpoint.
        Based on the results of these experiments we are able to make
          the following observations and conclusions.
        \begin{itemize}
            %% DATASET SAMPLING MATTERS
            %\item The baseline experiment shows that it is important to
            %  control for time, because near-duplicate images 
            %  the influence of near-duplicate images on matching
            %  accuracy.

            \item \textbf{Identification accuracy improves with more exemplars}:
            % NUM EXEMPLARS MATTERS
            The name scoring experiment and the $\K$ experiment show
              that the number of \exemplars{} per database \name{} is the
              most significant factor that impacts identification
              accuracy.

            \item \textbf{Foregroundness weighting reduce scenery matches}:
            % FOREGROUNDNESS GOOD
            Identification accuracy significantly improves by $2-4$
              percentage points when using foregroundness weighting.
            We have found that enabling foregroundness weighting
              eliminates nearly all failure cases due to scenery matches
              without significantly affecting other results.

            \item \textbf{Viewpoint and occlusion are the most difficult imaging challenges}:
            % VIEWPOINT HARD
            The viewpoint experiment and the failure cases show that
              there is a significant loss in accuracy when matching
              annotations from different viewpoints.
            Viewpoint seems to be the most difficult challenge across
              all species, however the failure cases show that occlusion
              is a more significant issue for plains zebras.
            This may be because Grevy's zebras and Masai giraffes have
              distinctive patterns in many places and thus can be matched
              when only part of the animal is visible.

            \item \textbf{Invariance settings are data dependent}:
                The invariance experiment shows
                % AFFINE GOOD FOR GZ CIRCLE GOOD FOR PZ
                affine invariance produces better results for Grevy's
                  zebras and Masai giraffes, whereas circular keypoints
                  lead to more accurate results for plains zebras.
                %This is likely because the circular keypoints are
                %  able to capture a larger portion of the coarse
                %  distinctive regions in plains zebras, whereas
                %  elliptical keypoints are better suited for smaller
                %  more distinctive textured patterns.
                % AQH GOOD FOR PZ
                This experiment also demonstrated that the query-side
                  rotation heuristic improves accuracy by adding a small
                  amount of orientation invariance to feature
                  localization, while using full orientation invariance
                  causes a drop in accuracy.
                %This means that it may not be feasible to use more than a
                %  single canonical viewpoint in the final population
                  %estimation.

              \item \textbf{The choice of \K{} has a minor impact}: 
                The $\K$ experiment shows that identification accuracy
                  is not significantly influenced by the choice of $\K$
                  for plains zebras, but for Grevy's zebras the most
                  accurate results were obtained with $\K\tighteq1$.
                This is likely because the features from plains zebras
                  are less distinguishing than features from Grevy's
                  zebras.
                Therefore, the correct match of a plains zebra feature
                  is less likely to be its closest neighbor.
                    %\item 
                        % CHOISE OF K DOES NOT MATTER TOO MUCH FOR LARGE DBS
                Furthermore, the size of the database does not seem to
                  strongly influence the optimal choice of $\K$.
                However, note that most tests were run on different
                  sizes of large databases, and the choice of $\K$ using
                  small databases was not investigated.
                %\end{itemize}

              \item \textbf{\Nsumprefix{} is slightly better than \csumprefix{} \namescoring{}}:
                % NSUM > CSUM
                The scoring mechanism experiment shows that the
                  \nsumprefix{} scoring mechanism is slightly more
                  accurate than the \csumprefix{} scoring mechanism.
                We hypothesize that this effect may be more significant
                  in conditions with more viewpoint variation.

              \item \textbf{LNBNN scores are not enough for automated decision}:
                % Separability
                The separability experiment shows that is a reasonable
                  separation between the scores of annotations seen from
                  the same viewpoint.
                Furthermore, if the effect of \photobombings{} can be
                  reduced it becomes possible to set an threshold to
                  automatically accept high scoring identification
                  results.
                However, there are still a significant number of
                  correct results with non-separable scores.
                Furthermore, it is still unclear to what degree the
                  scores depend on the database size.
                For these reasons we conclude that the decision
                  mechanism should be independent of the single image
                  identification algorithm.
                %More experimentation is needed before threshold levels can
                %  be confidently set for databases of arbitrary sizes.
        \end{itemize}


\section{Single query identification summary}\label{sec:staticsum}
    % NEED to turn this into a pure chapter summary

    In this section we have introduced an algorithm that ranks known
      database \names{} by their similarity to a single query annotation.
    This algorithm beings by extracting local patch-based features from
      cropped and normalized chips.
    Features from database annotations are indexed for fast nearest
      neighbor search using a kd-tree.
    An LNBNN-like mechanism is used to compute a matching score for
      each database annotation Based on these scores potential matches
      have their feature correspondences spatially verified and then are
      re-scored.
    We have shown how this algorithm can be applied to individual
      animal identification and demonstrated that in a majority of cases
      the correct match is ranked first by our algorithm.
    This remainder of this section summarizes the conclusions drawn
      about the static identification algorithm in a broad sense and
      discusses possible directions for future work.

    Because we have used the algorithm to curate the groundtruth we do
      not claim the reported accuracies in our experiments to be
      quantitatively absolute.
    However, the fact that we were able to use the algorithm to
      identify a significant number of individuals from different species
      is qualitative evidence for the algorithm's overall success.
    From our experiments we conclude that the algorithm is effective at
      identifying medium to high quality images of animals with
      distinguishing patterns when taken from the same viewpoint.
    %In addition we have comparative evidence that this problem is more
    %  difficult than location recognition.

    Further research and development is needed to improve the
      identification algorithm.
    Based on the evidence from our experiments, these improvements
      should:
    \begin{enumerate}
        %\item 
        %% Better invariance
        %A robust improvement to the level of feature invariance.
        %The query-side rotation heuristic should be replaced with a
        %  robust algorithm able to compromise between the gravity vector
        %  and full rotation invariance.
        %This may involve either constraining the orientation of a
        %  keypoint using its neighbors or dividing annotations into
        %  parts, each with their own ``global'' orientation
        %assignment.

        %\item Improving the separation between \groundtrue{} and
        %  \groundfalse{} \namescores{}.
        %    %Improvements to the scoring mechanism should focus on
        %    %  improving the separability of \groundtrue{} and \groundfalse{}
        %    %  cases.
        %    Ideally, we would set an operating point where automatic
        %      decisions can be made with close to $0\percent$ false
        %      positive and false negative rate, thus reducing the amount
        %      of necessary manual verification.
        %      %so manual review is only needed for a subset of the
        %      %identification results.
        %    Achieving this requires automatically detecting
        %      \photobomb{} and scenery match cases and reducing or
        %      adjusting the scores between non-distinct matches.

        %    Towards this goal, we must appropriately normalize the scores
        %      returned by the identification algorithm, but first further
        %      experiments are needed to understand the relationship between
        %      scores and database size.
        %    Such experiments would involve tracking the score of a specific
        %      query annotation while varying the size of the matching database
        %      (we do have the data to perform this experiment).
        %    If the scores flatten out after a certain database size then it
        %      becomes safe to apply score normalization even if the size of the
        %      database grows.

        %\item 
        %    % Quality classifier
        %    Detecting and possibly correcting low quality images.
        %    A significant number of failure cases in the experiments
        %      were caused by occluded, poorly lit, or low quality images.
        %    \Cref{fig:eqlighting} shows that issues due to lighting can
        %      be corrected for using adaptive histogram equalization.
        %    A useful improvement would be to train a classifier that
        %      can predict the quality of an image and potentially select
        %      normalizing preprocessing step.

        %      %  There is also work to be done in automatically selecting
        %      %normalization preprocessing steps to apply to chips on a per-chip
        %      %basis to reduceVthe errors caused by low illumination.
        %\eqlighting{}

        \item \textbf{Exploit information in multiple images}:
            % Name-to-Name
            %Identification using multiple pictures of an individual.
            One of the strongest results in the experiment section was
              that having more \exemplars{} per name in the database
              improves identification accuracy.
              %that having more images of an individual in the database
              %improves accuracy.
            The \namescoring{} mechanisms (which scores an annotation
              against a collection of annotations of an individual)
              described earlier in~\cref{subsec:namescore} has been
              designed for this purpose.
            However, this approach does not take information from
              multiple query images into account, nor does it account for
              negative evidence.
            %be extended from a
            %  annotation-vs-name approach into a name-vs-name

        \item \textbf{Make database independent identification decisions}:
          %information in multiple images}:
            % One-vs-one
            Another potential avenue is to develop an one-vs-one
              verification algorithm that can be applied on pairs of
              annotations (or pairs of annotation sets) independent of
              the current identification algorithm.
            This would lift the burden of a classification decision
              from the current identification algorithm, which is more
              suited for quickly searching and ranking database images.
            A dedicated one-vs-one mechanism would be able to be
              evaluated quantitatively in the absence of a database as
              well as make use of negative information --- the
              information that a distinctive feature is missing --- that
              is ignored by the current identification algorithm.
        \end{enumerate}
    The next chapter and outlines a proposal for making these
      improvements and discusses the associated challenges.



       \begin{comment}
           Chapter 3 setups of the rest of these:
           * Separate scores
           * improve overall matches

          Essence of Chapter4; 
          Given a graph, who are the same animals?

          Discussion points; 
          * What is wrong with an annotation?
           _ rigiditiy, mislabeings

           * What is wrong with the algo?
           _ how to improve scoring?

           * Take more data
           _ exploit lots of data

           Aggregate cm from multiple queries of the same individual

           We know from chapter 3 that having multiple instance works better

           Complete Graph of scores divide into names
           n_way cut 
           talk about approximantions

           https://en.wikipedia.org/wiki/Minimum_k-cut
      \end{comment}
