\edef\pzcode{PZ_PB_RF_TRAIN_2567}
\edef\gzcode{GZ_Master1_21589}
%\edef\gzcode{GZ_Master1_222}



\newcommand{\MatchStateExample}{
\begin{figure}[h] \centering
\begin{subfigure}[h]{0.26\textwidth} \centering
\includegraphics[height=160pt]{figures_pairclf/classesC.png}\caption{positive}\label{sub:classesC}
\end{subfigure}
\begin{subfigure}[h]{0.31\textwidth} \centering
\includegraphics[height=160pt]{figures_pairclf/classesA.png}\caption{negative}\label{sub:classesA}
\end{subfigure}
\begin{subfigure}[h]{0.31\textwidth} \centering
\includegraphics[height=160pt]{figures_pairclf/classesB.png}\caption{incomparable}\label{sub:classesB}
\end{subfigure}
\caption[\caplbl{MatchStateExample}Match state example]{\caplbl{MatchStateExample}
% ---
Examples of positive (\ref{sub:classesC}), negative  (\ref{sub:classesA}), and
  incomparable (\ref{sub:classesB}) pairs of annotations.
Local feature correspondences are superimposed over the pairs.
% ---
}
\label{fig:MatchStateExample}
\end{figure}
}


\newcommand{\PositiveHist}{
\begin{figure}[h]
\centering
\begin{subfigure}[h]{0.47\textwidth}
\centering
\includegraphics[width=\textwidth]{figures_pairclf/\expandafter\pzcode/score_hist_lnbnn.png}
\caption{Plains zebras LNBNN}\label{sub:pos_lnbnn_hist_pz}
\end{subfigure}
~
\begin{subfigure}[h]{0.47\textwidth}
\centering
\includegraphics[width=\textwidth]{figures_pairclf/\expandafter\pzcode/score_hist_pos_learn(sum,glob).png}
\caption{Plains zebras learned}\label{sub:pos_hist_pz}
\end{subfigure}
~
\begin{subfigure}[h]{0.47\textwidth}
\centering
\includegraphics[width=\textwidth]{figures_pairclf/\expandafter\gzcode/score_hist_lnbnn.png}
\caption{Grévy's zebras LNBNN}\label{sub:pos_lnbnn_hist_gz}
\end{subfigure}
~
\begin{subfigure}[h]{0.47\textwidth}
\centering
\includegraphics[width=\textwidth]{figures_pairclf/\expandafter\gzcode/score_hist_pos_learn(sum,glob).png}
\caption{Grévy's zebras learned}\label{sub:pos_hist_gz}
\end{subfigure}
%~

\caption[\caplbl{PositiveHist}Positive scores]{\caplbl{PositiveHist}
% ---
Positive scores of LNBNN (left) and pairiwse (right) matching for plains (top)
and Grévy's (bottom) zebras. In addition to being more interpretable than LNBNN
scores the learned probabilities exhibit better separability.
% ---
}
\label{fig:PositiveHist}
\end{figure}
}



\newcommand{\PositiveROC}{
\begin{figure}[h]
\centering
\begin{subfigure}[h]{0.47\textwidth}\centering\includegraphics[width=\textwidth]{figures_pairclf/\expandafter\pzcode/roc_match_state.png}\caption{Plains zebra}\label{sub:pos_roc_pz}\end{subfigure}
~
\begin{subfigure}[h]{0.47\textwidth}\centering\includegraphics[width=\textwidth]{figures_pairclf/\expandafter\gzcode/roc_match_state.png}\caption{Grévy's zebra}\label{sub:pos_roc_gz}\end{subfigure}
\caption[\caplbl{PositiveROC}Positive match-state ROC]{\caplbl{PositiveROC}
% ---
Positive match-state ROC.
% ---
}
\label{fig:PositiveROC}
\end{figure}
}


\begin{comment}
    python -m ibeis.scripts.thesis ExptChapter4.write_metrics --db GZ_Master1 --task-key=match_state
    python -m ibeis.scripts.thesis ExptChapter4.write_metrics --db PZ_PB_RF_TRAIN --task-key=match_state
\end{comment}
\newcommand{\ConfusionMatch}{
\begin{table}
    \centering
    \caption[\caplbl{ConfusionMatch}Match state confusion]{\caplbl{ConfusionMatch}
    % ---
    Multiclass match-state confusion for plains and Grévy's zebras.
    The rows are the real (groundtruth) state, and the columns are the predicted states.
    The final column indicates the number of examples for each class.
    % ---
    }
    \label{tbl:ConfusionMatch}
    \begin{subtable}[h]{\textwidth}\centering\input{figures_pairclf/\expandafter\pzcode/confusion_match_state.tex}\caption{Plains zebras match-state confusion}\end{subtable} %
    \begin{subtable}[h]{\textwidth}\centering\input{figures_pairclf/\expandafter\gzcode/confusion_match_state.tex}\caption{Grévy zebras match-state confusion}\end{subtable} %
\end{table}
}


\newcommand{\EvalMetricsMatch}{
\begin{table}
    \caption[\caplbl{EvalMetricsMatch}Match state metrics]{\caplbl{EvalMetricsMatch}
    % ---
    Multiclass match-state evaluation metrics for plains and Grévy's zebras
    computed from the confusion matrix.
    % ---
    }
    \label{tbl:EvalMetricsMatch}
    \centering
    \begin{subtable}[h]{\textwidth}\centering\input{figures_pairclf/\expandafter\pzcode/eval_metrics_match_state.tex}\caption{Plains zebras match-state metrics}\end{subtable} %
    \begin{subtable}[h]{\textwidth}\centering\input{figures_pairclf/\expandafter\gzcode/eval_metrics_match_state.tex}\caption{Grévy zebras match-state metrics}\end{subtable} %
\end{table}
}

\begin{comment}
    python -m ibeis.scripts.script_vsone report_classifier_importance --db PZ_PB_RF_TRAIN \
    --dpath ~/latex/crall-thesis-2017/ --save "figures_pairclf/wc_pz_clipwhite2.png" \
    --clipwhite --diskshow

    python -m ibeis.scripts.script_vsone report_classifier_importance --db GZ_Master1 \
    --dpath ~/latex/crall-thesis-2017/ --save "figures_pairclf/wc_gz_clipwhite2.png" \
    --clipwhite --diskshow
\end{comment}

\newcommand{\MatchWordCloud}{
\begin{figure}[h]
\centering
\begin{subfigure}[h]{0.47\textwidth}\centering\includegraphics[width=\textwidth]{figures_pairclf/\expandafter\pzcode/wc_match_state.png}\caption{Plains zebra}\label{sub:wc_pz}\end{subfigure}
~
\begin{subfigure}[h]{0.47\textwidth}\centering\includegraphics[width=\textwidth]{figures_pairclf/\expandafter\gzcode/wc_match_state.png}\caption{Grévy's zebra}\label{sub:wc_gz}\end{subfigure}
\caption[\caplbl{MatchWordCloud}Word cloud of important features for matching]{\caplbl{MatchWordCloud}
% ---
Word cloud of important features for predicting match-state.
For plains zebras, the difference in viewpoint (denoted as \pvar{global(view\_delta)}) has high importance
  because of its role in distinguishing incomparable cases.
% ---
}
\label{fig:MatchWordCloud}
\end{figure}
}

\newcommand{\ImportantMatchFeat}{
\begin{table}
    \centering
    \caption[\caplbl{ImportantMatchFeat}Important features for match prediction]{\caplbl{ImportantMatchFeat} 
    % ---
    Top $5$ most important features for predicting that matching state (positive, negative, incomparable) for a pair of
    annotations.
    %The sum of the importance for all $205$ feature dimensions is $1$.
    % ---
    }
    \label{tbl:ImportantMatchFeat}
    \begin{subtable}[h]{0.44\textwidth}
        \centering
        \begin{tabular}{l c}
            \toprule
            dimension & importance\\
            \midrule
            \input{figures_pairclf/\expandafter\pzcode/feat_importance_match_state.tex}
            \bottomrule
        \end{tabular}
        \caption{Plains zebra}
    \end{subtable} %
    ~~~~~~~~ 
    \begin{subtable}[h]{0.44\textwidth}
        \centering
        \begin{tabular}{l c}
            \toprule
            dimension & importance\\
            \midrule
            \input{figures_pairclf/\expandafter\gzcode/feat_importance_match_state.tex}
            \bottomrule
        \end{tabular}
        \caption{Grévy's zebra}
    \end{subtable} %
\end{table}
}




\begin{comment}
    python -m ibeis.scripts.thesis ExptChapter4.write_metrics --db GZ_Master1 --task-key=photobomb_state
    python -m ibeis.scripts.thesis ExptChapter4.write_metrics --db PZ_PB_RF_TRAIN --task-key=photobomb_state
\end{comment}
\newcommand{\ConfusionPhotobomb}{
\begin{table}[h]
    \caption[\caplbl{ConfusionPhotobomb}Photobomb state confusion]{\caplbl{ConfusionPhotobomb}
    % ---
    The photobomb-state confusion matrix. The columns indicate predicted classes, and the rows indicate real
    (groundtruth) classes.  The final column indicates the number of examples of each class.
    % ---
    }
    \label{tbl:ConfusionPhotobomb}
    \begin{subtable}[h]{\textwidth}\centering\input{figures_pairclf/\expandafter\pzcode/confusion_photobomb_state.tex}\caption{Plains zebra photobomb confusion}\end{subtable} %
    \begin{subtable}[h]{\textwidth}\centering\input{figures_pairclf/\expandafter\gzcode/confusion_photobomb_state.tex}\caption{Grévy's zebra photobomb confusion}\end{subtable} %
\end{table}
}


\newcommand{\EvalMetricsPhotobomb}{
\begin{table}
    \caption[\caplbl{EvalMetricsPhotobomb}Match state metrics]{\caplbl{EvalMetricsPhotobomb}
    % ---
    The photobomb-state evaluation metrics computed from the confusion matrix.
    % ---
    }
    \label{tbl:EvalMetricsPhotobomb}
    \centering
    \begin{subtable}[h]{\textwidth}
        \centering
        \input{figures_pairclf/\expandafter\pzcode/eval_metrics_photobomb_state.tex}
        \caption{Plains zebras photobomb metrics}
    \end{subtable} %

    \begin{subtable}[h]{\textwidth}
        \centering
        \input{figures_pairclf/\expandafter\gzcode/eval_metrics_photobomb_state.tex}
        \caption{Grévy zebras photobomb metrics}
    \end{subtable} %
\end{table}
}


\newcommand{\ImportantPBFeat}{
    \begin{table}
        \centering
        \caption[\caplbl{ImportantPBFeat}Important features for photobomb prediction]{\caplbl{ImportantPBFeat} 
        % ---
        Top $5$ most important features for predicting if a pair of annotations has a photobomb.  Notice that these
        features are related to the spatial location of the matches and to global information like speed that might
        preclude a match from occuring. Annotations taken at the same time and place are more likely to contain
        photobombs.
        % ---
        }
        \label{tbl:ImportantPBFeat}
        \begin{subtable}[h]{0.44\textwidth} 
            \centering
            \begin{tabular}{l c}
                \toprule dimension & importance\\
                \midrule
                \input{figures_pairclf/\expandafter\pzcode/feat_importance_photobomb_state.tex}
                \bottomrule
            \end{tabular}
            \caption{Plains zebra phototomb importance}
        \end{subtable} %
        ~~~~
        \begin{subtable}[h]{0.44\textwidth}
            \centering 
            \begin{tabular}{l c}
                \toprule
                dimension & importance\\ 
                \midrule
                \input{figures_pairclf/\expandafter\gzcode/feat_importance_photobomb_state.tex}
                \bottomrule
            \end{tabular}
            \caption{Grévy's zebra phototomb importance}
        \end{subtable} %
    \end{table}
}





\newcommand{\PairFailIN}{
\begin{figure}[h]
\centering
\includegraphics[width=\textwidth]{figures_pairclf/\expandafter\pzcode/cases_match_state/fail_Incomparable_Negative_383_503.jpg}
\caption[\caplbl{PairFailIN}Incomparable pairwise failure case]{\caplbl{PairFailIN}
% ---
The pair is incomparable, but the classifier predicts negative due to the small amount of incomparable training
data available.
% ---
}
\label{fig:PairFailIN}
\end{figure}
}

\newcommand{\PairFailNP}{
\begin{figure}[h]
\centering
\includegraphics[width=\textwidth]{figures_pairclf/\expandafter\pzcode/cases_match_state/fail_Negative_Positive_424_465.jpg}
\caption[\caplbl{PairFailNP}Negative pairwise failure case]{\caplbl{PairFailNP}
% ---
This pair is negative, but the classifier predicts positive because of
similarity in pose and coarse patterns. The classifier is unable to recognize
that the shoulder is the same region on both animals and contains distinctive
patterns that do not match.
% ---
}
\label{fig:PairFailNP}
\end{figure}
}

\newcommand{\PairFailPN}{
\begin{figure}[h]
\centering
\includegraphics[width=\textwidth]{figures_pairclf/\expandafter\pzcode/cases_match_state/fail_Positive_Negative_63_72.jpg}
\caption[\caplbl{PairFailPN}Positive pairwise failure case]{\caplbl{PairFailPN}
% ---
The pair is positive, but the classifier predicts negative because of occlusion and viewpoint variations.
% ---
}
\label{fig:PairFailPN}
\end{figure}
}


\newcommand{\LeftRightFace}{
\begin{figure}[h]
\centering
\includegraphics[width=\textwidth]{figures_pairclf/custom_match_leftrightface_5245_5161.jpg}
\caption[\caplbl{LeftRightFace}Face match]{\caplbl{LeftRightFace}
% ---
Even though this pair has different viewpoints, it is positive and comparable
because we can establish a distinctive correspondence in the face. 
% ---
}
\label{fig:LeftRightFace}
\end{figure}
}
