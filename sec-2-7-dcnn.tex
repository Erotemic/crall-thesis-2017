
\section{Deep convolutional neural networks}\label{sec:dcnn}
    % VGG-Net shows importance of small convolutions \cite{simonyan_very_2014}
    % Spatial pyramids in CNNs \cite{he_spatial_2014}
    Convolutional networks have been around for over more than two
      decades~\cite{lecun_gradient_based_1998,
      fukushima_neocognitron_1988}.
    However, they did not receive major attention from computer vision
      researchers until 2012 when a deep convolutional neural network
      (DCNN)~\cite{krizhevsky_imagenet_2012} outperformed the best
      support vector machines (SVMs)~\cite{vapnik_statistical_1998} by
      over $10\percent$ in the ImageNet category recognition
      challenge~\cite{russakovsky_imagenet_2014}.
    Since then, many successful category recognition techniques based
      on DCNNs have been published~\cite{simonyan_very_2014,
      chatfield_efficient_2014, chatfield_return_2014,
      oquab_learning_2014, szegedy_going_2014, long_convnets_2014,
      he_spatial_2014, dean_fast_2013}.
    DCNNs have also been shown excellent results when applied to other
      computer vision problems such as: %
    instance recognition~\cite{razavian_cnn_2014,
      razavian_baseline_2015, liu_learning_2015,
      held_deep_2015,arandjelovic_netvlad_2016,radenovic_cnn_2016}, %
    fine-grained recognition~\cite{branson_bird_2014,
      donahue_decaf_2013, catherine_wah_similarity_2014}, %
    detection~\cite{girshick_rich_2014, sermanet_overfeat_2013,
      li_wan_end_end_2015}, %
    face verification~\cite{huang_learning_2012, taigman_deepface_2014,
      sun_deep_2013}, %
    and learning similarity between feature
      patches~\cite{osendorfer_convolutional_2013, han_matchnet_2015,
      ng_exploiting_2015, zagoruyko_learning_2015, han_matchnet_2015}.
    The sudden success of deep nets has been attributed
    (1) a larger volume of available of training data, and
    (2) implementations using faster
      GPUs~\cite{krizhevsky_imagenet_2012}.

    %\chuckcomment{This section is a bit terse but we may need to live with this}
    %\chuckcomment{Make it clear that there are many different convolutions at each layer}
      
    Several techniques are employed to increase accuracy, reduce over-fitting,  and reduce training time.
    %There have also been advances in training with back propagation apart from
    %  faster GPUS and more training data \cite{dahl_improving_2013}.
    Data augmentation is used to artificially increase the amount of training data~\cite{ciresan_multi_column_2012,
    ciresan_high_performance_2011, simard_best_2003}. The dropout technique has been shown to reducing
    over-fitting~\cite{dahl_improving_2013, srivastava_dropout_2014}. At training time outputs of hidden units are
    randomly suppressed which forces the network to learn a more robust representation. It has been shown that dropout
    can be viewed as a form of model averaging~\cite{hinton_improving_2012}. Rectified Linear Units (ReLU) have been
    shown to be a faster alternative to the standard sigmoid activation functions~\cite{vinod_rectified_2010,
    dahl_improving_2013}. An ReLU is similar to a hinge function and simply outputs the signal of a unit if it is
    positive and outputs a zero otherwise.
    %  \max(0, x)$.
    %\begin{equation}
    %    f(x) = \max(0, x)
    %\end{equation}
    %\chuckcomment{give an intuition}
    Leaky Rectified Linear Units (LReLU) further improve network accuracy by including a ``leakiness'' term while
    maintaining the speed of ReLUs~\cite{maas_rectifier_2013}. While a ReLU strictly suppresses a feature activation if
    it is negative a LReLU returns a small negative signal (by multiplying by a constant) instead of zero.
    %The following is a common definition of a LReLU:
    %\begin{equation}
    %    f(x) = \fullbincases{x}{x > 0}{.01x}
    %\end{equation}
    %\chuckcomment{What is your point}

    A deep neural network is constructed by stacking several layers of units (neurons) together. Data is used to
    initialize the activations of an input layer, and the information is forward propagated through the network. Weights
    are chosen to optimize a loss function --- \eg{} categorical cross-entropy error or triplet
    loss~\cite{schroff_facenet_2015} --- which is chosen depending on the application. Optimization of the loss function
    is performed using back-propagation~\cite{rumelhart_learning_1986} --- typically using mini-batches and stochastic
    gradient descent with momentum~\cite{sutskever_importance_2013}. Traditionally each layer in a neural network is
    fully connected --- each pair of units between the previous layer this layer has its own edge weight ---  to the
    previous layer. However, in computer vision networks are constructed using convolutional layers.

    %Convolutional networks are used to extract features from
    A DCNN connects the input layer to a stack of convolutional layers~\cite{krizhevsky_imagenet_2012}.
    %A DCNN typically consists of several stacked convolutional layers
    %  connected to several stacked fully connected
    %  layers~\cite{krizhevsky_imagenet_2012}.
    A convolutional layer differs from a fully connected layer in that it is sparsely connected and that most of the
    edge weights between layers are shared~\cite{lecun_gradient_based_1998, fukushima_neocognitron_1988,
    serre_robust_2007}. Each convolutional layer is broken into several channels. Each channel is given its own weight
    matrix with a fixed width and height. This matrix of weights is convolved with the input layer to produce a feature
    activation map, one for each channel. Convolutional layers often use several pooling layers that aggregate
    information over a small area, reduce the size of the feature map, and increase robustness to transformations.
    Common pooling operations are max-pooling~\cite{serre_robust_2007, krizhevsky_imagenet_2012} and
    maxout~\cite{goodfellow_maxout_2013}.
    %The most common pooling operation is
    %  max-pooling~\cite{serre_robust_2007, krizhevsky_imagenet_2012}.
    %Similar to max pooling is a maxout layer where the input to a unit
    %  is the maximum output over all channels output by the previous
    %  convolution~\cite{goodfellow_maxout_2013}.
    %These shared weights act as a filter --- somewhat similar to a
    %  Gabor filter~\cite{gabor_theory_1946} --- that is applied
    %  convolutionally to produce a feature map.
    The convolutional layers may also be connected to a stack of fully connected layers. In this case, hierarchies of
    feature maps are built in the low level convolutional layers, and then fully connected layers learn decision
    boundaries between these features~\cite{zeiler_visualizing_2014}.
    %Convolutional neural networks are trained using back propagation

    %Deep convolutional architectures have been developed to learn
    %  visual similarity between images
    %  patches~\cite{osendorfer_convolutional_2013, han_matchnet_2015,
    %  ng_exploiting_2015, zagoruyko_learning_2015, han_matchnet_2015}.
    Because of weight sharing convolutional networks must learn significantly less parameters than fully connected
    networks. This allows convolutional networks to be trained much faster. Fewer weights also acts as a form of
    regularization for the network. Intuitively learned convolutional filters are similar to Gabor
    filters~\cite{gabor_theory_1946}, which are a naturally suited for extracting features from images. Even without
    learning weights, convolutions can be used to extract powerful features for matching~\cite{revaud_deep_2015}. The
    popular SIFT and HoG features~\cite{mahendran_understanding_2014} can even be implemented as convolutional networks.
    Despite the lack of hard theoretical insight into the inner workings of these networks, their empirical performance
    cannot be denied.

   % Formally a convolutional layer is defined as follows:
   % An activation function, $f^\ell$, defines a neural response using
   %   the input from the previous layer.
   % Let $C^{\ell}$ be the number of channels in the $\ell$\th{} layer.
   % Let $\mat{H}^\ell_c$ be the feature map of neural activations for
   %   the $c\th$ channel in the $\ell$\th{} layer.
   % Let $b^\ell_c$ be a bias of this channel.
   % Let $\mat{W}^\ell_{k,c} \in \Real^{n \times m}$ be a weight matrix
   %   between the $k$\th{} channel of layer $\ell - 1$ and the $c$\th{}
   %   channel of layer $\ell$ with width $n$ and height $m$.
   % The discrete convolution operator $\conv_{x, y}$ is applied with
   %   strides in the $x$ and $y$ direction.
   % Here, we overload all operations between scalars and tensors to be
   %   elementwise.
   % The activations of a convolutional layer is given by the following
   %   equation:
   % \begin{equation}
   %     \mat{H}_c^\ell = f^{\ell}\paren{
   %         \sum_{k=1}^{C^{\ell - 1}} 
   %       \paren{
   %         \mat{H}_k^{\ell-1} {\conv}_{x, y}
   %         \mat{W}_{k,c}^\ell 
   %       } 
   %         + b^\ell_{c}
   %     }
   % \end{equation}

   % To see the difference between a convolutional layer and a fully
   %   connected layer consider the activation of a single unit in a fully
   %   connected layer.
   % Let $L^{\ell}$ be the number of units in layer $\ell$.
   % Let $h_j^\ell$ be the activation of the $j$\th{} unit in the
   %   $\ell$\th{} layer.
   % Let $w^\ell_{i,j}$ be the weight between $o_i^{\ell - 1}$ and
   %   $h_j^\ell$.
   % Let $b^\ell_j$ be the bias of this unit.
   %\begin{equation}
   %     h_j^\ell = f^{\ell}\paren{
   %         %\paren{
   %         \sum_{i=1}^{L^{\ell - 1}}
   %                 h_i^{\ell - 1} w^\ell_{i,j}
   %         %}
   %         + b^\ell_j
   %         }
   % \end{equation}
   % The number of weights to learn for a fully connected layer is
   %   $\bigoh{L^{\ell - 1}L^{\ell}}$, whereas the number of weights to
   %   learn for a convolutional layer is $\bigoh{n m C^{\ell - 1}
   %   C^{\ell}}$.
   % The number weights into to a fully connected layer grows very fast
   %   with respect to the number of units.
   % In contrast, the number of weights into a convolutional layer
   %   depends only on the number of channels and size of the
   %   convolutional filter.
   % It does not depend on the number of input or output units.
   % Having fewer weights allows networks to be trained much faster.
   % Fewer weights also acts as a form of regularization for the
   %   network:
   % the same convolutional filters must be applicable to all locations
   %   in the image.

    %Careful initialization of weights is important. A good initialization
    %scheme is orthogonal initialization.
    %\subsection{Deep convolutional descriptors}
    %\label{sec:deepdesc}

    %    %Features extracted from deep networks are quickly outperforming hand
    %    %  crafted and even learned features in vision competitions
    %    %  \cite{krizhevsky_imagenet_2012,razavian_cnn_2014}.
    %    %This was most clearly demonstrated in the 2012 ImageNet competition
    %    %  where SVM based techniques were outperformed by deep convolution neural
    %    %  networks by a large margin \cite{deng_imagenet_2009,
    %    %  russakovsky_imagenet_2014,krizhevsky_imagenet_2012}.
    %    %For more details about convolutional networks see~\cref{sec:dcnn}.
    %    %Here we will discuss on the descriptors they can extract.



  \subsection{Discussion --- deep convolutional neural networks}\label{subsec:dcnndiscuss}
        Because of the astounding success of convolutional networks in almost every area in computer vision, we have
        investigated their use in animal identification. Specifically, we have investigated two approaches.

        The first approach used deep convolutional feature descriptors as a replacement for the
        SIFT~\cite{lowe_distinctive_2004} descriptor following the patch based scheme in~\cite{zagoruyko_learning_2015}.
        The basic idea is to have two patches fed through the same (Siamese)~\cite{chopra_learning_2005} architecture
        and then compare their resulting encodings. This comparison can be as simple as Euclidean distance, or as
        complex as a learned distance measure. Training can be performed on pairs of patches, labeled as correct or
        incorrect, using the discriminative loss function~\cite{lecun_loss_2005}. Unfortunately, due to issues with the
        quality and quantity of our training data our convolutional replacements for the SIFT descriptor have not been
        successful.

        The second approach aimed to use Siamese networks to directly compare two images of an animal to determine if
        they were the same or different, similar to the method used in Deep Face\cite{taigman_deepface_2014} for face
        verification. However, without the large training datasets and specialized alignment procedures used in Deep
        Face, we were unable to produce promising results.
        
        Due to these issues, this \thesis{} does not further pursue techniques based on DCNNs. We include this
        discussion to note on the potential of deep learning applied to animal identification and to strongly suggest
        further investigation of these techniques in the future research. Of particular interest for future research is
        the matching technique presented in \cite{rocco_convolutional_2017}. This method is particularly promising
        because it learns to match and align images by mimicking a classic computer vision pipeline while using only
        synthetic training data. This may be able to overcome the issues mentioned above, however further investigation
        is needed.

