
\begin{comment}
        python -m ibeis --tf show_chip --aid 9970 --db PZ_Master1 --affine-invariance=True --dpath ~/latex/crall-candidacy-2015/ --save figures5/backgroundlighting.jpg --figsize=10,4 --dpi 180 --clipwhite --diskshow
        python -m ibeis --tf show_chip --aid 9970 --db PZ_Master1 --show --adapteq=True --affine-invariance=True --dpath ~/latex/crall-candidacy-2015/ --save figures5/backgroundlighting1.jpg --figsize=10,4 --dpi 180 --clipwhite --diskshow
        python -m ibeis --tf show_chip --aid 9970 --db PZ_Master1 --show --histeq=True --affine-invariance=True --dpath ~/latex/crall-candidacy-2015/ --save figures5/backgroundlighting1.jpg --figsize=10,4 --dpi 180 --clipwhite --diskshow
\end{comment} 
\newcommand{\backgroundlighting}{
\begin{figure}[H]
\centering
~~% --
\begin{subfigure}[b]{0.47\textwidth}
\centering
\fbox{\includegraphics[width=\textwidth]{figures5/backgroundlighting.jpg}}\caption{}\label{sub:noeq}
\end{subfigure}
~~% --
\begin{subfigure}[b]{0.47\textwidth}
\centering
\fbox{\includegraphics[width=\textwidth]{figures5/backgroundlighting1.jpg}}\caption{}\label{sub:witheq}
\end{subfigure}
~~% --
\caption[Background lighting]{
% ---
An annotation with few features detected on the body due to lighting
  conditions is shown in \cref{sub:noeq}.
The same annotation with histogram equalization is shown in
  \cref{sub:witheq}, howver many more features are also detected on the
  background.
% ---
}
\label{fig:backgroundlighting}
\end{figure}
}


\begin{comment}
    % manually copied these
    python -m ibeis_cnn --tf pz_patchmatch --ds liberty --test --weights=liberty:current --arch=siaml2_128 --test --lightbg
    python -m ibeis_cnn --tf pz_patchmatch --ds pzmtest --test --weights=liberty:current --arch=siaml2_128 --testall  --lightbg  # NOQA

    cp /media/raid/work/PZ_MTEST/_ibsdb/_ibeis_cache/nets/patchmatch-_1686_ugtzhautrfzsiizh-/diagnostics/hist_eras009_epochs0121_txlcuebpovsypuyw/hlhtm3cz_00000000.jpg ~/latex/crall-candidacy-2015/figures5/patchmatch-roc-mtest-sift.jpg
    cp /media/raid/work/PZ_MTEST/_ibsdb/_ibeis_cache/nets/patchmatch-_1686_ugtzhautrfzsiizh-/diagnostics/hist_eras009_epochs0121_txlcuebpovsypuyw/wpwmify9_00000000.jpg ~/latex/crall-candidacy-2015/figures5/patchmatch-roc-mtest-cnn.jpg
    cp /home/joncrall/.config/ibeis_cnn/training/liberty/diagnostics/hist_eras009_epochs0121_txlcuebpovsypuyw/fig\(3\)_liberty\;dict\(detector\=\'dog\'\,_pairs\=250000\)_hist_eras009_epochs0121_txlcuebpovsypuyw__CNN_scores._#data\=20000_.jpg ~/latex/crall-candidacy-2015/figures5/patchmatch-roc-liberty-cnn.jpg
    cp /home/joncrall/.config/ibeis_cnn/training/liberty/diagnostics/hist_eras009_epochs0121_txlcuebpovsypuyw/fig\(4\)_liberty\;dict\(detector\=\'dog\'\,_pairs\=250000\)_hist_eras009_epochs0121_txlcuebpovsypuyw__SIFT_scores._#data\=20000_.jpg ~/latex/crall-candidacy-2015/figures5/patchmatch-roc-liberty-sift.jpg
\end{comment}
\newcommand{\siamlIIroc}{
\begin{figure}[H]
\centering
~~% --
\begin{subfigure}[b]{0.47\textwidth}
\centering
\fbox{\includegraphics[width=\textwidth]{figures5/patchmatch-roc-liberty-sift.jpg}}\caption{}\label{sub:siamlIIrocA}
\end{subfigure}
~~% --
\begin{subfigure}[b]{0.47\textwidth}
\centering
\fbox{\includegraphics[width=\textwidth]{figures5/patchmatch-roc-liberty-cnn.jpg}}\caption{}\label{sub:siamlIIrocB}
\end{subfigure}
~~% --
\begin{subfigure}[b]{0.47\textwidth}
\centering
\fbox{\includegraphics[width=\textwidth]{figures5/patchmatch-roc-mtest-sift.jpg}}\caption{}\label{sub:siamlIIrocC}
\end{subfigure}
~~% --
\begin{subfigure}[b]{0.47\textwidth}
\centering
\fbox{\includegraphics[width=\textwidth]{figures5/patchmatch-roc-mtest-cnn.jpg}}\caption{}\label{sub:siamlIIrocD}
\end{subfigure}
~~% --
\caption[Siamese network roc]{
    % ---
    A comparison between the score distributions and ROC curves of patch
      descriptor distances of SIFT and \siamLII{} CNN (trained on liberty) using a
      liberty test set and a zebra test set.
    \Cref{sub:siamlIIrocA,sub:siamlIIrocC} show SIFT results on liberty and
      plains zebras.
    \Cref{sub:siamlIIrocB,sub:siamlIIrocD} show \siamLII{} results on liberty
      and plains zebras.
    These plots suggests that
    (1) baseline SIFT distance classification of zebra patch correspondences
      is significantly more challenging than liberty building patch
      correspondences, and
    (2) the advantages of \siamLII{} based descriptors over SIFT descriptors
      seen in the liberty datasets do not transfer to our zebra datasets.
    %On the liberty test set the FPR95 score based on SIFT is $.23$
    %  (\cref{sub:siamlIIrocA}) , and the score for \siamLII is $.15$
    %  (\cref{sub:siamlIIrocB}).
    %However, using a sample of patches from zebra datasets the score based on
    %  SIFT is $.84$ (\cref{sub:siamlIIrocC}) , and the score for \siamLII is
    %  $.90$.
    % ---
}
\label{fig:siamlIIroc}
\end{figure}
}



\begin{comment}
python -m ibeis_cnn.draw_net --test-make_architecture_pydot_graph --show --netx-cnn-hack --figsize=18,2 --save=figures5/siaml2arch.png --dpath ~/latex/crall-candidacy-2015/  --clipwhite --diskshow
\end{comment}
\SingleImageCommand{siamLIIarch}{\textwidth}{siamLIIarch}{
    % ---
    Illustration of the main branch in the \siamLII{} architecture that computes
      a descriptor from an image patch.
    Durring training two images pushed through this network, the error is
      computed by comparing descriptor values and the weights in each layer are
      updated.
    The dropout layers are removed after training.
    % ---
}{figures5/siaml2arch.png}


\begin{comment}
        python -m ibeis_cnn.models.siam --test-constrastive_loss --show --save=figures5/contrastive.png --dpath ~/latex/crall-candidacy-2015/  --clipwhite --diskshow --figsize=10,4 --lightbg
\end{comment}
\SingleImageCommand{contrastive}{\textwidth}{contrastive}{
% ---
Contrastive loss objective function.
The correct and incorrect distance shows the loss of \groundtrue{} and
  \groundfalse{} pairs of training examples as a function of energy.
The energy of a pair of training examples is the Euclidean distance between
  their computed descriptors.
% ---
}{figures5/contrastive.png}


\begin{comment}
    python -m ibeis.control.IBEISControl --test-show_depc_annot_graph --reduced --save=figures5/reduced_annot_depc.png --dpath ~/latex/crall-candidacy-2015/  --clipwhite --diskshow --figsize=7,7  --dpi=220
\end{comment}

\begin{comment}
    python -m ibeis.control.IBEISControl --test-show_depc_image_graph --reduced --save=figures5/reduced_image_depc.png --dpath ~/latex/crall-candidacy-2015/  --clipwhite --diskshow  --figsize=4,4 --dpi=220
\end{comment}


\begin{comment}
        python -m ibeis.scripts.specialdraw double_depcache_graph --save=figures5/doubledepc.png --dpath ~/latex/cand/  --diskshow --figsize=20,10 --dpi=96 --testmode --show --clipwhite --arrow-width=5 --line-width=2
\end{comment}
\SingleImageCommand{doubledepc}{.9}{IBEIS \depcache{}}{
    % ---
    %The transitive reduction~\cite{aho_transitive_1972} of IBEIS image and
    %  annotation dependency cache.
    %We show the transitive reduction because the actual dependency graph has a
    %  messy visualization.
    %The transitive reduction more clearly shows the underlying dependency
    %  structure and more accurately reflects the order of execution (note that
    %  the transitive reduction when multi-edges are included has not yet been
    %  properly defined).
    The IBEIS annotation \depcache{} and image \depcache{}.
    The two graphs are connected by a user decision node.
    Root objects are shown in red.
    Note that the yellow node indicates a user decision and is not directly
      part of the \depcache{}.
    This node is placed to show that the results of one dependency cache can
      be used as primary objects of another.
    % ---
}{figures5/doubledepc.png}


\begin{comment}
python -m dtool.depcache_table show_input_graph --show --dpath=~/latex/cand --save=figures5/expandedinput.png --diskshow
\end{comment}
\SingleImageCommand{expandedinput}{.9}{Expanded input graph}{
% ---
This is a simplified example of the dependency cache for the selective match
  kernel algorithm.
It is a good example of the power of the dependency cache because there are
  two multi-input / model nodes (vocab and inverted index).
This figure shows how the ``source-most'' (shown in red) input in determined for
  the green table node.
It is possible to specify other inputs (pink nodes), but the ``source-most''
  input is minimal.
% ---
}{figures5/expandedinput.png}




% TODO: http://www.siafoo.net/article/15
% extend python lexer for doctest support
%\definecolor{mintedbackground}{rgb}{0.95,0.95,0.95}
\newmintedfile[pythonscript]{python}{
%fontfamily=tt,
linenos=true,
%numberblanklines=true,
%numbersep=12pt,
%numbersep=5pt,
gobble=0,
%frame=leftline,
%framerule=0.4pt,
%framesep=2mm,
funcnamehighlighting=true,
tabsize=4,
fontsize=\scriptsize{},
%obeytabs=false,
%mathescape=false
samepage=false, %with this setting you can force the list to appear on the same page
%showspaces=false,
%showtabs =false,
%texcl=false,
}

\newminted{pycon}{%
    linenos=false,
    gobble=0,
    funcnamehighlighting=true,
    tabsize=4,
    fontsize=\scriptsize{},
    samepage=false, %with this setting you can force the list to appear on the same page
}


\newminted{python}{
%fontfamily=tt,
linenos=false,
%numberblanklines=true,
%numbersep=12pt,
%numbersep=5pt,
gobble=0,
%frame=leftline,
%framerule=0.4pt,
%framesep=2mm,
funcnamehighlighting=true,
tabsize=4,
fontsize=\scriptsize{},
%obeytabs=false,
%mathescape=false
samepage=false, %with this setting you can force the list to appear on the same page
%showspaces=false,
%showtabs =false,
%texcl=false,
}
    

\newcommand{\egregister}{
%\begin{listing}[!ht]
\begin{figure}[!ht]
\centering
\RecustomVerbatimEnvironment{Verbatim}{BVerbatim}{}
\fbox{
%\inputminted[linenos,tabsize=4,fontsize=\footnotesize]{python3}{egregister.py}
\pythonscript{egregister.py}
}
\caption[\caplbl{egregister} Example of a registered \depcache{} function]{
% ---
\caplbl{egregister} A simplified version of the function that compute
  features given a chip.
This function is registered with the annotation \depcache{} in a
  declarative style using the current API{}.
This property can be accessed using the \depcache{} as shown in the
  doctest example.
%\depcache{}
% ---
}\label{fig:egregister}
\end{figure}
%\end{listing}
}


\begin{comment}
python -m ibeis.algo.graph.mixin_loops prob_any_remain --num_pccs=40 --size=2 --patience=20 --window=20 --dpi=300 --figsize=7.4375,3.0 '--dpath=~/latex/crall-thesis-2017' --save=figures5/poisson.png --diskshow
\end{comment}
\newcommand{\poisson}{
\begin{figure}[H]
    %
    \centering
    \includegraphics[width=\textwidth]{figures5/poisson.png}
    \caption[Convergence criteria]{\caplbl{poisson}
    %--
    The convergence criteria applied to a synthetic dataset with $40$ names and $2$ annotations per name.
    The red line indicates the fraction of meaningful reviews that remain undiscovered.
    The blue line is the probability that at least one of the next $20$ reviews will be meaningful.
    Notice that the blue line dips when the red line flattens.
    The process terminates once this probability drops below $0.1$.
    %--
} \label{fig:poisson}
\end{figure}
}
