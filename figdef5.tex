
\begin{comment}
python -m ibeis.scripts.specialdraw draw_graph_id \
    --dpath ~/latex/crall-thesis-2017/ --save "figures5/decisiongraph.jpg" \
    --figsize=12,8 --clipwhite --dpi=300 --diskshow
\end{comment}
\newcommand{\decisiongraph}{
\begin{figure}[t]
\centering
\includegraphics[width=\textwidth]{figures5/decisiongraph.jpg}
\caption[A synthetic decision graph]{\caplbl{decisiongraph}
% ---
An example of a consistent synthetic decision graph with positive,
  negative, and incomparable edges.
The color of each node represents the positive connected component (PCC)
  it belongs to.
% ---
}
\label{fig:decisiongraph}
\end{figure}
}





\begin{comment}
python -m ibeis.scripts.specialdraw draw_inconsistent_pcc --show
python -m ibeis.scripts.specialdraw draw_inconsistent_pcc \
    --dpath ~/latex/crall-thesis-2017/ --save "figures5/inconpcc.jpg" \
    --figsize=15,10 --clipwhite --dpi=300 --diskshow --saveparts
\end{comment}
%\MultiImageCommandII{inconpcc}{.4}{inconpcc}{
%}{figures5/inconpccA.jpg}{figures5/inconpccB.jpg}
\newcommand{\inconpcc}{
\begin{figure}[ht!]
\centering
\begin{subfigure}[h]{0.4\textwidth}\centering\includegraphics[width=\textwidth]{figures5/inconpccA.jpg}\caption{}\label{sub:inconpccA}\end{subfigure}
~~% --
\begin{subfigure}[h]{0.4\textwidth}\centering\includegraphics[width=\textwidth]{figures5/inconpccB.jpg}\caption{}\label{sub:inconpccB}\end{subfigure}
\caption[An inconsistent PCC]{\caplbl{inconpcc}
% ---
Any PCC containing at least one negative edge is inconsistent.
Subfigure \Cref{sub:inconpccA} shows an inconsistent PCC, and
\Cref{sub:inconpccB} shows the same PCC where the edges hypothesized to be
errors are highlighted.
% ---
}
\label{fig:inconpcc}
\end{figure}
}



  


\begin{comment}
python -m ibeis.scripts.specialdraw redun_demo2 --show
python -m ibeis.scripts.specialdraw redun_demo2 \
    --dpath ~/latex/crall-thesis-2017/ --save "figures5/kredun.jpg" \
    --figsize=10,5 --clipwhite --dpi=300 --saveparts --diskshow
\end{comment}
\newcommand{\kredun}{
\begin{figure}[h]
\centering
\begin{subfigure}[h]{0.31\textwidth}\centering\includegraphics[width=\textwidth]{figures5/kredunA.jpg}\caption{}\label{sub:kredunA}\end{subfigure}
~~% --
\begin{subfigure}[h]{0.31\textwidth}\centering\includegraphics[width=\textwidth]{figures5/kredunB.jpg}\caption{}\label{sub:kredunB}\end{subfigure}
~~%--
\begin{subfigure}[h]{0.31\textwidth}\centering\includegraphics[width=\textwidth]{figures5/kredunC.jpg}\caption{}\label{sub:kredunC}\end{subfigure}
~~%--
\begin{subfigure}[h]{0.31\textwidth}\centering\includegraphics[width=\textwidth]{figures5/kredunD.jpg}\caption{}\label{sub:kredunD}\end{subfigure}
~~%--
\begin{subfigure}[h]{0.31\textwidth}\centering\includegraphics[width=\textwidth]{figures5/kredunE.jpg}\caption{}\label{sub:kredunE}\end{subfigure}
~~%--
\begin{subfigure}[h]{0.31\textwidth}\centering\includegraphics[width=\textwidth]{figures5/kredunF.jpg}\caption{}\label{sub:kredunF}\end{subfigure}
\caption[Examples of $k$-redundant PCCs]{\caplbl{kredun}
%--%
Examples of positive (top) and negative (bottom) redundancy.
The positive edges are colored blue and the negative edges are colored red.
Choosing the level of redundancy is a trade-off between the number of required
  reviews and the confidence that the reviews are correct.
%In our current implementation we use $2$-redundancy.
%
} \label{fig:kredun}
\end{figure}
}



\begin{comment}
python -m ibeis.viz.viz_chip HARDCODE_SHOW_PB_PAIR --db PZ_Master1 --has_any=photobomb --index=1 --match \
    --dpath ~/latex/crall-thesis-2017/ --save "figures5/PhotobombExampleC.jpg" \
    --figsize=9,4 --clipwhite --dpi=180 --save

python -m ibeis.viz.viz_chip HARDCODE_SHOW_PB_PAIR --db PZ_Master1 --has_any=photobomb --index=1 \
    --dpath ~/latex/crall-thesis-2017/ --save "figures5/PhotobombExample.jpg" \
    --figsize=9,4 --clipwhite --dpi=300 --saveparts

python -m ibeis.core_annots --test-compute_one_vs_one --show
    
\end{comment}

\newcommand{\PhotobombExample}{
\begin{figure}[h]
\centering
\begin{subfigure}[h]{0.4\textwidth} \centering \includegraphics[height=100pt]{figures5/PhotobombExampleA.jpg}\caption{}\label{sub:PhotobombExampleA} \end{subfigure}
\begin{subfigure}[h]{0.4\textwidth} \centering \includegraphics[height=100pt]{figures5/PhotobombExampleB.jpg}\caption{}\label{sub:PhotobombExampleB} \end{subfigure}
%\begin{subfigure}[h]{0.5\textwidth}
%\centering
%\includegraphics[height=60pt]{figures/PhotobombExampleC.jpg}\caption{}\label{sub:PhotobombExampleC}
%\end{subfigure}
\caption[\caplbl{PhotobombExample}Photobomb example]{\caplbl{PhotobombExample}
% ---
A secondary animal in an annotation can cause a ``photobomb''.  Notice the
primary animal in~\cref{sub:PhotobombExampleA} appears in the background
of~\cref{sub:PhotobombExampleB}. 
%The matching regions are displayed in~\cref{sub:PhotobombExampleC}.
% ---
}
\label{fig:PhotobombExample}
\end{figure}
}



\begin{comment}
python -m ibeis.algo.graph.mixin_loops prob_any_remain --num_pccs=40 --size=2 --patience=20 --window=20 --dpi=300 --figsize=7.4375,3.0 '--dpath=~/latex/crall-thesis-2017' --save=figures5/poisson.png --diskshow
\end{comment}
\newcommand{\poisson}{
\begin{figure}[h]
    %
    \centering \includegraphics[width=\textwidth]{figures5/poisson.png} \caption[The convergence criteria on a
      synthetic dataset]{\caplbl{poisson}
    %--
    The convergence criteria applied to a synthetic dataset with $40$ names and $2$ annotations per name.
    The red line indicates the fraction of \meaningful{} reviews that remain undiscovered.
    The blue line is the probability that at least one of the next $20$ reviews will be \meaningful{}.
    Notice that the blue line dips when the red line flattens.
    The process terminates once this probability drops below a threshold, which is denoted by the green dotted
      line.
    %--
} \label{fig:poisson}
\end{figure}
}

% -----------
% Experiments
% -----------


\begin{comment}
python -m ibeis Chap5.draw error_tables PZ_Master1
python -m ibeis Chap5.draw error_tables GZ_Master1

python -m ibeis Chap5.draw dbstats PZ_Master1,GZ_Master1

\end{comment}
\newcommand{\TestTrainDBStats}{
\begin{table}[h]
    \centering
    \caption[\caplbl{TestTrainDBStats}Database statistics for graph identification experiments]{\caplbl{TestTrainDBStats}
    % ---
    Database statistics for the graph identification experiments.
    Each database is split into a training and a testing set.
    %We use the training set to learn the pairwise classifiers, and the simulation is run on the testing set.
    We report the number of names, the number of annotations, and the average number of annotations per name for
      each set.
    Additionally, we report how many edges (pairs of annotations) were used to train the classifiers.
    % ---
    }
    \label{tbl:TestTrainDBStats}
    \begin{subfigure}[h]{\textwidth}\centering\input{figures5/PZ_Master1/dbstats.tex}\caption{Plains zebra}\end{subfigure}
    \begin{subfigure}[h]{\textwidth}\centering\input{figures5/GZ_Master1/dbstats.tex}\caption{Grévy's zebra}\end{subfigure}
\end{table}
}


\begin{comment}
python -m ibeis Chap5.measure_simulation --db GZ_Master1 --show
python -m ibeis Chap5.measure_simulation --db PZ_Master1 --show

python -m ibeis Chap5.draw_simulation --db PZ_Master1 --diskshow
python -m ibeis Chap5.draw_simulation --db GZ_Master1 --diskshow
\end{comment}
\newcommand{\Simulation}{
\begin{figure}[t]
\centering
\begin{subfigure}[h]{\textwidth}\centering\includegraphics[width=\textwidth]{figures5/PZ_Master1/simulation.png}\caption{Plains zebra}\end{subfigure}
~
\begin{subfigure}[h]{\textwidth}\centering\includegraphics[width=\textwidth]{figures5/GZ_Master1/simulation.png}\caption{Grévy's zebra}\end{subfigure}
\caption[\caplbl{Simulation}Simulation experiment]{\caplbl{Simulation}
% ---
The user simulation experiment compares the three identification algorithms defined in this \thesis{}.
On the left indicates the identification accuracy using the number of remaining merges and the right counts the
  number of errors made (lower is better in both cases).
The best results are clearly achieved by \pvar{graph}.
% ---
}
\label{fig:Simulation}
\end{figure}
}


\begin{comment}
python -m ibeis Chap5.draw_refresh --db GZ_Master1 --diskshow
python -m ibeis Chap5.draw_refresh --db PZ_Master1 --diskshow
\end{comment}
\newcommand{\Refresh}{
\begin{figure}[ht]
\centering
\begin{subfigure}[h]{\textwidth}\centering\includegraphics[width=\textwidth]{figures5/PZ_Master1/refresh.png}\caption{Plains zebra}\end{subfigure}
~
\begin{subfigure}[h]{\textwidth}\centering\includegraphics[width=\textwidth]{figures5/GZ_Master1/refresh.png}\caption{Grévy's zebra}\end{subfigure}
\caption[\caplbl{Refresh}Measured refresh and termination probabilities]{\caplbl{Refresh}
% ---
The measured refresh and termination criteria.
The probability that the next reviews will be \meaningful{} ($\Pr{T\teq1}$) is high while new merges are discovered.
Once the probability falls under the threshold, positive redundancy is enforced (the flat areas) on existing
  PCCs, and then candidate edges are recomputed.
After an iteration with no \meaningful{} reviews, the process terminates.
%Identification converges in $4$ iterations for plains zebras and $3$ for Grevy's.
% ---
}
\label{fig:Refresh}
\end{figure}
}




\begin{comment}
python -m ibeis Chap5.draw error_tables PZ_Master1
python -m ibeis Chap5.draw error_tables GZ_Master1
\end{comment}

\newcommand{\ErrorSizeDetails}{
\begin{table}[h]
    \centering
    \caption[\caplbl{ErrorSizeDetails}Simulation error sizes]{\caplbl{ErrorSizeDetails}
    % ---
    Analysis of the simulation errors.
    We compare statistics of the predicted PCCs with statistics of the real groundtruth PCCs.
    In each category ``Pred PCCs'' is the number of predicted PCCs and ``Pred PCC size'' is the average number of
      annotations in those PCCs (measured as mean and standard deviation).
    The ``Real PCCs'' and ``Real PCC size'' columns are similarly defined.
    % ---
    }
    \label{tbl:ErrorSizeDetails}
    \begin{subfigure}[h]{\textwidth}\centering\input{figures5/PZ_Master1/error_size_details.tex}\caption{Plains zebra}\end{subfigure}
    \begin{subfigure}[h]{\textwidth}\centering\input{figures5/GZ_Master1/error_size_details.tex}\caption{Grévy's zebra}\end{subfigure}
\end{table}
}


\newcommand{\ErrorGroupDetails}{
\begin{table}[h]
    \centering
    \caption[\caplbl{ErrorGroupDetails}Simulation error group sizes]{\caplbl{ErrorGroupDetails}
    % ---
    Analysis of the error groups in the graph algorithm.
    A merge error group is a set of predicted PCCs that are incorrectly disconnected,  % 
    and a split error group is a set of real PCCs that are incorrectly connected.
    For each case we report the number of error groups, the average number of PCCs in each group, and the average
      size of the smallest and largest PCC in each group.
    % ---
    }
    \label{tbl:ErrorGroupDetails}
    \begin{subfigure}[h]{\textwidth}\centering\input{figures5/PZ_Master1/error_group_details.tex}\caption{Plains zebra}\end{subfigure}
    \begin{subfigure}[h]{\textwidth}\centering\input{figures5/GZ_Master1/error_group_details.tex}\caption{Grévy's zebra}\end{subfigure}
\end{table}
}


%-------------
% Error cases
%-------------

\newcommand{\SplitErrorsPZ}{
    \begin{figure}[h]
    \centering
    %\begin{subfigure}[h]{.7\textwidth}\centering
    \includegraphics[width=\textwidth]{figures5/PZ_Master1/errors/split_14,22_hdupvjwn_edge2134,2492.png}
    %\end{subfigure}
    %\begin{subfigure}[h]{.7\textwidth}\centering\includegraphics[width=\textwidth]{figures5/GZ_Master1/errors/split_4,4_swpbhhis_edge1600,2066.png}\end{subfigure}
    \caption[\caplbl{SplitErrorsPZ}Plains split case due to groundtruth error]{\caplbl{SplitErrorsPZ}
    % ---
    A reported split case from the plains zebra graph simulation.
    Further inspection shows that this split case is due to a groundtruth error.
    The automatic verification algorithm correctly predicted that this pair is positive.
    %The top shows the PCC that should be split into multiple PCCs.
    %Annotations are colored by their groundtruth label.
    %The highlighted edges have labels that disagree with the groundtruth.
    %The bottom shows one the annotations on a highlighted edge in more detail.
    %All $5$ split cases reported for plains zebras are discovered to be due to
    %errors in the groundtruth.
    % ---
    }
    \label{fig:SplitErrorsPZ}
    \end{figure}
}



\newcommand{\SplitErrorsGZ}{
    \begin{figure}[h]
    \centering
    %\begin{subfigure}[h]{.7\textwidth}
    \includegraphics[width=\textwidth]{figures5/GZ_Master1/errors/split_4,19_cymmmggo_edge1582,2371.png}
    %\begin{subfigure}[h]{.7\textwidth}\centering\includegraphics[width=\textwidth]{figures5/GZ_Master1/errors/split_4,4_swpbhhis_edge1600,2066.png}\end{subfigure}
    \caption[\caplbl{SplitErrorsGZ}Grévy's split case due to groundtruth error]{\caplbl{SplitErrorsGZ}
    % ---
    A reported split case from the Grévy's zebra graph simulation.
    Further inspection shows that this split case is due to a groundtruth error.
    The automatic verification algorithm correctly predicted that these annotations should be in the same PCC.
    % ---
    }
    \label{fig:SplitErrorsGZ}
    \end{figure}
}


\newcommand{\MergeErrorPZA}{

    \begin{figure}[h]
    \centering
    \includegraphics[width=\textwidth]{figures5/PZ_Master1/errors/merge_4,19_ybwvbppc_edge3420,3628.png}
    \caption[\caplbl{MergeErrorPZA}Plains merge case due to low probability]{\caplbl{MergeErrorPZA}
    % ---
    A  merge case from the plains zebra graph simulation.
    In this case the ranking algorithm generated an edge that could have connected the PCCs.
    Because the of the low positive probability due to viewpoint and occlusion, the termination criteria stopped
      the algorithm before this edge was reviewed.
    % ---
    }
    \label{fig:MergeErrorPZA}
    \end{figure}
}

\newcommand{\MergeErrorPZB}{
    \begin{figure}[h]
    \centering
    \includegraphics[width=\textwidth]{figures5/PZ_Master1/errors/merge_5,5,11_myqqzsxq_edge16523,17208.png}
    \caption[\caplbl{MergeErrorPZB}Plains merge case due to ranking failure]{\caplbl{MergeErrorPZB}
    % ---
    A  merge case from the plains zebra graph simulation.
    In this case there are three PCCs that should have been merged into one.
    Occlusion and pose variation prevented the ranking algorithm from generating candidate edges.
    However, the positive probability of the selected pair is $0.77$, which means that the pair would likely be
      reviewed if the ranking algorithm had found it.
    % ---
    }
    \label{fig:MergeErrorPZB}
    \end{figure}
}



\newcommand{\MergeErrorGZA}{
    \begin{figure}[h]
    \centering
    \includegraphics[width=\textwidth]{figures5/GZ_Master1/errors/merge_4,11_zzdqqwzs_edge930,1045.png}
    \caption[\caplbl{MergeErrorGZA}Grévy's merge case due to ranking failure]{\caplbl{MergeErrorGZA}
    % ---
    A merge case from the Grévy's zebra graph simulation.
    The ranking algorithm did not generate any candidate edge that could have merged these PCCs due to viewpoint
      variations.
    Even, if the ranking algorithm had selected this edge, the positive probability on the edge would not have
      given it a high enough priority to be reviewed.
    % ---
    }
    \label{fig:MergeErrorGZA}
    \end{figure}
}

\newcommand{\MergeErrorGZB}{
    \begin{figure}[h]
    \centering
    \includegraphics[width=\textwidth]{figures5/GZ_Master1/errors/merge_4,29_vwajntbm_edge2146,2601.png}
    \caption[\caplbl{MergeErrorGZB}Grévy's merge case due to low probability]{\caplbl{MergeErrorGZB}
    % ---
    A merge case from the Grévy's zebra graph simulation.
    The ranking algorithm selected this edge a a candidate, but due to occlusion from scenery and other animals
      the predicted positive probability is low.
    Therefore, the termination criteria stopped the algorithm before this edge was reviewed.
    % ---
    }
    \label{fig:MergeErrorGZB}
    \end{figure}
}

% MERGE GZ
%file:///home/joncrall/latex/crall-thesis-2017/figures5/PZ_Master1/errors/merge_4,14_ggywrmal_edge1913,2487.png
%file:///home/joncrall/latex/crall-thesis-2017/figures5/PZ_Master1/errors/merge_4,19_ybwvbppc_edge3420,3628.png
%file:///home/joncrall/latex/crall-thesis-2017/figures5/PZ_Master1/errors/merge_5,5,11_myqqzsxq_edge16523,17208.png
%file:///home/joncrall/latex/crall-thesis-2017/figures5/PZ_Master1/errors/merge_5,5,23_xwnmudnl_edge17126,17345.png
%file:///home/joncrall/latex/crall-thesis-2017/figures5/PZ_Master1/errors/merge_5,34_ymcmfzhc_edge9962,16092.png

%file:///home/joncrall/latex/crall-thesis-2017/figures5/GZ_Master1/errors/merge_4,11_zzdqqwzs_edge930,1045.png
%file:///home/joncrall/latex/crall-thesis-2017/figures5/GZ_Master1/errors/merge_4,19_xwsktmfx_edge2881,2927.png
%file:///home/joncrall/latex/crall-thesis-2017/figures5/GZ_Master1/errors/merge_4,29_vwajntbm_edge2146,2601.png
