%! TEX root = main.tex
%\documentclass[chap]{thesis}
\documentclass{thesis}

%References: http://tex.stackexchange.com/questions/58098/what-are-all-the-font-styles-i-can-use-in-math-mode
\usepackage[hang]{caption} 
\usepackage[T1]{fontenc}

\usepackage{times}
\usepackage{verbatim} 
\ifusegraphics{}
    \usepackage{graphicx}
\else
    \usepackage[draft]{graphicx}
\fi
\usepackage{amsmath}
\usepackage{mathrsfs} % Fancy script mathscr
\usepackage{amsthm}
\usepackage{amssymb}
\usepackage[usenames,dvipsnames,svgnames,table]{xcolor}

\usepackage{xstring}

\usepackage{mathtools}
\usepackage{algorithm}

\usepackage{bbm}

\usepackage{ulem}
\normalem{}

\usepackage[binary-units=true,group-separator={,}]{siunitx} % gives \num{3002432}

\usepackage{subcaption} % REMOVE?
\captionsetup{compatibility=false}


\newlength{\subfigheight} % REMOVE?
\setlength{\subfigheight}{1in} % REMOVE?

\usepackage[english]{babel} % REMOVE?

\newif\ifreadable{}
\readablefalse{}
\ifreadable{}
    \usepackage[none]{hyphenat} % No Hyphenations
    \raggedright{} % No Justification
    \usepackage[ left=2in, right=2in, top=2in, bottom=2in, paperwidth=8.5in,
    paperheight=16383pt, textwidth=280pt,
    %marginparsep=20pt, marginparwidth=100pt,
    marginpar=1cm,
    textheight=16263pt, footskip=40pt] {geometry} % One Page
\fi
\usepackage[marginpar=2cm] {geometry} 

% Include other packages here, before hyperref.
% If you comment hyperref and then uncomment it, you should delete egpaper.aux
% before re-running latex. 
% http://en.wikibooks.org/wiki/LaTeX/Hyperlinks
\usepackage[backref=page,%
    %pagebackref=true,%
    breaklinks=true,%
    %letterpaper=true,%
    bookmarks=true,%
    bookmarksnumbered=true,%
    bookmarksopen=true,%
    hyperfigures=true,%
    pdfauthor={Jon Crall},%
    colorlinks,
    allcolors=black % % Comment in for thesis submission
]{hyperref}
    

%\usepackage[T1]{fontenc}
\usepackage{blindtext}
\usepackage{needspace} 

\usepackage{ifxetex}
\ifxetex{}
    \usepackage{fontspec}
    %\setmainfont{OpenDyslexic-Regular} %\setmainfont{Arial} %\setmonofont{Lucida Console}
    %\setsansfont{Comic Sans MS} %\setmainfont[Scale=.8]{Mono Dyslexic}
    \setmainfont[Scale=.815]{OpenDyslexic-Regular}
\fi

\usepackage{multirow}

\usepackage{cleveref}
\crefname{subfigure}{subfigure}{subfigures}
\Crefname{subfigure}{Subfigure}{Subfigures}
\crefname{subfigure}{}{}
\Crefname{subfigure}{}{}
% Remove the subfigure from cleveref and add parenthsis
% http://tex.stackexchange.com/questions/170408/how-to-put-numbers-in-parenthesis-using-cleveref
\creflabelformat{subfigure}{(#2#1#3)}
% chktex-file 9
% chktex-file 10
% chktex-file 17
\crefmultiformat{subfigure}{(#2#1#3}{ and~#2#1#3)}{, #2#1#3}{ and~#2#1#3)}
%http://tex.stackexchange.com/questions/17874/cleveref-and-titlesec-section-issues
\crefname{appsec}{appendix}{appendices}
\Crefname{appsec}{Appendix}{Appendices}

\newif\iftempvar{}
%\redundefcitetrue{}
%\tempvarfalse{}
\tempvartrue{}
\iftempvar{}
    % http://tex.stackexchange.com/questions/300022/replace-undefined-reference-with-custom-text/300063#300063
    % chktex-file 26
    \usepackage{etoolbox}
    %\newcommand*{\undeffmt}[1]{\textcolor{red}{ref (#1)}}%
    \newcommand*{\undeffmt}[1]{\textcolor{red}{#1}}%
    \newcommand*{\patchsuccess}[1]{\typeout{Etoolbox patch: Success for \string#1}}
    \newcommand*{\patchfailure}[1]{\typeout{Etoolbox patch: Failure for \string#1}}

\makeatletter

\patchcmd{\@setcref}%
  {\nfss@text{\reset@font\bfseries ??}}
  {\nfss@text{\reset@font\undeffmt{#1}}}
  {\patchsuccess{\@setcref}}
  {\patchfailure{\@setcref}}

\patchcmd{\@setcrefrange}%
  {\nfss@text{\reset@font\bfseries ??}}%
  {\nfss@text{\reset@font\undeffmt{#1}}}
  {\patchsuccess{\@setcrefrange}}
  {\patchfailure{\@setcrefrange}}
\patchcmd{\@setcrefrange}%
  {\nfss@text{\reset@font\bfseries ??}}%
  {\nfss@text{\reset@font\undeffmt{#2}}}
  {\patchsuccess{\@setcrefrange}}
  {\patchfailure{\@setcrefrange}}
\patchcmd{\@setcrefrange}%
  {\nfss@text{\reset@font\bfseries ??}}%
  {\nfss@text{\reset@font\undeffmt{#1}}}
  {\patchsuccess{\@setcrefrange}}
  {\patchfailure{\@setcrefrange}}
\patchcmd{\@setcrefrange}%
  {\nfss@text{\reset@font\bfseries ??}}%
  {\nfss@text{\reset@font\undeffmt{#2}}}
  {\patchsuccess{\@setcrefrange}}
  {\patchfailure{\@setcrefrange}}

\patchcmd{\@setnamecref}%
  {\nfss@text{\reset@font\bfseries ??}}%
  {\nfss@text{\reset@font\undeffmt{#1}}}
  {\patchsuccess{\@setnamecref}}
  {\patchfailure{\@setnamecref}}

\patchcmd{\@setcpageref}%
  {\nfss@text{\reset@font\bfseries ??}}%
  {\nfss@text{\reset@font\undeffmt{#1}}}
  {\patchsuccess{\@setcpageref}}
  {\patchfailure{\@setcpageref}}

\patchcmd{\@setcpagerefrange}%
  {\nfss@text{\reset@font\bfseries ??}}%
  {\nfss@text{\reset@font\undeffmt{#1}}}
  {\patchsuccess{\@setcpagerefrange}}
  {\patchfailure{\@setcpagerefrange}}
\patchcmd{\@setcpagerefrange}%
  {\nfss@text{\reset@font\bfseries ??}}%
  {\nfss@text{\reset@font\undeffmt{#2}}}
  {\patchsuccess{\@setcpagerefrange}}
  {\patchfailure{\@setcpagerefrange}}
\patchcmd{\@setcpagerefrange}%
  {\nfss@text{\reset@font\bfseries ??}}%
  {\nfss@text{\reset@font\undeffmt{#1}}}
  {\patchsuccess{\@setcpagerefrange}}
  {\patchfailure{\@setcpagerefrange}}
\patchcmd{\@setcpagerefrange}%
  {\nfss@text{\reset@font\bfseries ??}}%
  {\nfss@text{\reset@font\undeffmt{#2}}}
  {\patchsuccess{\@setcpagerefrange}}
  {\patchfailure{\@setcpagerefrange}}

\makeatother

\fi

% For space after ensuremath
\usepackage{xspace}
\usepackage{lipsum}

\usepackage{enumerate}
%\usepackage{ifplatform} 

\usepackage[toc,page]{appendix} 
\usepackage[toc,xindy]{glossaries} 

%http://tex.stackexchange.com/questions/60135/path-to-external-files-in-nested-input
\usepackage{import}
\usepackage{rotating}

\usepackage[numbers,sort]{natbib}

\usepackage{silence}
\WarningFilter{latex}{Text page}

\usepackage{booktabs}
\usepackage{listings}

%\usepackage{fontspec}
%\usepackage{minted}
%\setsansfont{Calibri}
%\setmonofont{Consolas}
%\setmonofont{monofur}
%\setmonofont{Inconsolata}

%\usepackage[newfloat=true]{minted}
%\usepackage{shellesc} 
\usepackage[chapter]{minted}
%\usepackage{minted}
\newenvironment{code}{\captionsetup{type=listing}}{}


\usepackage[inline]{enumitem}
\newlist{itemln}{itemize}{3}
\setlist[itemln]{label=\textbullet,noitemsep,nolistsep}

\newlist{enumln}{enumerate}{3}
\setlist[enumln,1]{label=(\arabic*),noitemsep,nolistsep}
\setlist[enumln,2]{label=(\arabic{enumlni}.\arabic*),noitemsep,nolistsep}
\setlist[enumln,3]{label=(\arabic{enumlni}.\arabic{enumlnii}.\arabic*),noitemsep,nolistsep}

\newlist{enumin}{enumerate*}{2}
\setlist[enumin]{label=(\arabic*),noitemsep,nolistsep}


% Upside Down Text
% References;
% http://tex.stackexchange.com/questions/28861/text-upside-down-characters-rotated-along-baseline
\usepackage{forloop}
\newcounter{idx}
\newcounter{posx}
\DeclareRobustCommand{\rotraise}[1]{%
  \StrLen{#1}[\slen]
  \forloop[-1]{idx}{\slen}{\value{idx}>0}{%
    \StrChar{#1}{\value{idx}}[\crtLetter]%
    \IfSubStr{tlQWERTZUIOPLKJHGFDSAYXCVBNM}{\crtLetter}
      {\raisebox{\depth}{\rotatebox{180}{\crtLetter}}}
      {\raisebox{1ex}{\rotatebox{180}{\crtLetter}}}}%
}

%%%%%%%%%%%%%%%
% Cross platform stuff

%https://en.wikipedia.org/wiki/Wikipedia:LaTeX_symbols
\ifwindows{}
\newcommand{\wincommand}[2]{\renewcommand{#1}{#2}}
\newcommand{\lincommand}[2]{\newcommand{#1}{#2}}
\newcommand{\linonlycommand}[2]{}
\else
\newcommand{\wincommand}[2]{\newcommand{#1}{#2}}
\newcommand{\lincommand}[2]{\renewcommand{#1}{#2}}
\newcommand{\linonlycommand}[2]{\newcommand{#1}{#2}}
\fi
%%%%%%%%%%%%%%%
% MACROS
%
\newcommand{\bb}{\mathbb}

%\newcommand{\zspace}{\xspace}
\newcommand{\zspace}{}

\newcommand{\Naive}{Na\"{\i}ve\zspace}
\newcommand{\Naively}{Na\"{\i}vely\zspace}
\newcommand{\naive}{na\"{\i}ve\zspace}
\newcommand{\naively}{na\"{\i}vely\zspace}
\newcommand{\iid}{i.i.d.\zspace}
% WACV 
% Add a period to the end of an abbreviation unless there's one already
\makeatletter
\DeclareRobustCommand\onedot{\futurelet\@let@token\@onedot}
\def\@onedot{\ifx\@let@token.\else.\null\fi\xspace}

\def\eg{\emph{e.g}\onedot} \def\Eg{\emph{E.g}\onedot}
\def\ie{\emph{i.e}\onedot} \def\Ie{\emph{I.e}\onedot}
\def\cf{\emph{c.f}\onedot} \def\Cf{\emph{C.f}\onedot}
\def\etc{\emph{etc}\onedot} \def\vs{\emph{vs}\onedot}
%\def\wrt{w.r.t\onedot}
\def\dof{d.o.f\onedot}
\def\etal{\emph{et al}\onedot}
\makeatother

\newcommand{\wrt}{with respect to\xspace{}}


%\newcommand{\one}{1\zspace{}}
%\newcommand{\two}{2\zspace{}}
%\newcommand{\three}{3\zspace{}}
%\newcommand{\four}{4\zspace{}}
%\newcommand{\five}{5\zspace{}}

\newcommand{\one}{one\zspace{}}
\newcommand{\two}{two\zspace{}}
\newcommand{\three}{three\zspace{}}
\newcommand{\four}{four\zspace{}}
\newcommand{\five}{five\zspace{}}


% Definition of context sensitive articles
% http://stackoverflow.com/questions/4233707/a-an-substitution-in-latex
% http://tex.stackexchange.com/questions/43200/extract-first-last-characters-of-macro-argument
%http://tex.stackexchange.com/questions/132248/test-if-the-first-character-of-a-string-is-a
%\makeatletter
%\newcommand\aan[1]{%
%  a%
%  \@for\@vowel:=a,e,i,o,u,y,A,E,I,O,U,Y\do{%
%    \expandafter\ifx\@vowel#1%
%      n%
%    \fi
%  } % keep this space
%  #1%
%}
%\newcommand\Aan[1]{%
%  A%
%  \@for\@vowel:=a,e,i,o,u,y,A,E,I,O,U,Y\do{%
%    \expandafter\ifx\@vowel#1%
%      n%
%    \fi
%  } % keep this space
%  #1%
%}
%\makeatother


\makeatletter
\newcommand\aan[1]{%
\StrLeft{#1}{1}[\firstchar]%
a\IfStrEq{\firstchar}{a}{n}{%
\IfStrEq{\firstchar}{e}{n}{%
\IfStrEq{\firstchar}{i}{n}{%
\IfStrEq{\firstchar}{o}{n}{%
\IfStrEq{\firstchar}{u}{n}{%
\IfStrEq{\firstchar}{y}{n}{%
\IfStrEq{\firstchar}{A}{n}{%
\IfStrEq{\firstchar}{E}{n}{%
\IfStrEq{\firstchar}{I}{n}{%
\IfStrEq{\firstchar}{O}{n}{%
\IfStrEq{\firstchar}{U}{n}{%
\IfStrEq{\firstchar}{Y}{n}{}}}}}}}}}}}} #1}
\makeatother

\makeatletter
\newcommand\Aan[1]{%
\StrLeft{#1}{1}[\firstchar]%
A\IfStrEq{\firstchar}{a}{n}{%
\IfStrEq{\firstchar}{e}{n}{%
\IfStrEq{\firstchar}{i}{n}{%
\IfStrEq{\firstchar}{o}{n}{%
\IfStrEq{\firstchar}{u}{n}{%
\IfStrEq{\firstchar}{y}{n}{%
\IfStrEq{\firstchar}{A}{n}{%
\IfStrEq{\firstchar}{E}{n}{%
\IfStrEq{\firstchar}{I}{n}{%
\IfStrEq{\firstchar}{O}{n}{%
\IfStrEq{\firstchar}{U}{n}{%
\IfStrEq{\firstchar}{Y}{n}{}}}}}}}}}}}} #1}
\makeatother

\newcommand{\DIM}{D\zspace}
\newcommand{\see}[1]{see \cref{#1}}
\newcommand{\See}[1]{See \cref{#1}}
\newcommand{\Refin}[1]{Referenced in \cref{#1}}
\newcommand{\refin}[1]{referenced in \cref{#1}}
\newcommand{\Imgcite}[1]{Image from~\cite{#1}}

\newcommand{\topic}[1]{\subsubsection{#1}}

%\newcommand{\todo}[1]{\textbf{TODO: #1} \cite{crall2013hotspotterWACV}} 

\renewcommand{\cal}[1]{\mathcal{#1}}
\newcommand{\superscript}[1]{\ensuremath{^{\textrm{#1}}}}
\newcommand{\subscript}[1]{\ensuremath{_{\textrm{#1}}}}
\newcommand{\ds}{\displaystyle}
\newcommand{\dsi}{\displaystyle{} \hspace*{5mm}}
\newcommand{\ind}{\ensuremath{\hspace*{5mm}}}
\newcommand{\txt}[1]{\textrm{#1}}
\newcommand{\txtbf}[1]{\textbf{#1}}
\newcommand{\ttxt}[1]{{\tt #1}}
\renewcommand{\th}{\superscript{th}\zspace}
\newcommand{\rd}{\superscript{rd}\zspace}
\newcommand{\nd}{\superscript{nd}\zspace}
% st is such a weird command. Doesnt seem to want to work unless it is in 
% the document on windows. it is pdfcomment's fault
%\linonlycommand{\st}{\superscript{st}\zspace}
\newcommand{\st}{\superscript{st}\zspace}
\newcommand{\bincase}[1]{
    \ensuremath{
    \begin{cases}
    1 & \txt{if } #1 \\
    0 & \txt{otherwise}
    \end{cases}
    }}
\newcommand{\bincases}[2]{
    \ensuremath{
    \begin{cases}
    #1 & \txt{if } #2 \\
    0 & \txt{otherwise}
    \end{cases}
    }}

\newcommand{\fullbincases}[3]{
    \ensuremath{
    \begin{cases}
    #1 & \txt{if } #2 \\
    #3 & \txt{otherwise}
    \end{cases}
    }}

\newcommand{\tightpad}{\mkern-6mu}
\newcommand{\tight}[1]{\mkern-6mu#1\mkern-6mu }
\newcommand{\ttimes}{\tight{\times}}
\newcommand{\tighteq}{\tight{=}}
\newcommand{\teq}{\tight{=}}

%\newcommand{\slice}[2]{{[#1:#2]}}
\newcommand{\slice}[2]{{[#1\mkern-3mu:\mkern-3mu#2]}}

%%%%%%%%%%%%%%%
% Constants
\newcommand{\PTIME}{{\tt PTIME}}
\newcommand{\TAU}{2\pi}

%%%%%%%%%%%
% MATH MACROS
\newcommand{\paren}[1]{\left(#1\right)}
\newcommand{\brak}[1]{\left[#1\right]}
\newcommand{\curly}[1]{\left\{#1\right\}}
\newcommand{\func}[2]{\mathbin{#1}\paren{#2}}

\newcommand{\crly}[1]{\curly{#1}}
\newcommand{\brk}[1]{\brak{#1}}

%%%%%%%%%%%%%%%
% Operators
% 
% Logic
\newcommand{\assign}{:=}
\newcommand{\eq}{=}
%\newcommand{\where}{\;\big\vert\;}
\newcommand{\whereI}{\;\big\vert\;}
\newcommand{\whereII}{\mid}
\newcommand{\where}{\whereII}
%\newcommand{\where}{\whereI}
%\newcommand{\given}{\;\big\vert\;}
\newcommand{\given}{\mid}
\newcommand*\AND{\txtbf{ and } }
\newcommand*\OR{\txtbf{ or } }
\newcommand*\xor{\mathbin{\oplus}}
\newcommand{\union}{\cup}
\newcommand{\symdiff}{\Delta}
\newcommand{\setdiff}{-}
\newcommand{\Union}{\bigcup}
\newcommand{\intersect}{\cap}
\newcommand{\isect}{\cap}
\renewcommand{\mod}{\txt{mod}}
\newcommand{\Normal}{\cal{N}}
\newcommand{\Real}{\mathbb{R}}
\newcommand{\Int}{\mathbb{Z}}
\newcommand{\eye}{\mat{I}}

\newcommand{\opname}[1]{\operatorname{#1}}

% Probability Operator
%\newcommand{\probop}{\mathbin{\opname{\mathbb{P}}}}
%\newcommand{\probop}{\mathbin{\opname{Pr}}}
\newcommand{\probop}{\mathbin{\opname{P}}}
\newcommand{\oddsop}{\mathbin{\opname{O}}}
%\newcommand{\oddsop}{\mathbin{\opname{odds}}}
%\newcommand{\oddsop}{\mathbin{\mathcal{O}}}
\newcommand{\expectop}{\mathbb{E}}
\newcommand{\parzenop}{\mathbin{\hat{\probop}}}
\newcommand{\Parz}[1]{\parzenop\paren{#1}}
\newcommand{\logop}{\mathbin{\opname{log}}}
\newcommand{\lnop}{\mathbin{\opname{ln}}}
\newcommand{\expop}{\mathbin{\opname{exp}}}
\newcommand{\distri}{\sim}


\RenewDocumentCommand\ln{g}{\IfValueTF{#1}{\logop\paren{#1}}{\logop}}
\NewDocumentCommand\asin{g}{\IfValueTF{#1}{\opname{asin}\paren{#1}}{\opname{asin}}}


% Convolution
\newcommand{\conv}{\mathop{\scalebox{1.5}{\raisebox{-0.2ex}{$\ast$}}}}
\newcommand{\laplace}{\nabla^2} 
\newcommand{\del}{\Delta}
% Probability
%\renewcommand{\Pr}[1]{\ensuremath{\txt{Pr} \left(#1\right)}}
%\renewcommand{\Pr}{\mathbb{P}}
\renewcommand{\Pr}[1]{\func{\probop}{#1}}
\newcommand{\Odds}[1]{\func{\oddsop}{#1}}  % this is logit
\newcommand{\Prs}[2]{\func{\probop_{#1}}{#2}}
\newcommand{\Ex}[1]{\ensuremath{\expectop \left[#1\right]}}
\newcommand{\Err}[1]{\func{\opname{Err}}{#1}}
\newcommand{\ExUnder}[2]{\ensuremath{\underset{#1}{\expectop \left[#2\right]}}}
\newcommand{\ExSub}[2]{\ensuremath{\expectop_{#1} \left[#2\right]}}

\newcommand{\logit}[1]{\func{{\mathbin{\opname{logit}}}}{#1}}
\newcommand{\logitI}[1]{\ensuremath{-\ln{\frac{1}{#1} - 1}}}
\newcommand{\logitII}[1]{\ensuremath{\ln{\frac{#1}{1 - #1}}}}
\newcommand{\expit}[1]{\func{{\mathbin{\opname{expit}}}}{#1}}
\newcommand{\expitI}[1]{\ensuremath{\ln{\frac{1}{1 + \exp{#1}}}}}

% Optimization
\newcommand{\argmax}[1]{\underset{#1}{\opname{argmax}}\;}
\newcommand{\argmin}[1]{\underset{#1}{\opname{argmin}}\;}
\newcommand{\argsort}[1]{\underset{#1}{\opname{argsort}}\;}
\newcommand{\localmax}[1]{\underset{#1}{\opname{localmax}}}
\newcommand{\arglocalmax}[1]{\underset{#1}{\opname{localmax}}}
\newcommand{\arglocalextrema}[1]{\underset{#1}{\opname{argextrema}}}
\newcommand{\argextrema}[1]{\underset{#1}{\opname{argextrema}}}
% Calculation
\newcommand{\expI}[1]{\txt{exp}\paren{#1}}
\newcommand{\expII}[1]{\ensuremath{e^{#1}}}
\renewcommand{\exp}[1]{\expI{#1}}
%\renewcommand{\ln}[1]{\txt{ln}\paren{#1}}

\newcommand{\overbar}[1]{\mkern1.5mu\overline{\mkern-1.5mu#1\mkern-1.5mu}\mkern1.5mu}
%\newcommand{\complement}[1]{\overbar{#1}}
\newcommand{\setcomp}[1]{\overbar{#1}}
\newcommand{\indicator}{\mathbbm{1}}


% Linear Algebra
%\newcommand{\sqrtm}[1]{\opname{sqrtm}(#1)}
\newcommand{\sqrtm}[1]{#1^{\frac{1}{2}}}
\newcommand{\inv}[1]{#1^{-1}}

\newcommand{\tr}{\opname{Tr}}
\renewcommand{\det}{\opname{Det}}
\newcommand{\modop}{\opname{mod}}
\newcommand{\modfn}[2]{#1\; \modop{} \;\,#2}

\newcommand{\trop}{\txt{Tr}}
\newcommand{\detop}{\txt{Det}}

\newcommand{\trfn}[1]{\func{\trop}{#1}}
\newcommand{\detfn}[1]{\func{\detop}{#1}}

\newcommand{\arctantwo}{\opname{arctan2}}
\newcommand{\arctanII}{\opname{arctan2}}
\newcommand{\atantwo}[1]{\func{\arctantwo}{#1}}
\newcommand{\atan}[1]{\func{\arctan}{#1}}

\newcommand{\cov}{\Sigma}
\newcommand{\ltwonormvec}[1]{\frac{#1}{\elltwo{#1}}} 
\newcommand{\card}[1]{|#1|} 
\newcommand{\braket}[2]{\left<#1|#2\right>} 
\newcommand{\bra}[1]{\left<#1|} 
\newcommand{\ket}[2]{|#1\right>} 
%overbar\newcommand{\card}[1]{\txt{card}(#1)} 
%\newcommand{\sqrtm}[1]{#1^(.5)}


%%%%%%%%%%%%%%%
% Variables
% 

% Better ensure math with a space  Actually no. just use \cmd\
%\newcommand{\enmath}[1]{\ensuremath{#1}\zspace}

\newcommand{\mat}[1]{\ensuremath{\mathbf{#1}}} %Should always be capital
\newcommand{\rand}[1]{\ensuremath{\mathbf{#1}}}  %Should always be capital
\newcommand{\set}[1]{\ensuremath{\mathcal{#1}}} 
%\newcommand{\set}[1]{\ensuremath{\mathpzc{#1}}} 
\newcommand{\multiset}[1]{\ensuremath{\mathscr{#1}}} 
%\newcommand{\multiset}[1]{\ensuremath{\mathfrak{#1}}} 
%\newcommand{\multiset}[1]{\ensuremath{\mathrsfs{#1}}} 

%\let\arrowvec\vec % Keep old arrowvec functionality
\renewcommand{\vec}[1]{\ensuremath{\mathbf{#1}}} %Should always be lowercase
%\newcommand{\greekvec}[1]{\ensuremath{\boldsymbol{#1}}} %Should always be lowercase
%\newcommand{\greekvec}[1]{#1}
\newcommand{\greekvec}[1]{\mbox{\boldmath\ensuremath{#1}}}
%
%boldsymbol
% An explicit vector 
%  e.g. \VEC{ x \\ y \\ z}
\newcommand{\VEC}[1]{\ensuremath{
    \Bigl[\negthinspace \begin{smallmatrix} #1
    \end{smallmatrix} \negthinspace \Bigr] }}
\newcommand{\BVEC}[1]{
    \begin{bmatrix}
        #1
    \end{bmatrix}}
\newcommand{\BIGVEC}[1]{\BVEC{#1}}
% An explicit matrix 
% e.g. \MAT{ a & b \\ c & d}
\newcommand{\MAT}[1]{\ensuremath{
    \Bigl[ \begin{smallmatrix} #1
    \end{smallmatrix} \Bigr] }}

\newcommand{\BMAT}[1]{\ensuremath{
    \left[ \begin{matrix} #1
    \end{matrix} \right] }}
\newcommand{\BIGMAT}[1]{\BMAT{#1}}


\newcommand{\REAL}{\mathbb{R}}
%\renewcommand{\Pr}{\color{red} \txtbf{{Pr}} \color{black}}

\newcommand{\ellp}[2]{||#1||_{#2}}
%\newcommand{\elltwo}[1]{\ellp{#1}{2}}
\newcommand{\elltwo}[1]{\ellp{#1}{}}
%\newcommand{\elltwosqrd}[1]{\ellp{#1}{2}^2}
\newcommand{\elltwosqrd}[1]{||#1||^2}
\newcommand{\ellone}[1]{\ellp{#1}{1}}
\newcommand{\ltwo}[1]{||#1||_2}
\newcommand{\lone}[1]{||#1||_1}
\newcommand{\len}[1]{|#1|}
\newcommand{\abs}[1]{|#1|}

%\newcommand{\tau}[1]{|#1|}


%%%%%%%%%%%%%%%%%%
% Common notation
\newcommand{\boldgreek}[1]{\mbox{\boldmath{$ #1 $}}}

\newcommand{\eps}{\epsilon}
\newcommand{\prefers}{\succ}
\newcommand{\preferseq}{\succeq}

%\newcommand{\binom}[2]{{#1 \choose #2}}


\newcommand{\floor}[1]{\left\lfloor{} #1 \right\rfloor}
\newcommand{\bigoh}[1]{\func{\mathcal{O}}{#1}}

%\newcommand{\rotMATRIXtwo}[1]{\MAT{-\sin{#1} & \cos{#1}\\\;\;\;\cos{#1} & \sin{#1}}}
\newcommand{\mspc}{\;\;\;}
\newcommand{\rotMatII}[1]{\MAT{-\sin{#1} & \cos{#1}\\\mspc\cos{#1} & \sin{#1}}}
\newcommand{\rotMatIII}[1]{
    \MAT{
             \cos{#1} &     -\sin{#1} & 0\\
             \sin{#1} & \mspc\cos{#1} & 0\\ 
                    0 &             0 & 1
        }
}
\newcommand{\rotBigMatIII}[1]{
    \BIGMAT{
             \cos{#1} &     -\sin{#1} & 0\\
             \sin{#1} &      \cos{#1} & 0\\ 
                    0 &             0 & 1
        }
}
\newcommand{\invrotMatIII}[1]{
    \MAT{
        \mspc\cos{#1} &      \sin{#1} & 0\\
            -\sin{#1} &      \cos{#1} & 0\\ 
                    0 &             0 & 1
        }
}
%\newcommand{\invrotMatIII}[1]{\MAT{-\sin{#1} & -\cos{#1}  & 0\\-\cos{#1} & \mspc\sin{#1} & 0\\  0 & 0 &  1}}

\newcommand{\transMatIII}[2]{
    \MAT{
        1 & 0 & #1\\
        0 & 1 & #2\\
        0 & 0 & 1
        }
}

\newcommand{\transBigMatIII}[2]{
    \BIGMAT{
        1 & 0 & #1\\
        0 & 1 & #2\\
        0 & 0 & 1
        }
}


\newcommand{\sciE}[1]{\ensuremath{\times 10^{#1}}}
\newcommand{\E}[1]{\sciE{#1}}

\newcommand{\OnTheOrderOf}[1]{\ensuremath{\sim\mkern-6mu 10^{#1}}}

\newcommand{\suchthat}{\quad\txt{s.t. }}
\newcommand{\subjectto}{\suchthat}
%\newcommand{\suchthat}{\txt{ such that }}
%\newcommand{\subjectto}{\txt{ subject to }}

% Elementwise product / Hadamard Product
\newcommand{\elemprod}{\circ}

\newcommand{\concat}[2]{\brak{\paren{#1^T}, \paren{#2^T}}^T}

\newcommand{\percent}{\%}

\newcommand{\degrees}{^{\circ}}
%\renewcommand{\degree}{\degrees}
%\newcommand{\degree}{^{\circ}}


%\newcommand{\baseIdx}{0}
%\newcommand{\lastIdx}[1]{(#1 - 1)}
\newcommand{\baseIdx}{1}
\newcommand{\nextIdx}{2}
\newcommand{\lastIdx}[1]{#1}

\newcommand{\xdotseqIII}[4]{#1_{#2}, \ldots{}, #1_{#3}, \ldots, #1_{#4}}
\newcommand{\xdotseqII}[3]{#1_{#2}, \ldots, #1_{#3}}


\newcommand{\dotsubseq}[3]{#1_{#2}, \ldots{}, #1_{\lastIdx{#3}}}
\newcommand{\dotseqxII}[2]{\xdotseqII{#1}{\baseIdx{}}{\lastIdx{#2}}}
\newcommand{\dotseqIII}[3]{#1_\baseIdx{}, \ldots{}, #1_{#2}, \ldots, #1_{\lastIdx{#3}}}
\newcommand{\dotseqIV}[3]{#1_\baseIdx{}, #1_\nextIdx{} \ldots, #1_{#2}, #1_{#2 + 1}, \ldots, #1_{\lastIdx{#3}}}


\newcommand{\xdotseqIIITwoD}[5]{\xdotseqIII{#1}{\baseIdx,\baseIdx}{#2,#3}{\lastIdx{#4},\lastIdx{#5}}}


\newcommand{\dotseqII}[2]{#1_\baseIdx{}, \ldots{}, #1_{\lastIdx{#2}}}

\newcommand{\dotsubarr}[3]{\brak{\dotsubseq{#1}{#2}{#3}}}
\newcommand{\dotarrII}[2]{\brak{\dotseqxII{#1}{#2}}}
\newcommand{\dotarrIII}[3]{\brak{\dotseqIII{#1}{#2}{#3}}}
\newcommand{\dotarrIV}[3]{\brak{\dotseqIV{#1}{#2}{#3}}}

\newcommand{\dotsubset}[3]{\curly{\dotsubseq{#1}{#2}{#3}}}
\newcommand{\dotsetII}[2]{\curly{\dotseqxII{#1}{#2}}}
\newcommand{\dotsetIII}[3]{\curly{\dotseqIII{#1}{#2}{#3}}}
\newcommand{\dotsetIV}[3]{\curly{\dotseqIV{#1}{#2}{#3}}}

\renewcommand{\emptyset}{\varnothing}


% https://en.wikipedia.org/wiki/Normal_distribution#Definition
%\newcommand{\NormalEqnI}{
%    \frac{1}{\sqrt{\sigma^2 \TAU}}
%    \exp{-\frac{(x - \mu)^2}{2\sigma^2}}
%}

\newcommand{\NormalVarEqnI}[2]{
    \ensuremath{
    \frac{1}{\sqrt{#2^2 \TAU}}
    \exp{\frac{-{(#1)}^2}{2#2^2}}
}
}

\newcommand{\NormalVarConst}[1]{
    \ensuremath{
    \frac{1}{\sqrt{#1^2 \TAU}}
}}

\newcommand{\NormalVarExp}[2]{
    \ensuremath{
    \exp{\frac{-{#1}^2}{2#2^2}}
}}

\newcommand{\NormalVarEqnII}[2]{
    \ensuremath{ \NormalVarConst{#2} \NormalVarExp{#1}{#2} }
}
\newcommand{\NormEqnII}[2]{\NormalVarEqnII{#1}{#2}}

\newcommand{\NormalEqnI}{\NormalVarEqnI{\ensuremath{x - \mu}}{\sigma}}

\newcommand{\StandardNormalI}{
    %\NormalVarEqnI{0}{1}
    %\frac{\expII{-\frac{x^2}{2}}}{\sqrt{\TAU}}
    \ensuremath{\frac{\expI{-(x^2) / 2}}{\sqrt{\TAU}}}
}

\newcommand{\StandardNormalKernelI}[1]{
    %\NormalVarEqnI{0}{1}
    %\frac{\expII{-\frac{x^2}{2}}}{\sqrt{\TAU}}
    \ensuremath{\frac{\expI{-(#1^2) / 2}}{\sqrt{\TAU}}}
}

\newcommand{\ParzenEqn}{
    \ensuremath{\frac{1}{nh} \sum_{i=1}^n \StandardNormalKernelI{(\frac{x - x_i}{h})}}
}

% https://en.wikipedia.org/wiki/Multivariate_normal_distribution
\newcommand{\NormalEqnII}{
    \ensuremath{
        \frac{1}{\sqrt{\detfn{\Sigma} {(\TAU)}^k}}
        \exp{-\frac{{(\vec{x} - \greekvec{\mu})}^T \Sigma^{-1} (\vec{x} - \greekvec{\mu})}{2}}
    }}


%\newcommand{\fboxII}[1]{\fbox{#1}}
\newcommand{\fboxII}[1]{#1}


% ------------


\newcommand{\ImageFigureDraft}[4]{ 
    \begin{figure}[ht!]
    \centering 
    \fboxII{\includegraphics[draft,width=#2]{#1}}
    \caption{\caplbl{#4}#3}\label{fig:#4} 
    \end{figure} 
} 


\makeatletter
\newcommand{\ImageCommandDraft}[4]{%
    \expandafter\newcommand\csname #4\endcsname{\ImageFigureDraft{#1}{#2}{#3}{#4}}%
}
\makeatother



\newcommand{\ImageFigure}[4]{ 
    \begin{figure}[ht!]
    \centering 
    \fboxII{\includegraphics[width=#2]{#1}}
    \caption{\caplbl{#4}#3}\label{fig:#4} 
    \end{figure} 
} 

\makeatletter
\newcommand{\ImageCommand}[4]{%
    \expandafter\newcommand\csname #4\endcsname{\ImageFigure{#1}{#2}{#3}{#4}}%
}
\makeatother


\newcommand{\CaptionedImageFigure}[5]{ 
    \begin{figure}[ht!]
    \centering 
    \fboxII{\includegraphics[width=#2]{#5}}
    \caption[\caplbl{#1}#3]{\caplbl{#1}#4}\label{fig:#1} 
    \end{figure} 
} 

\makeatletter
% {label}{textwidth_percent}{shortcap}{caption_str}{fpath}
\NewDocumentCommand\SingleImageCommand{mmmmm}{%
    \expandafter\newcommand\csname #1\endcsname{\CaptionedImageFigure{#1}{#2\textwidth}{#3}{#4}{#5}}
}
\makeatother




% ----------
% Super hacky def for multiple fig stuffs

%http://tex.stackexchange.com/questions/132956/reduce-space-between-subfigure-and-the-subfigure-captions

% TODO: change to use keyval arguments for multifigure
% http://tex.stackexchange.com/questions/180147/newcommand-for-tabular-entries-using-keyval

% command to specify up to 4 subfigures
% FIXME: if one of the IfValueTF goes to its false case it leaves an ungly space between the figure and the caption
% Not sure how to fix this robustly. Adding \vspace{-1\baselineskip} in the last else seems to help some.
% chktex-file 39
\NewDocumentCommand\MultiImageFigure{mmmgggg}{ 
    \begin{figure}[ht!]
        \centering 
        \IfValueTF{#4}{
            \begin{subfigure}[h]{#2\textwidth}
                \centering
                \fboxII{\includegraphics[width=\textwidth]{#4}}\caption{}\label{sub:#1A}\end{subfigure}
           ~~%
        }{}
        \IfValueTF{#5}{
            \begin{subfigure}[h]{#2\textwidth}
                \centering
                \fboxII{\includegraphics[width=\textwidth]{#5}}\caption{}\label{sub:#1B}\end{subfigure}
            ~~%
        }{}
        \IfValueTF{#6}{
            \begin{subfigure}[h]{#2\textwidth}
                \centering
                \fboxII{\includegraphics[width=\textwidth]{#6}}\caption{}\label{sub:#1C}\end{subfigure}
            ~~%
        }{}
        \IfValueTF{#7}{
            \begin{subfigure}[h]{#2\textwidth}
                \centering
                \fboxII{\includegraphics[width=\textwidth]{#7}}\caption{}\label{sub:#1D}\end{subfigure}
            ~~%
        }{
            \vspace{-.5\baselineskip}
        }
        %\IfValueTF{#8}{
        %    \begin{subfigure}[h]{#2\textwidth}
        %        \centering
        %        \fboxII{\includegraphics[width=\textwidth]{#8}}\caption{}\label{sub:#1E}\end{subfigure}
        %    ~~%
        %}{}
        %\IfValueTF{#9}{
        %    \begin{subfigure}[h]{#2\textwidth}
        %        \centering
        %        \fboxII{\includegraphics[width=\textwidth]{#9}}\caption{}\label{sub:#1F}\end{subfigure}
        %    ~~%
        %}{}
        %\IfValueTF{#10}{
        %    \begin{subfigure}[h]{#2\textwidth}
        %        \centering
        %        \fboxII{\includegraphics[width=\textwidth]{#10}}\caption{}\label{sub:#1G}\end{subfigure}
        %    ~~%
        %}{}
        %\IfValueTF{#11}{
        %    \begin{subfigure}[h]{#2\textwidth}
        %        \centering
        %        \fboxII{\includegraphics[width=\textwidth]{#11}}\caption{}\label{sub:#1H}\end{subfigure}
        %    ~~%
        %}{}
        %\IfValueTF{#12}{
        %    \begin{subfigure}[h]{#2\textwidth}
        %        \centering
        %        \fboxII{\includegraphics[width=\textwidth]{#12}}\caption{}\label{sub:#1I}\end{subfigure}
        %    ~~%
        %}{}
%\caption[#3]{\caplbl{#1}#3} 
\caption{\caplbl{#1}#3}  % TODO: use first sentence as the short title.
\label{fig:#1} 
    \end{figure} 
} 

% Set up the keys.  Only the ones directly under /myparbox
% can be accepted as options to the \myparbox macro.
%http://tex.stackexchange.com/questions/34312/how-to-create-a-command-with-key-values
%\pgfkeys{
% /MultiImageFigure/.is family, /MultiImageFigure,
% % Here are the options that a user can pass
% default/.style = 
%  {width = \textwidth, height = \baselineskip},
% width/.estore in = \myparboxWidth,
% height/.estore in = \myparboxHeight,
%}


\NewDocumentCommand\MultiImageFigureII{mmmmgggg}{ 
    \begin{figure}[ht!]
        \centering 
        \IfValueTF{#5}{
            \begin{subfigure}[h]{#2\textwidth}
                \centering
                \fboxII{\includegraphics[width=\textwidth]{#5}}\caption{}\label{sub:#1A}\end{subfigure}
        }{}
        \IfValueTF{#6}{
            \begin{subfigure}[h]{#2\textwidth}
                \centering
                \fboxII{\includegraphics[width=\textwidth]{#6}}\caption{}\label{sub:#1B}\end{subfigure}
        }{}
        \IfValueTF{#7}{
            \begin{subfigure}[h]{#2\textwidth}
                \centering
                \fboxII{\includegraphics[width=\textwidth]{#7}}\caption{}\label{sub:#1C}\end{subfigure}
        }{
            \vspace{-.5\baselineskip}
        }
        \IfValueTF{#8}{
            \begin{subfigure}[h]{#2\textwidth}
                \centering
                \fboxII{\includegraphics[width=\textwidth]{#8}}\caption{}\label{sub:#1D}\end{subfigure}
        }{}
\caption[\caplbl{#1}#3]{\caplbl{#1}#4}  % TODO: use first sentence as the short title.
\label{fig:#1} 
    \end{figure} 
} 


\NewDocumentCommand\MultiImageFigureDraft{mmmgggg}{ 
    \begin{figure}[ht!]
        \centering 
        \IfValueTF{#4}{
            \begin{subfigure}[h]{#2\textwidth}
                \centering
                \fboxII{\includegraphics[draft,width=\textwidth]{#4}}\caption{}\label{sub:#1A}\end{subfigure}
           ~~%
        }{}
        \IfValueTF{#5}{
            \begin{subfigure}[h]{#2\textwidth}
                \centering
                \fboxII{\includegraphics[draft,width=\textwidth]{#5}}\caption{}\label{sub:#1B}\end{subfigure}
            ~~%
        }{}
        \IfValueTF{#6}{
            \begin{subfigure}[h]{#2\textwidth}
                \centering
                \fboxII{\includegraphics[draft,width=\textwidth]{#6}}\caption{}\label{sub:#1C}\end{subfigure}
            ~~%
        }{}
        \IfValueTF{#7}{
            \begin{subfigure}[h]{#2\textwidth}
                \centering
                \fboxII{\includegraphics[draft,width=\textwidth]{#7}}\caption{}\label{sub:#1D}\end{subfigure}
            ~~%
        }{
            \vspace{-.5\baselineskip}
        }
        %\IfValueTF{#8}{
        %    \begin{subfigure}[h]{#2\textwidth}
        %        \centering
        %        \fboxII{\includegraphics[width=\textwidth]{#8}}\caption{}\label{sub:#1E}\end{subfigure}
        %    ~~%
        %}{}
        %\IfValueTF{#9}{
        %    \begin{subfigure}[h]{#2\textwidth}
        %        \centering
        %        \fboxII{\includegraphics[width=\textwidth]{#9}}\caption{}\label{sub:#1F}\end{subfigure}
        %    ~~%
        %}{}
        %\IfValueTF{#10}{
        %    \begin{subfigure}[h]{#2\textwidth}
        %        \centering
        %        \fboxII{\includegraphics[width=\textwidth]{#10}}\caption{}\label{sub:#1G}\end{subfigure}
        %    ~~%
        %}{}
        %\IfValueTF{#11}{
        %    \begin{subfigure}[h]{#2\textwidth}
        %        \centering
        %        \fboxII{\includegraphics[width=\textwidth]{#11}}\caption{}\label{sub:#1H}\end{subfigure}
        %    ~~%
        %}{}
        %\IfValueTF{#12}{
        %    \begin{subfigure}[h]{#2\textwidth}
        %        \centering
        %        \fboxII{\includegraphics[width=\textwidth]{#12}}\caption{}\label{sub:#1I}\end{subfigure}
        %    ~~%
        %}{}
%\caption[#3]{\caplbl{#1}#3} 
\caption{\caplbl{#1}#3}  % TODO: use first sentence as the short title.
\label{fig:#1} 
    \end{figure} 
} 

\makeatletter
\NewDocumentCommand\MultiImageCommand{mmmgggg}{%
    \expandafter\newcommand\csname #1\endcsname{\MultiImageFigure{#1}{#2}{#3}{#4}{#5}{#6}{#7}}
    %\expandafter\newcommand\csname #1\endcsname{\MultiImageFigure{#1}{.1}{#3}{#4}{#5}{#6}{#7}}
}
\makeatother


\makeatletter
\NewDocumentCommand\MultiImageCommandII{mmmmgggg}{%
    \expandafter\newcommand\csname #1\endcsname{\MultiImageFigureII{#1}{#2}{#3}{#4}{#5}{#6}{#7}{#8}}
    %\expandafter\newcommand\csname #1\endcsname{\MultiImageFigure{#1}{.1}{#3}{#4}{#5}{#6}{#7}}
}
\makeatother

%\expandafter\newcommand\csname #1\endcsname{\MultiImageFigure{#1}{#2}{#3}{#4}{#5}{#6}{#7}{#8}{#9}{#10}{#11}{#12}}


% chktex-file 9
\def\lbrak{[}
\def\rbrak{]}
\newcommand{\rangeexex}[1]{\lparen#1\rparen}
\newcommand{\rangeinex}[1]{\lbrak#1\rparen} 
\newcommand{\rangeexin}[1]{\lparen#1\rbrak}
\newcommand{\rangeinin}[1]{\lbrak#1\rbrak}

%\newcommand{\rangeexex}[1]{(#1)}
%\newcommand{\rangeinex}[1]{[#1)} 
%\newcommand{\rangeexin}[1]{(#1]}
%\newcommand{\rangeinin}[1]{[#1]}

%\newcommand{\getitem}[1]{[#1]}
\newcommand{\getitem}[1]{\ensuremath{_{#1}}}


\newcommand{\gammav}{\greekvec{\gamma}}
\newcommand{\deltav}{\greekvec{\delta}}
\newcommand{\onevec}{\vec{1}}
\newcommand{\partfrac}[2]{\frac{\partial{} #1}{\partial{} #2}}

\newcommand{\rarrow}{\ensuremath{\rightarrow}}

%\newcommand{\rmultiarrow}{\ensuremath{*\mkern-10mu\rightarrow}}
%\newcommand{\rmultiarrow}{*\rightarrow}
%\newcommand{\rmultiarrow}{*\mskip-10mu\rightarrow}
\newcommand{\rmultiarrow}{\ensuremath{\overset{*}{\rarrow}}}


\begin{comment}

https://www.cs.rpi.edu/academics/grad/phd.html
http://gradoffice.rpi.edu/update.do
http://www.rpi.edu/dept/grad/docs/Dissertation%20Checklist.pdf
http://www.rpi.edu/dept/grad/docs/Thesis%20Manual.pdf

Submit here: http://www.etdadmin.com/cgi-bin/school?siteId=489

    rsync -arvhzP --include='figures*/***' --exclude='*' joncrall@hyrule.cs.rpi.edu:latex/crall-thesis-2017/ ~/latex/crall-thesis-2017
    rsync -arvhzP --include='figures*/***' --exclude='*' joncrall@lev.cs.rpi.edu:latex/crall-thesis-2017/ ~/latex/crall-thesis-2017

    rsync -arvhzP joncrall@hyrule.cs.rpi.edu:latex/crall-thesis-2017/figures5/PZ_Master1/dbstats.tex ~/latex/crall-thesis-2017/figures5/PZ_Master1/dbstats.tex

    rsync -arvhzP --include='figures*/***' --exclude='*' ~/latex/crall-thesis-2017 joncrall@hyrule.cs.rpi.edu:latex/crall-thesis-2017/ 

    rsync -arvhzP ~/latex/crall-thesis-2017/figures2/ joncrall@hyrule.cs.rpi.edu:latex/crall-thesis-2017/figures2
    rsync -arvhzP ~/latex/crall-thesis-2017/figures1/ joncrall@hyrule.cs.rpi.edu:latex/crall-thesis-2017/figures1

    ./texfix.py --fpaths main.tex --outline --asmarkdown --numlines=999 -w --ignoreinputstartswith=def,Crall,header,colordef,figdef
    ./texfix.py --fpaths main.tex --outline --numlines=0 -w --ignoreinputstartswith=def,Crall,header,colordef,figdef
    ./texfix.py --fpaths main.tex --outline --numlines=0 --ignoreinputstartswith=def,Crall,header,colordef,figdef
\end{comment}

\author{Jonathan P. Crall}

\thesistitle{Identifying Individual Animals using Ranking, Verification, and Connectivity}        

\degree{Doctor of Philosophy}        
\department{Computer Science} 
     
\signaturelines{4}     %max number of signature lines is 7        
\thadviser{Dr.\ Charles Stewart}
\memberone{Dr.\ Barbara Cutler}        
\membertwo{Dr.\ Bülent Yener}        
\memberthree{Dr.\ Richard Radke}

\submitdate{June 2017\\(For Graduation August 2017)}        
%\submitdate{June 2016}
%\copyrightyear{2017}   % if omitted, current year is used.        


\begin{document}

\titlepage{}

%\copyrightpage{}         % optional           
\tableofcontents{}        

\listoftables{}          % required if there are tables
\listoffigures{}         % required if there are figures

\specialhead{ACKNOWLEDGMENT}

% http://matt.might.net/articles/phd-school-in-pictures/
Over the past seven years, I have devoted nearly all of my time and energy into learning the skills and
  developing the systems that have allowed me to make this contribution.
I would've been unable to achieve this feat without the support and guidance of others.

First and foremost, I must thank my partner: Dana Cardona.
Her love, support, and feedback has motivated and sustained me during my time as a graduate student.
In our five years together I have found more stability and purpose than I thought was possible.

I am grateful to my parents for their ongoing encouragement and for fostering my curiosity.
Thank you for all that you have done and continue to do.

Also, to my friend Lucas Cotterell:
it was in a conversation with Luc that I decided that I could and would pursue a doctorate degree.
His long-lasting friendship has helped me achieve this goal.

Next, I thank my labmates Peter Honig, Zach Jablons, Jason Parham, and Hendrik Weideman.
Our discussions have been invaluable and are responsible for a significant portion of my understanding of
  computer vision, machine learning, and neural networks.

I graciously thank my advisor --- Chuck Stewart --- whose guidance helped me navigate my research, grow
  professionally and personally, and whose writing style has caused me to develop a fondness for em dashes.

To my committee members:
Barb Cutler, Rich Rake, and Bülent Yener, whose suggestions, comments, and challenges successfully pushed me from
  my candidacy to the completion of my dissertation.
I am thankful to them and to the other RPI professors who have taken the time to share their insight and discuss
  research problems with me.

I am immensely grateful to Bill and Naomi Hoffman for all they have done to open this path for me.
I am filled with appreciation for them and the many others at Kitware who acted as teachers and mentors to me.


\specialhead{ABSTRACT}

In this thesis we address the problem of identifying individual animals using images in the context of assisting
  an ecologist in performing a population census.
We are motivated by events like the ``Great Zebra Count'' where thousands of images of zebras and giraffes were
  collected in Nairobi National Park over two days.
By grouping all images that contain the same individual we can census these populations.
This problem is challenging because images are collected outdoors and contain occlusion, lighting, and quality
  variations and because the animals exhibit viewpoint and pose variations.

Our first contribution is an algorithm that ranks a database of images by their similarity to a query.
A manual reviewer inspects only the top few results for each query --- significantly reducing the search space
  --- and determines if the animals match.
Using this algorithm alone, we analyzed the images from the Great Zebra Count and performed a population census.
Our second contribution is a verification algorithm that determines the probability that two images are from the
  same animal, that they are not, or that there is not enough to decide.
This algorithm is used with the ranking algorithm to re-rank results and automatically verify high confidence
  image pairs.

Our third contribution is a semi-automatic graph identification algorithm.
The approach represents each image as a node in the graph and incrementally forms edges between nodes determined
  to the same animal.
The ranking and verification algorithms are used to search for candidate edges and estimate their probability of
  matching.
Based on these probabilities, edges are prioritized for review and placed in the graph when they are
  automatically verified or manually reviewed.
Redundant connections are added to detect and recover from errors.
A termination criterion determines when identification is finished.
Using the graph algorithm we perform a population census on the scale of the Great Zebra Count using less than
  $25\%$ of the manual reviews required by the original method.

%\newcommand{\GZC}{GZGC\zspace}
%\newcommand{\GZCFull}{Great Zebra and Giraffe Count\zspace}

%\newcommand{\zspace}{\xspace}
%\newcommand{\zspace}{}

\newcommand{\GZC}{GZC\zspace}
\newcommand{\GGR}{GGR\zspace}
\newcommand{\GGRFull}{\txt{Great Grevys Rally}}
\newcommand{\GZCFull}{Great Zebra Count\zspace}

\newcommand{\groundfalse}{incorrect}
\newcommand{\groundtrue}{correct}
\newcommand{\gtname}{\groundtrue{} \name{}}
\newcommand{\gtnames}{\groundtrue{} \names{}}


%%%% TERMINOLOGY %%%%%%

\newcommand{\eyetwo}{\MAT{1 & 0\\0 & 1}}

%\newcommand{\nsum}{``name sum''}
%\newcommand{\gridcov}{``grid coverage''}

\newcommand{\occurrence}{occurrence\zspace}
\newcommand{\Occurrence}{Occurrence\zspace}
\newcommand{\occurrences}{\occurrence{}s\zspace}
\newcommand{\Occurrences}{\Occurrence{}s\zspace}

\newcommand{\intraoccurrence}{intra-\occurrence{}\zspace}
\newcommand{\Intraoccurrence}{Intra-\occurrence{}\zspace}
%\newcommand{\vsexemplar}{vs-exemplar\zspace}
%\newcommand{\encountermatching}{\encounter{} matching\zspace}
\newcommand{\vsexemplar}{master database\zspace}
\newcommand{\masterdatabase}{master database\zspace}
\newcommand{\Masterdatabase}{Master database\zspace}
\newcommand{\Vsexemplar}{Master database\zspace}

%\newcommand{\mastername}{master-name\zspace}
%\newcommand{\mastername}{marked-individual\zspace}
%\newcommand{\mastername}{marked-individual\zspace}
\newcommand{\mastername}{master-name\zspace}
\newcommand{\exemplar}{exemplar\zspace}
%\newcommand{\occurrencename}{occurrence-name\zspace}
\newcommand{\tempname}{temporary-name\zspace}
\newcommand{\tempnames}{temporary-names\zspace}
\newcommand{\encounter}{encounter\zspace}
\newcommand{\occurrencename}{encounter\zspace}
\newcommand{\tempexemplar}{temporary-exemplar\zspace}
\newcommand{\annotscore}{annotation score\zspace}
\newcommand{\namescore}{name score\zspace}
%\newcommand{\annotscore}{\fbox{annotation score}\zspace}
%\newcommand{\namescore}{\fbox{name score}\zspace}

\newcommand{\annot}{annotation\zspace}
\newcommand{\annots}{annotations\zspace}

\newcommand{\correspondence}{correspondence\zspace}
%\newcommand{\match}{match\zspace}
%\newcommand{\matches}{\match{}es\zspace}
%\newcommand{\matching}{\match{}ing\zspace}

\newcommand{\matchvec}{match vector\zspace}
\newcommand{\matchvecs}{\matchvec\zspace}

\newcommand{\name}{name\zspace}
\newcommand{\Name}{Name\zspace}


% Matches when talking about names
\newcommand{\nmatch}{name match\zspace}

% Matches when talking about features
\newcommand{\fmatch}{feature match\zspace}
\newcommand{\Fmatch}{Feature match\zspace}

% Plural versions
\newcommand{\occurrencenames}{\occurrencename{}s\zspace}
\newcommand{\encounters}{\encounter{}s\zspace}
\newcommand{\exemplars}{\exemplar{}s\zspace}
\newcommand{\masternames}{\mastername{}s\zspace}
\newcommand{\tempexemplars}{\tempexemplar{}s\zspace}
\newcommand{\namescores}{\namescore{}s\zspace}
\newcommand{\annotscores}{\annotscore{}s\zspace}
\newcommand{\correspondences}{\correspondence{}s\zspace}
\newcommand{\names}{\name{}s\zspace}

%\newcommand{\masterexemplar}{master-exemplar\zspace}
%\newcommand{\masterexemplars}{\masterexemplar{}s\zspace}



% mood='indicative', aspect='imperfect'
\newcommand{\namescoring}{name scoring\zspace}
\newcommand{\corresponding}{corresponding\zspace}


\newcommand{\Nsumprefix}{Feature-based\zspace}
\newcommand{\nsumprefix}{feature-based\zspace}
\newcommand{\csumprefix}{annotation-based\zspace}
\newcommand{\nscore}{\nsumprefix{} \namescore{}\zspace}
\newcommand{\cscore}{\csumprefix{} \namescore{}\zspace}
\newcommand{\nscoring}{\nsumprefix{} \namescoring{}\zspace}
\newcommand{\cscoring}{\csumprefix{} \namescoring{}\zspace}


\newcommand{\pvar}[1]{{\tt{#1}}\zspace}
%\newcommand{\pvar}[1]{#1\zspace}

%\renewcommand{\nsum}{\pvar{nsum}}
%\newcommand{\csum}{\pvar{csum}}
\newcommand{\nsum}{\pvar{fmech}}
\newcommand{\csum}{\pvar{amech}}

%\newcommand{\fmatching}{\fmatch{}ing\zspace}
%\newcommand{\fmatches}{\fmatch{}es\zspace}
%\newcommand{\fmatching}{\fmatch{}ing\zspace}
%\newcommand{\Fmatches}{\Fmatch{}es\zspace}
%\newcommand{\Fmatching}{\Fmatch{}ing\zspace}

%\newcommand{\namescore}{name score\zspace}

% http://tex.stackexchange.com/questions/28704/defining-a-newcommand-with-variable-name-inside-another-newcommand
%\makeatletter
%\newcommand{\pluralize}[1]{%
%    %\expandafter\newcommand\csname #1s\endcsname{#1{}s}%
%    \@namedef{#1s}{\\#1s\zspace}
%}
%\makeatother
%\expandafter\newcommand\csname #1s\endcsname{#1{}s}%
%\pluralize{namescore}


% TODO: http://tex.stackexchange.com/questions/259287/how-to-define-a-newcommand-that-expands-into-another-newcommand/259293#259293



%%%%%%%% MATH %%%%%

    \newcommand{\kp}{\vec{p}}
    \newcommand{\desc}{\vec{d}}
    \newcommand{\feat}{f}
    %\newcommand{\name}{{\tt{nid}}}
    %\newcommand{\nid}{{\tt{nid}}}

    %\newcommand{\nidnew}{\nid_{\tt{new}}}
    \newcommand{\nid}{\set{N}}
    \newcommand{\nids}{\multiset{N}}
    %\newcommand{\nid}{\set{C}}
    %\newcommand{\nids}{\multiset{C}}
    \newcommand{\notnid}{\setcomp{\nid}}
    \newcommand{\nidnew}{\nid_{\tt{new}}}

    \newcommand{\aid}{\X}
    \newcommand{\qaid}{\X}
    \newcommand{\daid}{\Y}

    %\newcommand{\aid}{{\tt{aid}}}
    %\newcommand{\qaid}{{\tt{qaid}}}
    %\newcommand{\daid}{{\tt{daid}}}
    %\newcommand{\annot}{{\tt{aid}}}
    %\newcommand{\qannot}{{\tt{qaid}}}
    %\newcommand{\dannot}{{\tt{daid}}}
    \newcommand{\Feats}{\set{F}}
    \renewcommand{\fg}{\tt{fg}}
    \newcommand{\img}{I}
    \newcommand{\rawimg}{I_{\tt{raw}}}
    %\newcommand{\pureimg}{I_{\tt{continuous}}}
    \renewcommand{\K}{\ensuremath{K}}
    %
    % TODO: Chuck says dont use subscripts as name

    %\newcommand{\norm}{_{\tt{norm}}}
    %\newcommand{\Knorm}{\ensuremath{K_{\norm}}}
    \newcommand{\Knorm}{\ensuremath{\K^{*}}}
    \newcommand{\descnorm}{\ensuremath{\desc^{*}}}
    %\newcommand{\descnorm}{\ensuremath{\desc_{\norm}}}

    \newcommand{\query}{\tt{query}}
    \newcommand{\data}{\tt{data}}
    %\newcommand{\NN}{\tt{NN}}
    \newcommand{\NN}{\opname{NN}}

    %\newcommand{\NormDB}{\multiset{D}_{\tt{norm}}}
    \newcommand{\AnyDB}{\set{D}}
    \newcommand{\NormDB}{\set{D}_{\tt{norm}}}
    \newcommand{\ExempDB}{\set{D}_{\tt{exemp}}}
    \newcommand{\OccurDB}{\set{D}_{\tt{occurr}}}

    \newcommand{\LNBNN}{\tt{LNBNN}}
    \newcommand{\Matches}{\set{M}}
    % Keypoint Notation
    \newcommand{\kpts}{\set{P}}
    \newcommand{\pt}{\vec{x}}
    \newcommand{\x}{\vec{x}}
    \newcommand{\ptcolvec}{\VEC{ x\\ y}}
    \newcommand{\scale}{\sigma}

    \newcommand{\vmat}{\mat{A}}
    \newcommand{\invvmat}{\inv{\vmat}}
    % invV ALWAYS maps from u-circle onto an ellipse
    %    V ALWAYS maps from ellipse onto the u-circle
    % ellmat should be the the invV shape matrix. 
    % but in section 2 I use it as ellmat as V, FUUUU. Fixed...
    \newcommand{\vMATRIX}{\MAT{a & 0\\c & d}}

    \newcommand{\VMatII}{\MAT{a & 0\\c & d}}
    \newcommand{\VMatIII}{\MAT{a & 0 & 0\\c & d & 0 \\ 0 & 0 & 1}}
    \newcommand{\VBigMatIII}{\BIGMAT{a & 0 & 0\\c & d & 0 \\ 0 & 0 & 1}}

    \newcommand{\ellMatrixIII}{\MAT{a & 0 & x\\c & d & y\\ 0 & 0 & 1}}


    \newcommand{\ellMATRIXTres}{\MAT{a & 0 & 0\\c & d & 0\\0 & 0 & 1}}
    \newcommand{\transMATTres}[2]{\MAT{1 & 0 & #1\\0 & 1 & #2\\0 & 0 & 1}}
    %\newcommand{\rotMATTres}[1]{\MAT{-\sin{#1} & \cos{#1} & 1 \\ \cos{#1} & \sin{#1} & 0 \\0 & 0 & 1}}
    %%% 
    % SV definitions
    \newcommand{\aff}{{\tt aff}}
    \newcommand{\HmgMat}{\mat{H}}

    %\newcommand{\AffMat}{\mat{H}_{\aff}}
    %\newcommand{\AffMatij}{\mat{H}_{\aff, i, j}}

    %\newcommand{\rvmat}{\circled{\mat{A}}}
    \newcommand{\rvmat}{\mat{B}}
    \newcommand{\invvrmat}{\inv{\rvmat}}

    \newcommand{\ori}{\theta}
    \newcommand{\momentmat}{\ensuremath{\mat{M}(\pt, \scale)}}
    \newcommand{\warpedmomentmat}[1]{\ensuremath{\mat{M}(#1\pt, \scale)}}
    \newcommand{\hessMAT}{\ensuremath{\mat{H}(\pt, \scale)}}
    \newcommand{\momentmatNOARG}{\mat{M}}



    % SMK
    \newcommand{\X}{\set{X}}
    \newcommand{\Y}{\set{Y}}
    \newcommand{\C}{\set{C}}
    %\newcommand{\K}{K}
    %\newcommand{\V}{V}
    \newcommand{\Vh}{\hat{V}}
    %\renewcommand{\M}{M}
    \newcommand{\M}{M}




%\newcommand{\devcomment}[1]{\darkgreen \emph{/* --- #1 --- */} \black}
%\newcommand{\chuckcomment}[1]{\chuckblue \emph{/* --- #1 --- */} \black}
%\renewcommand{\textwidth}{4in}
%\renewcommand{\caption}[1]{}

\newcommand{\glossterm}[1]{\emph{#1}}
%\newcommand{\glossterm}[1]{\textbf{#1}}


% For concepts directly relating to a code implementation
\newcommand{\codeobj}[1]{\emph{#1}}
%\newcommand{\codeobj}[1]{#1}
\newcommand{\coderef}[1]{#1}

\newcommand{\OrigCref}[1]{\Cref{#1}}
\renewcommand{\cref}[1]{\Cref{#1}}



\renewcommand{\c}{\vec{c}}
\newcommand{\support}{\tt{support}}
%\newcommand{\support}{\tt{gt}}

\newcommand{\TP}{{\tt{TP}}}
\newcommand{\TN}{{\tt{TN}}}
\newcommand{\FN}{{\tt{FP}}}
\newcommand{\FP}{{\tt{FN}}}
\newcommand{\KDE}{{\opname{KDE}}}
\newcommand{\KDEop}{{\opname{KDE}}}
%\newcommand{\TP}{y\tighteq1}
%\newcommand{\TN}{y\tighteq-1}
%\newcommand{\FP}{\tt{FP}}
%\newcommand{\FN}{\tt{TN}}

\newcommand{\TPsupport}{\set{P}}
\newcommand{\TNsupport}{\set{N}}


%\newcommand{\eqv}{=}
\newcommand{\eqv}{\equiv}


\newcommand{\visitset}{\set{C}}
%\newcommand{\visitset}{\set{V}}
\newcommand{\poptotal}{T}
\newcommand{\nvisit}{c}
\newcommand{\resight}{m}


\newcommand{\Photobomb}{Photobomb\xspace}
\newcommand{\photobomb}{photobomb\xspace}
\newcommand{\photobombing}{photobombing\xspace}
\newcommand{\Photobombing}{Photobombing\xspace}
\newcommand{\Photobombings}{Photobombings\xspace}
\newcommand{\photobombings}{photobombings\xspace}


%\newcommand{\keywords}[1]{\noindent\textbf{\emph{Index terms}} --- #1}
%\newcommand{\relatedto}[1]{\noindent\textbf{\emph{Related to}} --- #1}
%\newcommand{\outline}[1]{\noindent\textbf{\emph{Outline}} --- {\emph{#1}}}
\newcommand{\keywords}[1]{}
\newcommand{\relatedto}[1]{}
\newcommand{\outline}[1]{}

\renewcommand{\E}{\opname{E}}  % Energy


\renewcommand{\F}{\Phi}
\renewcommand{\E}{\opname{E}}  % Energy
\renewcommand{\P}{\opname{P}}
\newcommand{\Phat}{\hat{\P}}

\renewcommand{\L}{\set{L}}
\renewcommand{\A}{\set{A}}
\newcommand{\ellvec}{\greekvec{\ell}}
\newcommand{\ellv}{\greekvec{\ell}}

\newcommand{\view}{v}

\renewcommand{\g}{\vec{g}}
\newcommand{\gps}{\vec{p}}
%\newcommand{\lat}{a}
%\newcommand{\lon}{o}
\newcommand{\lat}{\varphi}
\newcommand{\lon}{\lambda}
\renewcommand{\time}{t}
\newcommand{\haversineFULL}[1]{\sin^{2}\paren{\frac{#1}{2}}}
\newcommand{\haversine}[1]{\func{\opname{hav}}{#1}}

\newcommand{\qualJunk}{\emph{junk}}
\newcommand{\qualPoor}{\emph{poor}}
\newcommand{\qualOk}{\emph{ok}}
\newcommand{\qualGood}{\emph{good}}
\newcommand{\qualExcellent}{\emph{excellent}}



\newcommand{\vpFront}{\emph{front}}
\newcommand{\vpLeft}{\emph{left}}
\newcommand{\vpBack}{\emph{back}}
\newcommand{\vpRight}{\emph{right}}
\newcommand{\vpFrontLeft}{\emph{front-left}}
\newcommand{\vpBackLeft}{\emph{back-left}}
\newcommand{\vpFrontRight}{\emph{front-right}}
\newcommand{\vpBackRight}{\emph{back-right}}

\newcommand{\vpUp}{\emph{up}}
\newcommand{\vpDown}{\emph{down}}

\newcommand{\vpF}{{\tt F}}
\newcommand{\vpL}{{\tt L}}
\newcommand{\vpB}{{\tt B}}
\newcommand{\vpR}{{\tt R}}
\newcommand{\vpFL}{{\tt FL}}
\newcommand{\vpBL}{{\tt BL}}
\newcommand{\vpFR}{{\tt FR}}
\newcommand{\vpBR}{{\tt BR}}

\newcommand{\vpU}{{\tt U}}
\newcommand{\vpD}{{\tt D}}

\newcommand{\vpUF}{{\tt UF}}
\newcommand{\vpUL}{{\tt UL}}
\newcommand{\vpUB}{{\tt UB}}
\newcommand{\vpUR}{{\tt UR}}

\newcommand{\vpDF}{{\tt DF}}
\newcommand{\vpDL}{{\tt DL}}
\newcommand{\vpDB}{{\tt DB}}
\newcommand{\vpDR}{{\tt DR}}

\newcommand{\vpUFL}{{\tt UFL}}
\newcommand{\vpDFL}{{\tt DFL}}
\newcommand{\vpUBL}{{\tt UBL}}
\newcommand{\vpDBL}{{\tt DBL}}

\newcommand{\vpUFR}{{\tt UFR}}
\newcommand{\vpDFR}{{\tt DFR}}
\newcommand{\vpUBR}{{\tt UBR}}
\newcommand{\vpDBR}{{\tt DBR}}


%\newcommand{\edges}{E}
%\newcommand{\edges}{\Epsilon}
\newcommand{\edges}{\set{E}}


\newcommand{\vpvsI}[1]{#1{\tt{vs}}#1}
\newcommand{\vpvsB}[2]{#1{\tt{vs}}#2}
%\newcommand{\vpvsI}[1]{#1-{\tt{vs}}-#1}
%\newcommand{\vpvsB}[2]{#1-{\tt{vs}}-#2}
\newcommand{\FvF}{\vpvsI{\vpF}}
\newcommand{\FRvFR}{\vpvsI{\vpFR}}
\newcommand{\FLvFL}{\vpvsI{\vpFL}}
\newcommand{\RvR}{\vpvsI{\vpR}}
\newcommand{\LvL}{\vpvsI{\vpL}}
\newcommand{\BRvBR}{\vpvsI{\vpBR}}
\newcommand{\BLvBL}{\vpvsI{\vpBL}}
\newcommand{\BvB}{\vpvsI{\vpB}}

\newcommand{\FvFR}{\vpvsB{\vpF}{\vpFR}}
\newcommand{\FvFL}{\vpvsB{\vpF}{\vpFL}}
\newcommand{\FRvR}{\vpvsB{\vpFR}{\vpR}}
\newcommand{\FLvL}{\vpvsB{\vpFL}{\vpL}}
\newcommand{\LvBL}{\vpvsB{\vpL}{\vpBL}}
\newcommand{\RvBR}{\vpvsB{\vpR}{\vpBR}}
\newcommand{\BRvB}{\vpvsB{\vpBR}{\vpB}}
\newcommand{\BLvB}{\vpvsB{\vpBL}{\vpB}}


\renewcommand{\fs}{s}
\newcommand{\fsv}{\vec{s}}

%\renewcommand{\qaid}{\X}
%\renewcommand{\daid}{\Y}

\newcommand{\ischosen}{\opname{chosen}}
\newcommand{\isdupop}{\opname{isgrouped}}

%\newcommand{\inMI}{}
\newcommand{\iI}{i}
\newcommand{\iII}{i'}
\newcommand{\jI}{j}
\newcommand{\jII}{j'}
\newcommand{\mI}{m}
\newcommand{\mII}{m'}

\newcommand{\warp}[1]{{#1}'}

%\newcommand{\inI}[1]{{#1_{1}}}
%\newcommand{\inII}[1]{{#1_{2}}}

%\newcommand{\idxI}{1}
%\newcommand{\idxII}{2}
\newcommand{\idxI}{k}
\newcommand{\idxII}{\ell}
\newcommand{\inI}[1]{{#1_{\idxI}}}
\newcommand{\inII}[1]{{#1_{\idxII}}}

%\newcommand{\inI}[1]{#1}
%\newcommand{\inII}[1]{#1'}

%\newcommand{\annotI}{\inI{\X}}
%\newcommand{\annotII}{\inII{\X}}
%\newcommand{\annot}{\X}
\newcommand{\annotI}{\X_{1}}
\newcommand{\annotII}{\X_{2}}

\newcommand{\kpI}{\inI{\kp}}
\newcommand{\kpII}{\inII{\kp}}

\newcommand{\ptI}{\inI{\pt}}
\newcommand{\ptII}{\inII{\pt}}

\newcommand{\scaleI}{\inI{\scale}}
\newcommand{\scaleII}{\inII{\scale}}

\newcommand{\oriI}{\inI{\ori}}
\newcommand{\oriII}{\inII{\ori}}

\newcommand{\rvmatI}{\inI{\rvmat}}
\newcommand{\rvmatII}{\inII{\rvmat}}

\newcommand{\invvrmatI}{\inI{\invvrmat}}
\newcommand{\invvrmatII}{\inII{\invvrmat}}

\newcommand{\xythresh}{t_{\pt}}
%\newcommand{\orithresh}{t_{{\tt{ori}}}}
\newcommand{\orithresh}{t_{{\ori}}}
\newcommand{\scalethresh}{t_{{\scale}}}

\newcommand{\tohmg}[1]{\func{\mu}{#1}}
\newcommand{\unhmg}[1]{\func{\mu^{-1}}{#1}}

%\newcommand{\refptI}{\inI{\hat{\pt}}}
\newcommand{\refptI}{\inI{\vec{r}}}
\newcommand{\ptres}{\tilde{\vec{r}}}
\newcommand{\xres}{\tilde{r}_x}
\newcommand{\yres}{\tilde{r}_y}

\newcommand{\isinlierop}{\opname{isinlier}}
%\newcommand{\isinlierop}{\delta}
%\newcommand{\HmgMatBest}{\HmgMat_{\tt{homog}}^*}
\newcommand{\HmgMatBest}{\hat{\HmgMat}}


\newcommand{\rhov}{\greekvec{\rho}}
\newcommand{\dist}{\rho}
\newcommand{\distiX}[1]{\rho(\desc_i, #1)}
\newcommand{\distijI}{\dist(\desc_i,\desc_{j_1})}
\newcommand{\distij}{\dist(\desc_i,\desc_{j})}
\newcommand{\distkij}[1]{\ensuremath{\dist_{#1}(\desc_i,\desc_{j})}}
%\newcommand{\distkijI}[1]{\ensuremath{\dist^{#1}(\desc_i,\desc_{j_1})}}
%\newcommand{\distkijI}[1]{\ensuremath{\dist_{\nid_#1}(\desc_i,\desc_{j_1})}}
\newcommand{\distkijI}[1]{\ensuremath{\dist_{#1}(\desc_i,\desc_{j_1})}}


\newcommand{\matches}[2]{#1\txt{ matches }#2}
\newcommand{\notmatches}[2]{#1\txt{ does not match }#2}
\newcommand{\same}[2]{#1\txt{ is }#2}
\newcommand{\diff}[2]{#1\txt{ is not }#2}
\newcommand{\comparable}[2]{#1\txt{ comparable }#2}
\newcommand{\notcomparable}[2]{#1\txt{ non-comparable }#2}

%\newcommand{\matches}[2]{#1\txt{ matches }#2}
%\newcommand{\notmatches}[2]{#1\txt{ does not match }#2}
%\newcommand{\same}[2]{\ell_{#1} \eq{} \ell_{#2}}
%\newcommand{\diff}[2]{\ell_{#1} \neq{} \ell_{#2}}
%\newcommand{\comparable}[2]{#1\txt{ comparable }#2}
%\newcommand{\notcomparable}[2]{#1\txt{ incomparable }#2}

\newcommand{\matchXY}{\matches{\X}{\Y}}
\newcommand{\notmatchXY}{\notmatches{\X}{\Y}}
\newcommand{\sameXY}{\same{\X}{\Y}}
\newcommand{\diffXY}{\diff{\X}{\Y}}
\newcommand{\compXY}{\comparable{\X}{\Y}}
\newcommand{\notcompXY}{\notcomparable{\X}{\Y}}

\newcommand{\annoti}{\X_{i}}
\newcommand{\annotj}{\X_{j}}
\newcommand{\matchij}{\matches{\annoti}{\annotj}}
\newcommand{\notmatchij}{\notmatches{\annoti}{\annotj}}
\newcommand{\sameij}{\same{\annoti}{\annotj}}
\newcommand{\diffij}{\diff{\annoti}{\annotj}}
\newcommand{\compij}{\comparable{\annoti}{\annotj}}
\newcommand{\notcompij}{\notcomparable{\annoti}{\annotj}}

%\newcommand{\matches}[2]{\txt{matches}}
%\newcommand{\same}[2]{\txt{is}}
%\newcommand{\comparable}[2]{\txt{comparable}}

\newcommand{\sameXX}{\ell_i = \ell_j}
\newcommand{\compXX}{C_{ij}}
\newcommand{\notcompXX}{\overbar{C_{ij}}}
\newcommand{\evidXX}{M_{ij}}
\newcommand{\bgprobXX}{\phi_{ij}}

\newcommand{\idengraph}{identification graph}

%\newcommand{\lset}{labeled set}
%\newcommand{\uset}{unlabeled set}
\newcommand{\lset}{known set}
\newcommand{\uset}{unknown set}


\newcommand{\pymod}[1]{{\tt{#1}}}
\newcommand{\utool}{\pymod{utool}}
\newcommand{\vtool}{\pymod{vtool}}
\newcommand{\guitool}{\pymod{guitooltool}}
\newcommand{\plottool}{\pymod{plottool}}

\newcommand{\depcache}{dependency cache}
\newcommand{\depcaches}{dependency caches}
\newcommand{\Depcache}{Dependency cache}
\newcommand{\Depcaches}{Dependency caches}

\newcommand{\IBEIS}{IBEIS}
%\newcommand{\IBEIS}{{\tt{IBEIS}}}

\newcommand{\gia}{graph identification algorithm}
\newcommand{\Gia}{Graph identification algorithm}

\newcommand{\rpipe}{\ensuremath{\Rightarrow}}


% A "w" question
\newcommand{\wquest}[1]{\emph{#1}} 
%\newcommand{wquest}[1]{``#1''}



\newcommand{\pzmasterI}{PZ\xspace}
\newcommand{\gzall}{GZ\xspace}
\newcommand{\girmmasterI}{GIRM\xspace}
%\renewcommand{\st}{\superscript{st}\xspace}
\newcommand{\QRH}{QRH\xspace{}}
\newcommand{\QRHCirc}{QRH\xspace{}}
\newcommand{\QRHEll}{QRH+Affine\xspace{}}
\newcommand{\AIAlone}{Affine\xspace{}}
\newcommand{\RIAlone}{Rotation\xspace{}}
\newcommand{\AIRI}{Affine+Rotation\xspace{}}
\newcommand{\NoInvar}{No Invariance\xspace{}}
\newcommand{\viewdiff}{\pvar{viewdiff}}
\newcommand{\ctrl}{\pvar{ctrl}}
\newcommand{\timectrl}{\pvar{timectrl}}

%\newcommand{\thesis}{candidacy document\zspace}
%\newcommand{\Thesis}{Candidacy document\zspace}
\newcommand{\thesis}{thesis\zspace}
\newcommand{\Thesis}{Thesis\zspace}

\newcommand{\xdesc}{\vec{x}}
\newcommand{\ydesc}{\vec{y}}


%Names: IBEIS-PZ-1348 viewpoint issue different and viewing conditions
%IBEIS-PZ-1421 Pose issues where the zebras are fighting
%\newcommand{\captionlabel}[2]{\caption{#1 - #2}\label{#2}}

\begin{comment}
    python -m ibeis.scripts.gen_cand_expts --exec-parse_latex_comments_for_commmands --fname figdef1.tex

    sudo apt-get install pngquant
    pngquant --quality=0-70 figures1/*.png
    pngquant --quality=0-70 figures2/*.png
    pngquant --quality=0-70 figures3/*.png
    pngquant --quality=0-70 figures4/*.png
    pngquant --quality=0-70 figures5/*.png
    pngquant --quality=0-70 *.png
    pngquant --quality=0-70 *.png
\end{comment}


\begin{comment}
python -m ibeis.viz.viz_name show_multiple_chips --db NNP_Master3 --aids=6416,7458,13339,10170 \
    --no-inimage --notitle --adjust=.05 --rc=1,4 --saveparts \
    --dpath ~/latex/crall-thesis-2017/ --save "figures1/PlainsFigure.jpg" --figsize=9,4 --dpi=300 --diskshow 
\end{comment}
\newcommand{\PlainsFigure}{
\begin{figure}[h]
\centering
\begin{subfigure}[h]{0.44\textwidth}\centering\includegraphics[height=110pt]{figures1/PlainsFigureA.jpg}\caption{}\label{sub:PlainsFigureA}\end{subfigure}%
\begin{subfigure}[h]{0.48\textwidth}\centering\includegraphics[height=110pt]{figures1/PlainsFigureB.jpg}\caption{}\label{sub:PlainsFigureB}\end{subfigure}
\begin{subfigure}[h]{0.44\textwidth}\centering\includegraphics[height=110pt]{figures1/PlainsFigureD.jpg}\caption{}\label{sub:PlainsFigureD}\end{subfigure}%
\begin{subfigure}[h]{0.48\textwidth}\centering\includegraphics[height=110pt]{figures1/PlainsFigureC.jpg}\caption{}\label{sub:PlainsFigureC}\end{subfigure}%
\caption[\caplbl{PlainsFigure}Distinguishing features for plains zebras]{\caplbl{PlainsFigure}
% ---
For a plains zebra, the most distinguishing features tend to be located on the upper shoulder.
Other distinguishing features are typically be found on the face and side.
Image pairs~\cref{sub:PlainsFigureA,sub:PlainsFigureB} and~\cref{sub:PlainsFigureC,sub:PlainsFigureD} depict the
  same individual.
% ---
}
\label{fig:PlainsFigure}
\end{figure}
}


\begin{comment}
python -m ibeis.viz.viz_name show_multiple_chips --aids=163,253,286,449 --db=NNP_MasterGIRM_core \
    --no-inimage --adjust=.05 --no-figtitle --notitle --rc=1,4  --dpi=300 --figsize=9,4 \
    --dpath ~/latex/crall-thesis-2017/ --save "figures1/GirMasaiFigure.jpg"  \
    --saveparts --diskshow
%./main.py --db NNP_MasterGIRM_core --query 164 --daids-mode=all -y
\end{comment}
\newcommand{\GirMasaiFigure}{
\begin{figure}[h]
\centering
\begin{subfigure}[h]{0.48\textwidth}\centering\includegraphics[height=265pt]{figures1/GirMasaiFigureA.jpg}\caption{}\label{sub:GirMasaiFigureA}\end{subfigure}%
\begin{subfigure}[h]{0.48\textwidth}\centering\includegraphics[height=265pt]{figures1/GirMasaiFigureB.jpg}\caption{}\label{sub:GirMasaiFigureB}\end{subfigure}
\begin{subfigure}[h]{0.48\textwidth}\centering\includegraphics[height=180pt]{figures1/GirMasaiFigureC.jpg}\caption{}\label{sub:GirMasaiFigureC}\end{subfigure}%
\begin{subfigure}[h]{0.48\textwidth}\centering\includegraphics[height=180pt]{figures1/GirMasaiFigureD.jpg}\caption{}\label{sub:GirMasaiFigureD}\end{subfigure}
\caption[\caplbl{GirMasaiFigure}Distinguishing features for Masai giraffes]{\caplbl{GirMasaiFigure}
% ---
Masai giraffes have an abundance of features distinctive to each individual.
There are two individuals seen in images pairs~\cref{sub:GirMasaiFigureA,sub:GirMasaiFigureB}
  and~\cref{sub:GirMasaiFigureC,sub:GirMasaiFigureD}.
Note that the numerous features make it initially difficult for a human to match giraffes.
In contrast, this is easier for algorithms.
% ---
}
\label{fig:GirMasaiFigure}
\end{figure}
}


\begin{comment}
python -m ibeis.viz.viz_name show_multiple_chips --aids=923,1013,823,960 --db=GZ_ALL  \
    --no-figtitle --notitle --no-inimage --rc=2,2 \
    --dpath ~/latex/cand/ --save "figures1/GrevysFigure.jpg"  --saveparts \
    --figsize=9,4  --dpi=300 --diskshow
%ib 
./main.py --db GZ_ALL --query 923 --daids-mode=all -y
\end{comment}
\newcommand{\GrevysFigure}{
\begin{figure}[h]
\centering
\begin{subfigure}[h]{0.48\textwidth}\centering\includegraphics[height=90pt]{figures1/GrevysFigureA.jpg}\caption{}\label{sub:GrevysFigureA}\end{subfigure}%
\begin{subfigure}[h]{0.48\textwidth}\centering\includegraphics[height=90pt]{figures1/GrevysFigureB.jpg}\caption{}\label{sub:GrevysFigureB}\end{subfigure}
\begin{subfigure}[h]{0.48\textwidth}\centering\includegraphics[height=90pt]{figures1/GrevysFigureC.jpg}\caption{}\label{sub:GrevysFigureC}\end{subfigure}%
\begin{subfigure}[h]{0.48\textwidth}\centering\includegraphics[height=90pt]{figures1/GrevysFigureD.jpg}\caption{}\label{sub:GrevysFigureD}\end{subfigure}
\caption[\caplbl{GrevysFigure}Distinguishing features for Grévy's zebras]{\caplbl{GrevysFigure}
% ---
A Grévy's zebra's most distinctive features are above the front and rear legs.
Useful, but less distinctive information can be seen on the side of the body.
Image pairs~\cref{sub:GrevysFigureA,sub:GrevysFigureB} and~\cref{sub:GrevysFigureC,sub:GrevysFigureD} depict the
  same individual.
% ---
}
\label{fig:GrevysFigure}
\end{figure}
}


\begin{comment}
python -m ibeis.viz.viz_name show_multiple_chips --aids=7,2,8 --db=humpbacks_fb --dpath ~/latex/crall-thesis-2017/ --save "figures1/HumpbackFig.jpg" --figsize=9,4  --dpi=300 --no-inimage --adjust=.05,.05,.15 --no-figtitle --notitle --diskshow --saveparts 
\end{comment}
\newcommand{\HumpbackFig}{
\begin{figure}[h]
\centering
\begin{subfigure}[h]{0.33\textwidth}\centering\includegraphics[height=88pt]{figures1/HumpbackFigA.jpg}\caption{}\label{sub:HumpbackFigA}\end{subfigure}%
\begin{subfigure}[h]{0.33\textwidth}\centering\includegraphics[height=88pt]{figures1/HumpbackFigB.jpg}\caption{}\label{sub:HumpbackFigB}\end{subfigure}%
\begin{subfigure}[h]{0.33\textwidth}\centering\includegraphics[height=88pt]{figures1/HumpbackFigC.jpg}\caption{}\label{sub:HumpbackFigC}\end{subfigure}
\caption[\caplbl{HumpbackFig}Distinguishing features for humpback whales]{\caplbl{HumpbackFig}
% ---
A humpback whale can be identified by the texture patterns on the fluke or using the shape of the notches along
  the edge of the fluke.
Note that some humpbacks (like the on seen in \ref{sub:HumpbackFigB}) do not have any texture patterns on their
  fluke.
The pair of images \cref{sub:HumpbackFigA,sub:HumpbackFigC} depict the same individual.
% ---
}
\label{fig:HumpbackFig}
\end{figure}
}


\begin{comment}
python -m ibeis.viz.viz_name show_multiple_chips --db NNP_Master3 --aids=13285,12598,6563,9332 --no-inimage --notitle --rc=1,4 --dpath ~/latex/crall-thesis-2017/ --save "figures1/HardCaseFigure.jpg" --figsize=9,4  --dpi=300 --diskshow --saveparts
\end{comment}
\newcommand{\HardCaseFigure}{
\begin{figure}[h]
\centering
\begin{subfigure}[h]{0.45\textwidth}\centering\includegraphics[height=110pt]{figures1/HardCaseFigureB.jpg}\caption{}\label{sub:HardCaseFigureB}\end{subfigure}%
\begin{subfigure}[h]{0.55\textwidth}\centering\includegraphics[height=110pt]{figures1/HardCaseFigureA.jpg}\caption{}\label{sub:HardCaseFigureA}\end{subfigure}
\begin{subfigure}[h]{0.45\textwidth}\centering\includegraphics[height=110pt]{figures1/HardCaseFigureC.jpg}\caption{}\label{sub:HardCaseFigureC}\end{subfigure}%
\begin{subfigure}[h]{0.55\textwidth}\centering\includegraphics[height=110pt]{figures1/HardCaseFigureD.jpg}\caption{}\label{sub:HardCaseFigureD}\end{subfigure}
\caption[\caplbl{HardCaseFigure}Visually similar plains zebras]{\caplbl{HardCaseFigure}
% ---
Different plains zebras sometimes have visual similarities that can be difficult to distinguish.
There are three individuals in these four images.
The images in~\cref{sub:HardCaseFigureB,sub:HardCaseFigureD} depict the same individual.
Dissimilarities can be seen on the lower thigh of images~\cref{sub:HardCaseFigureC,sub:HardCaseFigureD}, as well
  as on the front shoulder of images~\cref{sub:HardCaseFigureA,sub:HardCaseFigureB}.
% ---
}
\label{fig:HardCaseFigure}
\end{figure}
}




\begin{comment}
python -m ibeis.viz.viz_name --test-show_multiple_chips --db NNP_Master3 --aids=6524,6540,6571,6751 --no-inimage --notitle --dpath ~/latex/crall-thesis-2017/ --save "figures1/BacksFigure.jpg" --figsize=9,4  --dpi=300 --rc=1,4 --diskshow --saveparts
\end{comment}
\newcommand{\BacksFigure}{
\begin{figure}[h]
\centering
\begin{subfigure}[h]{0.24\textwidth}\centering\includegraphics[height=165pt]{figures1/BacksFigureA.jpg}\caption{}\label{sub:BacksFigureA}\end{subfigure}%
\begin{subfigure}[h]{0.24\textwidth}\centering\includegraphics[height=165pt]{figures1/BacksFigureB.jpg}\caption{}\label{sub:BacksFigureB}\end{subfigure}
\begin{subfigure}[h]{0.24\textwidth}\centering\includegraphics[height=165pt]{figures1/BacksFigureC.jpg}\caption{}\label{sub:BacksFigureC}\end{subfigure}%
\begin{subfigure}[h]{0.24\textwidth}\centering\includegraphics[height=165pt]{figures1/BacksFigureD.jpg}\caption{}\label{sub:BacksFigureD}\end{subfigure}
\caption[\caplbl{BacksFigure}Back viewpoints of plains zebras]{\caplbl{BacksFigure}
% ---
The backs of plains zebras have very little distinguishing information.
All the above images are different individuals.
% ---
}
\label{fig:BacksFigure}
\end{figure}
}



\begin{comment}
python -m ibeis.viz.viz_name --test-show_multiple_chips --db PZ_Master0 --aids=5878,5885,5886,5888,5890,5904 --no-inimage --notitle --dpath ~/latex/crall-thesis-2017/ --save "figures1/ThreeSixtyFigure.jpg" --figsize=9,4  --dpi=300 --diskshow --saveparts

./dev.py --db PZ_Master0 --eval="','.join(list(map(str, ibs.search_annot_notes('360'))))"
\end{comment}
\newcommand{\ThreeSixtyFigure}{
\begin{figure}[h]
\centering
%\begin{subfigure}[h]{0.26\textwidth}\centering\includegraphics[height=110pt]{figures1/ThreeSixtyFigureF.jpg}\caption{}\label{sub:ThreeSixtyFigureF}\end{subfigure}%
\begin{subfigure}[h]{0.48\textwidth}\centering\includegraphics[height=155pt]{figures1/ThreeSixtyFigureA.jpg}\caption{}\label{sub:ThreeSixtyFigureA}\end{subfigure}
\begin{subfigure}[h]{0.28\textwidth}\centering\includegraphics[height=155pt]{figures1/ThreeSixtyFigureB.jpg}\caption{}\label{sub:ThreeSixtyFigureB}\end{subfigure}
\begin{subfigure}[h]{0.18\textwidth}\centering\includegraphics[height=155pt]{figures1/ThreeSixtyFigureC.jpg}\caption{}\label{sub:ThreeSixtyFigureC}\end{subfigure}
\begin{subfigure}[h]{0.48\textwidth}\centering\includegraphics[height=155pt]{figures1/ThreeSixtyFigureD.jpg}\caption{}\label{sub:ThreeSixtyFigureD}\end{subfigure}
\begin{subfigure}[h]{0.48\textwidth}\centering\includegraphics[height=155pt]{figures1/ThreeSixtyFigureE.jpg}\caption{}\label{sub:ThreeSixtyFigureE}\end{subfigure}
\caption[\caplbl{ThreeSixtyFigure}Examples of viewpoint variations]{\caplbl{ThreeSixtyFigure}
% ---
Viewpoint variations of an individual Grévy's zebra.
It would not be possible to match image~\cref{sub:ThreeSixtyFigureA,sub:ThreeSixtyFigureE} without information
  from images showing intermediate views.
% ---
}
\label{fig:ThreeSixtyFigure}
\end{figure}
}


\begin{comment}
python -m ibeis.viz.viz_name --test-show_multiple_chips --aids=11081,13057,15897,15249,8081,13758 --db NNP_Master3 --dpath ~/latex/crall-thesis-2017 --save figures1/PoseFigure.jpg  --figsize=9,4  --dpi=300 --no-figtitle --notitle --diskshow --no-draw_lbls --zoom=.5 --saveparts

--adjust=.05,.05,.05,.15 
\end{comment}
\newcommand{\PoseFigure}{
\begin{figure}[h]
\centering
\begin{subfigure}[h]{0.32\textwidth}\centering\includegraphics[height=100pt]{figures1/PoseFigureA.jpg}\caption{}\label{sub:PoseFigureA}\end{subfigure}
\begin{subfigure}[h]{0.32\textwidth}\centering\includegraphics[height=100pt]{figures1/PoseFigureE.jpg}\caption{}\label{sub:PoseFigureE}\end{subfigure}
\begin{subfigure}[h]{0.31\textwidth}\centering\includegraphics[height=100pt]{figures1/PoseFigureD.jpg}\caption{}\label{sub:PoseFigureD}\end{subfigure}
\begin{subfigure}[h]{0.32\textwidth}\centering\includegraphics[height=100pt]{figures1/PoseFigureB.jpg}\caption{}\label{sub:PoseFigureB}\end{subfigure}
\begin{subfigure}[h]{0.32\textwidth}\centering\includegraphics[height=100pt]{figures1/PoseFigureC.jpg}\caption{}\label{sub:PoseFigureC}\end{subfigure}
\begin{subfigure}[h]{0.31\textwidth}\centering\includegraphics[height=100pt]{figures1/PoseFigureF.jpg}\caption{}\label{sub:PoseFigureF}\end{subfigure}
\caption[\caplbl{PoseFigure}Examples of challenging pose variations]{\caplbl{PoseFigure}
% ---
Animals can appear in a wide variety of poses.
To clearly see the different poses, the images are shown with surrounding context.
During identification the image is cropped to the bounding box shown around each animal.
% ---
}
\label{fig:PoseFigure}
\end{figure}
}



\begin{comment}
python -m ibeis.viz.viz_name --test-show_multiple_chips --dpath ~/latex/crall-thesis-2017 --save 'figures1/OccludeFigure.jpg' --no-figtitle --notitle --db NNP_Master3 --figsize=9,4 --dpi=300 --no-inimage --aids=13870,13740,7735,13233,13603,13906,7354,9776 --rc=2,4 --diskshow --saveparts
\end{comment}
\newcommand{\OccludeFigure}{
\begin{figure}[h]
\centering
\begin{subfigure}[h]{0.44\textwidth}\centering\includegraphics[height=95pt]{figures1/OccludeFigureC.jpg}\caption{}\label{sub:OccludeFigureC}\end{subfigure}
\begin{subfigure}[h]{0.26\textwidth}\centering\includegraphics[height=95pt]{figures1/OccludeFigureB.jpg}\caption{}\label{sub:OccludeFigureB}\end{subfigure}
\begin{subfigure}[h]{0.26\textwidth}\centering\includegraphics[height=95pt]{figures1/OccludeFigureH.jpg}\caption{}\label{sub:OccludeFigureH}\end{subfigure}
\begin{subfigure}[h]{0.38\textwidth}\centering\includegraphics[height=95pt]{figures1/OccludeFigureD.jpg}\caption{}\label{sub:OccludeFigureD}\end{subfigure}
\begin{subfigure}[h]{0.30\textwidth}\centering\includegraphics[height=95pt]{figures1/OccludeFigureE.jpg}\caption{}\label{sub:OccludeFigureE}\end{subfigure}
\begin{subfigure}[h]{0.30\textwidth}\centering\includegraphics[height=95pt]{figures1/OccludeFigureF.jpg}\caption{}\label{sub:OccludeFigureF}\end{subfigure}

%\begin{subfigure}[h]{0.3\textwidth}\centering\includegraphics[height=80pt]{figures1/OccludeFigureA.jpg}\caption{}\label{sub:OccludeFigureA}\end{subfigure}%
%\begin{subfigure}[h]{0.24\textwidth}\centering\includegraphics[height=60pt]{figures1/OccludeFigureG.jpg}\caption{}\label{sub:OccludeFigureG}\end{subfigure}%
\caption[\caplbl{OccludeFigure}Examples of occlusion and distractors]{\caplbl{OccludeFigure}
% ---
Animals under varying degrees of occlusion from scenery and other secondary animals.
Occlusions can obfuscate or remove distinctive feature entirely.
Secondary animals can introduce new distinctive features that do not belong to the primary animal.
Images like this can cause other images of the secondary animal to the primary animal.
% ---
}
\label{fig:OccludeFigure}
\end{figure}
}


\begin{comment}
python -m ibeis.viz.viz_name --test-show_multiple_chips --dpath ~/latex/crall-thesis-2017 --save figures1/IlluminationFigure.jpg --no-figtitle --notitle --db NNP_Master3 --figsize=9,3 --no-inimage --aids=6466,10161,10634,9458,12472,14728  --diskshow  --saveparts --dpi=300
\end{comment}
\newcommand{\IlluminationFigure}{
\begin{figure}[ht!]
\centering
%\begin{subfigure}[h]{0.2\textwidth}\centering\includegraphics[height=65pt]{figures1/IlluminationFigureA.jpg}\caption{}\label{sub:IlluminationFigureA}\end{subfigure}%
%\begin{subfigure}[h]{0.2\textwidth}\centering\includegraphics[height=65pt]{figures1/IlluminationFigureB.jpg}\caption{}\label{sub:IlluminationFigureB}\end{subfigure}%
\begin{subfigure}[h]{0.36\textwidth}\centering\includegraphics[height=110pt]{figures1/IlluminationFigureC.jpg}\caption{}\label{sub:IlluminationFigureC}\end{subfigure}
\begin{subfigure}[h]{0.60\textwidth}\centering\includegraphics[height=110pt]{figures1/IlluminationFigureF.jpg}\caption{}\label{sub:IlluminationFigureF}\end{subfigure}
\begin{subfigure}[h]{0.50\textwidth}\centering\includegraphics[height=110pt]{figures1/IlluminationFigureD.jpg}\caption{}\label{sub:IlluminationFigureD}\end{subfigure}
\begin{subfigure}[h]{0.46\textwidth}\centering\includegraphics[height=110pt]{figures1/IlluminationFigureE.jpg}\caption{}\label{sub:IlluminationFigureE}\end{subfigure}
\caption[\caplbl{IlluminationFigure}Examples of different lighting conditions]{\caplbl{IlluminationFigure}
% ---
The effects of outdoor and illumination.
Shadow and illumination can cause variations in the underlying image intensity and gradients.
This can make it more difficult to localize repeatable keypoints and describe the underlying texture patterns.
% ---
}
\label{fig:IlluminationFigure}
\end{figure}
}



\begin{comment}
python -m ibeis.viz.viz_name --test-show_multiple_chips \
    --db NNP_Master3 --aids=6416,8227,6262,10705,15417 \
    --no-figtitle --notitle  --figsize=9,4 --dpi=300 \
    --adjust=.05  --rc=2,5  --no-draw_lbls --doboth --trydrawline --qualtitle  --grouprows \
    --dpath ~/latex/crall-thesis-2017 --save 'figures1/QualityFigure.jpg'  --diskshow  --saveparts 
\end{comment}
\newcommand{\QualityFigure}{
\begin{figure}[ht!]
\centering
\begin{subfigure}[h]{0.19\textwidth}\centering\includegraphics[height=150pt]{figures1/QualityFigureA.jpg}\caption{}\label{sub:QualityFigureA}\end{subfigure}
\begin{subfigure}[h]{0.19\textwidth}\centering\includegraphics[height=150pt]{figures1/QualityFigureB.jpg}\caption{}\label{sub:QualityFigureB}\end{subfigure}
\begin{subfigure}[h]{0.19\textwidth}\centering\includegraphics[height=150pt]{figures1/QualityFigureC.jpg}\caption{}\label{sub:QualityFigureC}\end{subfigure}
\begin{subfigure}[h]{0.19\textwidth}\centering\includegraphics[height=150pt]{figures1/QualityFigureD.jpg}\caption{}\label{sub:QualityFigureD}\end{subfigure}
\begin{subfigure}[h]{0.19\textwidth}\centering\includegraphics[height=150pt]{figures1/QualityFigureE.jpg}\caption{}\label{sub:QualityFigureE}\end{subfigure}
\caption[\caplbl{QualityFigure}Examples of different image qualities]{\caplbl{QualityFigure}
% ---
An individual seen in images of different qualities.
The bottom row shows the cropped images that correspond to the bounding boxes in the top row.
Each column shows different qualities:
\Cref{sub:QualityFigureA} an excellent quality image taken from a short distance, \Cref{sub:QualityFigureB} a
  good quality image with minor shadow and taken from a medium distance, \Cref{sub:QualityFigureC} an ok quality
  image due to minor occlusion, \Cref{sub:QualityFigureD} a poor quality image due to major occlusion,
  \Cref{sub:QualityFigureE} a junk quality image due to considerable blur.
% ---
}
\label{fig:QualityFigure}
\end{figure}
}


\begin{comment}
python -m ibeis.viz.viz_name --test-show_name --name=08_106 --db PZ_MTEST --save "figures1/Age.jpg" --dpath ~/latex/crall-thesis-2017/ --figsize=9,4  --dpi=300 --no-figtitle --notitle --diskshow --no-draw_lbls --no-inimage --saveparts
\end{comment}
\newcommand{\AgeFigure}{
\begin{figure}[h]
\centering
\begin{subfigure}[h]{0.48\textwidth}\centering\includegraphics[height=100pt]{figures1/AgeA.jpg}\caption{}\label{sub:AgeA}\end{subfigure}%
\begin{subfigure}[h]{0.48\textwidth}\centering\includegraphics[height=100pt]{figures1/AgeB.jpg}\caption{}\label{sub:AgeB}\end{subfigure}
\caption[\caplbl{AgeFigure}Examples of visual differences caused by age]{\caplbl{AgeFigure}
% ---
The left and right images show the adult and juvenile appearance of the
  same individual.
As an animal ages its appearance changes mainly in color and texture
  with some minor shape and scale differences.
% ---
}
\label{fig:AgeFigure}
\end{figure}
}


\begin{comment}
python -m ibeis.viz.viz_name --test-show_multiple_chips --db PZ_Master0 --aids=4020,4839 --no-inimage --notitle --adjust=.05 --dpath ~/latex/crall-thesis-2017/ --save "figures1/GashFigure.jpg" --figsize=9,4  --dpi=300 --diskshow --saveparts
\end{comment}
\newcommand{\GashFigure}{
\begin{figure}[h]
\centering
\begin{subfigure}[h]{0.48\textwidth}\centering\includegraphics[height=100pt]{figures1/GashFigureA.jpg}\caption{}\label{sub:GashFigureA}\end{subfigure}%
\begin{subfigure}[h]{0.48\textwidth}\centering\includegraphics[height=100pt]{figures1/GashFigureB.jpg}\caption{}\label{sub:GashFigureB}\end{subfigure}
\caption[\caplbl{GashFigure}Examples of visual differences caused by injuries]{\caplbl{GashFigure}
% ---
Injuries can obscure features on an animal as well as creating new ones.
The left image shows a wounded animal, and the right image shows an animal
  with a distinguishing scaring pattern.
% ---
}
\label{fig:GashFigure}
\end{figure}
}


\begin{comment}
python -m ibeis.viz.viz_image --test-show_multi_images --db NNP_Master3 --gids=7409,7448,4670,7497,7496,7464 --adjust=.05 --dpath ~/latex/crall-thesis-2017/ --save "figures1/DetectFigure.jpg" --figsize=9,4  --dpi=300 --diskshow --saveparts
\end{comment}
\newcommand{\DetectFigure}{
\begin{figure}[ht!]
\centering
\begin{subfigure}[h]{0.32\textwidth}\centering\includegraphics[height=80pt]{figures1/DetectFigureA.jpg}\caption{}\label{sub:DetectFigureA}\end{subfigure}
\begin{subfigure}[h]{0.32\textwidth}\centering\includegraphics[height=80pt]{figures1/DetectFigureB.jpg}\caption{}\label{sub:DetectFigureB}\end{subfigure}
\begin{subfigure}[h]{0.32\textwidth}\centering\includegraphics[height=80pt]{figures1/DetectFigureC.jpg}\caption{}\label{sub:DetectFigureC}\end{subfigure}
\begin{subfigure}[h]{0.32\textwidth}\centering\includegraphics[height=80pt]{figures1/DetectFigureD.jpg}\caption{}\label{sub:DetectFigureD}\end{subfigure}
\begin{subfigure}[h]{0.32\textwidth}\centering\includegraphics[height=80pt]{figures1/DetectFigureE.jpg}\caption{}\label{sub:DetectFigureE}\end{subfigure}
\begin{subfigure}[h]{0.32\textwidth}\centering\includegraphics[height=80pt]{figures1/DetectFigureF.jpg}\caption{}\label{sub:DetectFigureF}\end{subfigure}
\caption[\caplbl{DetectFigure}Detection of plains zebras]{\caplbl{DetectFigure}
% ---
Images from the \GZC{} with detections of plains zebras.
Detections were automatically suggested and manually verified before
  being accepted.
% ---
}
\label{fig:DetectFigure}
\end{figure}
}

\begin{comment}
python -m ibeis.viz.viz_name --test-show_multiple_chips --dpath ~/latex/crall-thesis-2017 --save figures1/OccurrenceComplementFigure.jpg --no-figtitle --notitle --db NNP_Master3 --figsize=9,4 --dpi=300 --no-inimage --aids=15288,15333,15797 --diskshow --saveparts
\end{comment}
\newcommand{\OccurrenceComplementFigure}{
\begin{figure}[h]
\centering
\begin{subfigure}[h]{0.33\textwidth}\centering\includegraphics[height=85pt]{figures1/OccurrenceComplementFigureA.jpg}\caption{}\label{sub:OccurrenceComplementFigureA}\end{subfigure}
\begin{subfigure}[h]{0.32\textwidth}\centering\includegraphics[height=85pt]{figures1/OccurrenceComplementFigureB.jpg}\caption{}\label{sub:OccurrenceComplementFigureB}\end{subfigure}
\begin{subfigure}[h]{0.30\textwidth}\centering\includegraphics[height=85pt]{figures1/OccurrenceComplementFigureC.jpg}\caption{}\label{sub:OccurrenceComplementFigureC}\end{subfigure}
\caption[\caplbl{OccurrenceComplementFigure}Multiple images in an occurrence]{\caplbl{OccurrenceComplementFigure}
% ---
Images taken within an occurrence that demonstrate redundant and complementary features.
Features on the shoulders are somewhat redundant in
  images~\cref{sub:OccurrenceComplementFigureA,sub:OccurrenceComplementFigureB,sub:OccurrenceComplementFigureC}
  because they are all under approximately constant illumination and are seen from the same angle.
Images~\cref{sub:OccurrenceComplementFigureA,sub:OccurrenceComplementFigureC} have complementary features because
  the viewpoint of the animal has shifted slightly.
% ---
}
\label{fig:OccurrenceComplementFigure}
\end{figure}
}

\begin{comment}
python -m ibeis.viz.viz_qres show_qres --db=PZ_MTEST --qaid=45 --top-aids=5 --simplemode --sidebyside --annot_mode=0 --notitle --no-viz_name_score --max_nCols=3 --adjust=.02 --figsize=9,4 --show --dpi=300 '--dpath=~/latex/crall-thesis-2017' --save=figures1/RankFigure2.jpg --diskshow --saveparts
\end{comment}
\newcommand{\RankFigure}{
\begin{figure}[h]
\centering
\begin{subfigure}[h]{0.7\textwidth}\centering\includegraphics[width=\textwidth]{figures1/RankFigure2A.jpg}\caption{Rank 1}\label{sub:RankFigure2A}\end{subfigure}
\begin{subfigure}[h]{0.7\textwidth}\centering\includegraphics[width=\textwidth]{figures1/RankFigure2B.jpg}\caption{Rank 2}\label{sub:RankFigure2B}\end{subfigure}
\begin{subfigure}[h]{0.7\textwidth}\centering\includegraphics[width=\textwidth]{figures1/RankFigure2C.jpg}\caption{Rank 3}\label{sub:RankFigure2C}\end{subfigure}
%\begin{subfigure}[h]{0.19\textwidth}\centering\includegraphics[width=\textwidth]{figures1/RankFigure2D.jpg}\caption{}\label{sub:RankFigure2D}\end{subfigure}%
%\begin{subfigure}[h]{0.19\textwidth}\centering\includegraphics[width=\textwidth]{figures1/RankFigure2E.jpg}\caption{}\label{sub:RankFigure2E}\end{subfigure}%
\caption[Examples of top ranked matches]{\caplbl{RankFigure}
% ---
A ranked list image pairs.
Each pair is a one-vs-one comparison.
All the left images are the same query image.
Each image on the right is a candidate match.
The match in \cref{sub:RankFigure2A} is correct and the other matches are incorrect.
However, a ranked list may contain more than one correct match.
% ---
}
\label{fig:RankFigure}
\end{figure}
}



\begin{comment}
        python -m ibeis.scripts.gen_cand_expts --exec-parse_latex_comments_for_commmands --fname figdef2.tex
\end{comment}



\begin{comment}
# Fig for scale space can be seen here
http://opticalengineering.spiedigitallibrary.org/article.aspx?articleid=1089124
http://opticalengineering.spiedigitallibrary.org/data/Journals/OPTICE/22119/017204_1_1.png
\end{comment}


% --------


\begin{comment}
python -m vtool.patch --test-draw_kp_ori_steps \
 --fname=zebra.png --fx=121 --stride=2 \
 --dpath ~/latex/crall-thesis-2017/ --save figures2/testfindkpdirection.jpg \
 --saveparts --diskshow --figsize=10,5 --dpi=300 --hspace=.4 --top=.9


this is bugged in mpl 2.0.2, but works in 2.0.0


python -m vtool.patch --test-draw_kp_ori_steps --fname=zebra.png --fx=121 --show 
python -m vtool.patch --test-draw_kp_ori_steps --fname=zebra.png --fx=121 --dpath . --save KpOri.jpg  --figsize=10,5 --dpi=300  --diskshow  --top=0.8 --saveparts
python -m vtool.patch --test-draw_kp_ori_steps --fname=zebra.png --fx=121 --dpath . --save KpOri.jpg  --figsize=10,5 --dpi=300  --diskshow  --top=0.9 --hspace=.4
--left=.04 --bottom=.05 --wspace=.2 --hspace=.3
\end{comment}
\newcommand{\testfindkpdirection}{
\begin{figure}[ht!]
\centering
\begin{subfigure}[h]{0.23\textwidth}\centering\includegraphics[height=60pt]{figures2/testfindkpdirectionA.jpg}\caption{}\label{sub:testfindkpdirectiona}\end{subfigure}
~~%--
\begin{subfigure}[h]{0.23\textwidth}\centering\includegraphics[height=100pt]{figures2/testfindkpdirectionB.jpg}\caption{}\label{sub:testfindkpdirectionb}\end{subfigure}
~~%--
\begin{subfigure}[h]{0.23\textwidth}\centering\includegraphics[height=100pt]{figures2/testfindkpdirectionC.jpg}\caption{}\label{sub:testfindkpdirectionc}\end{subfigure}
~~%--
\begin{subfigure}[h]{0.23\textwidth}\centering\includegraphics[height=100pt]{figures2/testfindkpdirectionD.jpg}\caption{}\label{sub:testfindkpdirectiond}\end{subfigure}
~~%--
\begin{subfigure}[h]{0.23\textwidth}\centering\includegraphics[height=100pt]{figures2/testfindkpdirectionE.jpg}\caption{}\label{sub:testfindkpdirectione}\end{subfigure}
~~%--
\begin{subfigure}[h]{0.23\textwidth}\centering\includegraphics[height=100pt]{figures2/testfindkpdirectionF.jpg}\caption{}\label{sub:testfindkpdirectionf}\end{subfigure}
~~%--
\begin{subfigure}[h]{0.23\textwidth}\centering\includegraphics[height=100pt]{figures2/testfindkpdirectionG.jpg}\caption{}\label{sub:testfindkpdirectiong}\end{subfigure}
~~%--
\begin{subfigure}[h]{0.23\textwidth}\centering\includegraphics[height=100pt]{figures2/testfindkpdirectionH.jpg}\caption{}\label{sub:testfindkpdirectionh}\end{subfigure}
~~%--
\begin{subfigure}[h]{1\textwidth}\centering\includegraphics[width=\textwidth]{figures2/testfindkpdirectionI.jpg}\caption{}\label{sub:testfindkpdirectioni}\end{subfigure}
\caption[\caplbl{testfindkpdirection} Computing the dominant gradient orientation]{\caplbl{testfindkpdirection} 
% ---
Visualization of the steps involved in computing the dominant gradient orientations.
The top rows shows:
\cref{sub:testfindkpdirectiona} the input image with a single elliptical keypoint,
  \cref{sub:testfindkpdirectionb} the normalized keypoint, \cref{sub:testfindkpdirectionc,sub:testfindkpdirectiond}
  the squared x and y image derivatives.
The middle row shows:
\cref{sub:testfindkpdirectione} the gradient magnitudes, \cref{sub:testfindkpdirectionf} the Gaussian weighted
  gradient magnitude, \cref{sub:testfindkpdirectiong,sub:testfindkpdirectionh} the orientation at each pixel.
The final row~\cref{sub:testfindkpdirectioni} shows the histogram of weighted orientations.
The starred positions show the dominant gradient orientations localized to sub-orientation accuracy.
% ---
}
\label{fig:testfindkpdirection}
\end{figure}
}


\begin{comment}
python -m plottool.viz_featrow draw_feat_row --fname zebra.png --fx=121 \
    --dpath ~/latex/crall-thesis-2017/ --save figures2/vizfeatrow.jpg \
    --figsize=6,3 --dpi 300 --diskshow --saveparts

python -m plottool.viz_featrow --test-draw_feat_row --fname zebra.png --fx=121 --save foo.jpg --figsize=6,3 --dpi 300  --diskshow
\end{comment}
\newcommand{\vizfeatrow}{
\begin{figure}[h]
\centering
\begin{subfigure}[h]{0.47\textwidth}\centering\includegraphics[width=\textwidth]{figures2/vizfeatrowA.jpg}\caption{}\label{sub:vizfeatrowA}\end{subfigure}
~~% --
\begin{subfigure}[h]{0.47\textwidth}\centering\includegraphics[width=\textwidth]{figures2/vizfeatrowB.jpg}\caption{}\label{sub:vizfeatrowB}\end{subfigure}
\caption[A SIFT descriptor]{\caplbl{vizfeatrow}
% ---
\Cref{sub:vizfeatrowA} shows a SIFT feature superimposed over pixels it
  describes.
\Cref{sub:vizfeatrowB} shows the same SIFT descriptor as a flat histogram.
Notice the correspondence between the colors of the histogram bars. 
% ---
}
\label{fig:vizfeatrow}
\end{figure}
}




\begin{comment}
python -m plottool.draw_sv --test-show_sv_simple --dpath ~/latex/crall-thesis-2017/ --save figures2/figSVInlier.jpg --figsize=12,6 --dpi 300 --clipwhite --diskshow
\end{comment}
\newcommand{\figSVInlier}{
\begin{figure}[ht!]
\centering
\includegraphics[width=.6\textwidth]{figures2/figSVInlier.jpg}
\caption[Visualization of spatial verification]{\caplbl{figSVInlier}
% ---
Matches before and after spatial verification.
Inconsistent matches are shown in red.
Consistent matches are shown in blue.
Notice that not all spatially consistent matches are correct.
% ---
}
\label{fig:figSVInlier}
\end{figure}
}



\begin{comment}
wget http://xphilipp.developpez.com/contribuez/scalespace.png -O ~/latex/cand/figures2/ScaleSpaceFigure.png
python -m ibeis.scripts.specialdraw scalespace --dpath ~/latex/crall-thesis-2017/ --save figures2/ScaleSpaceFigure.png --dpi 300 --clipwhite --diskshow
\end{comment}
\SingleImageCommand{ScaleSpaceFigure}{.8}{A scale space pyramid}{
% ---
A Gaussian pyramid is used as a scale space representation of an image.
A property of scale space is that doubling the scale is equivalent to
  downsampling the image by half.
The set of images of a specific size correspond to an octave.
The images within each octave are the intervals.
% ---
}{figures2/ScaleSpaceFigure.png}


\begin{comment}
        python -m ibeis.scripts.gen_cand_expts --exec-parse_latex_comments_for_commmands --fname figdef3.tex
\end{comment}

            
\begin{comment}
ibeis ChipMatch.show_ranked_matches --qaid 79 --db PZ_MTEST --show --heatmask=True
ibeis ChipMatch.show_ranked_matches --qaid 79 --db PZ_MTEST \
    --clip-top=4 --colorbar_=False --show_aid=False --score_precision=2 --stack_larger=True --noshow_truth --draw_lbl=False --stack_side=True --show_timedelta=False \
    --dpath ~/latex/crall-thesis-2017 --save figures3/rankedmatches.jpg \
    --diskshow --saveparts --dpi=300 --figsize=14,14 
\end{comment}
\newcommand{\rankedmatches}{
\begin{figure}[h]
\centering
\begin{subfigure}[h]{0.7\textwidth}\centering\includegraphics[width=\textwidth]{figures3/rankedmatchesA.jpg}\caption{}\label{sub:rankedmatchesa}\end{subfigure}
\begin{subfigure}[h]{0.7\textwidth}\centering\includegraphics[width=\textwidth]{figures3/rankedmatchesB.jpg}\caption{}\label{sub:rankedmatchesb}\end{subfigure}
\begin{subfigure}[h]{0.7\textwidth}\centering\includegraphics[width=\textwidth]{figures3/rankedmatchesC.jpg}\caption{}\label{sub:rankedmatchesc}\end{subfigure}
\caption[\caplbl{rankedmatches}Ranked matches]{\caplbl{rankedmatches}
    % ---
    The top three ranked results from the ranking algorithm.
    Each results shows the matches to a particular \name{}.
    The top-ranked match in \cref{sub:rankedmatchesa} is correct.
    The other ranks in \cref{sub:rankedmatchesb,sub:rankedmatchesc} are incorrect.
    In each result, the query annotation is on the left and the matching exemplars for the name are on the right.
    The overall matching score is shown on the top of each result.
    The feature matches are overlaid on each result and colored by the feature correspondence score.
    Notice that each database \name{} may have a different number exemplars.
    % ---
}
\label{fig:rankedmatches}
\end{figure}
}



\begin{comment}
python -m ibeis.viz.viz_name --test-show_multiple_chips --db GZ_Master1 --aids 2811 2810 --show --notitle --no-inimage  --dpath ~/latex/crall-candidacy-2015/ --save figures3/SceneryMatch.jpg --diskshow --clipwhite --figsize=12,6 --dpi 300
python -m ibeis.viz.viz_name --test-show_multiple_chips --db GZ_Master1 --tags SceneryMatch --index 5 --show --notitle --no-inimage 

python -m ibeis.scripts.specialdraw simple_vsone_matches \
    --db GZ_Master1 --aids=2811,2810 \
    --figsize=12,6 --dpi 300 \
    --dpath ~/latex/crall-thesis-2017/ --save figures3/SceneryMatch2.jpg \
    --diskshow

\end{comment}
\newcommand{\SceneryMatch}{
\begin{figure}[ht!]
\centering
\includegraphics[width=\textwidth]{figures3/SceneryMatch2.jpg}
\caption[A scenery match]{\caplbl{SceneryMatch}
% ---
An example of two different animals with appearing in front of the same
distinctive background, illustrating the importance of background
downweighting. The matching regions are highlighted.
% ---
}
\label{fig:SceneryMatch}
\end{figure}
}



\begin{comment}
python -m ibeis gen_featweight_worker --dpath ~/latex/crall-candidacy-2015/ --saveparts --save figures3/genfeatweight.png --figsize=12,3 --dpi=180 --adjust=.15,.15,.1 --diskshow --clipwhite --label genfeatweight --db PZ_MTEST

python -m ibeis.scripts.specialdraw featweight_fig --db PZ_MTEST --aid=1 \
--dpath ~/latex/crall-thesis-2017/ --save figures3/fgweight.png \
--figsize=12,3 --dpi=300 --saveparts --diskshow 
\end{comment}
\newcommand{\fgweight}{
\begin{figure}[ht!]
\centering
\begin{subfigure}[h]{0.32\textwidth}\centering\includegraphics[width=\textwidth]{figures3/fgweightA.png}\caption{}\label{sub:fgweightA}\end{subfigure}
\begin{subfigure}[h]{0.32\textwidth}\centering\includegraphics[width=\textwidth]{figures3/fgweightB.png}\caption{}\label{sub:fgweightB}\end{subfigure}
\begin{subfigure}[h]{0.32\textwidth}\centering\includegraphics[width=\textwidth]{figures3/fgweightC.png}\caption{}\label{sub:fgweightC}\end{subfigure}
\caption[Foregroundness weights]{\caplbl{fgweight}
% ---
Generation of foregroundness feature weights. \Cref{sub:fgweightA} shows the annotation's cropped chip.
This chip is passed to the species detector. \Cref{sub:fgweightB} shows the species detector outputs an
intensity image indicating the likelihood that each pixel belongs to the foreground. \Cref{sub:fgweightC}
shows the weighted sum of the intensity under each feature is used as that feature's foregroundness score.
}
\label{fig:fgweight}
\end{figure}
}



\begin{comment}
python -m ibeis.viz.viz_nearest_descriptors --test-show_nearest_descriptors --db PZ_MTEST --qaid 3 --qfx 1062 --usetex --texknormplot --show 

python -m ibeis.viz.viz_nearest_descriptors --test-show_nearest_descriptors --db PZ_MTEST --qaid 3 --qfx 1062 --usetex --texknormplot --diskshow --saveparts --save figures3/knorm.png --dpi=256 --figsize 30 40  --dpath ~/latex/crall-candidacy-2015/ --hspace .1 --labelsize=42 --reshape 2

879?
%python -m ibeis.viz.viz_nearest_descriptors --test-show_nearest_descriptors --db PZ_MTEST --qaid 3 --qfx 'special' --usetex --texknormplot --show 
python -m ibeis.viz.viz_nearest_descriptors --test-show_nearest_descriptors --db testdb1  --show --qfx 1 

python -m ibeis.viz.interact.interact_matches --test-testdata_match_interact --show --db PZ_MTEST --qaid 3
\end{comment}
\newcommand{\knorm}{
\begin{figure}[ht!]
\centering
\begin{subfigure}[h]{0.18\textwidth}\centering\includegraphics[width=\textwidth]{figures3/knormA.png}\caption{}\label{sub:knorma}\end{subfigure}
\begin{subfigure}[h]{0.18\textwidth}\centering\includegraphics[width=\textwidth]{figures3/knormC.png}\caption{}\label{sub:knormb}\end{subfigure}
\begin{subfigure}[h]{0.18\textwidth}\centering\includegraphics[width=\textwidth]{figures3/knormE.png}\caption{}\label{sub:knormc}\end{subfigure}
\begin{subfigure}[h]{0.18\textwidth}\centering\includegraphics[width=\textwidth]{figures3/knormG.png}\caption{}\label{sub:knormd}\end{subfigure}
\begin{subfigure}[h]{0.18\textwidth}\centering\includegraphics[width=\textwidth]{figures3/knormI.png}\caption{}\label{sub:knorme}\end{subfigure}
\begin{subfigure}[h]{0.18\textwidth}\centering\includegraphics[width=\textwidth]{figures3/knormB.png}\caption{}\label{sub:knormf}\end{subfigure}
\begin{subfigure}[h]{0.18\textwidth}\centering\includegraphics[width=\textwidth]{figures3/knormD.png}\caption{}\label{sub:knormg}\end{subfigure}
\begin{subfigure}[h]{0.18\textwidth}\centering\includegraphics[width=\textwidth]{figures3/knormF.png}\caption{}\label{sub:knormh}\end{subfigure}
\begin{subfigure}[h]{0.18\textwidth}\centering\includegraphics[width=\textwidth]{figures3/knormH.png}\caption{}\label{sub:knormi}\end{subfigure}
\begin{subfigure}[h]{0.18\textwidth}\centering\includegraphics[width=\textwidth]{figures3/knormJ.png}\caption{}\label{sub:knormj}\end{subfigure}
\caption[\caplbl{knorm}LNBNN feature correspondence scoring]{
% ---
\caplbl{knorm} The four nearest neighbors of a distinctive query feature~\cref{sub:knormf}. The bottom row shows
the warped and normalized features with their SIFT descriptors overlaid. The top row shows the annotation from
which each feature was extracted. The first two neighbors~\cref{sub:knormg,sub:knormh} are correct matches, the
third neighbor~\cref{sub:knormi} is an incorrect match, and the fourth neighbor~\cref{sub:knormj} is used as an
LNBNN normalizer to score the first three matches. 
% ---
}
\label{fig:knorm}
\end{figure}
}



\begin{comment}
python -m ibeis.algo.hots.chip_match --test-show_single_namematch --qaid 1 \
    --noshow_truth --show_timedelta=False --show_aid=False \
    --dpath ~/latex/crall-thesis-2017 --save figures3/namematch.jpg --diskshow --dpi=300 --figsize=5,5
#
python -m ibeis.algo.hots.chip_match --test-show_single_namematch --qaid 2 --dpath ~/latex/crall-candidacy-2015 --save figures3/namematch.jpg --diskshow --dpi=180 --clipwhite
python -m ibeis.algo.hots.chip_match --test-show_single_namematch --qaid 3 --dpath ~/latex/crall-candidacy-2015 --save figures3/namematch.jpg --diskshow --dpi=180 --clipwhite
python -m ibeis.algo.hots.chip_match --test-show_single_namematch --qaid 4 --dpath ~/latex/crall-candidacy-2015 --save figures3/namematch.jpg --diskshow --dpi=180 --clipwhite
--verbose 
\end{comment}
\SingleImageCommand{namematch}{1}{Name scoring}{
    % ---
    \Nsumprefix{} \namescoring{}.
    The query annotation is at the top left.
    Each query feature matches at most one feature in the exemplars for a name.
    Each line denotes a feature correspondence colored by its matching score.
    In the top right of each database annotation is its annotation score.
    Feature scores from multiple views are combined into a name score shown on top.
    % ---
}{figures3/namematch.jpg}


\begin{comment}
ibeis sver_single_chipmatch -t default:refine_method=cv2-lmeds-homog,full_homog_checks=True -a default --qaid 18 --dpath ~/latex/crall-candidacy-2015 --save figures3/sverkpts.jpg --label sver --dpi=300 --clipwhite --diskshow --saveparts --figsize=10,10 --norefinelbl
\end{comment}
\newcommand{\sver}{
\begin{figure}[h]
\centering
\begin{subfigure}[h]{0.25\textwidth}\centering\includegraphics[height=130pt]{figures3/sverkptsA.jpg}\caption{}\label{sub:svera}\end{subfigure}
\begin{subfigure}[h]{0.25\textwidth}\centering\includegraphics[height=130pt]{figures3/sverkptsB.jpg}\caption{}\label{sub:sverb}\end{subfigure}
\begin{subfigure}[h]{0.25\textwidth}\centering\includegraphics[height=130pt]{figures3/sverkptsC.jpg}\caption{}\label{sub:sverc}\end{subfigure}
\begin{subfigure}[h]{0.35\textwidth}\centering\includegraphics[width=\textwidth]{figures3/sverkptsD.jpg}\caption{}\label{sub:sverd}\end{subfigure}
\begin{subfigure}[h]{0.35\textwidth}\centering\includegraphics[width=\textwidth]{figures3/sverkptsE.jpg}\caption{}\label{sub:svere}\end{subfigure}
\begin{subfigure}[h]{0.35\textwidth}\centering\includegraphics[width=\textwidth]{figures3/sverkptsF.jpg}\caption{}\label{sub:sverf}\end{subfigure}
\begin{subfigure}[h]{0.35\textwidth}\centering\includegraphics[width=\textwidth]{figures3/sverkptsG.jpg}\caption{}\label{sub:sverg}\end{subfigure}
\caption[Spatial verification]{
    % ---
    An example of spatial verification process. The three images on the top show~\cref{sub:svera} the original
    matches, \Cref{sub:sverb} the best set of inliers from affine hypothesis generation, and \Cref{sub:sverc} the
    final set of homography inliers. The images on the bottom show~\cref{sub:sverd,sub:sverf} the matching images
    warped and superimposed by both the best affine \Cref{sub:svere} and estimated homography
    transformation~\cref{sub:sverg}.
    % ---
}
\label{fig:sver}
\end{figure}
}


\begin{comment}
ALL DATABASE INFO
python -m ibeis Chap3.measure dbstats --dbs=PZ_Master1,GZ_Master1,GIRM_Master1,humpbacks_fb
python -m ibeis Chap3.agg_dbstats
\end{comment}

\newcommand{\DatabaseInfo}{
\begin{table}[ht!]
\centering
\caption[Database statistics]{Database statistics.}
\label{tbl:DatabaseStatistics}
\input{figuresY/agg-enc.tex}
\end{table}
%--
\begin{table}[h!]
\centering
\caption[Annotations per quality]{Annotations per quality.}
\label{tbl:AnnotationsPerQuality}
\input{figuresY/agg-qual.tex}
\end{table}
%--
\begin{table}[h!]
\centering
\caption[Annotations per viewpoint]{Annotations per viewpoint.}
\label{tbl:AnnotationsPerViewpoint}
\input{figuresY/agg-view.tex}
\end{table}
}

%-------------
% TimeDeltas
\begin{comment}
python -m ibeis Chap3.measure time_distri --dbs=GIRM_Master1,GZ_Master1,PZ_Master1,humpbacks_fb --diskshow
python -m ibeis Chap3.draw time_distri --dbs=GIRM_Master1,GZ_Master1,PZ_Master1,humpbacks_fb --diskshow
\end{comment}
\newcommand{\timedist}{
\begin{figure}[ht!] \centering
\begin{subfigure}[h]{\textwidth}\centering\includegraphics[width=.9\textwidth]{figuresY/PZ_Master1/timedist.png}\caption{plains zebras}\end{subfigure}
\begin{subfigure}[h]{\textwidth}\centering\includegraphics[width=.9\textwidth]{figuresY/GZ_Master1/timedist.png}\caption{Grévy's zebras}\end{subfigure}
\begin{subfigure}[h]{\textwidth}\centering\includegraphics[width=.9\textwidth]{figuresY/GIRM_Master1/timedist.png}\caption{Masai giraffes}\end{subfigure}
\begin{subfigure}[h]{\textwidth}\centering\includegraphics[width=.9\textwidth]{figuresY/humpbacks_fb/timedist.png}\caption{Humpbacks}\end{subfigure}
\caption[\caplbl{timedist}Distribution of image timestamps]{\caplbl{timedist}
% ---
Distribution of image timestamps.
The y-axis is plotted on a square-root scale to emphasize times when only a few images were taken.
For plains zebras and Grévy's zebras images were collected over many years.
For Masai giraffes all data was collected immediately before and during the \GZC{}.
% ---
}
\label{fig:timedist}
\end{figure}
}


% -------------------------------
% --- Baseline Experiments ---
% -------------------------------


\begin{comment}                                                                                                                                       
\end{comment}
                                                                                                                                                      
\begin{comment}                                                                                                                                       
python -m ibeis Chap3.draw_agg_baseline --diskshow

python -m ibeis Chap3.draw_all --dbs=GZ_Master1,PZ_Master1,GIRM_Master1
python -m ibeis Chap3.draw_all --db GZ_Master1
python -m ibeis Chap3.draw_all --db PZ_Master1
python -m ibeis Chap3.draw_all --db GIRM_Master1
\end{comment}

\newcommand{\BaselineExpt}{
    \begin{figure}[ht!]\centering
        \begin{subfigure}[h]{\textwidth}\centering\includegraphics[width=\textwidth]{figuresY/agg-baseline.png}\end{subfigure}
        \caption[\caplbl{BaselineExpt}Baseline experiment]{\caplbl{BaselineExpt}
    % ---
    The baseline experiment is a high-level indicator of the ranking accuracy of each species.
    We measure ranking accuracy using a single query and database annotation --- selected from different
      encounters --- per individual.
    The number of query annotations (\pvar{qsize}) and database annotations (\pvar{dsize}) are given for each
      species in the legend.
    % ---
        }
        \label{fig:BaselineExpt}
    \end{figure}
}

\begin{comment}
python -m ibeis Chap3.draw_all --dbs=GZ_Master1,PZ_Master1
\end{comment}



\begin{comment}

ibeis Chap3.measure smk --db=GZ_Master1
ibeis Chap3.draw smk --db=GZ_Master1 --diskshow

ibeis Chap3.measure smk --db=PZ_Master1
ibeis Chap3.draw smk --db=PZ_Master1 --diskshow

ibeis Chap3.measure smk --dbs=GZ_Master1,PZ_Master1
ibeis Chap3.draw smk --dbs=GZ_Master1,PZ_Master1 --diskshow
\end{comment}
\newcommand{\SMKExpt}{
\begin{figure}[ht!]\centering
    \begin{subfigure}[h]{\textwidth}\centering\includegraphics[width=\textwidth]{figuresY/PZ_Master1/smk.png}\caption{plains zebras}\label{sub:SMKExptA}\end{subfigure}
    \begin{subfigure}[h]{\textwidth}\centering\includegraphics[width=\textwidth]{figuresY/GZ_Master1/smk.png}\caption{Grévy's zebras}\label{sub:SMKExptB}\end{subfigure}
    \caption[\caplbl{SMKExpt}SMK experiment]{\caplbl{SMKExpt}
    % ---
    The (VLAD based) SMK algorithm compared to our LNBNN ranking algorithm.
    The results demonstrate that LNBNN outperforms the ranking accuracy of SMK.
    The number of query/database annotations (\pvar{qsize} / \pvar{dsize}) are
    shown in the lower left.
    % ---
    }
    \label{fig:SMKExpt}
\end{figure}
}



\begin{comment}
python -m ibeis Chap3.measure foregroundness --dbs=GZ_Master1,PZ_Master1
python -m ibeis Chap3.draw foregroundness --dbs=GZ_Master1,PZ_Master1 --diskshow

python -m ibeis -e draw_rank_cmc --db GZ_Master1   -a timectrl   -t baseline:fg_on=[True,False]  --show
\end{comment}
\newcommand{\ForegroundExpt}{
    \begin{figure}[ht!]\centering
        \begin{subfigure}[h]{\textwidth}\centering\includegraphics[width=\textwidth]{figuresY/PZ_Master1/foregroundness.png}\caption{plains zebras}\label{sub:ForegroundExptA}\end{subfigure}
        \begin{subfigure}[h]{\textwidth}\centering\includegraphics[width=\textwidth]{figuresY/GZ_Master1/foregroundness.png}\caption{Grévy's zebras}\label{sub:ForegroundExptB}\end{subfigure}
        \caption[\caplbl{ForegroundExpt}Foregroundness experiment]{\caplbl{ForegroundExpt}
            % ---
            Weighting the score of the feature correspondences using foregroundness results in more accurate
              identifications.
            % ---
        }
        \label{fig:ForegroundExpt}
    \end{figure}
}


\newcommand{\FGIntraExpt}{
    \begin{figure}[ht!]\centering
        \begin{subfigure}[h]{\textwidth}\centering\includegraphics[width=\textwidth]{figuresY/PZ_Master1/foregroundness_intra.png}\caption{plains zebras}\end{subfigure}
        \begin{subfigure}[h]{\textwidth}\centering\includegraphics[width=\textwidth]{figuresY/GZ_Master1/foregroundness_intra.png}\caption{Grévy's zebras}\end{subfigure}
        \caption[\caplbl{FGIntraExpt}Foregroundness experiment]{\caplbl{FGIntraExpt}
            % ---
            Applying foregroundness weights to feature correspondences improves the identification accuracy at
              the top rank by filtering matches in scenery.
            This experiment was performed by matching annotations within serveral occurrences.
            Thus, in this experiment \pvar{qsize} is a sum and \pvar{dsize} is an average.
            % ---
        }
        \label{fig:FGIntraExpt}
    \end{figure}
}

% -------------------------------
% --- Invariance Experiments ----
% -------------------------------


\newcommand{\InvarExpt}{
    \begin{figure}[ht!]\centering
        \begin{subfigure}[h]{\textwidth}\centering\includegraphics[width=\textwidth]{figuresY/PZ_Master1/invar.png}\caption{plains zebras}\label{sub:InvarExptA}\end{subfigure}
        \begin{subfigure}[h]{\textwidth}\centering\includegraphics[width=\textwidth]{figuresY/GZ_Master1/invar.png}\caption{Grévy's zebras}\label{sub:InvarExptB}\end{subfigure}
        \caption[\caplbl{InvarExpt}Feature invariance experiment]{\caplbl{InvarExpt}
            % ---
            Results of the feature invariance experiment, testing the effect of affine invariance (AI) and the
              query-side rotation heuristic (QRH).
            For plains zebras circular keypoints with the QRH are the most accurate.
            For Grévy's zebras enabling affine invariance works the best.
            The number of query/database annotations (\pvar{qsize} / \pvar{dsize}) are shown in the lower left.
            % ---
        }
        \label{fig:InvarExpt}
    \end{figure}
}



% TODO; http://tex.stackexchange.com/questions/75014/is-it-possible-to-make-a-reference-to-a-subfigure-to-appear-figure-2a-with-cle

\begin{comment}
    python -m ibeis.viz.viz_chip --test-show_chip --aid 44 \
        --weight_label=None --ecc --dpi=300 --draw_lbls=False \
        --ellalpha=.8 --ell_linewidth=1.4 --notitle \
        --dpath ~/latex/crall-thesis-2017/figures3 --save=pzaffkpts.jpg --diskshow --darken

    python -m ibeis.viz.viz_chip --test-show_chip --aid 44 \
        --weight_label=None --ecc --dpi=300 --draw_lbls=False \
        --ellalpha=.8 --ell_linewidth=1.4 --notitle \
        --affine-invariance=False --augment_orientation=True --ori \
        --dpath ~/latex/crall-thesis-2017/figures3 --save=pzcirckpts.jpg --diskshow --darken

    python -m ibeis.viz.viz_chip --test-show_chip --db GZ_Master1 --aid 1000 \
        --weight_label=None --ecc --dpi=300 --draw_lbls=False \
        --ellalpha=.8 --ell_linewidth=1.4 --notitle \
        --dpath ~/latex/crall-thesis-2017/figures3 --save=gzaffkpts.jpg --diskshow --darken

    python -m ibeis.viz.viz_chip --test-show_chip --db GZ_Master1 --aid 1000 \
        --weight_label=None --ecc --dpi=300 --draw_lbls=False \
        --ellalpha=.8 --ell_linewidth=1.4 --notitle \
        --affine-invariance=False --augment_orientation=True --ori \
        --dpath ~/latex/crall-thesis-2017/figures3 --save=gzcirckpts.jpg --diskshow --darken 
\end{comment}
\newcommand{\kptstype}{
    \begin{figure}[ht!]\centering
        \begin{subfigure}[h]{.48\textwidth}\centering\includegraphics[width=\textwidth]{figures3/pzaffkpts.jpg}\caption{}\label{sub:kptstypeA}\end{subfigure}
        \begin{subfigure}[h]{.48\textwidth}\centering\includegraphics[width=\textwidth]{figures3/pzcirckpts.jpg}\caption{}\label{sub:kptstypeB}\end{subfigure}
        \begin{subfigure}[h]{.48\textwidth}\centering\includegraphics[width=\textwidth]{figures3/gzaffkpts.jpg}\caption{}\label{sub:kptstypeC}\end{subfigure}
        \begin{subfigure}[h]{.48\textwidth}\centering\includegraphics[width=\textwidth]{figures3/gzcirckpts.jpg}\caption{}\label{sub:kptstypeD}\end{subfigure}
        \caption[\caplbl{kptstype}Examples of keypoint invariance]{\caplbl{kptstype}
            % ---
            Many affine keypoints detected on plains zebras tend to encompass only one or two stripes. The distinctive stripe
            patterns on Grévy's zebras are well captured by affine keypoints, whereas circular keypoints are more spread out.
            For visibility this figure shows a random sample of all keypoints on a darkened image. Elliptical keypoints
            in~\cref{sub:kptstypeA,sub:kptstypeC} are colored by eccentricity and circular keypoints
            in~\cref{sub:kptstypeB,sub:kptstypeD} are colored by scale.
            % ---
        }
        \label{fig:kptstype}
    \end{figure}
}

% -------------------------------
% --- Namescore Experiments ----
% -------------------------------

\begin{comment}
python -m ibeis Chap3.measure nsum --dbs=GZ_Master1
python -m ibeis Chap3.measure nsum --dbs=GZ_Master1,PZ_Master1
python -m ibeis Chap3.draw nsum --dbs=GZ_Master1,PZ_Master1 --diskshow
\end{comment}

\newcommand{\NScoreExpt}{
    \begin{figure}[ht!]\centering
        \begin{subfigure}[h]{\textwidth}\centering\includegraphics[width=\textwidth]{figuresY/PZ_Master1/nsum.png}\caption{plains zebras}\label{sub:NScoreExptA}\end{subfigure}
        \begin{subfigure}[h]{\textwidth}\centering\includegraphics[width=\textwidth]{figuresY/GZ_Master1/nsum.png}\caption{Grévy's zebras}\label{sub:NScoreExptB}\end{subfigure}
        \caption[\caplbl{NScoreExpt}Name scoring experiment]{\caplbl{NScoreExpt}
            % ---
            Results of the name scoring mechanism experiment.
            There is a clear separation between identification accuracy when the number of exemplars per name is
              $1$ compared to when it is $3$.
            Feature based name scoring (\nsum{}) is slightly more accurate than scoring using the annotation
              based name scoring (\csum{}).
            The number of query /database annotations (\pvar{qsize} / \pvar{dsize}) are shown in the lower left.
            Database size was normalized using confusors.
            %Note that the scores reported here are higher than the baseline for
            %  the same reasons as explained in~\cref{fig:DBSizeExpt}.
            % ---
        }
        \label{fig:NScoreExpt}
    \end{figure}
}



% -------------------------------
% --- K Experiments
% -------------------------------

\begin{comment}
python -m ibeis Chap3.measure kexpt --dbs=GZ_Master1,PZ_Master1
python -m ibeis Chap3.draw kexpt --dbs=GZ_Master1,PZ_Master1 --diskshow
\end{comment}
\newcommand{\KExptA}{
    \begin{figure}[ht!]\centering
        \centering\includegraphics[width=\textwidth]{figuresY/PZ_Master1/kexpt.png}
        \caption[\caplbl{KExptA}The $K$ experiment for plains zebras]{\caplbl{KExptA}
            % ---
            Identification accuracy for plains zebras using different values of $\K$ (the number of nearest
              neighbors assigned to each query feature), different numbers of exemplars (\pvar{dpername}), and
              different database sizes (\pvar{dsize}).
            %Note that the scores reported here are higher than the baseline for
            %  the same reasons as explained in~\cref{fig:DBSizeExpt}.
            % ---
        }
        \label{fig:KExptA}
    \end{figure}
}
\newcommand{\KExptB}{
    \begin{figure}[ht!]\centering
        \centering\includegraphics[width=\textwidth]{figuresY/GZ_Master1/kexpt.png}
        \caption[\caplbl{KExptB}The $K$ experiment for Grévy's zebras]{\caplbl{KExptB}
            % ---
            Identification accuracy for Grévy's zebras using different values of $\K$ (the number of nearest
              neighbors assigned to each query feature), different numbers of exemplars (\pvar{dpername}), and
              different database sizes (\pvar{dsize}).
            %Note that the scores reported here are higher than the baseline for
            %  the same reasons as explained in~\cref{fig:DBSizeExpt}.
            % ---
        }
        \label{fig:KExptB}
    \end{figure}
}


% --- Photobomb

\begin{comment}
python -m ibeis.dev -e draw_cases -a timectrl -t best --filt :fail=True,with_tag=Photobomb,sortdsc=gfscore --db PZ_Master1 \
--qaid=3727 --cmdaug="FailPhotobomb" --hargv=match --render
\end{comment}
\SingleImageCommand{FailPhotobomb}{1}{
    Photobomb failure case
}{
% ---
A photobombing animal in the background of the query annotation cause LNBNN to return the incorrect result (on
  the left) at rank $1$.
The correct match (on the right), has a significant number of matches, but there is a difference of $1$ day
  between the pair.
On the other hand, the annotations in the photobomb pair were taken within minutes of each other and therefore
  have much higher visual similarity.
% ---
}{figuresC/case_FailPhotobomb.png}


% --- Scenery Match

\begin{comment}
python -m ibeis.dev -e draw_cases -a timectrl -t best --filt :fail=True,with_tag=SceneryMatch,sortdsc=gfscore --db GZ_Master1 \
    --qaid 1988 \
    --hargv=match --render  --cmdaug="FailScenery" \
    --cappref="Failure case due to a scenery match. Most scenery match cases have small timedeltas between the images."

# 2811
python -m ibeis.dev -e draw_cases -a timectrl -t best --filt :fail=True,with_tag=SceneryMatch,sortdsc=gfscore --db GZ_Master1 --qaid 1988 --show

python -m ibeis.dev -e draw_cases -a timectrl -t best --filt :fail=True,with_tag=SceneryMatch,sortdsc=gfscore --db GZ_Master1 --qaid 1988 --show
python -m ibeis.dev -e draw_cases -a timectrl -t best:sv_on=False --filt :fail=True,with_tag=SceneryMatch,sortdsc=gfscore --db GZ_Master1 --qaid 1988 --show
python -m ibeis.dev -e draw_cases -a timectrl -t best:sv_on=False,AI=False --filt :fail=True,with_tag=SceneryMatch,sortdsc=gfscore --db GZ_Master1 --qaid 1988 --show
\end{comment}
\SingleImageCommand{FailScenery}{1}{
    Scenery failure case
}{
% ---
The incorrect pair of annotations (on the left) was returned at rank $1$ because of strong matches in the
  background scenery.
The correct pair was returned at rank $2$ and did not produce matches in the front leg due to pose variations.
The annotations in the scenery match pair were taken $8$ seconds appart in the same location causing their
  backgrounds to be near duplicates.
The foregroundness measure was disabled to produce this example, enabling it addresses nearly all scenery match
  cases.
% ---
}{figuresC/case_FailScenery.png}

% --- QUALITY

\begin{comment}
python -m ibeis.dev -e draw_cases -a timectrl -t best --filt :fail=True,with_tag=Quality,sortdsc=gfscore --db GIRM_Master1 --qaid 639 \
    --hargv=match --render --cmdaug="FailQuality" --vert=False
\end{comment}
\SingleImageCommand{FailQuality}{1}{
Quality failure case
}{
% ---
The low resolution of the query annotation and the overall viewpoint difference causes the correct pair of
  annotations (on the right) to be returned at rank $75$.
The incorrect pair of annotation (on the left) did not recieve a particularly high score, but it was returned at
  rank $1$ because there were no feature correspondences established to the correct match.
% ---
}{figuresC/case_FailQuality.png}


% --- OCCLUSION
\begin{comment}
python -m ibeis.dev -e draw_cases -a timectrl -t best --filt :fail=True,with_tag=Occlusion,sortdsc=gfscore --db PZ_Master1 --qaid 3812 \
    --hargv=match --render  --cmdaug="Occlusion" \
    --cappref="Failure case due to occlusion. "
\end{comment}
\SingleImageCommand{FailOcclusion}{1}{
Occlusion failure case
}{
% ---
The plants occluding both the query and database annotations inhibit the creation of feature correspondences,
  causing the correct pair of annotations (on the right) to be returned at rank $2$.
This is exacerbated by pose and viewpoint variations.
The incorrect pair of annotation (on the left) at rank $1$ are relatively distinctive by coincidence.
% ---
}{figuresC/case_Occlusion.png}



% --- VIEWPOINT
\begin{comment}
python -m ibeis.dev -e draw_cases -a timectrl -t best --filt :fail=True,with_tag=Viewpoint,sortdsc=gtscore --db GZ_Master1  --qaid 2787 \
    --hargv=match --render  --cmdaug="FailViewpoint" \
    --overwrite
%--qaid 2660 \
\end{comment}
\SingleImageCommand{FailViewpoint}{1}{
Unaligned failure case
}{
% ---
Due to pose and viewpoint variations, the correctly matching pair of annotations (on the right) is returned at
  rank $2$ while the incorrect pair of annotations (on the left) is returned at rank $1$.
In the correct pair, the features on the front leg are not aligned and failed to match.
In the incorrect pair, the heads of the animals are in a similar pose and thus creating several correspondences
  that are distinctive by coincidence.
% ---
}{figuresC/case_FailViewpoint.png}

%\edef\pzcode{PZ_PB_RF_TRAIN_2567}
%\edef\gzcode{GZ_Master1_21589}
%%\edef\gzcode{GZ_Master1_222}


%\newcommand{\PairFeatVec}{
%\begin{figure}
%\begin{minted}[gobble=4]{python}
%    OrderedDict([('global(qual_min)',    3),
%                 ('global(qual_max)',    nan),
%                 ('global(qual_delta)',  nan),
%                 ('global(gps_delta)',   5.79),
%                 ('len(matches)',        20),
%                 ('sum(ratio)',          10.05),
%                 ('mean(ratio)',         0.50),
%                 ('std(ratio)',          0.09)])
%\end{minted}
%\caption[\caplbl{PairFeatVec}A pairwise feature vector]{\caplbl{PairFeatVec} %
%Example of a small pairwise feature vector containing local and global information.
%In practice the pairwise feature vectors include many more dimensions.
%}
%\label{fig:PairFeatVec}
%\end{figure}
%}

\newcommand{\PairDBStats}{
\begin{table}[b]
    \centering
    \caption[\caplbl{PairDBStats}Database statistics for the pairwise experiment]{\caplbl{PairDBStats}
    % ---
    Database statistics for the pairwise experiment.
    Starting with a database of annotations with name labels, we sample a set of annotation pairs to evaluate our
      pairwise classifiers with.
    % ---
    }
    \label{tbl:PairDBStats}
    \begin{tabular}{lrrrrrr}
    \toprule
               {}   & {Names} & {Annots} & {Positive} & {Negative} & {Incomparable} & {Photobombs} \\
    \midrule
      Plains zebras &            1202 &             5720 &   16583   & 30376    & 353  & 286 \\
     Grévy's zebras &             771 &             2283 &   5002    & 13008    & 0    & 76  \\
    \bottomrule
    \end{tabular}
\end{table}
}


\newcommand{\MatchStateExample}{
\begin{figure}[h] \centering
\begin{subfigure}[h]{0.26\textwidth}\centering\includegraphics[height=160pt]{figures4/classesC.png}\caption{Positive}\label{sub:classesC}\end{subfigure}
\begin{subfigure}[h]{0.31\textwidth}\centering\includegraphics[height=160pt]{figures4/classesA.png}\caption{Negative}\label{sub:classesA}\end{subfigure}
\begin{subfigure}[h]{0.31\textwidth}\centering\includegraphics[height=160pt]{figures4/classesB.png}\caption{Incomparable}\label{sub:classesB}\end{subfigure}
\caption[\caplbl{MatchStateExample}Match-state example]{\caplbl{MatchStateExample}
% ---
Examples of positive \cref{sub:classesC}, negative  \cref{sub:classesA}, and incomparable \cref{sub:classesB}
pairs of annotations. Local feature correspondences are superimposed over the pairs.
% ---
}
\label{fig:MatchStateExample}
\end{figure}
}


\newcommand{\LeftRightFace}{
\begin{figure}[h]
\centering
\includegraphics[width=\textwidth]{figures4/custom_match_leftrightface_5245_5161.jpg}
\caption[\caplbl{LeftRightFace}A comparable pair with different viewpoints]{\caplbl{LeftRightFace}
% ---
Even though this pair has different viewpoints, it is positive and comparable
because we can establish a distinctive correspondence in the face. 
% ---
}
\label{fig:LeftRightFace}
\end{figure}
}


% -------------------
% --- Experiments ---
% -------------------

\begin{comment}
    python -m ibeis Chap4.measure_all --db PZ_PB_RF_TRAIN
    python -m ibeis Chap4.measure_all --db GZ_Master1 && python -m ibeis Chap4.measure_all --db PZ_Master1
    python -m ibeis Chap4.measure_all --db PZ_MTEST

    python -m ibeis Chap4.draw_all --db PZ_MTEST
    python -m ibeis Chap4.draw_all --db PZ_Master1
    python -m ibeis Chap4.draw_all --db GZ_Master1
\end{comment}


\newcommand{\PositiveHist}{
\begin{figure}[h]
\centering
\begin{subfigure}[h]{0.47\textwidth}\centering\includegraphics[width=\textwidth]{figures4/PZ_Master1/score_hist_lnbnn.png}\caption{Plains zebras LNBNN}\label{sub:pos_lnbnn_hist_pz}\end{subfigure}
\begin{subfigure}[h]{0.47\textwidth}\centering\includegraphics[width=\textwidth]{figures4/PZ_Master1/score_hist_pos_learn(sum,glob).png}\caption{Plains zebras learned}\label{sub:pos_hist_pz}\end{subfigure}
\begin{subfigure}[h]{0.47\textwidth}\centering\includegraphics[width=\textwidth]{figures4/GZ_Master1/score_hist_lnbnn.png}\caption{Grévy's zebras LNBNN}\label{sub:pos_lnbnn_hist_gz}\end{subfigure}
\begin{subfigure}[h]{0.47\textwidth}\centering\includegraphics[width=\textwidth]{figures4/GZ_Master1/score_hist_pos_learn(sum,glob).png}\caption{Grévy's zebras learned}\label{sub:pos_hist_gz} \end{subfigure}
%~
\caption[\caplbl{PositiveHist}Positive score histogram experiment]{\caplbl{PositiveHist}
% ---
Positive scores of LNBNN (left) and the pairwise algorithm (right) for pairs of plains (top) and Grévy's (bottom)
  zebras.
The learned probabilities are more separable and more interpretable than LNBNN scores.
%In addition to being more interpretable than LNBNN scores the learned probabilities exhibit better separability.
Note that in this plot, negative refers to annotation pairs with a non-positive match-state label.
% ---
}
\label{fig:PositiveHist}
\end{figure}
}



\newcommand{\PositiveROC}{
\begin{figure}[h]
\centering
\begin{subfigure}[h]{0.47\textwidth}\centering\includegraphics[width=\textwidth]{figures4/PZ_Master1/roc_match_state.png}\caption{Plains zebra}\end{subfigure}
\begin{subfigure}[h]{0.47\textwidth}\centering\includegraphics[width=\textwidth]{figures4/GZ_Master1/roc_match_state.png}\caption{Grévy's zebra}\end{subfigure}
\caption[\caplbl{PositiveROC}Positive match-state ROC experiment]{\caplbl{PositiveROC}
% ---
The positive match-state ROC for scores computed by the pairwise classifier and LNBNN.
The pairwise classifier significantly improves the separation of positive and non-positive pairs.
%The scores from the pairwise classifier are better at separating positive and non-positive cases.
%Additionally, operating points exist where the true positive rate is high and
%the false positive rate is near zero.
%can be selected to automatically review a significant number of positive cases
%  while making only a few errors.
% ---
}
\label{fig:PositiveROC}
\end{figure}
}


\newcommand{\ReRank}{
\begin{figure}[h]
\centering
\begin{subfigure}[h]{\textwidth}\centering\includegraphics[width=\textwidth]{figures4/PZ_Master1/rerank.png}\caption{Plains zebra}\end{subfigure}
\begin{subfigure}[h]{\textwidth}\centering\includegraphics[width=\textwidth]{figures4/GZ_Master1/rerank.png}\caption{Grévy's zebra}\end{subfigure}
\caption[\caplbl{ReRank}Re-ranking experiment]{\caplbl{ReRank}
% ---
Re-ranking the top LNBNN results using the positive probabilities from the match-state classifier improves the
  number of correct matches at rank $1$ for both plains and Grévy's zebra.
% ---
}
\label{fig:ReRank}
\end{figure}
}


\begin{comment}
    python -m ibeis.scripts.thesis ExptChapter4.write_metrics --db GZ_Master1 --task-key=match_state
    python -m ibeis.scripts.thesis ExptChapter4.write_metrics --db PZ_Master1 --task-key=match_state
\end{comment}
\newcommand{\ConfusionMatch}{
\begin{table}[b]
    \centering
    \caption[\caplbl{ConfusionMatch}Match-state experiment confusion matrix]{\caplbl{ConfusionMatch}
    % ---
    Multiclass match-state confusion for plains and Grévy's zebras.
    The rows are the real (groundtruth) state, and the columns are the predicted states.
    %The final column indicates the number of examples for each class.
    %A pair is classified as the state (positive, negative, or incomparable) with the maximum predicted
    %  probability.
    Each pair is classified as positive, negative, or incomparable depending on which state has the maximum
      probability.
    % ---
    }
    \label{tbl:ConfusionMatch}
    \begin{subtable}[h]{\textwidth}\centering\input{figures4/PZ_Master1/confusion_match_state.tex}\caption{Plains zebras match-state confusion matrix}\end{subtable} %
    \begin{subtable}[h]{\textwidth}\centering\input{figures4/GZ_Master1/confusion_match_state.tex}\caption{Grévy's zebras match-state confusion matrix}\end{subtable} %
\end{table}
}


\newcommand{\EvalMetricsMatch}{
\begin{table}[b]
    \caption[\caplbl{EvalMetricsMatch}Match-state experiment evaluation metrics]{\caplbl{EvalMetricsMatch}
    % ---
    Multiclass match-state evaluation metrics for plains and Grévy's zebras computed from the confusion matrix.
    These metrics demonstrate that our match-state classifiers have strong predictive power.
    %A pair is classified as positive, negative, or incomparable depending on which state has the maximum
    %  probability.
    % ---
    }
    \label{tbl:EvalMetricsMatch}
    \centering
    \begin{subtable}[h]{\textwidth}\centering\input{figures4/PZ_Master1/eval_metrics_match_state.tex}\caption{Plains zebras match-state metrics}\end{subtable} %
    \begin{subtable}[h]{\textwidth}\centering\input{figures4/GZ_Master1/eval_metrics_match_state.tex}\caption{Grévy's zebras match-state metrics}\end{subtable} %
\end{table}
}

\begin{comment}
    python -m ibeis.scripts.thesis ExptChapter4.write_metrics --db GZ_Master1 --task-key=photobomb_state
    python -m ibeis.scripts.thesis ExptChapter4.write_metrics --db PZ_Master1 --task-key=photobomb_state
\end{comment}
\newcommand{\ConfusionPhotobomb}{
\begin{table}[h]
    \caption[\caplbl{ConfusionPhotobomb}Photobomb-state experiment confusion matrix]{\caplbl{ConfusionPhotobomb}
    % ---
    The photobomb-state confusion matrix. The columns indicate predicted classes, and the rows indicate real
    (groundtruth) classes.  The final column indicates the number of examples of each class.
    A pair is classified as a photobomb if its probability is greater than $0.5$.
    % ---
    }
    \label{tbl:ConfusionPhotobomb}
    \begin{subtable}[h]{\textwidth}\centering\input{figures4/PZ_Master1/confusion_photobomb_state.tex}\caption{Plains zebra photobomb confusion matrix}\end{subtable} %
    \begin{subtable}[h]{\textwidth}\centering\input{figures4/GZ_Master1/confusion_photobomb_state.tex}\caption{Grévy's zebra photobomb confusion matrix}\end{subtable} %
\end{table}
}


\newcommand{\EvalMetricsPhotobomb}{
\begin{table}[h]
    \caption[\caplbl{EvalMetricsPhotobomb}Photobomb-state experiment evaluation metrics]{\caplbl{EvalMetricsPhotobomb}
    % ---
    The photobomb-state evaluation metrics computed from the confusion matrix.
    A pair is classified as a photobomb if its probability is greater than $0.5$.
    % ---
    }
    \label{tbl:EvalMetricsPhotobomb}
    \centering
    \begin{subtable}[h]{\textwidth}\centering\input{figures4/PZ_Master1/eval_metrics_photobomb_state.tex}\caption{Plains zebras photobomb metrics} \end{subtable} %
    \begin{subtable}[h]{\textwidth}\centering\input{figures4/GZ_Master1/eval_metrics_photobomb_state.tex}\caption{Grévy's zebras photobomb metrics}\end{subtable} %
\end{table}
}


\newcommand{\ConfusionPhotobombII}{
\begin{table}[p]
    \caption[\caplbl{ConfusionPhotobombII}Photobomb-state adjusted confusion matrix]{\caplbl{ConfusionPhotobombII} 
    % ---
    The photobomb-state confusion matrix after adjusting the probability threshold to maximize the MCC.
    The columns indicate predicted classes, and the rows indicate real (groundtruth) classes.
    The final column indicates the number of examples of each class.
    % ---
    }
    \label{tbl:ConfusionPhotobombII}
    \begin{subtable}[h]{\textwidth}\centering\input{figures4/PZ_Master1/confusion2_photobomb_state.tex}\caption{Plains zebra photobomb adjusted confusion matrix}\end{subtable} %
    \begin{subtable}[h]{\textwidth}\centering\input{figures4/GZ_Master1/confusion2_photobomb_state.tex}\caption{Grévy's zebra photobomb adjusted confusion matrix}\end{subtable} %
\end{table}
}


\newcommand{\EvalMetricsPhotobombII}{
\begin{table}[p]
    \caption[\caplbl{EvalMetricsPhotobombII}Photobomb-state adjusted evaluation metrics]{\caplbl{EvalMetricsPhotobombII}
    % ---
    The photobomb-state evaluation metrics computed from the confusion matrix after adjusting the probability
      threshold to maximize the MCC.
    % ---
    }
    \label{tbl:EvalMetricsPhotobombII}
    \centering
    \begin{subtable}[h]{\textwidth}\centering\input{figures4/PZ_Master1/eval_metrics2_photobomb_state.tex}\caption{Plains zebras adjusted photobomb metrics} \end{subtable} %
    \begin{subtable}[h]{\textwidth}\centering\input{figures4/GZ_Master1/eval_metrics2_photobomb_state.tex}\caption{Grévy's zebras adjusted photobomb metrics}\end{subtable} %
\end{table}
}



\begin{comment}
    python -m ibeis.scripts.script_vsone report_classifier_importance --db PZ_Master1 \
    --dpath ~/latex/crall-thesis-2017/ --save "figures4/wc_pz_clipwhite2.png" \
    --clipwhite --diskshow

    python -m ibeis.scripts.script_vsone report_classifier_importance --db GZ_Master1 \
    --dpath ~/latex/crall-thesis-2017/ --save "figures4/wc_gz_clipwhite2.png" \
    --clipwhite --diskshow
\end{comment}

\newcommand{\MatchWordCloud}{
\begin{figure}[h]
\centering
\begin{subfigure}[h]{0.47\textwidth}\centering\includegraphics[width=\textwidth]{figures4/PZ_Master1/wc_match_state.png}\caption{Plains zebra}\label{sub:wc_pz}\end{subfigure}
\begin{subfigure}[h]{0.47\textwidth}\centering\includegraphics[width=\textwidth]{figures4/GZ_Master1/wc_match_state.png}\caption{Grévy's zebra}\label{sub:wc_gz}\end{subfigure}
\caption[\caplbl{MatchWordCloud}Word cloud of important features for match-state prediction]{\caplbl{MatchWordCloud}
% ---
Word cloud of important features for predicting match-state.
For plains zebras, the difference in viewpoint (denoted as \pvar{global(view\_delta)}) has high importance
  because of its role in distinguishing incomparable cases.
% ---
}
\label{fig:MatchWordCloud}
\end{figure}
}


\newcommand{\MatchPrune }{
\begin{figure}[h]
\centering
%\begin{subfigure}[h]{0.47\textwidth}\centering\includegraphics[width=\textwidth]{figures4/PZ_PB_RF_TRAIN/wc_match_state.png}\caption{Plains zebra}\label{sub:wc_pz}\end{subfigure}
\begin{subfigure}[h]{0.49\textwidth}\centering\includegraphics[width=\textwidth]{figures4/PZ_Master1/prune.png}\caption{Plains zebra}\end{subfigure}
\begin{subfigure}[h]{0.49\textwidth}\centering\includegraphics[width=\textwidth]{figures4/GZ_Master1/prune.png}\caption{Grévy's zebra}\end{subfigure}
\caption[\caplbl{MatchPrune}Pruning feature dimensions for match classification]{\caplbl{MatchPrune}
% ---
The effect of pruning the least important feature dimensions on the MCC of the match-state classifier.
We find that a reduced subset of feature dimensions results in a slight increase in classification accuracy over
  the original $131$ features.
However, there is a point at which reducing the number of feature dimensions significantly degrades performance.
% ---
}
\label{fig:MatchPrune}
\end{figure}
}


\newcommand{\ImportantMatchFeat}{
\begin{table}[p]
    \centering
    \caption[\caplbl{ImportantMatchFeat}Important features for match-state prediction]{\caplbl{ImportantMatchFeat} 
    % ---
    Top $10$ most important features for predicting the match-state (positive, negative, incomparable) for a pair
      of annotations.
    The sum of the importance for all $131$ feature dimensions is $1$.
    % ---
    }
    \label{tbl:ImportantMatchFeat}
    \begin{subtable}[h]{1.0\textwidth}
        \centering
        \begin{tabular}{l c}
            \toprule
            Dimension & Importance\\
            \midrule
            \input{figures4/PZ_Master1/feat_importance_match_state.tex}
            \bottomrule
        \end{tabular}
        \caption{Plains zebra}
    \end{subtable} %
    %~~~~~~~~ 
    \begin{subtable}[h]{1.0\textwidth}
        \centering
        \begin{tabular}{l c}
            \toprule
            Dimension & Importance\\
            \midrule
            \input{figures4/GZ_Master1/feat_importance_match_state.tex}
            \bottomrule
        \end{tabular}
        \caption{Grévy's zebra}
    \end{subtable} %
\end{table}
}


\newcommand{\ImportantMatchFeatPrune}{
\begin{table}[p]
    \centering
    \caption[\caplbl{ImportantMatchFeatPrune}Important features for match-state prediction]{\caplbl{ImportantMatchFeatPrune} 
    % ---
    Top $10$ most important feature dimensions for predicting the match-state (positive, negative, incomparable)
      for a pair of annotations, after removing the least important dimensions.
    %The sum of the importance for all $131$ feature dimensions is $1$.
    % ---
    }
    \label{tbl:ImportantMatchFeatPrune}
    \begin{subtable}[h]{1.0\textwidth}
        \centering
        \begin{tabular}{l c}
            \toprule
            Dimension & Importance\\
            \midrule
            \input{figures4/PZ_Master1/pruned_feat_importance_match_state.tex}
            \bottomrule
        \end{tabular}
        \caption{Plains zebra}
    \end{subtable} %
    %~~~~~~~~ 
    \begin{subtable}[h]{1.0\textwidth}
        \centering
        \begin{tabular}{l c}
            \toprule
            Dimension & Importance\\
            \midrule
            \input{figures4/GZ_Master1/pruned_feat_importance_match_state.tex}
            \bottomrule
        \end{tabular}
        \caption{Grévy's zebra}
    \end{subtable} %
\end{table}
}


\newcommand{\ImportantPBFeat}{
    \begin{table}[h]
        \centering
        \caption[\caplbl{ImportantPBFeat}Important features for photobomb-state prediction]{\caplbl{ImportantPBFeat} 
        % ---
        The top $10$ most important features for predicting if a pair of annotations has a photobomb.
        Features like speed and GPS delta are important because photobombs are more common in pairs of
          annotations taken at the same time and place.
        Features related to the spatial distribution of the feature correspondences are important because
          photobombing animals often appear off to one side of an annotation.
        %they might preclude a match from occurring,
        %  and because annotations taken
        % ---
        }
        \label{tbl:ImportantPBFeat}
        \begin{subtable}[h]{\textwidth} 
            \centering
            \begin{tabular}{l c}
                \toprule Dimension & Importance\\
                \midrule
                \input{figures4/PZ_Master1/feat_importance_photobomb_state.tex}
                \bottomrule
            \end{tabular}
            \caption{Plains zebra photobomb importance}
        \end{subtable} %
        %~~~~
        \begin{subtable}[h]{\textwidth}
            \centering 
            \begin{tabular}{l c}
                \toprule
                Dimension & Importance\\ 
                \midrule
                \input{figures4/GZ_Master1/feat_importance_photobomb_state.tex}
                \bottomrule
            \end{tabular}
            \caption{Grévy's zebra photobomb importance}
        \end{subtable} %
    \end{table}
}


% ---------------------
% --- Failure Cases ---
% ---------------------



\newcommand{\PairFailPN}{
\begin{figure}[h]
\centering
\begin{subfigure}[h]{.65\textwidth}\centering\includegraphics[width=\textwidth]{figures4/PZ_Master1/cases_match_state/fail_match_nomatch_835_5325.jpg}\end{subfigure}
\begin{subfigure}[h]{.65\textwidth}\centering\includegraphics[width=\textwidth]{figures4/PZ_Master1/cases_match_state/fail_match_nomatch_1022_7845.jpg}\end{subfigure}
\begin{subfigure}[h]{.65\textwidth}\centering\includegraphics[width=\textwidth]{figures4/GZ_Master1/cases_match_state/fail_match_nomatch_1511_2145.jpg}\end{subfigure}
\begin{subfigure}[h]{.65\textwidth}\centering\includegraphics[width=\textwidth]{figures4/GZ_Master1/cases_match_state/fail_match_nomatch_1453_2042.jpg}\end{subfigure}
\caption[\caplbl{PairFailPN}Positive pairwise failure case]{\caplbl{PairFailPN}
% ---
These pairs are all positive, but the match-state classifier predicts each as negative.
These failures can be attributed to poor image quality, occlusion, and viewpoint variations.
Notice that the positive probability is well above zero in all but one case.
%The pair is positive, but the classifier predicts negative because of occlusion and viewpoint variations.
% ---
}
\label{fig:PairFailPN}
\end{figure}
}



\newcommand{\PairFailNP}{
\begin{figure}[h]
\centering
\begin{subfigure}[h]{.7\textwidth}\centering\includegraphics[width=\textwidth]{figures4/PZ_Master1/cases_match_state/fail_nomatch_match_2157_2240.jpg}\end{subfigure}
\begin{subfigure}[h]{.7\textwidth}\centering\includegraphics[width=\textwidth]{figures4/PZ_Master1/cases_match_state/fail_nomatch_match_3550_5250.jpg}\end{subfigure}
\begin{subfigure}[h]{.7\textwidth}\centering\includegraphics[width=\textwidth]{figures4/GZ_Master1/cases_match_state/fail_nomatch_match_1260_2902.jpg}\end{subfigure}
\begin{subfigure}[h]{.7\textwidth}\centering\includegraphics[width=\textwidth]{figures4/GZ_Master1/cases_match_state/fail_nomatch_match_1418_1419.jpg}\end{subfigure}
\caption[\caplbl{PairFailNP}Negative pairwise failure case]{\caplbl{PairFailNP}
% ---
These pairs are negative, but the classifier predicts positive.
Notice that the negative probability in each case is not close to zero.
While the classifier can recognize that the matches may be weak, it is not able to explicitly recognize that the
  same region on two animals contains different distinctive patterns.
Photobomb and scenery matches also contribute to negative failure cases.
% ---
}
\label{fig:PairFailNP}
\end{figure}
}


\newcommand{\PairFailIN}{
\begin{figure}[h]
\centering
\begin{subfigure}[h]{.6\textwidth}\centering\includegraphics[width=\textwidth]{figures4/PZ_Master1/cases_match_state/fail_notcomp_nomatch_1806_16228.jpg}\end{subfigure}
\begin{subfigure}[h]{.6\textwidth}\centering\includegraphics[width=\textwidth]{figures4/PZ_Master1/cases_match_state/fail_notcomp_nomatch_1195_16215.jpg}\end{subfigure}
\begin{subfigure}[h]{.6\textwidth}\centering\includegraphics[width=\textwidth]{figures4/PZ_Master1/cases_match_state/fail_notcomp_nomatch_2847_16301.jpg}\end{subfigure}
\begin{subfigure}[h]{.6\textwidth}\centering\includegraphics[width=\textwidth]{figures4/PZ_Master1/cases_match_state/fail_notcomp_match_5245_5676.jpg}\end{subfigure}
\caption[\caplbl{PairFailIN}Incomparable pairwise failure case]{\caplbl{PairFailIN}
% ---
These pairs are incomparable, but the classifier predicted either positive or negative.
In part this is due to a small amount of available incomparable training data.
In the top two examples the confidence in the incorrect negative prediction is low.
In the bottom two examples, scenery matches and photobombing animals hinder the classifier's ability to predict
  incomparable.
% ---
}
\label{fig:PairFailIN}
\end{figure}
}

\newcommand{\MatchLabelErrors}{
\begin{figure}[h]
\centering
\begin{subfigure}[h]{.7\textwidth}\centering\includegraphics[width=\textwidth]{figures4/PZ_Master1/cases_match_state/fail_notcomp_match_646_1725.jpg}\end{subfigure}
\begin{subfigure}[h]{.7\textwidth}\centering\includegraphics[width=\textwidth]{figures4/PZ_Master1/cases_match_state/fail_match_nomatch_4771_4846.jpg}\end{subfigure}
\begin{subfigure}[h]{.7\textwidth}\centering\includegraphics[width=\textwidth]{figures4/GZ_Master1/cases_match_state/fail_nomatch_match_1349_3087.jpg}\end{subfigure}
\begin{subfigure}[h]{.7\textwidth}\centering\includegraphics[width=\textwidth]{figures4/GZ_Master1/cases_match_state/fail_nomatch_match_1535_2549.jpg}\end{subfigure}
\caption[\caplbl{MatchLabelErrors}Errors in the match-state groundtruth]{\caplbl{MatchLabelErrors}
% ---
Groundtruth errors in the database are the reason for several match-state failure cases.
In these examples the classifier picks the correct answer even though the groundtruth is incorrect.
Note that the probability assigned to the true state of each pair is close to $1.0$.
% ---
}
\label{fig:MatchLabelErrors}
\end{figure}
}


\newcommand{\PhotobombROC}{
\begin{figure}[h]
\centering
\begin{subfigure}[h]{0.47\textwidth}\centering\includegraphics[width=\textwidth]{figures4/PZ_Master1/roc_photobomb_state.png}\caption{Plains zebra}\end{subfigure}
\begin{subfigure}[h]{0.47\textwidth}\centering\includegraphics[width=\textwidth]{figures4/GZ_Master1/roc_photobomb_state.png}\caption{Grévy's zebra}\end{subfigure}
\caption[\caplbl{PhotobombROC}Photobomb ROC experiment]{\caplbl{PhotobombROC}
% ---
The photobomb-state ROC curve demonstrate that both photobomb-state classifiers
are indeed learning from the data.
% ---
}
\label{fig:PhotobombROC}
\end{figure}
}


\newcommand{\PBThreshMCC}{
\begin{figure}[h]
\centering
\begin{subfigure}[h]{0.47\textwidth}\centering\includegraphics[width=\textwidth]{figures4/PZ_Master1/mcc_thresh_photobomb_state.png}\caption{Plains zebra}\end{subfigure}
\begin{subfigure}[h]{0.47\textwidth}\centering\includegraphics[width=\textwidth]{figures4/GZ_Master1/mcc_thresh_photobomb_state.png}\caption{Grévy's zebra}\end{subfigure}
\caption[\caplbl{PBThreshMCC}Photobomb ROC experiment]{\caplbl{PhotobombROC}
% ---
Because there are not many labeled photobomb pairs, the probabilities returned by the photobomb-state classifier
  are low.
However, good classification can be achieved by choosing an operating point that maximizes the MCC.
In each plot the legend indicates the threshold corresponding to the maximum MCC.
% ---
}
\label{fig:PBThreshMCC}
\end{figure}
}


\newcommand{\PBFailures}{
\begin{figure}[h]
\centering
\begin{subfigure}[h]{.63\textwidth}\centering\includegraphics[width=\textwidth]{figures4/PZ_Master1/cases_photobomb_state/fail_notpb_pb_3844_4160.jpg}\end{subfigure}
\begin{subfigure}[h]{.63\textwidth}\centering\includegraphics[width=\textwidth]{figures4/PZ_Master1/cases_photobomb_state/fail_pb_notpb_529_1785.jpg}\end{subfigure}
\begin{subfigure}[h]{.63\textwidth}\centering\includegraphics[width=\textwidth]{figures4/GZ_Master1/cases_photobomb_state/fail_pb_notpb_1241_1242.jpg}\end{subfigure}
\begin{subfigure}[h]{.63\textwidth}\centering\includegraphics[width=\textwidth]{figures4/GZ_Master1/cases_photobomb_state/fail_pb_notpb_1413_1414.jpg}\end{subfigure}
\caption[\caplbl{PBFailures}Photobomb failure cases]{\caplbl{PBFailures}
% ---
In the top example the classifier incorrectly predicts photobomb due to the alignment of the annotations.
In the next case down, the classifier incorrectly predicts photobomb, but no matches were made between the
  photobombing animals.
The last two cases the classifier incorrectly predicts not photobomb, but confidence of the prediction is low.
% ---
}
\label{fig:PBFailures}
\end{figure}
}



\newcommand{\PBLabelErrors}{
\begin{figure}[h]
\centering
\begin{subfigure}[h]{.75\textwidth}\centering\includegraphics[width=\textwidth]{figures4/PZ_Master1/cases_photobomb_state/fail_notpb_pb_1063_1072.jpg}\end{subfigure}
\begin{subfigure}[h]{.75\textwidth}\centering\includegraphics[width=\textwidth]{figures4/PZ_Master1/cases_photobomb_state/fail_notpb_pb_3928_4880.jpg}\end{subfigure}
\begin{subfigure}[h]{.75\textwidth}\centering\includegraphics[width=\textwidth]{figures4/GZ_Master1/cases_photobomb_state/fail_notpb_pb_1377_1378.jpg}\end{subfigure}
\begin{subfigure}[h]{.75\textwidth}\centering\includegraphics[width=\textwidth]{figures4/GZ_Master1/cases_photobomb_state/fail_notpb_pb_1184_1185.jpg}\end{subfigure}
\caption[\caplbl{PBLabelErrors}Errors in the photobomb-state groundtruth]{\caplbl{PBLabelErrors}
% ---
Groundtruth errors in the database are the reason for several photobomb-state failure cases.
It is encouraging that the photobomb-state classifier is able to detect errors in the groundtruth even given only
  a few examples of photobomb pairs.
% ---
}
\label{fig:PBLabelErrors}
\end{figure}
}



\begin{comment}
python -m ibeis.scripts.specialdraw draw_graph_id \
    --dpath ~/latex/crall-thesis-2017/ --save "figures_graph/decisiongraph.jpg" \
    --figsize=12,8 --clipwhite --dpi=300 --diskshow
\end{comment}
\newcommand{\decisiongraph}{
\begin{figure}[t]
\centering
\includegraphics[width=\textwidth]{figures_graph/decisiongraph.jpg}
\caption[A synthetic decision graph]{\caplbl{decisiongraph}
% ---
An example of a consistent synthetic decision graph with positive,
  negative, and incomparable edges.
The color of each node represents the positive connected compoment (PCC)
  it belongs to.
% ---
}
\label{fig:decisiongraph}
\end{figure}
}





\begin{comment}
python -m ibeis.scripts.specialdraw draw_inconsistent_pcc --show
python -m ibeis.scripts.specialdraw draw_inconsistent_pcc \
    --dpath ~/latex/crall-thesis-2017/ --save "figures_graph/inconpcc.jpg" \
    --figsize=15,10 --clipwhite --dpi=300 --diskshow --saveparts
\end{comment}
%\MultiImageCommandII{inconpcc}{.4}{inconpcc}{
%}{figures_graph/inconpccA.jpg}{figures_graph/inconpccB.jpg}
\newcommand{\inconpcc}{
\begin{figure}[ht!]
\centering
\begin{subfigure}[h]{0.4\textwidth}\centering\includegraphics[width=\textwidth]{figures_graph/inconpccA.jpg}\caption{}\label{sub:inconpccA}\end{subfigure}
~~% --
\begin{subfigure}[h]{0.4\textwidth}\centering\includegraphics[width=\textwidth]{figures_graph/inconpccB.jpg}\caption{}\label{sub:inconpccB}\end{subfigure}
\caption[An inconsistent PCC]{\caplbl{inconpcc}
% ---
Any PCC containing at least one negative edge is inconsistent.
Subfigure \Cref{sub:inconpccA} shows an inconsistent PCC, and
\Cref{sub:inconpccB} shows the same PCC where the edges hypothesized to be
errors are highlighted.
% ---
}
\label{fig:inconpcc}
\end{figure}
}



  


\begin{comment}
python -m ibeis.scripts.specialdraw redun_demo2 --show
python -m ibeis.scripts.specialdraw redun_demo2 \
    --dpath ~/latex/crall-thesis-2017/ --save "figures_graph/kredun.jpg" \
    --figsize=10,5 --clipwhite --dpi=300 --saveparts
\end{comment}
\newcommand{\kredun}{
\begin{figure}[h]
\centering
\begin{subfigure}[h]{0.31\textwidth}\centering\includegraphics[width=\textwidth]{figures_graph/kredunA.jpg}\caption{}\label{sub:kredunA}\end{subfigure}
~~% --
\begin{subfigure}[h]{0.31\textwidth}\centering\includegraphics[width=\textwidth]{figures_graph/kredunB.jpg}\caption{}\label{sub:kredunB}\end{subfigure}
~~%--
\begin{subfigure}[h]{0.31\textwidth}\centering\includegraphics[width=\textwidth]{figures_graph/kredunC.jpg}\caption{}\label{sub:kredunC}\end{subfigure}
~~%--
\begin{subfigure}[h]{0.31\textwidth}\centering\includegraphics[width=\textwidth]{figures_graph/kredunD.jpg}\caption{}\label{sub:kredunD}\end{subfigure}
~~%--
\begin{subfigure}[h]{0.31\textwidth}\centering\includegraphics[width=\textwidth]{figures_graph/kredunE.jpg}\caption{}\label{sub:kredunE}\end{subfigure}
~~%--
\begin{subfigure}[h]{0.31\textwidth}\centering\includegraphics[width=\textwidth]{figures_graph/kredunF.jpg}\caption{}\label{sub:kredunF}\end{subfigure}
\caption[Examples of $k$-redundant PCCs]{\caplbl{kredun}
%--%
Examples of positive (top) and negative (bottom) redundancy.
The positive edges are colored blue and the negative edges are colored red.
Choosing the level of redundancy is a tradeoff between the number of required
  reviews and the confidence that the reviews are correct.
%In our current implementation we use $2$-redundancy.
%
} \label{fig:kredun}
\end{figure}
}



\begin{comment}
python -m ibeis.viz.viz_chip HARDCODE_SHOW_PB_PAIR --db PZ_Master1 --has_any=photobomb --index=1 --match \
    --dpath ~/latex/crall-thesis-2017/ --save "figures_graph/PhotobombExampleC.jpg" \
    --figsize=9,4 --clipwhite --dpi=180 --save

python -m ibeis.viz.viz_chip HARDCODE_SHOW_PB_PAIR --db PZ_Master1 --has_any=photobomb --index=1 \
    --dpath ~/latex/crall-thesis-2017/ --save "figures_graph/PhotobombExample.jpg" \
    --figsize=9,4 --clipwhite --dpi=300 --saveparts

python -m ibeis.core_annots --test-compute_one_vs_one --show
    
\end{comment}

\newcommand{\PhotobombExample}{
\begin{figure}[h]
\centering
\begin{subfigure}[h]{0.4\textwidth} \centering \includegraphics[height=100pt]{figures_graph/PhotobombExampleA.jpg}\caption{}\label{sub:PhotobombExampleA} \end{subfigure}
\begin{subfigure}[h]{0.4\textwidth} \centering \includegraphics[height=100pt]{figures_graph/PhotobombExampleB.jpg}\caption{}\label{sub:PhotobombExampleB} \end{subfigure}
%\begin{subfigure}[h]{0.5\textwidth}
%\centering
%\includegraphics[height=60pt]{figures/PhotobombExampleC.jpg}\caption{}\label{sub:PhotobombExampleC}
%\end{subfigure}
\caption[\caplbl{PhotobombExample}Photobomb example]{\caplbl{PhotobombExample}
% ---
A secondary animal in an annotation can cause a ``photobomb''.  Notice the
primary animal in~\cref{sub:PhotobombExampleA} appears in the background
of~\cref{sub:PhotobombExampleB}. 
%The matching regions are displayed in~\cref{sub:PhotobombExampleC}.
% ---
}
\label{fig:PhotobombExample}
\end{figure}
}



\begin{comment}
python -m ibeis.algo.graph.mixin_loops prob_any_remain --num_pccs=40 --size=2 --patience=20 --window=20 --dpi=300 --figsize=7.4375,3.0 '--dpath=~/latex/crall-thesis-2017' --save=figures5/poisson.png --diskshow
\end{comment}
\newcommand{\poisson}{
\begin{figure}[h]
    %
    \centering
    \includegraphics[width=\textwidth]{figures5/poisson.png}
    \caption[The convergence criteria on a synthetic dataset]{\caplbl{poisson}
    %--
    The convergence criteria applied to a synthetic dataset with $40$ names and $2$ annotations per name.
    The red line indicates the fraction of meaningful reviews that remain undiscovered.
    The blue line is the probability that at least one of the next $20$ reviews will be meaningful.
    Notice that the blue line dips when the red line flattens.
    The process terminates once this probability drops below a thresold,
    denoted by the green dotted line.
    %--
} \label{fig:poisson}
\end{figure}
}


\begin{comment}
python -m ibeis Chap5.measure_simulation --db GZ_Master1 --show
python -m ibeis Chap5.measure_simulation --db PZ_Master1 --show

python -m ibeis Chap5.draw_simulation --db PZ_Master1 --diskshow
python -m ibeis Chap5.draw_simulation --db GZ_Master1 --diskshow
\end{comment}
\newcommand{\Simulation}{
\begin{figure}[t]
\centering
\begin{subfigure}[h]{\textwidth}\centering\includegraphics[width=\textwidth]{figuresGraph/PZ_Master1/simulation.png}\caption{Plains zebra}\end{subfigure}
~
\begin{subfigure}[h]{\textwidth}\centering\includegraphics[width=\textwidth]{figuresGraph/GZ_Master1/simulation.png}\caption{Grévy's zebra}\end{subfigure}
\caption[\caplbl{Simulation}Simulation experiment]{\caplbl{Simulation}
% ---
The user simulation experiment compares the three identification algorithms defined in this \thesis{}.
On the left indicates the identifiation accuracy using the number of remaining merges and the right counts the
  number of errors made (lower is better in both cases).
The best results are clearly achieved by \pvar{graph}.
% ---
}
\label{fig:Simulation}
\end{figure}
}


\begin{comment}
python -m ibeis Chap5.draw_refresh --db GZ_Master1 --diskshow
python -m ibeis Chap5.draw_refresh --db PZ_Master1 --diskshow
\end{comment}
\newcommand{\Refresh}{
\begin{figure}[ht]
\centering
\begin{subfigure}[h]{\textwidth}\centering\includegraphics[width=\textwidth]{figuresGraph/PZ_Master1/refresh.png}\caption{Plains zebra}\end{subfigure}
~
\begin{subfigure}[h]{\textwidth}\centering\includegraphics[width=\textwidth]{figuresGraph/GZ_Master1/refresh.png}\caption{Grévy's zebra}\end{subfigure}
\caption[\caplbl{Refresh}Measured refresh and termination probabilities]{\caplbl{Refresh}
% ---
The measured refresh and termination criteria.
The probability that the next reviews will be meaningful ($\Pr{T\teq1}$) is high while new merges are discovered.
Once the probability falls under the threshold, positive redundancy is enforced (the flat areas) on existing
  PCCs, and then candidated edges are recomputed.
After an iteration with no meaningful reviews the process terminates.
Identification converges in $3$ iterations for plains zebras and $3$ for Grevy's.
% ---
}
\label{fig:Refresh}
\end{figure}
}




    \begin{comment}
    python -m ibeis Chap5.print_measures --db GZ_Master1 --diskshow
    python -m ibeis Chap5.print_measures --db PZ_Master1 --diskshow
    \end{comment}


\newcommand{\SimDetails}{
\begin{table}[h]
    \centering
\caption[\caplbl{SimDetails}Simulation error details]{\caplbl{SimDetails} 
% ---
Detailed analysis of the errors for the simulations.
We compare statistics the predicted PCCs with statistics of the real ground-truth PCCs.
In each category ``pred \#'' is the number of predicted PCCs and ``pred size'' is the average number of
  annotations in those PCCs (measured as mean and standard deviation).
The ``real \# PCCs'' and ``real size'' columns are similarly defined.
Lastly, the ``small size'' and ``large size'' columns are the average size of the smallest and largest PCCs in
  each error group.
%The sum of the importance for all $205$ feature dimensions is $1$.
% ---
}
\label{tbl:SimDetails}
    \begin{subfigure}[h]{\textwidth}\centering\input{figuresGraph/PZ_Master1/error_size.tex}\caption{Plains zebra}\end{subfigure}
    \begin{subfigure}[h]{\textwidth}\centering\input{figuresGraph/GZ_Master1/error_size.tex}\caption{Grévy's zebra}\end{subfigure}
\end{table}
}


\begin{comment}
    ./texfix.py --outline --fpaths chapter1-intro.tex
    ./texfix.py --fpaths chapter1-intro.tex --outline --asmarkdown --numlines=999  -w
    ./texfix.py --grep "\\\\[A-Za-z]*[^{a-zA-Z]"
    ./texfix.py --reformat --fpaths figdef1.tex
\end{comment}


\chapter{Introduction}\label{chap:intro}

\section{Image based identification applied to population ecology}

    Population ecology relies on estimating the number of individual animals that inhabit an
    area~\cite{krebs_ecological_1999}. Estimating a population size is done in two phases: data collection and analysis.
    Data are collected as a set of \glossterm{sighting} and \glossterm{resighting} observations. A sighting is the first
    observation of an individual, and a resighting is a subsequent observation of a previously sighted individual. The
    observed data are then analyzed using software such as program MARK~\cite{white_program_1999,
    schwarz_jollyseber_2006} or Wildbook that applies statistical models such as the Lincoln-Petersen
    index~\cite{seber_estimation_1982}, Jolly-Seber model,~\cite{jolly_explicit_1965, seber_note_1965}, or other related
    models~\cite{cormack_estimates_1964, chao_estimating_1987,kenneth._h._pollock_statistical_1990}. For an ecologist
    recording that an individual has been observed is simple, but determining if that observation is a sighting or a
    resighting can be challenging. This requires the ecologist to identify the individual against all other observations
    in the data set.

    Current methods to estimate a population size are limited by the data collection
    phase~\cite{sundaresan_network_2007, rubenstein_ecology_2010}. The statistical population models require an
    observation sample size that grows with the size of the population being studied~\cite{seber_estimation_1982}. As
    the number of observations increases so does the difficulty of determining identity. Thus the scope of a population
    study is limited by the number of raw observations that can be made, and by the rate of determining the individual
    identity within a set of observations. Overcoming these limitations is of particular importance to wildlife
    preservation because population statistics are necessary to guide conservation
    decisions~\cite{rubenstein_behavioral_1998}.

    Consider images as a source of sight-resight observations. There are numerous advantages. Many observations can be
    made rapidly and simultaneously, due to the simplicity and availability of cameras. Recording an observation is as
    cheap and simple as taking a picture. Camera traps can be employed for autonomous data collection. In a wildlife
    conservancy or national park, observations can be crowd-sourced by gathering images from safari tourists and citizen
    scientists. Images can be accumulated and stored in a large dynamic dataset of observations that grows by thousands
    of images each day. However, the challenge of identifying the individuals in the images remains. Manual methods are
    infeasible due to the rapid rate at which images can be collected. Therefore, we must turn towards computer vision
    based methods.

    This \thesis{} develops the foundation of the image analysis component of \IBEIS{} --- the Image Based Ecological
    Information System. The purpose of this system is to gain ecological insight from images using computer vision. We
    focus on estimating the size of a population of animals as just on example of ecological insight that might be
    gained from images. Thus we come to the core problem addressed in this \thesis{}: image based identification of
    individual animals.

\section{Challenges of animal identification}\label{sec:challenges}

    In animal identification we are given a database of images. This database may initially be empty. Each image is
    cropped to a bounding box around an animal of interest and labeled with that animal's identity. For a new query
    image, the goal is to determine whether or not any other images of the individual are in the database. If the query
    is matched, it is added to the database as a resighting of that individual. If the query is not matched, then it is
    added as a new individual.

    In this work we focus on identifying individuals of species with distinguishing textures, such as zebras,
    giraffes, humpback whales, lionfish, nautiluses, hyenas, whale sharks, wildebeest, wild dogs, jaguars,
    cheetahs, leopards, frogs, toads, snails, and seals. The primary species that we will consider in this
    \thesis{} are plains and Grévy's zebras, but we will maintain a secondary focus on Masai giraffes and humpback
    whales. The difficulty of animal identification depends on the distinctiveness of the visual patterns that
    distinguish an individual from others of its species. In addition, the images we identify are collected ``in
    the wild'' and therefore contain occlusion, distracting features, variations in viewpoint and quality.

    This section will present several examples to illustrate the challenges faced in animal identification. The
    discussion will begin with the challenges posed by the three primary species. Then the problems common to all
    species will be described. These will be illustrated using plains zebras because they are the most challenging
    species considered in this \thesis{}.

    \subsection{Distinguishing textures of each species}
        The plains zebra --- shown in~\cref{fig:PlainsFigure} --- is challenging to visually identify because
        individuals have relatively few distinguishing texture features. For most plains zebras, the majority of distinctive
        information lies in a small area on the front shoulder. \Cref{fig:HardCaseFigure} illustrates that the patterns
        that distinguish two individuals can be subtle, even when the features are clearly visible. The matching
        difficulty greatly increases when features are partially occluded, the viewpoint changes, or the image quality
        is poor.

        In contrast, Masai giraffes and Grévy's zebras, shown in~\cref{fig:GirMasaiFigure}
        and~\cref{fig:GrevysFigure} respectively, have an abundance of distinctive features. Distinctive textures
        that are unique to each individual are spread across the entire body of a Masai giraffe. For a Grévy's
        zebra there is a high density of distinguishing information above both front and back legs, as well as a
        moderate density of distinctive textures along the side of the body. The high density of distinctive
        textures in Masai giraffes and Grévy's zebras increases the likelihood that the same distinctive features
        can be seen from different viewpoints, but these factors still increase the difficulty of the problem.

        There are some species, like Humpback whales, where some individuals may contain distinguishing textures
          while other may lack them entirely.
        This means that only a subset of humpback whales will be able to be identified with the texture based
          techniques that we will consider in this thesis.
        However, other cues --- like the shape of the notches along the trailing edge of the fluke --- can be
          used to distinguish between different individuals.
        The work of Hendrick Weideman~\cite{hendrik} addresses identifying humpback whales using shape features.
        The example in~\cref{fig:HumpbackFig} illustrates individual humpback whales with and without distinctive
          textures.

        \PlainsFigure{}

        \HardCaseFigure{}

        \GirMasaiFigure{}

        \GrevysFigure{}

        \HumpbackFig{}

    \subsection{Viewpoint and pose}
        One of the most difficult challenges faced in the animal identification problem is viewpoint. Animals are seen
        in a variety of poses and viewpoints, which can cause distinctive features to appear distorted. The patterns on
        the left and right sides of animals are almost always asymmetric. Therefore, matches can only be established
        using overlapping viewpoints and only if the viewpoints are distinctive. Some viewpoints, such as the backs of
        plains zebras, lack distinguishing information as shown in~\cref{fig:BacksFigure}. The effect of pose and
        viewpoint variation can be seen in~\cref{fig:ThreeSixtyFigure} and~\cref{fig:PoseFigure}.

        \BacksFigure{}

        \ThreeSixtyFigure{}

        \PoseFigure{}

    \subsection{Occluders and distractors}
        Because images of animals are often taken ``in the wild'', other objects in the image can act as
        \glossterm{occluders} or \glossterm{distractors}. Objects such as grass, bushes, trees or other animals, can act
        as occluders by partially obscuring the features that distinguish one individual from another. The appearance of
        the other animals nearby can be distracting because features from these animals will match different animals in
        the database. These \glossterm{distractors} may also be from non-animal features when multiple pictures are
        taken against the same background as animals move through the same field of view. Several examples of occlusions
        and distractors are illustrated in~\cref{fig:OccludeFigure}

        \OccludeFigure{}

    \subsection{Image quality}
        Image quality is influenced by lighting, shadows, the camera used, image resolution, and the size of the animal
        in the image. Outdoor images will naturally have large variations in illumination. Different cameras can produce
        visual differences between images of an object. Images taken out of focus, from far away, or with a non-steady
        camera can cause animals to appear blurred. The effects of outdoor shadow and illumination are illustrated
        in~\cref{fig:IlluminationFigure}. \Cref{fig:QualityFigure} illustrates five categories of image quality that
        will be described later in~\cref{sub:viewqual}.

        \IlluminationFigure{}

        \QualityFigure{}

    \subsection{Aging and injuries}
        The appearance of an individual changes over time due to aging and other factors including injuries. An example
        of the difference between a juvenile and adult zebra is shown in~\cref{fig:AgeFigure}. An example of how
        injuries can both remove distinctive features and add new ones is shown in~\cref{fig:GashFigure}.

        \AgeFigure{}

        \GashFigure{}

\section{The Great Zebra Count}\label{sec:introgzc}

    To further illustrate the problems addressed in this \thesis{}, we consider the Great Zebra Count (\GZC{}), held at
    Nairobi National Park on March 1\st{} and 2\nd{}, 2015. This event was designed with two purposes in mind: (1) to
    involve citizens in the scientific data collection effort, thereby increasing their interest in conservation, and
    (2) to determine the number of plains zebras and Masai giraffes in the park.

    \subsection{Data collection}
        Volunteer participants --- each with his or her own camera --- arrived by car at the park. Some cars had more
        than one photographer. Each car was assigned a route to drive through the park. We attached a GPS dongle to each
        car to record time and location throughout the drive. Correlating this with the time stamp on each image (after
        adding a correction offset for each camera) allowed us to determine the geolocation of each image. Each
        photographer was given instructions guiding them toward taking quality images of the left sides of the animals
        they saw. When the cars returned --- some after just an hour or two, others after the whole day --- the images
        were copied from the cameras, a small sample of each photographer's images was immediately processed to
        illustrate what we would do with the data, and the entire set of images was stored for further processing. The
        result of this crowd-sourced collection event was a 48GB dataset consisting of $9406$ images.

    \subsection{Data processing}\label{subsec:introdataprocess}

        After the event, the entire collection of images was processed using a preliminary version of the system in
        order to generate the final count. The preliminary system followed the workflow of: %
        \begin{enumin}
            %\item ingest images  %
            \item \occurrence{} grouping  %
            \item animal detection %
            \item viewpoint and quality labeling  %
            \item \intraoccurrence{} matching %
            \item \vsexemplar{} identification %
            \item consistency checks  %
            \item population estimation.  %
        \end{enumin}
        %\Cref{chap:application} discusses this workflow
        %in greater detail. 
        Here, we provide a brief overview of each step involved in the processing of the \GZC{} image data, and then we
        will describe the challenges that arose.

        \subsubsection{Occurrence grouping}
            The images were first divided into \glossterm{\occurrences{}} --- a standard term defined by the Darwin
            Core~\cite{wieczorek_darwin_2012} to denote a collection of evidence (\eg{} images) that an organism exists
            within defined location and time-frame. In the scope of this application, an \occurrence{} is a cluster of
            images taken within a small window of time and space. Images are grouped into \occurrences{} using the GPS
            and time data. Details are provided in~\cref{sec:occurgroup}.

            There are several benefits to first grouping images into an \occurrence{}. One benefit is that an
            \occurrence{} can be used as a semantic processing unit to distribute manageable chunks of work to users of
            the system. Another is that \occurrences{} can be used to improve the results of identification. Typically
            there will be only a small number of individuals within an \occurrence{}, and it is not uncommon for each
            individual to photographed multiple times and from multiple viewpoints. This redundancy in images will be
            exploited in \Cref{chap:graphid}.

        \subsubsection{Animal Detection}
            Before matching begins each image is cropped to focus on a particular animal and remove background
            distractors. A detection algorithm localizes animals within the images. Each verified detection generates an
            \glossterm{\annot{}} --- a bounding box around a single animal in an image. An example illustrating
            detection of plains zebras is shown in~\cref{fig:DetectFigure}. In the \GZC{} each detection was manually
            verified before becoming an \annot{}, but recent work introduces an automatic verification mechanism and
            reduces the need for complete manual review. The details of the detection algorithm are beyond the scope of
            this \thesis{}, and will be described in the work of Jason Parham~\cite{parham_photographic_2015}.

            \DetectFigure{}

        \subsubsection{Viewpoint and quality labeling}\label{sub:viewqual}
            When determining the number of animals in a population it is important to account for factors that can lead
            to over-counting. If two \annots{} of the same individual are not matched, then that individual will be
            counted twice. This could happen due to factors such as viewpoint and quality. For example, one \annot{}
            showing the only left side of an animal and another \annot{} showing only the right side the same animal
            cannot be matched because they are \glossterm{incomparable}. The two \annots{} are comparable when they
            share regions with distinguishing patterns that can be put in correspondence. Viewpoint is the primary
            reason that two \annots{} are not comparable. However, other factors like image quality and heavy occlusion
            can corrupt distinguishing patterns rendering the \annot{} unidentifiable --- not comparable with any other
            \annot{}. We must define what it means for two \annots{} to be comparable before we can estimate a
            population size.

            Determining if an individual can be identified is analogous to the
            notion of a marked-individual~\cite{seber_estimation_1982}. For an
            \annot{} to be identifiable the patterns that can distinguish it
            from the rest of the population must be clear and visible, otherwise
            the \annot{} may not be able find or be compared to potential
            matches. This means an \annot{} is only identifiable if
            \begin{enumin}
                \item the image quality is high enough, and %
                \item it has a viewpoint that is comparable to all potential
                matches. %
            \end{enumin}
            
            To address this challenge we label each \annot{} with $5$ discrete quality labels and $8$ discrete viewpoint
            labels. The quality labels we define are: \qualJunk{}, \qualPoor{}, \qualOk{}, \qualGood{}, and
            \qualExcellent{}. The \qualJunk{} label is given to \annots{} that almost certainly will not be able to be
            identified, and \qualPoor{} labels are given to \annots{} that will likely be unidentifiable for a computer
            vision algorithm. The $\qualGood{}$ and \qualExcellent{} labels are given to clear well illuminated
            \annots{} with little to no occlusion with \qualExcellent{} being reserved for the best of the best. All
            other \annots{} are labeled as $\qualOk$. The viewpoint labels we define are: \vpFront{}, \vpFrontLeft{},
            \vpLeft{}, \vpBackLeft{}, \vpBack{}, \vpBackRight{}, \vpBack{}, and \vpFrontRight{}. Note, that additional
            viewpoint labels like $\vpUp{}$ and $\vpDown{}$ may be necessary for animals such as lionfish or turtles.
            However, the $8$ labels we use are sufficient for animals like zebras and giraffes because they are most
            commonly seen in upright positions.

            In an effort to ensure that all \annots{} used in the \GZC{} were comparable, we did not include any
            \annot{} that had junk or poor qualities. We also did not include \annots{} not labeled with a left or
            frontleft viewpoint to account for limitations in the initial ranking algorithm. All labelings of viewpoint
            and quality were generated manually. Since then we have trained viewpoint and quality classifiers using this
            manual data. Automatic detection of quality and viewpoint is discussed in Jason Parham's
            work~\cite{parham_photographic_2015}.

        \subsubsection{Matching within each \occurrence{}} %
            Animals often have multiple redundant views within an \occurrence{}, which can be the same, better, or
            complementary to other views. The images in~\cref{fig:OccurrenceComplementFigure} illustrate redundant and
            complementary views of an individual in an \occurrence{}. Merging all of an individual's views is a
            challenge, but also potentially an advantage as we can exploit redundancy to better handle missing features,
            subtle viewpoint changes, and occlusions.

            We exploit this redundancy to gain the benefit of complimentary views by matching all \annots{} within an
            \occurrence{} in a process called \glossterm{\intraoccurrence{} matching}. In the \GZC{}, each \annot{} was
            queried against all other \annots{} in its \occurrence{}, returning a ranked list of candidate matches. The
            person running the software made the final decisions about which \annots{} match. Details about the ranking
            algorithm are given in~\cref{chap:ranking}.

            The result of \intraoccurrence{} matching is a set of \glossterm{\encounters{}}. \Aan{\encounter{}} is a
            group of \annots{} that were matched within an \occurrence{}. Each \encounter{} is either (1) the first
            sighting an individual or a (2) resighting. The task now becomes to determine which of these is the case by
            identifying each \encounter{} against a \masterdatabase{}.

            \OccurrenceComplementFigure{}
 
        \subsubsection{Matching against the \masterdatabase{}} %
            To determine if \aan{\encounter{}} is a new sighting or a resighting of an individual, it is matched against
            the \masterdatabase{} in a process called \glossterm{\vsexemplar{} matching}. Before matching begins the
            \masterdatabase{} is prepared for search. For each \name{} in the \masterdatabase{} a subset of
            \glossterm{\exemplar{}} \annots{} are chosen to represent the appearance of that individual. The
            \exemplars{} are indexed using a search data structure.

            After the \masterdatabase{} has been prepared, the ranking algorithm is able to issue a subset of the
            \encounter{}'s \annots{} as a query. The result is a ranked list of \exemplars{} that are visually similar
            to the \encounter{}. The top \exemplars{} in the ranked list are used as candidate matches. The candidate
            matches are reviewed, and the \encounter{} is either merged into an existing \mastername{} or added to the
            \masterdatabase{} as a new \mastername{}.

        \subsubsection{Consistency checks}
            When merging \encounters{} into the \masterdatabase{} it is possible that mistakes were made.  There are two
            error cases that commonly occur.
            %%%
            \begin{enumln}
            \item  A \glossterm{split case} occurs when a set of \annots{} (from two or more different animals) is
            incorrectly labeled with the same \name{}.  The main cause of this error is when distracting features are
            matched causing the \annots{} to appear visually similar.
            %%%
            \item A \glossterm{merge case} occurs when two sets of \annots{} (from the same animal) are incorrectly
            labeled with different \names{}.  This is caused by an algorithm or human error where a query \encounter{}
            was not correctly matched to the database \exemplars{}.
            \end{enumln}
            %%%
            This is usually because the query and database \annots{} have a low degree of \emph{comparability} (\eg{}
            differences in viewpoint or low quality).  Of course, if no visual overlap exists between the two sets ---
            such as one set exclusively from the left side and another exclusively from the right --- nothing can be
            done.  This is why the animal must be seen from a predetermined view in order to be counted.  In the \GZC{}
            this is the left side.

            In the \GZC{} suspect individuals were flagged for split checks using various heuristics such as the number
            of \annots{} in the \name{} or speed of the animal estimated using GPS and time data. To check a flagged
            individual we used the ranking algorithm to search for pairs of \annots{} with low matching scores that
            belong to the flagged \name{}. Low similarity between two \annots{} within a \name{} suggested that an error
            had occurred. These low scoring results were then manually reviewed. When breaking apart split cases, care
            was taken to account for the fact that right and left images should not match. Likewise, care was taken to
            ensure that an intermediate \annot{} linking two disjoint \annots{} has enough information to establish the
            link. Merge checks issue the all \exemplars{} as queries against all other \exemplars{}. High similarity
            between two different \names{} suggested that a match that was missed. These high scoring results were
            manually reviewed. More sophisticated error detection and recovery will be discussed in \Cref{sec:incon}.

        \subsubsection{Population estimation}
            The final step for the \GZC{} workflow was to estimate the number of animals in the park.  Using the
            identification algorithm we defined which \annots{} were sightings and which were resightings. Because we
            were using a preliminary version of the system we were conservative in defining when an animal was sighted
            by only using the left and frontleft \annots{} with quality labels of ok, good, or excellent.  Each
            individual that met this criteria was counted as a sighting.  If a sighted individual had an \annot{} from
            both days, then we counted that individual as resighted.

    \subsection{Processing challenges}
        Our experience with the Great Zebra Count has highlighted a number of challenges that must be addressed if this
        system is to be applied in future events. These challenges include the number of manual reviews required, the
        detection of and recovery from manual errors, and the overall lack of a systematic identification framework.

        Perhaps the greatest challenge faced during the \GZC{} was the
          considerable amount time that was required to manually verify
          identification results.
        It can take several seconds to manually verify if a pair of \annots{}
          is a correct match even if the results are presented in a ranked list.
        This task is illustrated in~\cref{fig:RankFigure}.
        Requiring the manual verification of each result is untenable for a
          system that accepts thousands of new images a day.
        The lack of a systematic approach for identification meant that
          whenever two \annots{} were matched, the name labels of all
          annotations of those names were changed.
        This made it difficult to tease apart errors when they occurred.
        Furthermore, manual errors (likely caused by fatigue from the large
          number of manual reviews) resulted in numerous identification errors
          cases that were not able to be detected and resolved until the end of
          the process.
        Reviews of results were also done in order of matching scores
          regardless of previous decisions, causing the manual reviewer to
          inefficiently review redundant results between the same individual.
        Additionally no stopping criteria for reviews was defined resulting in
          an ad-hoc approach to determining when all matches were found.

        Motivated by these observations we seek to develop a semi-automatic approach to animal identification. This
        approach will should be governed by a system that reduces the number of manual reviews and is able to detect and
        recover from errors, and determine when to stop searching for new matches.

        %Furthermore, as new \exemplars{} are added to the system the search
        %  data structure must be updated before additional queries can be made.
        %Rebuilding this data structure is another source of delays.
        %We consider addressing this problem as two separate challenges.
        %The first challenge is algorithmic, and the second challenge is system
        %  based.
        %We will use these challenges to motivate the development a system that
        %is able to dynamically detect and identify individual animals in large
        %volumes of images.
        %The algorithmic challenge is to develop a confidence based decision
        %  mechanism.

        %We will use these challenges to motivate a verification mechanism that
        %automatically accepts or dismisses candidate matches. 
        %Only a subset of the most difficult identification results should be
        %  manually reviewed, the rest should be handled automatically.
        %This motivates developing a 

        %On the system side, the challenge is to dynamically update the search
        %  data structure.
        %This involves intelligent bookkeeping because the image analysis
        %  system is designed as a stateless API{}.
        %Statelessness is essential if multiple users are to access the same
        %  instance of image analysis and makes the system compatible with web
        %  technologies.
        %A stateless API is allowed to cache results, but it cannot maintain a
        %  single canonical object such as an indexer.
        %Instead the API{} works by accepting and responding to requests.
        %This has the effect of enforcing that objects are immutable, but also
        %  eliminates bugs due to race conditions, gives the program a large
        %  degree of thread safety, and encourages extensible and testable coding
        %  practices.
        %Updating search structures dynamically is a challenging problem in a
        %  stateless framework, but it can be addressed with careful system
        %  design.
              
        \RankFigure{}

\section{Approach}
    The problem addressed in this \thesis{} is to identify individual animals ``in the wild'' and to count the
    individuals in a population. We are given a set images containing \annots{} of the same species. The images are
    collected in an uncontrolled environment and likely contain imaging challenges such as occlusion, distracting
    features, viewpoint variations, pose variations, and quality variations. Furthermore, the images may be collected
    either over many years or over just a few days as in the \GZC{}. Each \annot{} is labeled with time, GPS, quality,
    and viewpoint. We may also be given an initial partial \name{} labeling of the annotations --- \eg{} in the case
    where we identify a new set of annotations against a previously identified set --- but this need not be the case.
    Our task is label each \annot{} with a \glossterm{\name{}} that uniquely identifies the individual. Once this is
    complete it forms the data needed to estimate the size of the population using techniques from sight-resight
    statistics.

    The first step of the identification process is a ranking algorithm. The inputs to the algorithm are a single query
    \annot{} and a set of database \annots{}. Sparse patch-based features are localized in all \annots{}, and a
    descriptor vector is extracted for each feature. The descriptors of the database \annots{} are indexed for fast
    nearest neighbor search. We then find a set of matches in the database for each descriptor in the query \annot{}.
    The matches are scored based on visual similarity, distinctiveness within the database, and likelihood of belonging
    to the foreground. Matches are combined across multiple \exemplar{} \annots{} to produce a matching score for each
    \name{} in the database, resulting in a ranked list of results for each query.

    We then extend ranking algorithm by developing a classifier able to automatically review its results. We construct a
    pairwise feature that captures relationships between two annotations using local feature correspondence and global
    properties such as time and GPS. We learn a classifier to predict if a pair of annotations --- \ie{} a result in the
    ranked list --- is correct or incorrect.

    In the final part of our approach, we place the problem of animal identification in a graph framework able to
    systematically guide the identification process. This is done by placing each annotation in a graph as a vertex and
    placing labeled edges between annotations to represent how they are related. Using the graph framework we will be
    able to detect and recover from errors by taking advantage of multiple images seen of each individual.

    We evaluate the ranking, verification, and graph identification algorithm by performing experiments two main
    databases of plains zebras and Grévy's zebras. Some additional experiments are also performed on databases of Masai
    Giraffes (and maybe others?). First, the ranking experiments test the algorithm's ability to find potential matches
    of an individual animal over large periods of time, different viewpoints, different sized databases, and different
    numbers of \exemplars{}. Then, the verification experiments will test the extent to the correct results from the
    ranking algorithm can be separated from the incorrect results using our learned classifier. Finally, the graph
    identification experiments will demonstrate the algorithm's ability to reduce the number of required manual reviews
    and recover from errors. We determine the configuration of each algorithm that works best for identifying each
    species.
    
    %To do this we
    %develop both a suite of algorithms and a software system. The algorithms
    %will allow us to infer properties about images and \annots{}. The system
    %will allow us to maintain the images, \annots{}, algorithms, and inferred
    %properties in a controlled and reproducible manner.
    
    %We build a workflow on top of the matching algorithm.
    %This workflow accepts new \annots{} in groups defined by \occurrences{}.
    %The matching algorithm groups \annots{} within the \occurrence{}, and
    %  then leverages redundant and multiple viewpoints to perform identification
    %  against the database.
    %As the database grows we handle multiple views of each \exemplar{} by
    %  maintaining a set of \exemplars{} for each \name{}.
    %We develop methods for recovering from any errors in identification when
    %  multiple individuals are grouped into the same \exemplar{} as well as when
    %  multiple \exemplars{} actually represent the same individual.

    %To address the challenges introduced by this workflow we extend the core
    %  matching algorithm using a probabilistic graph based inference algorithm.
    %We will learn the probability of matching given two \annots{} as well as a
    %  confidence in that estimate.
    %We will use this information build a weighted graph of potential matches.
    %To perform inference on this graph we propose to to develop a decision
    %  mechanism that will make probabilistic decisions about \intraoccurrence{}
    %  matching, \vsexemplar{} matching, and consistency checks.

    %To support continuous and dynamic use of the system we develop a caching
    %scheme that supports seamless invalidation of outdated data, computes
    %requested data on the fly, and disallows duplicate data. We use this scheme
    %to dynamically update the underlying data structures as more data is added
    %to the system. This is all accomplished in a stateless framework which
    %allows for the image analysis software to be used concurrently by web-based
    %frameworks.

\section{\Thesis{} organization} %
    This \thesis{} is organized as follows:
    %
    \Cref{chap:relatedwork} describes related work with a focus the details of
      techniques used in the system as well as an overview of those which are
      indirectly related.
    %
    \Cref{chap:ranking} describes a ranking algorithm for identifying
      individual animals, one \annot{} at a time, against a database of
      \exemplars{}.
    This chapter includes an experimental evaluation of the ranking algorithm.
    This is the algorithm that was used in the \GZC{}.
    \Cref{chap:pairclf} addresses the problem of semi-automatic verification
      of results from the ranking algorithm.
    %
    \Cref{chap:graphid} combines the ranking and verification algorithm into a
      semi-automatic framework that detects and corrects errors while reducing
      the number of manual reviews .
    %
    \Cref{chap:conclusion} concludes this \thesis{} and summarizes its contributions.

\begin{comment}
    ./texfix.py --fixcap --dryrun
    ./texfix.py --findcite --unused-important
    ./texfix.py --findcite --close-keys
    fixtex --fpaths chapter2-related-work.tex --outline --asmarkdown --numlines=999 --shortcite -w && ./checklang.py outline_chapter2-related-work.md
\end{comment}

\chapter{Related work}\label{chap:relatedwork} 

    To address the individual animal identification problem we draw upon related research in %
    feature detection~\cite{mikolajczyk_comparison_2005,tuytelaars_local_2007, perdoch_efficient_2009}, %
    feature description~\cite{lowe_distinctive_2004, mikolajczyk_performance_2005,simonyan_learning_2014,
      winder_picking_2009,zagoruyko_learning_2015,han_matchnet_2015}, %
    approximate nearest neighbor search~\cite{silpa_anan_optimised_2008, muja_fast_2009}, %
    instance recognition~\cite{sivic_efficient_2009,nister_scalable_2006,
      philbin_object_2007,jegou_hamming_2008,bo_efficient_2009, jegou_aggregating_2012, tolias_aggregate_2013}, %
    face verification~\cite{chopra_learning_2005,huang_labeled_2007,berg_tom_vs_pete_2012,
      chen_blessing_2013,taigman_deepface_2014} %
    fine-grained recognition~\cite{parkhi_cats_2012,berg_poof_2013, gavves_local_2014}, %
    category recognition~\cite{lazebnik_beyond_2006,zhang_local_2006,
      mccann_local_2012,boiman_defense_2008,sanchez_compressed_2013}, %
    and convolutional neural networks~\cite{krizhevsky_imagenet_2012, razavian_cnn_2014,
      zagoruyko_learning_2015,han_matchnet_2015,arandjelovic_netvlad_2016}.

    At a high level main questions addressed by the aforementioned research can be summarized as: How should image
    features be detected? How should detected image features be represented? How much invariance should features
    have? Should image features be quantized? How should image feature be matched? How should feature matches be
    aggregated? How should feature matches be scored? How should all of this be done accurately? How should all of
    this be done efficiently? Answers to these questions address many of the challenges to animal identification
    previously introduced in~\cref{sec:challenges}.

    This chapter summarizes literature relevant to addressing these questions in the context of animal
    identification. The outline of this chapter is as follows: \Cref{sec:featuredetect} will discuss feature
    detection. \Cref{sec:featuredescribe} will discuss feature description. \Cref{sec:ann} will discuss approximate
    nearest neighbor algorithms. \Cref{sec:ir,sec:cr,sec:fgr}, will discuss approaches to recognition.
    \Cref{sec:dcnn} will discuss convolutional neural networks.


\section{Image feature detection}\label{sec:featuredetect}

    Before an image can be analyzed, it must be broken down into smaller components. An image's visual appearance
    can be captured using a combination of local image patterns --- \glossterm{patch-based features}. The most
    informative patch-based features are typically centered on simple image structures such as junctions, corners,
    edges, and blobs~\cite{tuytelaars_local_2007}. If these features can be reliably detected, localized, and
    described then image matching can be posed as a problem in matching sets of features. This section describes
    work related to detecting features in an image, and~\cref{sec:featuredescribe} will discuss how detected
    features are then described.

    The region where a feature is detected is called a \glossterm{keypoint}. The simplest definition of a keypoint
    is just an $xy$-location in an image. However, images contain information at multiple scales; therefore a
    keypoint is typically associated with a scale. The scale of a keypoint is a non-negative real number that
    defines the level of detail at which to interpret the underlying image information. A keypoint with a scale can
    be thought of as a circular region with a radius that is the scale multiplied by some constant % 
    (\eg{} $3\sqrt{3}$ is the constant used to determine a keypoint's radius in~\cite{perdoch_efficient_2009}). To
    account for changes in viewpoint and pose, it is also common to augment features with an orientation and shape.
    Adding these properties is said to add invariance to the feature. Invariant features can provide similar
    descriptions of the same semantic image region under different viewing conditions. However, adding invariance
    can cause features to lose distinguishing information~\cite{mikolajczyk_comparison_2005,
    tuytelaars_local_2007, perdoch_efficient_2009, lowe_distinctive_2004}.

    Many detectors have been developed to detect patch-based feature keypoints~\cite{mikolajczyk_comparison_2005,
    tuytelaars_local_2007}. Algorithms such as Harris, SUSAN, and FAST detect corners~\cite{harris_combined_1988,
    mikolajczyk_indexing_2001, smith_susannew_1997, rosten_machine_2006}. Blobs and corners can be detected with
    Hessian~\cite{beaudet_rotationally_1978, lindeberg_shape_adapted_1994} or difference of
    Gaussians~\cite{gaussier_neural_1992, lowe_distinctive_2004} detectors. There are also region-based detectors:
    maximally stable extremal regions~\cite{matas_robust_2004}, saliency based
    methods~\cite{buoncompagni_saliency_based_2015} and superpixel-based methods~\cite{ren_learning_2003,
    mori_recovering_2004}. Some applications choose to skip keypoint detection and use a uniform grid of dense
    features~\cite{liu_sift_2008, revaud_deep_2015, iscen_comparison_2015}. Other applications, such as face
    recognition, use specialized keypoint detectors~\cite{dantone_real_time_2012, berg_tom_vs_pete_2012}. There
    currently exists no principled method for selecting the appropriate feature detector. Different feature
    detectors perform differently given the application~\cite{tuytelaars_local_2007}.

    This section describes the representation of an image over multiple scales, the detection of features to
    sub-pixel and sub-scale accuracy, and the adaption of features to specify orientation and shape. We focus on
    the Hessian-based keypoint because it has been experimentally shown to be a reliable choice for instance
    recognition~\cite{tuytelaars_local_2007}.

   \subsection{Scale-space}
        Scale-space theory describes image features as existing at multiple
        scales~\cite{lindeberg_scale_space_1993}. The same point on an object seen close up appears quite different
        compared to when it is at a distance. For example, a zebra in the distance may appear to have two stripes
        that are connected, but when the animal appears closer it becomes clear that the stripes are actually
        disconnected.
        %This problem is addressed by detecting features at multiple scales.
        Multi-scale detection is formalized by the theory of scale-space, which parameterizes a continuous signal,
        $f$, with a scale, $\scale$. The original signal is said to exist at scale $0$. Convolving the original
        signal with a Gaussian kernel produces coarser scales.

        Let $f$ be a continuous $2$-dimensional signal that defines an image. Let vector $\pt=\ptcolvec$ be a
        location in the image. The function $g(\scale)$
        %= \frac{1}{\TAU\scale^2} \exp{-(\vec{i} \cdot \vec{i})/2\scale^2}$
        is the isotropic 2D Gaussian kernel. The scale-space representation of a continuous image (for any non-zero
        scale) takes the form: $\img(\pt, \scale) = g(\scale) \conv f(\pt)$, where $\conv$ is the convolution
        operator. However, we do not have access to a continuous representation of an image. Therefore, in
        practice, the continuous Gaussian kernel is replaced with the discrete Gaussian kernel. This can be
        efficiently implemented as a discrete convolution with the $1$-dimensional discrete Gaussian kernel in the
        $x$-direction and then in the $y$-direction, because the discrete Gaussian kernel is separable in
        orthogonal directions~\cite{lindeberg_scale_space_1993}. Using the definition of an image at a single scale
        the next step is to represent an image a multiple scales.

       \paragraph{Gaussian pyramid}

           \newcommand{\downsamp}[2]{#1[::\tightpad#2,::\tightpad#2]}
            % See page 39 of Scale-space theory.
            %The continuous image signal is defined to be the zeroth scale $\img(\pt, 0) = \rawimg(\pt)$. 
            The discrete scale-space representation of an image is efficiently implemented using a Gaussian
            pyramid. A scale-space pyramid consists of $L$ levels. Each level covers an octave. Starting from the
            base of each level with scale parameter $\sigma$ the next octave is reached when $\sigma$ doubles.
            There are $s$ intervals represented within each octave. A Gaussian pyramid is illustrated
            in~\cref{fig:ScaleSpaceFigure}.

            \ScaleSpaceFigure{}

            The pyramid's base, %
            $\img(\pt, 1) = g( 1 ) \conv \rawimg(\pt)$ %
            is the $\ell=0\th{}$ level of the pyramid, and is computed by blurring the original image (sometimes
            with small initial blurring) with $\sigma=1$. Subsequent levels of the pyramid are produced by doubling
            sigma, thus the $\ell\th{}$ level of the pyramid is $\img(\pt, 2^\ell)$.

            A property of discrete scale-space is that after appropriate smoothing downsampling the image by half
            is equivalent to doubling sigma. Let
            % ---
            $\img_\ell(\pt) = \downsamp{\rawimg}{2^{\ell}}({\pt} / {2^{\ell}})$ 
            % ---
            denote the raw image downsampled by a factor of $2^{\ell}$ using Lanczos resampling. Now, each level of
            the pyramid can be written as %
            $\img(\pt, 2^\ell) = g( 1 ) \conv \rawimg^\ell$. Given the raw image at level $\ell$, the scale
            corresponding to $\sigma$ can be written as a relative scale
            % ---
            $\sigma_\ell = \sigma / 2^\ell$.
            % ---
            Thus, a discrete image at any scale can be efficiently computed as:
            \begin{equation}
                \img(\pt, \sigma) =
                    g(\sigma_\ell) \conv \img_\ell(\pt)
            \end{equation}
            Discrete convolution is applied using a window of size
              $\floor{6\sigma_\ell + 1} + (1 -
              (\modfn{\floor{6\sigma_\ell + 1}}{2}))$.
            Interpolation between discrete values of $\pt$ is used to
              sample intensity at sub-pixel accuracy.

            A scale between two levels of the pyramid is called an interval. Typically, $s$ intervals --- with
            relative scales $2^{0/s}, 2^{1/s}, \ldots 2^{s/s}$ --- are computed to represent the octave between
            level $\ell$ and $\ell + 1$. If differences between scales are needed, then the scales $2^{-1/s}$ and
            $2^{1 + 1/s}$ are also computed~\cite{lowe_distinctive_2004}.

    \subsection{Hessian keypoint detection}

        Hessian-based keypoint detection searches for extrema of the Hessian operator in both space and
        scale~\cite{beaudet_rotationally_1978, lindeberg_shape_adapted_1994}. The Hessian detector can
        qualitatively be viewed as a blob detector, but it also detects corners which may appear as blobs in
        scale-space~\cite{tuytelaars_local_2007}. The Hessian keypoint detector will compute a response value for
        each point in scale space indicating how blob-like each pixel is. The extrema of this response defines a
        set of Hessian keypoints. Post processing removes non-robust keypoints and localizes all other keypoints to
        sub-pixel and sub-scale accuracy.

        \paragraph{Hessian response}
            Let subscripts denote the partial derivatives of the image intensity (\eg{} $\img_{x}$ is the first $x$
            derivative, $\img_{xx}$ is the second $x$ derivative, and $\img_{xy}$ is the first derivative in both
            $x$ and $y$). The Hessian is a matrix of second order partial derivatives and is defined at each point
            in scale space.
            \begin{equation}
                \hessMAT = 
                \BIGMAT{
                \img_{xx}(\pt, \scale) & \img_{xy}(\pt, \scale) \\
                \img_{xy}(\pt, \scale) & \img_{yy}(\pt, \scale) } 
            \end{equation}\label{eqn:hessianmatrix}  
            %(derivatives are computed in scale-space over an integration scale %$\scale_I$). 

            The initial response of the detector at each point is the determinant of the Hessian matrix. This
            response is computed for level and every pixel in the scale-space pyramid. At coarser scales the
            Hessian response weakens, so to ensure that responses between scales are comparable, the initial
            response is scale normalized by multiplying with $\sigma^2$. (See~\cite{lindeberg_feature_1998} for
            more details about the choice of this normalization factor.) The extrema of this space defines a set of
            candidate keypoints, $\kpts'$.
            \begin{equation}
                \kpts' = \argextrema{\pt,\scale} \paren{\scale^2 \detfn{\hessMAT}} 
            \end{equation}
            %\devcomment{Is this 2D or 3D extrema detection.  What would need to change in my math if it was 3D?} 
            A point in this 3D space is a maxima/minima if its scale normalized value is greater/less than the
            scale normalized values of all its neighbors in the pyramid --- \ie{} the $8$ neighbors in its current
            interval, its $9$ neighbors in the next interval, and its $9$ neighbors in the previous interval.

        \paragraph{Edge filtering}
            Edge responses are not robust --- \ie{} the same point can not be localized reliably in two views of the
            same scene --- due to their elongated nature. Because of this, the extrema that appear too edge-like
            are filtered using a threshold $t_{\tt{edge}}$ which is compared to the ratio of the Hessian's squared
            trace and the determinant.
            \begin{equation}
                \kpts = \{(\pt, \scale) \in \kpts' \where
                  \frac{\trfn{\hessMAT}^2}{\detfn{\hessMAT}} > t_{\tt{edge}}\}
            \end{equation}

        \paragraph{Sub-pixel and sub-scale localization}
            To compensate for the discrete nature of pixel images, each keypoint detection is localized to
            sub-pixel and sub-scale accuracy. The importance of feature localization is demonstrated
            in~\cite{ke_pca_sift_2004}, where descriptors were computed on the normalized vectors of patch
            gradients using only principal component analysis (PCA)~\cite{jolliffe_principal_2014}. Despite the
            simplicity of the descriptors the authors were still able to effectively match two images due to the
            robust localization of the features.

            Sub-pixel and sub-scale localization transforms a keypoint $\kp_0$ into $\kp^*$ using an iterative
            process. At each iteration $i$, a $2\nd{}$ order Taylor expansion, centered at %
            $\kp_i = (\pt_i, \scale_i)$, approximates the scale normalized Hessian response: %
            $T_i(\pt, \scale) \approx \scale^2 \detfn{\hessMAT}$. The keypoint is updated to the position of the
            maximum response of the Taylor expansion: $\kp_{i + 1} = \argmax{\kp} T_i(\kp)$. This process iterates
            until convergence. If the process does not converge before a threshold number of iterations, the
            keypoint is deemed not robust and thrown out.

    \subsection{Affine adaptation}
        So far, the keypoints we have described correspond to circular regions where the pixel radius is some
        multiple of the scale. To account for small affine changes seen in non-planar objects (like zebras), the
        shape of each circular keypoint is adapted into an ellipse.

        An affine shape $\vmat=\vMATRIX$ is estimated (as a lower triangular matrix) for each keypoint using an
        iterative technique involving the second moment
        matrix~\cite{lindeberg_shape_adapted_1997,baumberg_reliable_2000,mikolajczyk_comparison_2005}. The affine
        shape matrix transforms an ellipse into a unit circle. Note that because the matrix is lower triangular one
        of its eigenvectors points in the downward direction. Thus, the shape has no influence on the orientation
        of the keypoint. For each point in scale space the second moment matrix is evaluated as:
        \begin{equation}\label{eqn:secondmoment}
                \momentmat 
                \tighteq 
                \MAT{ 
                \img_x^2(\pt, \scale)      & \img_x(\pt, \scale) \img_y(\pt, \scale) \\
                \img_x(\pt, \scale) \img_y(\pt, \scale) & \img_y^2(\pt, \scale) }
        \end{equation}

        The goal is to ``stabilize'' each keypoint shape by searching for the transformation, $\vmat^*$, that
        causes the second moment matrix to have equal eigenvalues. For each keypoint, its elliptical shape is
        initialized as a circle $\vmat_0=\eyetwo$. For each iteration $i$:

        \begin{enumln}

            \item Compute the second moment matrix, $\warpedmomentmat{\vmat_i}$, at the warped image patch.

            \item Check if the keypoint shape is stable. A keypoint shape is stable if the eigenvalue ratio of the
            second moment matrix is close to $1$. If the keypoint has been stable for two consecutive iterations,
            then accept $\vmat^* \leftarrow \vmat_{i}$ and stop iteration. Otherwise, if the number of iterations,
            $i$, is greater than some threshold, then stop and discard the keypoint.

            \item Update the affine shape  using the rule
                % ---
                $\vmat_{i + 1} = \sqrtm{\warpedmomentmat{\vmat_i}} \vmat_i$.
        \end{enumln}

        The matrix $\vmat$ only defines the transformation from an ellipse to a circle. The standard representation
        of an ellipse is a conic of the form $\mat{E} = \vmat^T\vmat$. This means that $\vmat$ is only defined up
        to an arbitrary rotation~\cite{mikolajczyk_comparison_2005,perdoch_efficient_2009}. Thus, we can freely
        rotate $\vmat$ into a lower triangular matrix. This ensures that one of its eigenvectors is pointing
        downwards --- \ie{} in the direction of the ``gravity vector''~\cite{perdoch_efficient_2009}. Making use of
        the gravity vector removes a dimension of invariance. To allow for the specification of keypoint
        orientation, the keypoint representation can be extended with a parameter $\theta$ that defaults to $0$.

    \subsection{Orientation adaptation}

        The keypoint orientation is defined using the parameter $\theta$.
        By default, the orientation of a keypoint can be assumed to be aligned with the ``gravity vector'' ---
          \ie{} $\theta=0$~\cite{perdoch_efficient_2009}.
        Otherwise, an orientation must be computed.
        A common method for determining a keypoint's orientation is to use the dominant gradient orientation.
        In theory adapting the orientation to match the dominant gradient will cause a computed keypoint
          description to be invariant to rotations.

        To compute a keypoint's dominant orientation the pixels around a keypoint vote into a fine-binned
        orientation histogram~\cite{lowe_distinctive_2004}. A pixel's vote is weighted by its gradient magnitude
        multiplied by its Gaussian weighted distance to the keypoint center. The dominant orientation %
        $\ori \in \rangeinex{0,\TAU}$ is chosen as the peak of this histogram. If there is more than a single peak
        it is common to create a copy of the keypoint for each maxima in this histogram. This process is
        illustrated in~\cref{fig:testfindkpdirection}.

        \testfindkpdirection{}

        \FloatBarrier{}

    \subsection{Discussion --- detector and invariance choices}
        To identify individual animals, features must be detected in distinguishing areas of an animal. For a
        feature to be useful, it must be detected in the multiple images of the same individual despite variations
        in viewpoint, pose, lighting, and quality. In our baseline algorithm we choose to use a Hessian based
        detector~\cite{perdoch_efficient_2009, lindeberg_feature_1998} because it generally produces a large number
        of features and has been experimentally shown to be repeatable, accurate, and adaptable to multiple degrees
        of invariance~\cite{tuytelaars_local_2007}.

        Once a keypoint is detected, it is described using a keypoint description algorithm. It is desirable for a
        keypoint description to be invariant to small changes in viewpoint, pose, and lighting. Accurate
        localization of a keypoint in scale and space helps to ensure that similar images contain similar features.
        Sometimes, it is beneficial to further localize a keypoint in shape and orientation, thus adding invariance
        to the feature. However, if too much invariance is used, it may not be possible to distinguish between
        semantically different features.

        It is a challenge to choose the correct level of invariance when computing features. Often an application
        chooses one of two extremes. Consider the computation of keypoint orientation. Standard methods for
        orientation invariance assume patches can freely rotate, when in fact they may be constrained to be
        consistent with the orientation of surrounding patches~\cite{lowe_distinctive_2004}. On the other side
        extreme is the ``gravity vector''~\cite{perdoch_efficient_2009}, which globally enforces all keypoint to
        have a downward orientation. This may be a safe assumption when working with features from images of rigid
        objects taken in an upright position. It may not be correct when dealing with non-rigid objects like
        zebras.
        %Orientation invariance assumes that orientation is a local property
        %  of a patch, but the orientation of a patch is usually not
        %  independent of its surrounding patches on an object.
        In our experiments in~\cref{sub:exptinvar} we test different degrees on invariance. This test includes a
        novel method that achieves a middle ground between full orientation invariance and the gravity vector.


\section{Image feature description}\label{sec:featuredescribe}  

    Once each feature has been localized its visual appearance must be described before it can be matched. The goal
    of feature description is to encode raw image data into a vector --- \ie{} a \glossterm{descriptor}. To
    represent the visual appearance of a keypoint a descriptor vector should have the following properties: (1) two
    visually similar patches produce vectors with a small metric distance and (2) visually dissimilar patches have
    vectors with large distances between them.

    Constructing such a descriptor vector has been a core problem throughout the history of computer vision.
    The first texture descriptor robust to small image transformations was the scale invariant feature
      transform (SIFT) descriptor first published in 1999~\cite{lowe_object_1999, lowe_distinctive_2004}.
    Since then, other hand-crafted algorithms have been proposed.
    However, results have always been at least comparable to the SIFT descriptor, and SIFT is still an effective
      and widely used hand-designed descriptors~\cite{mikolajczyk_performance_2005, calonder_brief_2010,
      bay_surf_2006, leutenegger_brisk_2011, alahi_freak_2012, jegou_triangulation_2014}.
    A promising direction for outperforming the SIFT descriptor is descriptor
      learning~\cite{simonyan_descriptor_2012, simonyan_learning_2014, winder_picking_2009}; specifically
      descriptor learning using deep neural networks~\cite{razavian_cnn_2014, bengio_representation_2013,
      russakovsky_imagenet_2014}.
    This section first describes the basic SIFT algorithm and then provides an overview of alternatives that have
      been proposed to SIFT{}.
    Work related to learning descriptor vectors using deep neural networks is discussed later in~\cref{sec:dcnn}.
      
    \subsection{SIFT}
        The {SIFT descriptor} is a $128$ dimensional vector that summarizes the spatial distribution of the
        gradient orientations in an image patch~\cite{lowe_distinctive_2004}. To describe a keypoint with a SIFT
        descriptor, the keypoint's image data is warped using the affine transform of the scale space gradient
        image into a normalized reference frame (typically $41 \times 41$ pixels). For a descriptor to be useful in
        matching it is important that the keypoint is properly localized before a descriptor is
        computed~\cite{ke_pca_sift_2004}. Because it is not always possible to perfectly localize a keypoint, the
        SIFT descriptor aggregates information into a soft-histogram. Allowing data to contribute to multiple bins
        helps the SIFT descriptor to be robust to small localization errors and viewpoint variations. Distance
        between two SIFT descriptors is typically computed using the Euclidean distance. The SIFT descriptor of a
        patch is visualized in~\cref{fig:vizfeatrow}.

        The structure of a SIFT descriptor is as follows: A $4\times4$ regular grid is superimposing over the
        normalized patch. Each of the $16$ spatial grid cells contains an orientation histogram discretized into
        $8$ bins. The SIFT descriptor is the concatenation of all orientation histograms, resulting in a single %
        $16 \times 8 = 128$ dimensional vector.

        The patch information populates the SIFT descriptor as follows: For every pixel, the patch gradient (the
        derivative in the $x$ and $y$ direction) is computed. Next, each pixel computes its gradient magnitude and
        orientation. Each pixel then casts a weighted vote. The bin that a pixel votes into is computed from its
        $xy$-location and gradient orientation. The weight of a pixel's vote is based on its gradient magnitude and
        Gaussian weighted distance to the patch center. To be robust to small localization errors, a pixel's vote
        is split via trilinear interpolation ($x$-location, $y$-location, and orientation) into the orientation
        histograms of the pixel's nearest grid cells as well as neighboring orientation bins in each grid cell's
        orientation histogram.

        Once voting is completed a SIFT descriptor is normalized to account for lighting differences between
        images. First, the vector is L2-normalized to unit length, which makes the descriptor invariant to linear
        changes in intensity. Then, a heuristic --- that truncates each dimension to a maximum value of $0.2$ ---
        is applied to increase robustness to non-linear changes in illumination. Finally, the vector is
        renormalized.

        For storage considerations the resulting $512$-byte floating-point (float32) descriptor is typically cast
          as an array of unsigned $8$-bit integers (uint8), resulting in a $128$-byte descriptor.
        To reduce the impact of this quantization a trick is to multiply by $512$ instead of $255$ and then
          truncate values to $255$ before converting from a float to a uint8.
        Even though each component is $8$-bits and therefore can only store a maximum value of $255$, but because
          of truncation, L2-normalization, and properties of natural images value overflow is not likely to occur.
    
       \vizfeatrow{}

    \subsection{Other descriptors and SIFT extensions}
        Even more than a decade after its original publication, SIFT remains a popular descriptor for patch-based
        matching because it is versatile, unsupervised, widely available, and easy to use. The principles used to
        guide the construction of the SIFT descriptor --- particularly the use of aggregated gradients --- have
        inspired many variants, extensions, and new techniques~\cite{mikolajczyk_performance_2005,
        dalal_histograms_2005, bay_surf_2006}. Hand crafted alternatives to SIFT have been developed that are
        faster to compute and more efficient to store, but these alternatives do not significantly outperform
        SIFT's matching accuracy on general data~\cite{lowe_distinctive_2004, mikolajczyk_performance_2005,
        alahi_freak_2012}. This subsection provides a brief overview of these alternatives.

        % ALTERNATIVES FOR DETECTION
        The use of aggregated gradient information in SIFT has been adapted for use in other computer vision
        problems such as detection and scene classification.
        % GIST
        The GIST descriptor is a low dimension descriptor used for scene classification that coarsely summarizes
        rough appearance of an entire image~\cite{oliva_modeling_2001, douze_evaluation_2009}.
        % HOG
        The histogram of oriented gradients (HoG) descriptor is a high dimensional descriptor used in detection.
        The HoG descriptors describes the shapes of objects in an image~\cite{dalal_histograms_2005}. Like the SIFT
        descriptor, the HoG descriptor illustrates the value of gradient-based image descriptions and has inspired
        extensions such as the discriminatively trained parts model~\cite{felzenszwalb_object_2010}.

        As a general single-scale patch-based descriptor, the matching accuracy of SIFT has not been significantly
        outperformed on general datasets.
        % GLOH
        One attempt at an improved general descriptor is the gradient location-orientation histogram (GLOH)
        descriptor~\cite{mikolajczyk_performance_2005}. GLOH uses a similar structure to SIFT but replaces the
        rectangular-bins with log-polar bins. GLOH did achieve higher matching accuracy on some datasets, but it
        was not by a significant margin.
        Despite the lack of generic success, hand-crafted SIFT variants have been successful when applied to
        specific tasks.
        % COLORED SIFT
        Colored SIFT variants such as opponent-SIFT are valuable in category recognition tasks, where a color
        difference could be the distinguishing factor between categories~\cite{van_de_sande_evaluating_2010}
        % SCALE-LESS SIFT
        Combining multiple SIFT descriptors over different scales has also shown moderate improvements. The
        scale-less SIFT descriptor combines SIFT descriptors computed at multiple scales into a single descriptor.
        It has been shown to produce more accurate dense correspondences than representing each scale with an
        individual descriptor~\cite{hassner_sifts_2012}.

        %Despite However, domain specific modifications have shown promising results.
        %The Rotation Invariant Feature Transform (RIFT) descriptor~\cite{lazebnik_sparse_2005} uses concentric
        %  circles to make a similar modification.
        %The RIFT descriptor are used in texture classification~\cite{lazebnik_sparse_2005}.

        % SURF
        Efficiency is one area where SIFT has been significantly outperformed.
        An approximation to SIFT called speeded up robust features (SURF) is a fast approximation to SIFT
          based on integral images that achieves similar accuracy using a smaller $64$ dimensional
          descriptor~\cite{bay_surf_2006}.
        % DAISY
        The DAISY descriptor uses a similar binning structure to GLOH, but uses convolutions with Gaussian
          kernels to quickly aggregate gradient histograms~\cite{tola_fast_2008}.
        % BINARY PATTERNS
        Binary descriptors such as local binary patterns (LBP)~\cite{ojala_comparative_1996, zhang_local_2010},
          local derivative patterns~\cite{heikkila_description_2009}, and their variants such as
          BRIEF~\cite{calonder_brief_2010}, BRISK~\cite{leutenegger_brisk_2011}, and FREAK~\cite{alahi_freak_2012}
          also quickly compute compact distinctive descriptors.
        Binary descriptors are built using multiple pairwise comparisons of average image intensity at
          predetermined locations.
        This results in a small descriptor that effectively represents aggregated gradient information.

        Machine learning is able to outperform the matching accuracy of SIFT, however these techniques require
        training data to adapt to each new problem domain. Learned descriptors make use of the same aggregated
        gradient information used in the construction of SIFT descriptors. The Liberty, Yosemite, and Notre-Dame
        buildings datasets are standard datasets for descriptor learning~\cite{brown_discriminative_2011}. Error on
        these datasets is measured using false positive rate at $95\percent$ recall (FPR95). The baseline SIFT
        error on this dataset is $27.02\percent$. The configuration of a DAISY descriptor is learned
        in~\cite{winder_picking_2009} and achieves an error of $15.16\percent$ on the buildings datasets.
        In~\cite{simonyan_learning_2014}, large scale non-convex optimization is used to learn a spatial pooling
        configuration of log-polar bins, a dimensionality reduction matrix, and a distance metric to further reduce
        the FPR95 error to $10.98\percent$. The current state-of-the-art error of $4.56\percent$ on the buildings
        dataset is achieved using a convolutional neural network~\cite{zagoruyko_learning_2015}.

    \subsection{Discussion --- descriptor choices}
        In our application we use the SIFT~\cite{lowe_distinctive_2004} as our
        baseline descriptor because it is one of the most widely used and well-known descriptors. SIFT describes
        images patches in such a way that small localization errors do not significantly impact the resulting
        representation. Exploration of alternative convolutional descriptors is discussed later in~\cref{sec:dcnn}.


\section{Approximate nearest neighbor search}\label{sec:ann}  

    In computer vision applications it is often necessary to search a database of high dimensional
    vectors~\cite{shakhnarovich_nearest_neighbor_2006, datar_locality_sensitive_2004, muja_fast_2009,
    kulis_kernelized_2012, weiss_spectral_2009}. Patch descriptor vectors like SIFT are constructed such that the
    distance (under some metric) between vectors is small for matching patches and large for non-matching patches.
    Thus, finding matching descriptor vectors is often framed as a nearest neighbor search
    problem~\cite{lowe_distinctive_2004}. It becomes prohibitively expensive to perform exact nearest neighbor
    search as the size of the database increases. Therefore, approximate algorithms --- which can trade off a small
    amount of accuracy for substantial speed-ups --- can be used instead.

    \subsection{Kd-tree}

        A \glossterm{kd-tree} is a data structure used to index high dimensional vectors for fast approximate
        nearest neighbor search~\cite{bentley_multidimensional_1975}. A kd-tree is an extension of a binary tree to
        multiple dimensions. Each non-leaf node of the tree is assigned a dimension and threshold value. The node
        splits data vectors between the left and right children by comparing the value of the data vector at the
        assigned dimension to the assigned threshold.

        \paragraph{Building a kd-tree index}
        Indexing a set of vectors involves first choosing a dimension and threshold to split the data into two
        partitions. Then this procedure is recursively to each of the partitions. A common measure for choosing the
        dimension is to choose the dimension with the greatest variance in the data. The threshold is then selected
        as the median value of the chosen dimension.

        \paragraph{Augmenting a kd-tree index}
        It is possible to augment an existing kd-tree by adding and removing vectors. Addition of a vector to a
        kd-tree is performed by appending the point to its assigned leaf. Removal of points from a kd-tree is done
        using lazy deletion --- \ie{} by masking the removed data. To avoid tree imbalance, a kd-tree is re-indexed
        after the number of points added or removed passes a threshold. Any masked point is deleted whenever the
        tree is re-indexed.

        \paragraph{Searching a kd-tree index}
        Searching for a query point's exact nearest neighbor in a kd-tree has been shown to take expected
        logarithmic time for low ($k < 16$) dimensional data~\cite{friedman_algorithm_1977}. However, for higher
        dimensional data this same method takes nearly linear time~\cite{sproull_refinements_1991}. This is because
        a query point and its nearest neighbor might be on opposite sides of a partition. Therefore, searching for
        nearest neighbors is typically done by approximate search using a priority queue~\cite{beis_shape_1997}. A
        priority queue orders nodes to further search based on their distance to the query vector. The search
        returns the best result after a threshold number of checks have been made. % $ ~\cite{beis_shape_1997}.

        Search accuracy is improved by using multiple randomized kd-trees~\cite{silpa_anan_optimised_2008}. If a
        single kd-tree has a probability of failure $p$, then $m$ independently constructed trees have a $p^m$
        probability of failure. For each kd-tree a random Householder matrix is used to efficiently rotate the
        data. Using a random rotation preserves distances between rotated vectors but does not preserve the
        dimension of maximum variance. This means that each of the $m$ kd-trees yields a different partitioning of
        the data, although it is not guaranteed to be independent. When searching multiple random kd-trees, a
        single priority queue keeps track of the next nearest bin boundaries to search over all the trees.

    \subsection{Hierarchical k-means}
        Another tree-based method for approximate nearest neighbor search is the hierarchical k-means. Each level
        in the hierarchical k-means tree partitions the data using the k-means algorithm~\cite{lloyd_least_1982}
        with a small value of $k$ (\eg{} 3). To query a new point it moves down the tree into the bin of the
        closest centroid at each level until it reaches a leaf node. Hierarchical k-means was one of the first
        techniques used to define a visual vocabulary~\cite{nister_scalable_2006} --- a structure used for indexing
        and quantizing large amounts of descriptors.
    
    \subsection{Locality sensitive hashing}
        A hashing-based method for approximate nearest neighbor search is locality-sensitive hashing (LSH). This
        method is able to search a dataset of vectors for approximate nearest neighbors in sub-linear
        time~\cite{charikar_similarity_2002, datar_locality_sensitive_2004, kulis_fast_2009, kulis_kernelized_2012,
        tao_locality_2013}. LSH trades off a small amount of accuracy for a large query speed-up. A database is
        indexed using $M$ hash tables. Each hash table uses its own randomly selected hash function. For each hash
        table, a query vector computes its hash and adds the database vectors it collided with to a shortlist. The
        shortlist is sorted by distance and returned as the approximate nearest neighbors.

    \subsection{FLANN}
        The fast library for approximate nearest neighbors (FLANN) is a software package built to quickly index and
        search high dimensional vectors~\cite{muja_fast_2009}. The FLANN package implements efficient algorithms
        for hierarchical k-means, kd-trees, and LSH{}. It also implements a hybrid between the k-means and kd-tree,
        as well as configuration optimization, to select the combination of algorithms that best reaches the
        desired speed/accuracy trade-off for a given dataset. Configuration optimization is performed using the
        Nelder-Mead downhill simplex method~\cite{nelder_simplex_1965} with cross-validation.

    \subsection{Product quantization}
        Product quantization is a method for speeding up approximate nearest neighbor search of a set of high
        dimensional vectors~\cite{jegou_product_2011,ge_optimized_2013}. Each vector is split up into a set of
        sub-vector components. For each component, the sub-vectors are separately quantized using a
        codebook/dictionary/vocabulary. The pairwise squared distances between centroids in the vocabulary are
        stored in a lookup table. To comparing the distance between two vectors first each vector is split into
        sub-vectors, next the sub-vectors are quantized, and then the squared distances between quantized
        sub-vectors are read from the lookup table. The approximated squared distance between these two vectors is
        the sum of the squared distances between the quantized sub-vectors.

    \subsection{Discussion --- choice of approximate nearest neighbor algorithm}
        In our single annotation identification algorithm a query descriptor searches for its nearest neighbor in a
        database containing all descriptors from all \exemplars{}. Each annotation contains \OnTheOrderOf{4}
        features, which are described with $128$-component SIFT descriptors. Searching exact nearest neighbors
        becomes prohibitive when hundreds or thousands of images are searched. Thus, we turn towards approximate
        nearest neighbor algorithms. In this \thesis{} all of our approximate nearest neighbors are found using the
        multiple kd-tree implementation in the FLANN package~\cite{muja_fast_2009}. Using the configuration
        optimization built into the FLANN package, we have found that multiple kd-trees provide more accurate
        feature matches for our datasets than those computed by hierarchical k-means trees or LSH{}. In addition to
        being fast and accurate, multiple kd-trees support efficient addition and removal of points, which is
        needed in a dynamic setting~\cite{silpa_anan_optimised_2008}.


\section{Instance recognition}\label{sec:ir}
    There are many variations on the problem of visual recognition such as: specific object recognition (\eg{}
    CD-covers)~\cite{lowe_distinctive_2004, sivic_efficient_2009, nister_scalable_2006},
      % --
    location recognition~\cite{jegou_hamming_2008,jegou_aggregating_2012,tolias_aggregate_2013},
      % --
    person re-identification~\cite{shi_embedding_2016,karanam_person_2015,wu_viewpoint_2015},
      % --
    face verification/recognition~\cite{chopra_learning_2005, huang_labeled_2007, berg_tom_vs_pete_2012,
    chen_blessing_2013, taigman_deepface_2014, schroff_facenet_2015},
      % --
    category recognition~\cite{lazebnik_beyond_2006,zhang_local_2006,mccann_local_2012,boiman_defense_2008},
      % --
    and fine-grained recognition~\cite{parkhi_cats_2012,berg_poof_2013, gavves_local_2014}.
      % --
    The different types of recognition problems lie on a spectrum of specificity \wrt{} the objects they attempt to
    recognize. On one end of the spectrum, \glossterm{instance recognition} techniques --- like scene recognition
    or face verification --- search for matches of the same exact object. On the other end of the spectrum category
    recognition algorithms --- like car, bird, dog, and plane detectors --- look for the same type of objects.
    Other problems sit --- like fine-grained recognition where the goal might be to recognize specific subspecies
    of dog (\eg{} German shepherd, golden retriever, boxer, beagle, \ldots{}) --- somewhere in the middle. Animal
    identification is closest to the instance recognition side of the spectrum, but the proposed solution draws
    upon techniques from other forms of recognition.

    The discussion in this section focuses on instance recognition.
    The next two sections will discuss category recognition and fine-grained recognition.

    \subsection{Spatial verification}\label{subsec:sverreview}
        Before discussing specific techniques in instance recognition, we describe work related to spatial
        verification. Most instance recognition techniques initially match local image features without using any
        spatial information~\cite{lowe_distinctive_2004, sivic_efficient_2009, philbin_object_2007,
        tolias_image_2015}. This results in pairs of images with spatially inconsistent feature correspondences.
        Spatially inconsistent matches are illustrated in~\cref{fig:figSVInlier}. Inconsistent features are removed
        using \glossterm{spatial verification}, a process based on the random sample consensus (RANSAC)
        algorithm~\cite{fischler_random_1981}.

        RANSAC has come to refer to a family of iterative techniques to sample inliers from a noisy dataset that
        are consistent with some model~\cite{fischler_random_1981, hartley_multiple_2003, chum_locally_2003,
        raguram_usac_2013}. In the context of spatial verification the model is an affine transformation matrix,
        and the dataset is a set of feature correspondences~\cite{lowe_distinctive_2004, sivic_video_2003,
        philbin_object_2007, chum_total_2011, arandjelovic_three_2012}. At each iteration of RANSAC a small subset
        of points is sampled from the original dataset and used to fit a hypothesis model. All other data points
        are tested for consistency with the hypothesis model. A score is assigned to the hypothesis model based on
        how well the out of sample data fit the model (\eg{} the number of transformed points that are within a
        threshold distance of their corresponding feature). After a certain number of iterations the process stops
        and returns the hypothesized model with the highest score as well as the inliers to that model.

        When RANSAC returns a large enough set of inliers (\wrt{} some threshold), the hypothesis model it is
        generally considered to be a ``good fit''. In this case a more complex model --- that may be more sensitive
        to outliers --- can be fit. In spatial verification, it is common to use the RANSAC-inliers to estimate a
        homography transformation~\cite[311--320]{szeliski_computer_2010}. The homography is then used to estimate
        a new set of refined inliers, and these are returned as the spatially verified feature correspondences.

        \figSVInlier{}

        \FloatBarrier{}

    \subsection{Lowe's object recognition}

        Lowe's introduction of SIFT descriptors includes an algorithm for recognizing objects in a training
        database and serves as an instance recognition baseline~\cite{lowe_distinctive_2004}. A single kd-tree
        indexes all database image descriptors. Approximate nearest neighbor search of the kd-tree is performed
        using the best-bin-first algorithm~\cite{beis_shape_1997}. For a query image, each keypoint is assigned to
        its nearest neighbor as a match. The next nearest neighbor (belonging to a different object) is used as a
        normalizer --- a feature used to measure the distinctiveness of a match. Any match with a ratio of
        distances to the match and the normalizer greater than threshold $t_{\tt{ratio}} \teq 0.8$ is filtered as
        not distinctive. Features likely to belong to the same object are clustered using a Hough Transform, and
        then clusters of features are spatially verified with a RANSAC approach~\cite{fischler_random_1981}.


    \subsection{Bag-of-words instance recognition}\label{subsec:bow}

        One of the most well-known techniques in instance recognition is the \glossterm{bag-of-words} model
        introduced to computer vision by Sivic and Zisserman~\cite{sivic_video_2003, sivic_efficient_2009}. The
        bag-of-words model addresses instance recognition using techniques from text-retrieval. An image is cast as
        a text document where the image patches (detected at keypoints and described with SIFT) are the words. The
        concept of a visual word is formalized using a visual vocabulary. A \glossterm{visual vocabulary} is
        defined by clustering feature descriptors traditionally constructed using k-means~\cite{lloyd_least_1982}
        (however more recent methods have learned vocabularies using neural
        networks~\cite{arandjelovic_netvlad_2016}). The centroids of the clusters represent the \glossterm{visual
        words} in the vocabulary. These centroids are used to quantize descriptor space. A feature in an image is
        assigned (quantized) to the visual word that is the feature's approximate nearest neighbor in the
        vocabulary. This means that each descriptor vector can be represented using just a single number --- \ie{}
        its index in the vocabulary. Vocabulary indices are used to construct an inverted index, which allows
        multiple feature correspondences to be made using a single lookup.

        Given a visual vocabulary, the bag-of-words algorithm consists of two high level steps: (1) an offline
        indexing step and (2) an online search step. The offline step indexes a database of images for fast search.
        First each descriptor in each database image is assigned to its nearest visual word. An inverted index is
        constructed to map each visual word to the set of database features assigned to that word. Each database
        feature is assigned a weight based on its term frequency (tf). Finally, each word in the vocabulary is
        assigned a weight based on its inverse document frequency (idf). The online step searches for the images in
        the database that are visually similar to the query image. First, each descriptor in the query image is
        assigned to its visual word, and the term frequency of each visual word in the query image is computed.
        Then, the inverted index is used to build a list of all images that share a visual word with the query. For
        each matching image, the sum of the tf-idf scores of the corresponding features is used as the image score.
        Finally, the ranked list of images is returned. These steps are now described in further detail.

        \paragraph{The inverted index}
            The visual vocabulary allows for a constant length image representation. An image is represented as a
            histogram of visual words called a bag-of-words. A bag-of-words histogram is sparse because each image
            contains only a handful of words from a vocabulary. The sparsity of these vectors allows for efficient
            indexing using an inverted index. An inverted index maps each word to the database images that contain
            the word. Therefore, when a feature in a new query image is quantized the inverted index looks up all
            the database features that it matches to. A new feature correspondence is created for each database
            feature the inverted index maps to. For each correspondence a feature matching score is computed.
            Because all the word assignments and feature correspondences are known, the scores of all matching
            images can be efficiently computed by summing the scores of their respective feature correspondences.

        \paragraph{Vocabulary tf-idf weighting}
            Each word in the database is weighted by its inverse document frequency (idf), and each individual
            descriptor is weighted by its term-frequency (tf)~\cite{sivic_efficient_2009}. The idea behind the idf
            weight is that words appearing infrequently in the database are discriminative and should receive
            higher weight. The idea behind the tf weight is that words occurring more than once in the same image
            are more important.

        \paragraph{Formal bag-of-words scoring}
            Let $\X$ be the set of descriptor vectors in an image. We also use $\X$ to refer to the image in
            general. Descriptor space is quantized using a visual vocabulary where $\C$ is the set of word
            centroids and $w_\c$ is the weight of a specific word. Let $\X_\c \subset \X$ be the set of descriptors
            in an image assigned to visual word $\c$. Let $q(\x)$ be the function that maps a vector to a visual
            word. We overload notation to also let $q(\X)$ map a set of descriptors into a set of visual words.

            The tf-idf weighting of a single word $\c$ in the vocabulary is computed as follows: Let $N$ be the
            number of images in the database. Let $N_\c$ be the number of images in the database that contain word
            $\c$. $\card{\X}$ is the number of descriptors in an image, and $\card{\X_\c}$ is the number of
            descriptors quantized to word $\c$ in that image. The idf weighting of word $\c$ is:
            \begin{equation}
                w_\c = \opname{idf}(\c) = \ln{N/N_\c}
            \end{equation}
            The tf weighting of a word $\c$ in an image $\X$ is:
            \begin{equation}
                \opname{tf}(\X, \c) = \frac{\card{\X_\c}}{\card{\X}}
            \end{equation}

            Similarity between bag-of-words vectors is computed using the weighted cosine similarity. It is only
            necessary to sum the scores of matching features, because the weight of a word that is not in both a
            query and database image is $0$. The tf-idf similarity between two images can be written as
            \begin{equation}
                \opname{sim}(\X, \Y) = \sum_{\c \in q(\X) \isect q(\Y)} \opname{tf}(\X, \c) \opname{tf}(\Y, \c) \opname{idf}(\c) 
            \end{equation}
            or equivalently
            \begin{equation}
                \opname{sim}(\X, \Y) = \frac{1}{\card{\X}\card{\Y}}\sum_{\c \in \C} w_\c \sum_{\xdesc \in \X_\c} \sum_{\ydesc \in \Y_\c} 1
            \end{equation}
            The second formulation unifies the bag-of-words model with other vocabulary-based methods in the SMK
            framework, which will be discussed later in~\cref{sec:smk}.

        \paragraph{Extensions to bag-of-words}
            The main strength and the main weakness of vocabulary-based matching is its use of quantization.
            Quantization allows for large databases of images to be searched very
            rapidly~\cite{nister_scalable_2006}. However, quantization causes raw descriptors to lose much of their
            discriminative information~\cite{philbin_lost_2008, boiman_defense_2008}. When a high-dimensional
            feature vector is quantized, it only encodes the presence of a word in a single number. This number is
            as descriptive as the partitioning of descriptor space, which is quite coarse in $128$ dimensions, even
            with a large vocabulary. Several methods have been developed to help reduce errors caused by
            quantization.

            Soft-assignment helps avoid quantization errors by mapping each raw descriptor to multiple
            words~\cite{philbin_lost_2008}. Another way to reduce quantization error is to use a finer partitioning
            of descriptors space~\cite{philbin_object_2007}. Approximate hierarchical clustering and approximate
            k-means have been used to build vocabularies with up to $1.6 \times 10^7$
            words~\cite{nister_scalable_2006, philbin_object_2007, mikulik_learning_2010}. Alternative similarity
            measures for descriptor quantization are also explored in~\cite{mikulik_learning_2010}. A projection
            matrix for SIFT descriptors is learned in~\cite{philbin_descriptor_2010} to preserve information that
            would be lost in quantization.

            Because the tf-idf weighting was original designed for text recognition, it does not take into account
            challenges that occur in image recognition such as bursty features --- a single feature that appears in
            an image with a higher than term expected frequency (\eg{} bricks on a wall or vertical stripes on a
            zebra). Strategies for accounting for burstiness involve penalizing frequently occurring features by
            removing multiple matches to the same feature, using inter-image normalization, and using intra-image
            normalization.~\cite{jegou_burstiness_2009}.
            
            Query Expansion is a way to increase the recall of retrieval techniques and recover from tf-idf
            failure~\cite{chum_total_2007, chum_total_2011, arandjelovic_three_2012, tolias_visual_2014}. After an
            initial query, all spatially verified feature correspondences are back-projected onto the query image.
            Then the query is then re-issued. A model of ``confusing features'' --- features more likely to belong
            to the background --- can be used to filter out matches that should not be back projected onto the
            query image. Query expansion enriches the query with intermediate information that may help retrieve
            other viewpoints of the query image. However, because this technique requires at least one correct
            result in the ranked list, it only improves recall for queries that already have high accuracy.

            One method to improve the performance of bag-of-words search is to remove non-useful features. It is
            found that as few as $4\percent$ of the features can be used in location recognition without loss in
            accuracy~\cite{turcot_better_2009}. This related work defines a useful feature as one that is robust
            enough to be matched with a corresponding feature and stable enough to exist in multiple viewpoints.
            Thus, these useful features are computed as those that produced a spatially verified match to a correct
            image. Any feature that does not produce at least one spatially verified match is removed. Removing
            non-robust features both saves space and improves matching accuracy.

    \subsection{Min hash}
        Min-hashing is the analog of locality-sensitive hashing for sets. Min-hashing has been applied as an
        instance recognition technique for near-duplicate image detection~\cite{chum_near_2008}, logo
        recognition~\cite{romberg_bundle_2013}, large scale image search~\cite{wang_semi_supervised_2012}, scene
        recognition~\cite{zhang_image_2011}, and unsupervised object discovery~\cite{chum_geometric_2009,
        chum_large_scale_2010}.

        The basic idea is to represent an image as a set of hashes based on permutations of a visual vocabulary.
        Recognition is performed performing a lookup for each hash. Collisions are returned as the recognition
        results. Like LSH, the primary advantage of using min hash for instance recognition is its speed.

    \subsection{Hamming embedding}
        Hamming embedding is an extension of the bag-of-words framework that reduces the information lost in
        quantization by assigning each descriptor a small binary vector~\cite{jegou_hamming_2008,
        jegou_burstiness_2009, jegou_improving_2010}. Each visual word $\c$, is assigned a $d_b \times d$ random
        orthogonal projection matrix $\mat{P}_\c$, where $d$ is the number of descriptor dimensions and $d_b$ is
        the length of the binary code. A set of $d_b$ thresholds, $\vec{t}_\c \in \Real^{d_b}$, is pre-computed for
        each word using the descriptors used to form the visual word cluster. These descriptors are projected using
        the word's random orthogonal matrix, and the median value of each dimension is chosen as that dimension's
        threshold.

        When any descriptor, $\desc$, is assigned to a word $\c$ it is also assigned a binary Hamming code,
        $\vec{b}$. To compute the binary Hamming code the descriptor is projected using the word's orthogonal
        matrix, $\vec{b}' = \mat{P}_\c \desc$, and then each dimension is binarized based on a threshold, %
        $b_i = (b'_i > t_{\c{}i})$.

        When a query descriptor, $\desc$, is assigned to a word, $\c$, it is matched to all database descriptors
        belonging to that word. Each match is then assigned a score. First, the Hamming distance, $h_d$, is
        computed between the binary signature of the query and database descriptors. If the Hamming distance of the
        match is not within threshold, $h_t$, the score of the match is $0$ and does not contribute to bag-of-words
        scoring. Otherwise, the score is the word's squared idf weight multiplied by a Gaussian falloff based on
        the Hamming distance. Using the inverted index, each image is scored by summing the scores of the
        descriptors that matched that image. The image scores are used to define a ranked list of results.
        %In~\cite{jegou_hamming_2008} only the idf weight is used 
        %In~\cite{jegou_burstiness_2009} squared idf and the Gaussian falloff
        %In~\cite{jegou_improving_2010} squared idf and the probability having a Hamming distance lower than or equal to a.
        %\begin{equation}
        %w_d(h_d) = -\log_2\paren{2^{-d_b} \sum_{i = 0}^{h_d} \binom{i}{d_b}}
        %\end{equation}

    \subsection{Fisher vector}
        A Fisher vector is an alternative to a bag-of-words~\cite{perronnin_large_scale_2010_1,
        jegou_aggregating_2010}. Like bag-of-words, Fisher vector representations have been used in both instance
        and category recognition~\cite{perronnin_fisher_2007, cinbis_image_2012, sun_large_scale_2013,
        sanchez_image_2013, juneja_blocks_2013, douze_combining_2011, ma_local_2012, murray_generalized_2014,
        gosselin_revisiting_2014}. Instead of training discrete visual vocabulary using the cluster centers of
        k-means, a Fisher vector encoding uses a continuous Gaussian mixture model (GMM). The number of Gaussian
        components in the GMM is  similar to the number of words in a vocabulary. An image is encoded using the GMM
        by computing the likelihood of each feature with respect to the GMM{}. Likelihoods for different components
        of the GMM are aggregated using a soft-max function. Often, each component of this vector $\vec{v}$ is then
        power law normalized with fixed constant $0 \leq \beta < 1$. Power law normalization is a simple post
        processing method written as $v_i = \txt{sign}\paren{v_i}\abs{v_i}^\beta$~\cite{jegou_aggregating_2012}.
        Fisher vectors produce a much richer representation than normal bag-of-words vector because each descriptor
        is assigned to a continuous mixture of words rather than a single word.

        It is noted in~\cite{perronnin_large_scale_2010_1} that using Fisher vectors for instance recognition is
        similar to tf-idf. Normalized Fisher vectors down-weight frequently occurring GMM components --- \ie{}
        words with low idf weights. Furthermore, Fisher vector representations are well suited for compression,
        which allows scaling to large image collections.

    \subsection{VLAD --- vector of locally aggregated descriptors} 

        A vector of locally aggregated descriptors (VLAD) is similar to a Fisher vector descriptor --- in fact it
        is a discrete analog of a Fisher vector~\cite{jegou_aggregating_2010, jegou_aggregating_2012}. Like Fisher
        vectors, VLAD has been used in the context of both instance and category
        recognition~\cite{jegou_negative_2012, delhumeau_revisiting_2013, arandjelovic_all_2013}. VLAD still
        computes a visual vocabulary and assigns each feature to its nearest word, but instead of only recording
        presence or absence of a word, each feature computes the residual vector from the centroid of its assigned
        word. The residual vectors are summed to produce one constant length vector per word. All summed residuals
        are concatenated to produce a constant length image representation. Aggregation of the residual vectors
        allows for an accuracy similar to bag-of-words methods to be obtained, but using a smaller vocabulary
        ($\OnTheOrderOf{1} - \OnTheOrderOf{2}$ words). Like Fisher vectors, VLAD descriptors are also power-law
        normalized~\cite{jegou_aggregating_2012}.

        There have been many extensions of the VLAD descriptor. The value of PCA, whitening, and negative evidence
        was shown in~\cite{jegou_negative_2012}. The MultiVLAD scheme is inspired by~\cite{torii_visual_2011}, and
        allows for retrieval of smaller objects that appear in larger images~\cite{arandjelovic_all_2013}. The
        basic idea is that VLAD descriptors are tiled in $3 \times 3$ grids. An integral
        image~\cite{viola_robust_2004} of unnormalized VLAD descriptors is used to represent many possible tiles.

        A vocabulary adaptation scheme is also introduced in~\cite{arandjelovic_all_2013}. The vocabulary is
        updated when a new image is added to the VLAD inverted index. This is performed by updating any word
        centroid $\c$ to $\c'$, where $\c'$ is the average of all the descriptors currently assigned to that
        word. The residuals of the affected words are recomputed and re-aggregated into updated VLAD descriptors.

        Recently, NetVLAD --- a convolutional variant of the VLAD descriptor --- has been
        introduced~\cite{arandjelovic_netvlad_2016,radenovic_cnn_2016}. NetVLAD uses deep learning with a triplet
        loss function to simultaneously learn both the patch-based descriptors and the vocabulary. This
        convolutional approach shows large improvements (a $19\percent$ improvement on Oxford 5k) over previous
        state-of-the-art image retrieval techniques.

    \subsection{SMK --- the selective match kernel}\label{sec:smk}
        The selective match kernel (SMK) encapsulates the vocabulary-based techniques such as bag-of-words, Hamming
        embedding, VLAD, and Fisher vectors into a unified framework~\cite{bo_efficient_2009,
        tolias_aggregate_2013, tolias_image_2015, jegou_triangulation_2014}. SMK provides a framework that
        ``bridges the gap'' between matching-based (here a match refers to a feature correspondence) approaches and
        aggregation-based approaches. The scores of matching-based approaches such as Hamming embedding and
        bag-of-words are based on establishing individual features correspondences. In contrast, the scores of
        aggregation approaches such as VLAD and Fisher vectors are computed from compressed image representations,
        where the individual features are not considered.

        An advantage of a matching-based approach like Hamming embedding is that it can define a selectivity
        function. A selectivity function down weights individual feature correspondence with low descriptor
        similarity. Aggregation schemes have been shown to have their own advantages. Aggregated approaches like
        VLAD allow for matching applications to scale to a large number of images because each image is indexed
        with a compressed representation. Furthermore, aggregation-based approaches have been shown to provide
        better matching results on many datasets because they implicitly down weight bursty
        features~\cite{tolias_aggregate_2013, tolias_image_2015}.

        In the SMK framework a matching function and selectivity function are chosen. Different selections of these
        functions can implement and blend desirable attributes of the aforementioned frameworks. The matching
        function assigns correspondences between query and database descriptors. The choice of the matching
        function determines whether the resulting kernel is aggregated or non-aggregated. The selectivity function
        weights a correspondence's contribution to image similarity. It also can apply either power-law like
        normalization or hard thresholding in order to down weights correspondences with low visual similarity. One
        advantage of the SMK framework is that the selectivity function can be used in aggregated matching. In this
        case the selectivity function is applied to all correspondences assigned to a particular word.

    \subsection{Face recognition and verification}
        Face recognition is a specific form of instance recognition with the goal of recognizing individual human
        faces~\cite{zhao_face_2003, huang_labeled_2007}. Related to face recognition is the problem of face
        verification. In contrast to face recognition, face verification takes two unlabeled images and decides if
        they show the same face or different faces~\cite{taigman_deepface_2014}. Clearly these techniques are
        complementary because highly ranked results from a face recognition algorithm can be verified as true or
        false by a face verification algorithm.

        Due to the specific nature of this problem specialized features detectors are often used. Facial feature
        detectors localize facial-landmarks such as the eye, mouth, and nose center and corner
        locations~\cite{dantone_real_time_2012, berg_tom_vs_pete_2012}. Local texture-based descriptors such as Gabor
        filters~\cite{liu_gabor_2002, zhang_histogram_2007, shen_review_2006} and local binary patterns
        (LBP)~\cite{ahonen_face_2006, chen_blessing_2013} are extracted at detected facial
        regions~\cite{belhumeur_localizing_2011}. Facial recognition researchers have also developed global
        descriptors --- such as eigenfaces~\cite{turk_eigenfaces_1991},
        Fisherfaces~\cite{belhumeur_eigenfaces_1997}, and neural network based
        descriptions~\cite{lawrence_face_1997, taigman_deepface_2014}. --- that represent the entire face.
        Recently, algorithms using both local and global representations computed using deep convolutional neural
        networks have shown state-of-the-art performance on both machine and human verification and recognition
        benchmarks~\cite{taigman_deepface_2014}.

        In face recognition, each face image is encoded into a single vector. A function is trained to classify an
        unseen test image as an individual from the database of known faces. Many techniques are used in the
        literature to retrieve or classify a face. Examples of these techniques are: neural
        networks~\cite{turk_eigenfaces_1991, taigman_deepface_2014}, sparse coding~\cite{wright_robust_2009,
        jiang_label_2013}, principal component analysis (PCA)~\cite{craw_face_1992}, Fisher linear discriminant
        (FLD)~\cite{liu_robust_2000}, linear discriminant analysis (LDA)~\cite{lu_face_2003}, and support vector
        machines (SVMs)~\cite{phillips_support_1998, levy_svm_minus_2013}.

        Before the neural network revolution~\cite{krizhevsky_imagenet_2012}, sparse coding was one of the most
        popular techniques to retrieve faces~\cite{aharon_k_svd_2006, wright_robust_2009, zhang_sparse_2011,
        jiang_label_2013}. Sparse coding attempts to reconstruct unlabeled test vectors by searching for a linear
        combination of basis vectors from an over-complete labeled training database. Coding-based techniques are
        very similar to vocabulary-based methods. A codebook, dictionary, and vocabulary all are used to build
        image-level vector representations by quantizing raw features.

        Another interesting technique is the Tom-vs-Pete classifier~\cite{berg_tom_vs_pete_2012}. Given a set of
        $N$ individuals (classes), a set of Tom-vs-Pete classifiers are used for both verification and indexing. At
        each facial landmark, $k$ Tom-vs-Pete classifiers are computed. A single Tom-vs-Pete classifier is a linear
        SVM trained on a single corresponding feature for a single pair of classes. \Eg{} all the nose descriptors
        from class $T$ and class $P$ make up the SVM training data, and the learned SVM classifies a new nose
        feature as $T$-ish or $P$-ish. A descriptor vector for a single face is made by selecting $5000$ out of the
        total $k\binom{N}{2}$ classifiers and concatenating the signed distances from all the classifiers'
        separating hyperplanes. This descriptor facilitates both search and verification. A pair of face descriptor
        vectors can be verified as either a correct or incorrect match by constructing a new vector. The new vector
        is constructed by concatenating the element-wise product and difference of the two descriptor vectors. Then
        this new vector is classified using a radial basis function SVM{}.

        One of the most recent advances in face verification and recognition is the DeepFace
          system~\cite{taigman_deepface_2014}.
        The DeepFace system implements face verification using the following pipeline:
        detect \rpipe{} align \rpipe{} represent \rpipe{} classify.
        Specialized facial point detectors and a 3D face model are used to register a 3D affine camera to an
          RGB-image.
        The image is then warped into a ``frontalized'' view using a piecewise affine transform.
        A face is represented as the $4096$ dimensional output of a deep $7$ layer convolutional neural network
          that exploits the aligned nature input images.
        An $8$\th layer is used in supervised training where each output unit corresponds to a specific
          individual.
        At test time the L2-normalized output of the network is used as the feature representation.
        In a supervised setting, a $\chi^2$-SVM is trained to recognize the individuals in a training dataset
          using the descriptor vectors produced by the network.
        In an unsupervised setting an ensemble of classifiers is used.
        The ensemble is composed of the output of a Siamese network~\cite{chopra_learning_2005} and 
          several non-linear SVM classifiers with different inputs.
        The inputs are deep representations --- the activations of a deep neural network's output layer --- of
          the 3D aligned RGB-image, the 2D aligned RGB-image (generated using a simpler model based on similarity
          transforms), and an image comprised of intensity, magnitude, and orientation channels.
        Each input was fed through four deep networks each with different initialization seeds.
        DeepFace achieves an accuracy of $0.9735$ on the Labeled Faces in the Wild
          dataset~\cite{huang_labeled_2007}, which is comparable to the human performance measured at $0.975$.
        When using unaligned faces the ROC score drops to $0.879$, which demonstrates that alignment is very
          important for handling the problem of viewpoint in face verification.

    \subsection{Person re-identification}
        %Radke dictionary learning ICCV~\cite{karanam_person_2015}
        %Radke pose priors TPAMI~\cite{wu_viewpoint_2015}
        %Deep model of person re-id~\cite{shi_embedding_2016}.
        The person re-identification problem is typically posed in the context of locating the same person within
          a few minutes or hours from low-resolution surveillance
          video~\cite{hirzer_relaxed_2012,karanam_person_2015,wu_viewpoint_2015,shi_embedding_2016}.
        Common approaches to person re-identification typically transform images into a fixed length mid-level
          vector representation and a learned distance metric is used to compare representations.
        Mid-level representations can be built from color and texture histograms or extracted using a
          convolutional neural network.
        The distance metric is commonly learned as a Mahalanobis distance using linear discriminant
          analysis~\cite{hirzer_relaxed_2012}.
        However, alternative approaches using dictionary learning ~\cite{karanam_person_2015} have also been
          shown to work well.
        Improvements to baseline can be achieved by conditioning person descriptors on viewpoint and
          pose~\cite{wu_viewpoint_2015}.
        Recently both features and distance metric have been learned using neural
          networks~\cite{shi_embedding_2016}.
        %The data for person re-identification typically is composed of
        %  low-resolution image captured by surveillance cameras.
        %The goal is often to identify images of people taken within minutes or
        %  hours of each other.
        %and
        %  therefore keypoint algorithms have typically proven most successful.
        %Therefore, additional work is needed to generalize to other species.
        %We have performed initial experiments that support this claim.

    \subsection{Discussion --- instance recognition}
        Most instance recognition techniques use an indexing scheme based on a visual
        vocabulary~\cite{tolias_image_2015, jegou_hamming_2008, philbin_object_2007, cao_learning_2012,
        arandjelovic_all_2013, jegou_negative_2012, chum_fast_2012, gong_multi_scale_2014}. However, our baseline
        approach for animal identification does not use a visual vocabulary. This is because a visual vocabulary
        quantizes the raw features in the image and thus removes some of their discriminative
        ability~\cite{philbin_lost_2008, boiman_defense_2008}. We have found this quantization to cause a
        noticeable drop in performance. Many aspects of our baseline algorithm are similar to Lowe's recognition
        algorithm~\cite{lowe_distinctive_2004}, which does not quantize descriptors. The guiding principles of
        matching, filtering based on distinctiveness, filtering based on spatial consistency, and scoring are
        shared with our approach. However, our approach features several improvements to this algorithm.
        Furthermore, animal identification is a dynamic problem with specific domain-based concerns --- such as
        quality and viewpoint in natural images --- and requires innovation beyond Lowe's recognition algorithm.

        % chktex-file 8
        Even though we would prefer to retain the discriminative information contained in raw descriptors,
          quantized image search has the ability to scale beyond our current suites~\cite{chum_fast_2012,
          perronnin_large_scale_2010_1, tolias_image_2015}.
        In the future it may be necessary to investigate a VLAD-based SMK framework as a quantized alternative to
          our matching algorithm.
        Techniques such as soft-assignment~\cite{philbin_lost_2008} and learned
          vocabularies~\cite{mikulik_learning_2010} could be used to reduce quantization errors.
        It is necessary to update the vocabulary as new images are added to the system.
        This issue could be addressed using the vocabulary adaptation technique in~\cite{arandjelovic_all_2013}.
        However, in this research we are more focused on the problem of verifying identifications to reduce
          manual effort.
        As such we leave the scalable search issue for future work.

        Facial recognition is similar to the problem of animal identification.
        %Technically is is a subset of the problem.
        Both problems seek to identify individuals. Some techniques used for face verification such as the Siamese
        network~\cite{chopra_learning_2005, taigman_deepface_2014} can be extended to the scope of animal
        identification. However, there is a much more mature literature on face recognition that has resulted in
        easily accessible and specialized algorithms for face feature detection and --- most importantly --- for
        face alignment. Individual animal identification does not have such a corpus of knowledge. We do not have
        access to highly specialized animal part detectors and alignment algorithms. Furthermore, we would like our
        algorithms to generalize to multiple species, so we would like to avoid over-specialized approaches. These
        are some reasons why convolutional neural networks will not make a prominent appearance in this \thesis{}.
        Other reasons involve the size of our datasets. The recent NetVLAD network was trained using training
        datasets with $10,000$ to $90,000$ images~\cite{arandjelovic_netvlad_2016}. We simply do not have this much
        labeled data. However, one goal of this \thesis{} is to develop techniques that will help bootstrap labeled
        datasets of this size. Future research should investigate these deep learning techniques so they can be
        used after enough data has been collected for a specific species.

        While the problem of animal identification and person re-identification are conceptually similar ---
        sharing challenges such as lighting, pose, and viewpoint variation --- differences in data collection
        creates the need for different solutions in practice. In contrast to the low-resolution image captured by
        surveillance cameras, the images used in animal identification are often manually captured by scientists in
        the field using high resolution DSLR cameras, and the goal is to match individuals over longer periods of
        time (years). Furthermore, re-identification techniques commonly focus on aggregate features that emphasize
        clothing, color, texture, and the presence of objects such as coats and backpacks, while in the animal id
        problem, it is often subtle localized variations in patterns on the skin and fur that distinguish
        individuals.


\section{Category recognition}\label{sec:cr}  

    Different types of image recognition lie at different points on a spectrum of specificity. If instance
    recognition is at one end of the spectrum, then category recognition is at the other. The goal of a
    \glossterm{category recognition} algorithm is to assign a categorical class label to a query
    image~\cite{everingham_pascal_2010, everingham_pascal_2015, russakovsky_imagenet_2014, deng_imagenet_2009,
    fei_fei_one_shot_2006, griffin_caltech_256_2007}. The categories often have visual appearances with a high
    degree of intra-class variance. \Eg{}, a recliner and a bench both belong to the chair category. Image
    representations and similarity measures are constructed to account for this. Despite this, techniques in
    category recognition have many similarities to instance recognition techniques. Until the neural network
    revolution~\cite{krizhevsky_imagenet_2012}, most category recognition techniques have been based on vocabulary
    methods~\cite{csurka_visual_2004, yang_linear_2009, sanchez_compressed_2013, russakovsky_imagenet_2014,
    krizhevsky_imagenet_2012} similar to those discussed in~\cref{subsec:bow}. This section first provides a brief
    overview of this literature. Then, we discuss naive Bayes classification
    techniques~\cite{boiman_defense_2008,mccann_local_2012} that play a large role in our baseline animal
    identification algorithms.

    \subsection{Vocabulary-based methods for category recognition}
        After vocabulary-based techniques demonstrated success in instance recognition, these techniques were
        quickly adapted and applied to category recognition~\cite{csurka_visual_2004}. Thus, there are many
        similarities --- and some differences --- in the techniques used to address these two problems. One
        difference is the size of the visual vocabulary. Instance recognition tends to require huge vocabularies
        ($\OnTheOrderOf{5}$ --- $\OnTheOrderOf{7}$ words) to achieve a fine sampling of descriptor
        space~\cite{nister_scalable_2006, philbin_object_2007}. In contrast, category recognition uses smaller
        vocabulary sizes ($\OnTheOrderOf{4}$ words) to more coarsely sample descriptor
        space~\cite{zhang_local_2006}. However, the vocabularies used in instance recognition have decreased in
        size with the advent of aggregated representations like VLAD and the Fisher
        vector~\cite{arandjelovic_all_2013, sanchez_compressed_2013}.

        A second difference is how similarity between images is computed. In instance recognition the similarity
        between bag-of-word vectors is computed using a weighted cosine similarity. However, in
        category-recognition intra-class variation requires more sophisticated similarity measurements. Here, image
        similarity is computed using SVMs with different either linear or non-linear kernels such as $\chi^2$,
        earth mover's distance, Hellinger, and Jensen-Shannon~\cite{zhang_local_2006, varma_learning_2007,
        vedaldi_efficient_2012}.

        A third difference is the way that spatial information is used. Instead of filtering correspondences using
        spatial verification, spatial information is incorporated into category recognition algorithms using
        spatial pyramids~\cite{grauman_pyramid_2005, lazebnik_beyond_2006}. A spatial pyramid sub-divides an image
        into a hierarchy of grids. Max pooling is often used to select only the strongest features in each spatial
        region~\cite{boureau_theoretical_2010, boureau_learning_2010}. Each section of the image is encoded using
        the vocabulary and images are scored based on matches in each region.

        \paragraph{Enhancements to category recognition}
        There are a wide variety of extensions and enhancements for image classification techniques based on
        bag-of-words, such as soft assignment of visual-words~\cite{liu_defense_2011} and vocabulary
        optimization~\cite{wang_locality_constrained_2010}. Numerous matching kernels --- both linear and
        non-linear --- have been developed such as kernel PCA, histogram intersection, and SVM square root
        bag-of-words vectors~\cite{vedaldi_multiple_2009, maji_classification_2008, perronnin_large_scale_2010}.

        % chktex-file 8
        Generalized coding schemes improve performance over a bag-of-words image encoding. Vocabularies can be seen
        as codebooks or dictionaries in coding-based image classification techniques such as sparse coding and
        locally constrained linear coding~\cite{jurie_creating_2005, yang_linear_2009, yang_supervised_2010,
        yang_efficient_2010, wang_locality_constrained_2010}. Many coding schemes learn both the centroids 
        and the function that quantizes a raw descriptor into a word~\cite{jurie_creating_2005, yang_linear_2009,
        yang_supervised_2010, yang_efficient_2010, wang_locality_constrained_2010, vedaldi_multiple_2009}.
        Techniques other than k-means are used to create vocabularies such as mean
        shift~\cite{jurie_creating_2005}, coordinate descent with the locally constrained linear code
        criterion~\cite{wang_locality_constrained_2010}, and random forests~\cite{perronnin_fisher_2007}. Fisher
        vectors with linear classifiers have been found to outperform non-linear bag-of-words based SVM classifiers
        by using an L1-based distance measure and careful L2 and power-law normalization of
        descriptors~\cite{perronnin_improving_2010, perronnin_large_scale_2010}.

    \subsection{\Naive{} Bayes classification}\label{sec:nbnn}  

        The \naive{} Bayes nearest neighbor (NBNN) classifier is a simple non-parametric algorithm for category
        recognition that does not quantize descriptor vectors~\cite{boiman_defense_2008}. Boiman responds to the
        dominance of complex non-linear category recognition algorithms in the field~\cite{varma_learning_2007,
        marszalek_learning_2007} by showing that simple techniques can compete with complex methods for category
        recognition. Boiman's paper also provides insight into the magnitude of information loss resulting from
        quantization.
          
        Previous to~\cite{boiman_defense_2008}, nearest neighbor classifiers had shown underwhelming accuracy in
        category recognition~\cite{varma_unifying_2004, lazebnik_beyond_2006, marszalek_learning_2007}. This was
        shown to be a result of using image-to-image distance. To remedy this, NBNN aggregates the information from
        multiple images by swapping the image-to-image distance for an image-to-class distance.

        In NBNN, features of each class are indexed for fast nearest neighbor search, typically with a
        kd-tree~\cite{bentley_multidimensional_1975}. For each feature, $\desc_i$, in a query image, the algorithm
        searches for the feature's nearest neighbors in each class, $\opname{NN}_C(\desc_i)$. The result of the
        algorithm is the class, $C$, that minimizes the image-to-class distance. In other words, the class of a
        query image is chosen by searching for the class that minimizes the total distance between each query
        descriptor and the nearest database descriptor in that class. This is expressed in the following equation:
        \begin{equation}
            C = \argmin{C} \sum_{i=1}^n ||\desc_i - \opname{NN}_C(\desc_i)||^2
        \end{equation}

        This formulation where each descriptor is assigned to only the single nearest neighbor has been shown to be
        a good approximation to the minimum image-to-class Kullback-Leibler divergence~\cite{boiman_defense_2008}
        --- a measure of how much information is lost when the query image is used to model the entire class.

    \subsection{Local \naive{} Bayes nearest neighbor}\label{sec:lnbnn}  

        Local \naive{} Bayes nearest neighbor (LNBNN) is an improved version of the NBNN algorithm in both accuracy
        and speed~\cite{mccann_local_2012}. In the original NBNN formulation a search is executed find each query
        descriptor's nearest neighbor in the database for each class separately. In contrast, the LNBNN
        modification searches all database descriptors simultaneously and ignores classes that do not return
        descriptor matches.
        
        Each descriptor $\desc_i$ in the query image searches for its nearest $K+1$ neighbors, %
        $\{\desc_1, \ldots, \desc_K, \desc_{K + 1}\}$ over all classes.
        The first $K$ neighbors are used as matches.
        The last neighbor is used as a normalizing term to weight the query descriptor's distinctiveness.
        Let $(\desc_i, \desc_j)$ be a matching descriptor pair, and let $C$ be the class of $\desc_j$.
        The score of each match is computed as the distance to the match subtracted from the distance to the
          normalizer.
        \begin{equation}
            s_{i, C} = \elltwo{\desc_j - \desc_K} - \elltwo{\desc_i - \desc_j}
        \end{equation}
        The score of a class $C$ is the sum of all the descriptor scores that match to it.

    \subsection{Discussion --- class recognition}

        Progress in category recognition is generally made using techniques that allow classes with high
        intra-class variance to have lower matching scores. This is of little value to an instance recognition
        application, therefore we do not investigate most of the techniques in this section. However, the
        LNBNN~\cite{mccann_local_2012} approach is interesting to us because it is a simple algorithm that does not
        suffer from quantization artifacts. NBNN and LNBNN~\cite{boiman_defense_2008,mccann_local_2012} never
        achieved state-of-the-art performance in image classification, however they have produced competitive
        results using simple techniques.

        The simplicity of the techniques allowed for the authors to gain insight into visual recognition. Due to
        its simplicity and the insight that quantization significantly reduces the descriptive power of SIFT
        features, we adopt LNBNN as the baseline algorithm for animal identification.


\section{Fine-grained recognition}\label{sec:fgr}  

    Fine-grained recognition is a problem more general than instance recognition, but more specific than category
    recognition~\cite{parkhi_cats_2012, berg_poof_2013, gavves_local_2014}. Given an object of a known category,
    such as a bird, the goal of fine-grained recognition is to sub-classify the object into a fine-grained category
    such as a blackbird or a raven~\cite{berg_how_2013}.

    Algorithms for fine-grained recognition typically start by localizing the object and its parts with a detection
    algorithm~\cite{dalal_histograms_2005} and parts-based models. Parts are segmented to remove background noise
    using algorithms like GrabCut~\cite{rother_grabcut_2004}. Classification is performed locally on aligned parts
    as well as globally on the entire body and aggregated to yield a final classification.

    Because fine-grained recognition lies on the same spectrum as instance recognition and category recognition it
    is not surprising that many of the same techniques --- like Fisher vectors --- are
    used~\cite{gosselin_revisiting_2014}.
    Recently convolutional models have been successfully applied to fine-grained
    recognition~\cite{catherine_wah_similarity_2014, branson_bird_2014, zongyuan_ge_modelling_2015,
    zhang_weakly_2015, xiao_application_2015}.

    \subsection{Discussion --- fine-grained recognition}
        The goal of fine-grained recognition is somewhat similar to animal identification. Fine-grained recognition
        localizes subtle information to distinguish between two similar species, whereas animal identification
        localizes subtle information to distinguish between two similar individuals. However, the domains of
        species and individuals are dissimilar enough that off the shelf techniques for fine-grained recognition
        would need to be adapted before identification could be performed. One interesting avenue of research would
        be to use a parts model~\cite{felzenszwalb_object_2010} as in~\cite{gavves_local_2014}, to align
        individuals before they are compared.


\section{Deep convolutional neural networks}\label{sec:dcnn}
    Convolutional networks have been around for over more than two decades~\cite{lecun_gradient_based_1998,
    fukushima_neocognitron_1988}. However, they did not receive major attention from computer vision researchers
    until 2012 when a deep convolutional neural network (DCNN)~\cite{krizhevsky_imagenet_2012} outperformed the
    best support vector machines (SVMs)~\cite{vapnik_statistical_1998} by over $10\percent$ in the ImageNet
    category recognition challenge~\cite{russakovsky_imagenet_2014}. Since then, many successful category
    recognition techniques based on DCNNs have been published~\cite{simonyan_very_2014, chatfield_efficient_2014,
    chatfield_return_2014, oquab_learning_2014, szegedy_going_2014, long_convnets_2014, he_spatial_2014,
    dean_fast_2013}. DCNNs have also been shown to produce excellent results when applied to other computer vision
    problems such as: %
    instance recognition~\cite{razavian_cnn_2014, razavian_baseline_2015, liu_learning_2015,
    held_deep_2015,arandjelovic_netvlad_2016,radenovic_cnn_2016}, %
    fine-grained recognition~\cite{branson_bird_2014, donahue_decaf_2013, catherine_wah_similarity_2014}, %
    detection~\cite{girshick_rich_2014, sermanet_overfeat_2013, li_wan_end_end_2015}, %
    face verification~\cite{huang_learning_2012, taigman_deepface_2014, sun_deep_2013}, %
    and learning similarity between feature patches~\cite{osendorfer_convolutional_2013, han_matchnet_2015,
    ng_exploiting_2015, zagoruyko_learning_2015, han_matchnet_2015}. The sudden success of deep nets has been
    attributed (1) a larger volume of available of training data, and (2) implementations using faster
    GPUs~\cite{krizhevsky_imagenet_2012}.
      
    Several techniques are employed to increase accuracy, reduce over-fitting,  and reduce training time.
    Data augmentation is used to artificially increase the amount of training
      data~\cite{ciresan_multi_column_2012, ciresan_high_performance_2011, simard_best_2003}.
    The dropout technique has been shown to reducing over-fitting~\cite{dahl_improving_2013,
      srivastava_dropout_2014}.
    At training time outputs of hidden units are randomly suppressed which forces the network to learn a more
      robust representation.
    It has been shown that dropout can be viewed as a form of model averaging~\cite{hinton_improving_2012}.
    Rectified linear units (ReLU) have been shown to be a faster alternative to the standard sigmoid activation
      functions~\cite{vinod_rectified_2010, dahl_improving_2013}.
    A ReLU is similar to a hinge function and simply outputs the signal of a unit if it is positive and outputs a
      zero otherwise.
    Leaky rectified linear units (LReLU) further improve network accuracy by including a ``leakiness'' term while
      maintaining the speed of ReLUs~\cite{maas_rectifier_2013}.
    While a ReLU strictly suppresses a feature activation if it is negative a LReLU returns a small negative
      signal (by multiplying by a constant) instead of zero.

    A deep neural network is constructed by stacking several layers of units (neurons) together. Data is used to
    initialize the activations of an input layer, and the information is forward propagated through the network.
    Weights are chosen to optimize a loss function --- \eg{} categorical cross-entropy error or triplet
    loss~\cite{schroff_facenet_2015} --- which is chosen to depend on the application. Optimization of the loss
    function is performed using back-propagation~\cite{rumelhart_learning_1986} --- typically using mini-batches
    and stochastic gradient descent with momentum~\cite{sutskever_importance_2013}. Traditionally each layer in a
    neural network is fully connected --- each pair of units between the previous layer and the current layer has
    its own edge weight ---  to the previous layer. However, in computer vision networks are constructed using
    convolutional layers.

    A DCNN connects the input layer to a stack of convolutional layers~\cite{krizhevsky_imagenet_2012}. A
    convolutional layer differs from a fully connected layer in that it is sparsely connected and that most of the
    edge weights between layers are shared~\cite{lecun_gradient_based_1998, fukushima_neocognitron_1988,
    serre_robust_2007}. Each convolutional layer is broken into several channels. Each channel is given its own
    weight matrix with a fixed width and height. This matrix of weights is convolved with the input layer to
    produce a feature activation map, one for each channel. Convolutional layers often use several pooling layers
    that aggregate information over a small area, reduce the size of the feature map, and increase robustness to
    transformations. Common pooling operations are max-pooling~\cite{serre_robust_2007, krizhevsky_imagenet_2012}
    and maxout~\cite{goodfellow_maxout_2013}. The convolutional layers may also be connected to a stack of fully
    connected layers. In this case, hierarchies of feature maps are built in the low level convolutional layers,
    and then fully connected layers learn decision boundaries between these
    features~\cite{zeiler_visualizing_2014}.

    Because of weight sharing, convolutional networks must learn significantly fewer parameters than fully
      connected networks.
    This allows convolutional networks to be trained much faster.
    Fewer weights also acts as a form of regularization for the network.
    Intuitively learned convolutional filters are similar to Gabor filters~\cite{gabor_theory_1946}, which are a
      naturally suited for extracting features from images.
    Even without learning weights, convolutions can be used to extract powerful features for
      matching~\cite{revaud_deep_2015}.
    The popular SIFT and HoG features~\cite{mahendran_understanding_2014} can even be implemented as
      convolutional networks.
    Despite the lack of hard theoretical insight into the inner workings of these networks, their empirical
      performance cannot be denied.

  \subsection{Discussion --- deep convolutional neural networks}\label{subsec:dcnndiscuss}
        Because of the astounding success of convolutional networks in almost every area in computer vision, we have
        investigated their use in animal identification. Specifically, we have investigated two approaches.

        The first approach used deep convolutional feature descriptors as a replacement for the
        SIFT~\cite{lowe_distinctive_2004} descriptor following the patch-based scheme
        in~\cite{zagoruyko_learning_2015}. The basic idea is to have two patches fed through the same
        (Siamese)~\cite{chopra_learning_2005} architecture and then compare their resulting encodings. This
        comparison can be as simple as Euclidean distance, or as complex as a learned distance measure. Training
        can be performed on pairs of patches, labeled as correct or incorrect, using the discriminative loss
        function~\cite{lecun_loss_2005}. Unfortunately, due to issues with the quality and quantity of our training
        data our convolutional replacements for the SIFT descriptor have not been successful.

        The second approach aimed to use Siamese networks to directly compare two images of an animal to
          determine if they were the same or different, similar to the method used in
          DeepFace~\cite{taigman_deepface_2014} for face verification.
        However, without the large training datasets and specialized alignment procedures used in DeepFace, we
          were unable to produce promising results.
        
        Due to these issues, this \thesis{} does not further pursue techniques based on DCNNs.
        We include this discussion to note the potential of deep learning applied to animal identification and to
          strongly suggest further investigation of these techniques in the future research.
        Of particular interest for future research is the matching technique presented
          in~\cite{rocco_convolutional_2017}.
        This method is particularly interesting because it learns to match and align images by mimicking a
          classic computer vision pipeline while using only synthetic training data.
        This may be able to overcome the issues mentioned above,
        however further investigation is needed.

% +--- CHAPTER --- 
\begin{comment}
    ./texfix.py --fpaths chapter3-matching.tex --outline --asmarkdown --numlines=999 -w
    ./texfix.py --fpaths chapter3-matching.tex --outline --asmarkdown --numlines=999 -w
    ./texfix.py --fpaths chapter3-matching.tex --reformat 
    # http://jaxedit.com/mark/
\end{comment}



\chapter{Identification using a ranking algorithm}\label{chap:ranking}

    This chapter addresses the problem of computer-assisted animal
      identification, where an algorithm suggests likely possibilities, but a
      human reviewer always makes the final decision.
    Given, a single annotation depicting an unknown animal and a database of
      previously identified annotations, the task is to determine a ranking of
      database individuals most likely to match the unknown animal.
    A manual reviewer determines which --- if any --- of the top ranked
      results are correct.

    An \glossterm{annotation} is a rectangular region  of interest around a
      specific animal within an image.
    Each known annotation is associated with a \name{} label denoting its
      individual identity.
    A \glossterm{\name} refers to a group of annotations known to be the same
      individual.
    The identification process assigns a \name{} label to an unknown
      annotation either as
    (1) the \name{} label of a matched database annotation or
    (2) a new \name{} label if no matches are found.

    %A static context is chosen for this chapter in order to introduce
    %  and determine the effectiveness of the algorithm that identifies a
    %  query by ranking the \names{} in the database.
    %In the static context, only a single query annotation is used to
    %  perform identification, all \name{} labels in the database are
    %  assumed to be correct, and annotations are not added to or removed
    %  from the database.
    %This is done to determine what properties of query annotations,
    %  configurations of the database, and parameters of the
    %  identification algorithm have the highest impact on identification
    %  accuracy.
    %In~\cref{chap:application}, the identification algorithm is
    %  extended for use in a dynamic context.
    %In the dynamic context, annotations are added and removed from the
    %  database, multiple annotations are used to perform identification,
    %  and the database is assumed to contain errors.

    The ranking algorithm is based on the feature correspondences between a
      query annotation and a set of database annotations.
    In each annotation a set of patch-based features is detected at keypoint
      locations.
    Then the visual appearance of each patch is described using
      SIFT~\cite{lowe_distinctive_2004}.
    A nearest neighbor algorithm establishes a set of feature correspondences
      between query and annotations in database.
    A scoring mechanism based on Local \Naive{} Bayes Nearest Neighbor
      (LNBNN)~\cite{mccann_local_2012} produces a score for each feature
      correspondence.
    These scores are then aggregated into a single score for each \name{} in
      the database, producing a ranked list of \names{}.
    Identification is performed by applying a classifier (decision algorithm)
      to the scores in this ranked list.
    If the top ranked \name{} has a ``high'' score it is likely to be the same
      individual depicted in the query annotation.
    If the top ranked \name{} has a ``low'' score it is likely that the query
      individual is not depicted in any database annotation.
    An example ranked list returned by the algorithm is illustrated
      in~\cref{fig:rankedmatches}.
    In the baseline algorithm this identification decision is left to a user.

    The outline of this chapter is as follows:
    \Cref{sec:annotrepr} discusses the initial processing of an annotation
      which involves image normalization, feature extraction, and feature
      weighting.
    \Cref{sec:baselineranking,sec:sver} describes the baseline ranking and
      scoring algorithm.
    The first of these sections focuses on establishing feature
      correspondences, and the second focuses on verifying the correspondences.
    \Cref{sec:exempselect} describes the process for selecting \exemplars{}.
    \Cref{sec:experiments} provides an experimental evaluation of the ranking
      algorithm.
    %These experiments inform the direction we take in our proposal to extend
    %  this algorithm.
    \Cref{sec:staticsum} summarizes this chapter.

    \rankedmatches{}

    \begin{comment}
    ./texfix.py --fpaths main.tex --outline --asmarkdown --numlines=999 -w --ignoreinputstartswith=def,Crall,header,colordef,figdef
    ./texfix.py --fpaths sec-3-1-annotrepr.tex --outline --asmarkdown --numlines=999  -w
\end{comment}


\section{Annotation representation}\label{sec:annotrepr}
  
    For each annotation in the database we
    (1) normalize the image geometry and intensity,
    (2) compute features,
    (3) and weight the features.
    % Chip Extract
    Image normalization rotates, crops, and resizes an annotation from
      its image.
    This helps to remove background clutter and roughly align the
      annotations in pose and scale.
    The extracted and normalized region is referred to as a
      \glossterm{chip}.
    % Feat Detect
    Then, a set of features ---  a keypoint and descriptor pair --- is
      computed.
    Keypoints are detected at multiple locations and scales within the
      chip, and a texture based descriptor vector is extracted at each
      keypoint.
    % Featweight
    Finally, each feature is assigned a probabilistic weight using a
      foregroundness classifier.
    This helps remove the influence of background features.

    \subsection{Chip extraction}

        % Bounding box + orientation
        Each annotation has a bounding box and an orientation specified
          in a previous detection step.
        For zebras and giraffes, the orientation of the chip is chosen
          such that the top of the bounding box is roughly parallel to
          the back of the animal.

        A chip is a extracted by jointly rotating, scaling, and
          cropping an annotation's parent image using Lanczos
          resampling~\cite{lanczos_applied_1988}.
        The scaling resizes the image such that the cropped chip has
          approximately $450^2$ pixels and it maintains the aspect ratio
          of the bounding box.
        If specified in the pipeline configuration, adaptive histogram
          equalization~\cite{pizer_adaptive_1987} is applied to the chip,
          however this is not used in the experimental evaluation
          presented later in this chapter.

    \subsection{Keypoint detection and description}

        Keypoints are detected within each annotation's chip using a
          modified implementation of the Hessian detector described
          in~\cite{perdoch_efficient_2009} and reviewed
          in~\cref{sec:featuredetect}.
        This produces a set of elliptical features localized in space,
          scale, shape, and orientation.
        Each keypoint is described using the
          SIFT~\cite{lowe_distinctive_2004} descriptor that was reviewed
          in~\cref{sec:featuredescribe}.
        The resulting keypoint-descriptor pairs are an annotation's
          features.
        Further details about the keypoint structure are given
          in~\cref{sec:kpstructure}.

        We choose a baseline feature detection algorithm that produces
          affine invariant keypoints with the gravity vector.
        Affine invariant (\ie{} shape adapted) keypoints detect
          elliptical patches instead of circular ones.
        We choose affine invariant keypoints because the animals we
          identify will be seen from many different viewpoints.
        Because all chips have been rotated into an upright position,
          we assign all keypoints a constant orientation --- this is the
          gravity vector assumption~\cite{perdoch_efficient_2009}.
        However, these baseline settings may not be appropriate for all
          species.

        It is important to select the appropriate level of invariance
          for each species we identify.
        % Shape 
        Our experiments in~\cref{sub:exptinvar} vary several parameters
          related to invariance in keypoint detection.
        To determine if affine invariance is appropriate for animal
          identification we experiment with both circular and elliptical
          keypoints.
        % Orientation 
        We also experiment with different levels of orientation
          invariance.
        The gravity vector assumption holds in the case of rigid
          non-poseable objects (\eg{} buildings), if the image is
          upright.
        Clearly, for highly poseable animals, this assumption is more
          questionable.
        However, full rotation invariance (using dominant gradient
          orientations) has intuitive problems.
        Patterns (like ``V'' and ``$\Lambda$'') that might contribute
          distinguishing evidence that two annotations match, would
          always appear identical under full rotation invariance.
        Ideally orientation selection would be made based on the pose
          of the animal.

        We introduce a simple orientation heuristic to help match keypoints
          from the same animal in the presence of small pose variations.
        Instead of extracting a single keypoint in the direction of gravity or
          multiple keypoints in the directions of the dominant gradient
          orientation we extract 3 descriptors at every keypoint:
        one in the direction of gravity, and the other two offset at
          $\pm15\degrees$.
        This provides a middle ground between rotation invariance and the
          gravity vector.
        Using this heuristic, it will be more likely to extract similar
          descriptors from two annotations of the same animal seen from slightly
          different poses.

        %In our experiments we will investigate replacing the SIFT descriptor
        %  features extracted from a deep convolutional neural network.

    \subsection{Feature weighting}
     
        In animal identification, there will often be many annotations
          containing the same background.
        Photographers may take many photos in a single place and camera
          traps will contribute many images with the same background.
        Without accurate background masking, regions of an annotation
          from different images containing the same background may
          strongly match and outscore matches to correct individuals.
        An example illustrating two different individuals seen in front
          of the same distinctive background is shown
          in~\cref{fig:SceneryMatch}.
        To account for this, each feature is given a weight based on
          its probability of belonging to the foreground --- its
          ``foregroundness''.
        This weight is used indicate the importance of a feature in
          scoring and spatial verification.

        Foregroundness is derived from a species detection algorithm
          developed by Jason Parham~\cite{parham_photographic_2015}.
        The input to the species detection algorithm is the
          annotation's chip, and the output is an intensity image.
        Each pixel in the intensity image represents the likelihood
          that it is part of a foreground object.

        A single feature's foregroundness weight is computed for each
          keypoint in an annotation as follows:
        The region around the keypoint in the intensity image is warped
          into a normalized reference frame.
        Each pixel in the normalized intensity patch is weighted using
          a Gaussian falloff based on the pixel's distance from the
          center of the patch.
        The sum of these weighted intensities is the feature's
          foregroundness weight.
        The steps of feature weight computation are illustrated
          in~\cref{fig:genfeatweight}.

        \SceneryMatch{}

        \genfeatweight{}

    \subsection{Keypoint structure overview}\label{sec:kpstructure}
        %Before we discuss the computation of the we review the structure of
        %  a keypoint.
        The keypoint of a feature is represented as:
        $\kp\tighteq(\pt, \vmat, \ori)$, %
        The vector $\pt\tighteq\ptcolvec$ is the feature's
          $xy$-location.
        The scalar $\theta$ is the keypoint orientation.
        The lower triangular matrix $\vmat\tighteq\VMatII$ encodes the
          keypoint's shape and scale.
        This matrix skews and scales a keypoint's elliptical shape into
          a unit circle.
        A keypoint is circular when $a\tighteq{}d$ and $c\tighteq0$.
        %If $c\tighteq0$, there is no skew and if $a\tighteq{}d$ the keypoint
        %  is circular.
        The keypoint scale is related to the determinant of this matrix
          and can be extracted as: %
        $\sigma = \frac{1}{\sqrt{\detfn{\vmat}}} =
          \frac{1}{\sqrt{ad}}$.
        All of this information can be encoded in a single affine
          matrix.

        \paragraph{Encoding keypoint parameters in an affine matrix}
        It will be useful to construct two transformations that encode
          all keypoint information in a single matrix.
        The first, $\rvmat$, maps a keypoint in an annotation into a
          normalized reference frame --- the unit circle.
        The second transformation, $\inv{\rvmat}$ is the inverse, which
          warps the normalized reference frame back onto the keypoint.
        To construct $\rvmat$, the keypoint is centered at the origin
          $(0, 0)$ using translation matrix, $\mat{T}$.
        Then $\vmat$ is used to skew and scale the keypoint into a unit
          circle.
        Finally, the keypoint's orientation is normalized by rotating
          $-\theta$ radians using a rotation matrix $\mat{R}$.
        \begin{equation}\label{eqn:RVTConstruct}
          %\rvmat=\paren{\invrotMatIII{\paren{-\ori}} \VMatIII \transMATIII{-x}{-y}}
            \rvmat=\mat{R} \vmat \mat{T} = \rotBigMatIII{\paren{-\ori}} \VBigMatIII \transBigMatIII{-x}{-y}
        \end{equation}
        The construction of $\inv{\rvmat}$ is performed similarly.
        % see vtool.keypoint.get_invVR_mats_oris
        \begin{equation}\label{eqn:invTVRConstruct}
             \inv{\rvmat} = \inv{\mat{T}} \inv{\vmat} \inv{\mat{R}} = 
             \transBigMatIII{x}{y}
             \BIGMAT{
                \frac{1}{a}     & 0               & 0\\
                -\frac{c}{a d}  & \frac{1}{d}     & 0\\
                0               & 0               & 1
                }
             \rotBigMatIII{\paren{\ori}}
        \end{equation}

        \paragraph{Extracting keypoint parameters from an affine matrix}
        During the spatial verification step, described
          in~\cref{sec:sver}, keypoints are warped from one image into
          the space of another.
        It will be useful to extract the keypoint parameters from an
          arbitrary keypoint matrix.
        This will allows us to directly compare properties of
          corresponding right side of~\cref{eqn:invTVRConstruct}.
        Given an arbitrary affine matrix $\inv{\rvmat}$ representing
          keypoint $\kp$, we show how the individual parameters $(\pt,
          \scale, \ori)$ can be extracted.
        First consider the components of $\inv{\rvmat}$ by simplifying
          the right side of~\cref{eqn:invTVRConstruct}.
        \begin{equation}\label{eqn:ArbInvRVTMat}
            \inv{\rvmat} = 
            \BIGMAT{
            e & f & x\\
            g & h & y\\
            0 & 0 & 1
            } = 
            \BIGMAT{
            \frac{1}{a} \cos{(\theta )}                                 & -\frac{1}{a} \sin{(\theta )}                                & x\\
            \frac{1}{d} \sin{(\theta )} - \frac{c}{a d} \cos{(\theta )} & \frac{1}{d} \cos{(\theta )} + \frac{c}{a d} \sin{(\theta )} & y\\
            0                                                           & 0                                                           & 1
            }
        \end{equation}
        %%---------
        The position, scale, and orientation can be extracted from an
          arbitrary affine keypoint shape matrix $\invvrmat$ as follows:
        \begin{equation}\label{eqn:affinewarp}
            \begin{aligned}
                \pt     &= \VEC{x\\y} \\
                \scale  &= \sqrt{\detfn{\invvrmat}}\\
                \ori    &= \modfn{\paren{-\atantwo{f, e}}}{\TAU}
            \end{aligned}
        \end{equation}

    %\newcommand{\annotscoreop}{\opname{annot\_score}}
%\newcommand{\namescoreop}{\opname{name\_score}}
\newcommand{\annotscoreop}{\opname{K_{\tt annot}}}
\newcommand{\amechscoreop}{\opname{K_{\csum}}}
\newcommand{\fmechscoreop}{\opname{K_{\nsum}}}

\section{Matching against a database of individual animals}\label{sec:baselineranking}

    To identify a query annotation, it is matched against a database of known \names{}.
    \Aan{\name{}} is a set of annotations known to depict the same animal.
    The basic matching pipeline can be summarized in \three{} steps:
    establish feature correspondences \rpipe{} %
    score feature correspondences \rpipe{} %
    aggregate feature correspondence scores across the \names{}.
    Correspondences between a query annotation's features and \emph{all} database annotation features are
      established using an approximate nearest neighbor algorithm.
    This step also establishes a normalizing feature which is used to measure the distinctiveness of a query
      feature.
    Each feature correspondence is scored based on the feature weights established in the previous section and a
      measure of the distinctiveness of the query feature.
    The feature correspondence scores are then aggregated into a \glossterm{\namescore{}} for each \name{} in the
      database.
    The \namescores{} induce a ranking on \names{} in the database where database \names{} with higher ranks are
      more likely to be correct matches.

   \subsection{Establishing initial feature correspondence}\label{sub:featmatch}

        \paragraph{Offline Indexing}
            Before feature correspondences can be established, an offline algorithm indexes descriptors from all
              database annotations for fast approximate nearest neighbor search.
            All database descriptor vectors are stacked into a single database array of vectors, %
            $\AnyDB$, % should this be removed?
              %
            and these descriptors are indexed by an inverted file.
            The inverted file maps each descriptor in the stacked array back to its original annotation and
              feature.
            This database array is indexed for nearest neighbor search using a forest of
              kd-trees~\cite{silpa_anan_optimised_2008} using the FLANN library~\cite{muja_fast_2009}, which were
              reviewed in~\cref{sec:ann}.
            This allows for the efficient implementation of a \codeobj{neighbor index function}  %
            $\NN(\AnyDB, \desc, \K)$  %
            %$\NN(\desc, \K)$  %
            that returns the indices in $\AnyDB$ of the $\K$ approximate nearest neighbors of a query feature's
              descriptor $\desc$.
            %that returns the database indices of the $\K$ approximate
            %  nearest neighbors of a query feature's descriptor
            %$\desc$.

        \paragraph{Approximate Nearest Neighbor Search}

            Matching begins by establishing multiple feature correspondences between each query feature and
              several visually similar database features.
            For each query descriptor vector $\desc_i \in \X$ the $\K + \Knorm$ approximate nearest neighbors are
              found using the \coderef{neighbor index function}.
            These neighbors sorted by ascending distance are:
            \begin{equation}
                \NN(\AnyDB, \desc_i, \K + \Knorm) \eqv \dotarrIII{j}{\K}{\K + \Knorm}
                %\NN(\desc_i, \K + \Knorm) \eqv \dotarrIII{j}{\K}{\K + \Knorm}
            \end{equation}
            The $\K$ nearest neighbors, $\dotsubarr{\desc}{{j_1}}{{j_\K}}$, are the initial feature
              correspondences to the $i$\th{} query feature.
            The remaining $\Knorm$ neighbors, $\dotsubarr{\desc}{{j_{\K + 1}}}{j_{\Knorm}}$, are candidate
              normalizers for use in LNBNN scoring.

        \paragraph{Normalizer selection}
            A single descriptor $\descnorm_{i}$ is selected from the $\Knorm{}$ candidate normalizers and used in
              computing the LNBNN score for all (up to $\K$) of the $i$\th{} query descriptor's correspondences in
              the database.
            The purpose of a normalizing descriptor is to estimate the local density of descriptor space, which
              can be interpreted as a measure of the query descriptor's distinctiveness \wrt{} the database.
            The normalizing descriptor is chosen as the most visually similar descriptor to the query that is not
              a correct match.
            In other words, the query descriptor's normalizer should be from an individual different from the
              query.
            The intuition is there will not be any features in the database that are close to distinctive
              features in the query except for the features that belong to the correct match.

            The selection process described in the original formulation of LNBNN is to simply choose the $\K +
              1$\th{} nearest neighbor, which amounts to setting $\Knorm=1$.
            The authors of LNBNN find that there is no benefit to using a higher value of
              $\Knorm$~\cite{mccann_local_2012}.
            However, this does not account for the case when the $\K + 1$\th{} nearest neighbor belongs to the
              same class as one of the nearest $\K$ neighbors.
            Therefore, we employ a slightly different selection process.
            To motivate our selection process, consider the case when there are more than $\K$ images of the same
              individual from the same viewpoint in the database and a distinctive feature from a new annotation of
              that individual is being scored.
            In this case $\K$ correspondences will be correctly established a distinctive the query feature and
              $\K$ database features.
            However, if the normalizer is chosen as the $\K + 1$ neighbor, then these correspondences will be
              inappropriately downweighted.
              
            % probably need to reword.
            Consider the example in~\cref{fig:knorm}.
            In this case there are two examples \Cref{sub:knorma,sub:knormb} of the query images in the database.
            The figure shows the case where $\K=3$ and $\Knorm=1$.
            Even though there is an incorrect match, the LNBNN scores of the correct matches are an order of
              magnitude higher than the score for the incorrect match.
            Now, consider the case where the number of correct matches in the database is greater than $\K$ by
              setting $\K=1$.
            In this case the normalizing descriptor is the ``same'' feature as the query feature and the nearest
              match drops from $0.066$ to $0.007$.

            \knorm{}

            To avoid this case, a normalizing feature is carefully chosen to reduce the possibility that it
              belongs to a potentially correct match.
            More formally, the normalizing descriptor is chosen to be the descriptor with the smallest distance
              to the query descriptor that is not from the same \name{} as any of the chosen correspondences.
            Let $\nid_j$ be the \name{} associated with the annotation containing descriptor $\desc_j$.
            Let %
            %$\multiset{N}_i \eqv \curly{\nid_j \quad \forall j \in \NN(\desc_i, \K + \Knorm)}$
            $\multiset{N}_i \eqv \curly{\nid_j \where j \in \NN(\AnyDB, \desc_i, \K + \Knorm)}$
              %\NN(\desc_i, \K + \Knorm)}$
              %
            be the set of \names{} matched by the $i$\th{} query feature.
            The descriptor that normalizes all matches of query descriptor $\desc_i$ is:
              \begin{equation}
                  \descnorm_{i} \eqv 
                  \argmin{\desc_j \in \dotsubarr{\desc}{{j_{\K + 1}}}{j_{\Knorm}}}
                  \elltwosqrd{\desc_j - \desc_i} \where \nid_j \notin \multiset{N}_i
              \end{equation}

    \subsection{Feature correspondence scoring}
        Each feature correspondence is given a score representing how likely it is to be a correct match.
        While the L2-distance between query and database descriptors is useful ranking feature correspondences
          based on visual similarity, the distinctiveness of the match is more useful for ranking the query
          annotation's similarity to a database annotation~\cite{lowe_distinctive_2004,
          arandjelovic_dislocation_2015, mccann_local_2012}.
        However, highly distinctive matches from other objects --- like background matches --- do not provide
          relevant information about a query annotation's identity and should not contribute to the final score.
        Therefore, each feature correspondence is scored using a mechanism that combines both distinctiveness and
          likelihood that the object belongs to the foreground.
        For each feature correspondence $m = (i, j)$ with query descriptor $\desc_i$ and matching database
          descriptor $\desc_j$, several scores are computed which are then combined into a single feature
          correspondence score $s_{i,j}$.

        \paragraph{LNBNN score}\label{sec:lnbnnscore}

            Using the normalizing feature, $\descnorm_{i}$, LNBNN compares a query feature's similarity to the
              match and query feature's similarity to the normalizer.
            This serves as an estimate of local feature density and measures the distinctiveness of the feature
              correspondence.
            A match is distinctive when the query-to-match distance is much smaller than the query-to-normalizer
              distance, \ie{} the local density of descriptor space around the query is sparse.
            The LNBNN score of a feature match is computed as follows:
            \begin{equation}\label{eqn:lnbnn}
                \fs_{\LNBNN} \eqv \frac{\elltwo{\desc_i - \descnorm_{i}} - \elltwo{\desc_i - \desc_j}}{Z}
            \end{equation}
            All descriptors used in this calculation are L2-normalized to unit length --- \ie{} to sit on the
              surface of a unit hypersphere.
            The $Z$ term normalizes the score to ensure that it is in the range $\rangeinin{0, 1}$.
            If descriptor vectors have only non-negative components (as in the case of
              SIFT~\cite{lowe_distinctive_2004}) then the maximum distance between any two L2-normalized
              descriptors is $Z\tighteq\sqrt{2}$.
            If descriptors vectors have negative components (like those that might extracted from a deep
              convolutional neural network~\cite{zagoruyko_learning_2015}) then the maximum distance between is
              $Z\tighteq2$.

        \paragraph{Foregroundness score}
            To reduce the influence of background matches, each feature correspondence is assigned a score based
              on the foregroundness of both the query and database features.
            The geometric mean of the foregroundness of query feature, $w_i$, and database feature, $w_j$, drives
              the score to $0$ if either is certain to be background.
            % show python -m ibeis --db testdb3 --query 325 -y
            \begin{equation}
                \fs_{{\tt fg}} \eqv \sqrt{w_i w_j}
            \end{equation}

        \paragraph{Final feature correspondence score}
            The final score of the correspondence $(i, j)$ captures both the distinctiveness of the match as well
              as the likelihood that the match is part of the foreground.
              \begin{equation}\label{eqn:featscore}
                  \fs_{i,j} \eqv \fs_{{\tt fg}} \fs_{\LNBNN} 
              \end{equation}

    \subsection{Feature score aggregation}\label{subsec:namescore}

        So far, each feature in a query annotation has been matched to several features in the database and a
          score has been assigned to each of these correspondences based on its distinctiveness and foregroundness.
        The next step in the identification process is to aggregate the scores from these patch-based
          correspondences into a single \glossterm{\namescore} for each \name{} in the database.
        Note that this \name-based definition of scoring is a key difference between animal identification and
          image retrieval, where a score is assigned to each image in the database.
        In animal identification the analogous concept is an \glossterm{\annotscore} --- a score assigned to each
          annotation in the database~\cite{philbin_object_2007}.
        This distinction between a score from a query annotation to a database annotation is important because
          the goal of the application is to classify a new query annotation as either a known \name{} or as a new
          \name{}, not to determine which annotations are most similar.

        This subsection presents two mechanisms to compute \namescores{}.
        The first mechanism is \csumprefix{} and computes a \namescore{} in two steps.
        This mechanism aggregates feature correspondences scores into an \annotscore{} for each annotation in the
          database.
        Then the \annotscores{} are aggregated into a score for each \name{} in the database.
        The second mechanism is \nsumprefix{}.
        This mechanism aggregates feature correspondences scores matching multiple database annotations directly
          into a \namescore{}.
        These mechanisms are respectively similar to the image-to-image distance and the image-to-class distance
          described in~\cite{boiman_defense_2008}.

        \paragraph{The set of all feature correspondences}
        All scoring mechanisms presented in this subsection are based on aggregating scores from features
          correspondences.
        The set of all feature correspondences for a query annotation $\X$ is expressed as:
        \begin{equation}
            %\Matches \eqv \{(i, j) \where \desc_i \in \X \AND j \in \NN(\desc_i, \K)\}
            \Matches \eqv \{(i, j) \where \desc_i \in \X \AND j \in \NN(\AnyDB, \desc_i, \K)\}
        \end{equation}

        % FIXME: SAME AS IMAGE-TO-IMAGE
        \paragraph{Annotation scoring}
            An \annotscore{} is a measure of similarity between two annotations.
            An \annotscore{} between a query annotation and a database annotation is defined as the sum of the
              feature correspondence scores matching to the features from that database annotation.
            %However, the \annotscore{} will allow us to compare our
            %  techniques against other instance recognition techniques.
            Let $\daid_j$ be the database annotation containing feature $j$.
            Let
            %
            $\Matches_{\daid} \eqv \curly{(i, j) \in \Matches \where \daid_j = \daid}$
            %
            denote all of the correspondences to a particular database annotation.
            The \annotscore{} between the query annotation $\qaid$ and database annotation $\daid$ is:
            \begin{equation}
                \annotscoreop(\qaid, \daid) \eqv \sum_{(i, j) \in \Matches_{\daid}} \fs_{i, j}
            \end{equation}

        % FIXME: SAME AS IMAGE-TO-CLASS
        \paragraph{Name scoring (1) --- \csumprefix{}} %

            The \cscoring{} mechanism aggregates \annotscores{} into \namescores{} by taking as the score highest
              scoring annotation for each \name{}.
            In our experiments we refer to this version of \namescoring{} as \csum{}.
            Let $\nid$ be the set of database annotations with the same \name{}.
            The \cscore{} between a query annotation and a database \name{} is the maximum over all annotations
              scores in that \name{}:
            \begin{equation}
                \amechscoreop(\qaid, \nid) 
                \eqv
                \max_{\daid \in \nid}
                \paren{
                    \annotscoreop\paren{\qaid, \daid}
                }
                %\opname{annot\_score}(\qaid, \daid) \eqv \sum_{m_a \in \Matches_{\daid}} \fs_{i, j}
            \end{equation}

         \paragraph{Name scoring (2) --- \nsumprefix{}} %
            The \cscoring{} mechanism accounts for the fact that animals will be seen multiple times, but it does
              not take advantage of complementary information available when \aan{\name{}} has multiple
              annotations.
            The following aggregation mechanism combines scores on a feature level to correct for this.
            It allows each query feature at a specific location to vote for a given \name{} at most once.
            Thus, when a query feature (or multiple query features at the same location) corresponds to database
              features from multiple views of the same animal, only the best correspondence for that feature will
              contribute to the score.
            In our experiments we refer to this version of \namescoring{} as \nsum{}.

            The first step of computing \aan{\namescore{}} for a specific \name{} is grouping the feature
              correspondences.
            Two feature correspondences are in the same group if the query features have the same location and
              the database features belong to the same \name{}.
            The next step is to choose the highest scoring correspondence within each group.
            The sum of the chosen scores is the score for \aan{\name{}}.
            This procedure is illustrated in~\cref{fig:namematch}.

            \newcommand{\MatchesGroup}{\Matches^{G}}

            Formally, consider two feature correspondences $\mI\tighteq(\iI, \jI)$ and $\mII\tighteq(\iII,
              \jII)$.
            Let $\pt_{\iI}$ and $\pt_{\iII}$ be the $xy$-location of the query feature in a correspondence.
            Let $\nid_{\jI}$ and $\nid_{\jII}$ be the \name{} of the database annotations containing the matched
              features.
            The group that contains feature correspondence $\mI$ is defined as:
            \begin{equation}
                \MatchesGroup_{\mI} \eqv \curly{\mII \in \Matches  \where
                \paren{
                    \paren{\pt_{\iI} \eq \pt_{\iII}} \AND 
                    \paren{\nid_{\jI} \eq \nid_{\jII}}
                }
            }
            \end{equation}
            The correspondence with the highest score in each connected component is flagged as chosen.
            Ties are broken arbitrarily.
            \begin{equation}
                \ischosen(\mI) \eqv 
                \bincase{
                \paren{
                    \fs_{\mI} > \fs_{\mII} 
                    \quad \forall \mII  \in \MatchesGroup_{\mI}
                } 
                \OR
                \card{\MatchesGroup_{\mI}} \eq 1
                }
            \end{equation}

            Let $\Matches_{\nid} \eqv \{(i, j) \in \Matches \where
              \nid_j = \nid\}$ denote all of the correspondences to a particular
              \name{}.
            The \nscore{} of \aan{\name{}} is:
            \begin{equation}
                \fmechscoreop(\qaid, \nid) 
                \eqv 
                \sum_{m \in \Matches_{\nid}} \ischosen(m) \; \fs_m
            \end{equation}

            \namematch{}

    
\section{Spatial verification}\label{sec:sver}

    The basic matching algorithm treats each annotation as an orderless set of feature descriptors (with a small
      exception in name scoring, which has used a small amount of spatial information).
    This means that many of the initial feature correspondences will be spatially inconsistent.
    Spatial verification removes these spatially inconsistent feature correspondences.
    Determining which features are inconsistent is done by first estimating an affine transform between the two
      annotations.
    Then a projective transform is estimated using the inliers to the affine transform.
    Finally any correspondences that do not agree with the projective transform transformation are
      removed~\cite{fischler_random_1981, philbin_object_2007}.
    We have reviewed related work in spatial verification in~\cref{subsec:sverreview}.

    %In our problem, the animals are seen in a wide variety of poses,
    %  and projective transforms may not always be sufficient to capture
    %  all correctly corresponding features.
    %Yet, without strong spatial constraints on matching, many
    %  background features will be spatially verified.
    %For now, we proceed with standard techniques for spatial
    %  verification and evaluate if more sophisticated methods are needed.

    \subsection{Shortlist selection}
        Standard methods for spatial verification are defined on the feature correspondences between two
          annotations (images).
        Normally, a shortlist of the top ranked annotations are passed onto spatial verification.
        However, in our application we rank \names{}, which may have multiple annotations.
        In our baseline approach we simply apply spatial verification to the top $N_{\tt{nameSL}}=40$ \names{}
          and the top $N_{\tt{annotSL}}=3$ annotations within those \names{}.

    \subsection{Affine hypothesis estimation}
        Here, we will compute an affine transformation that will remove a majority of the spatially inconsistent
          feature correspondences.
        Instead of using random sampling of the feature correspondences as in the original RANSAC
          algorithm~\cite{hartley_multiple_2003}, we estimate affine hypotheses using a deterministic method
          similar to~\cite{philbin_object_2007, chum_homography_2012}.
        Given a set of matching features between annotation $\annotI$ and $\annotII$, the shape, scale,
          orientation, and position of each pair of matching keypoints are used to estimate a hypothesis affine
          transformation.
        Each hypothesis transformation warps keypoints from annotation $\annotI$ into the space of $\annotII$.
        Inliers are estimated by using the error in position, scale, and orientation between each warped keypoint
          and its correspondence.
        The transformation with the most inliers determines the final affine transform.

        % vmat = V here
        % V maps from ellipse to u-circle
        \newcommand{\AffMat}{\mat{A}}
        \newcommand{\HypothSet}{\set{A}}
        \newcommand{\AffMatij}{\mat{A}_{i, j}}
        \newcommand{\HypothAffMat}{\hat{\mat{A}}}

        \subsubsection{Enumeration of affine hypotheses}
            %The deterministic set of hypothesis transformations mapping from
            %  query annotation $\annotI$ to database annotation $\annotII$ is
            %  computed for each feature correspondence in a match from  to an
            %  annotation.
            Let $\Matches_{\annotII}$ be the set of all correspondences between features from query annotation
              $\annotI$ to database annotation $\annotII$.
            For each feature correspondence $(i, j) \in \Matches_{\annotII}$, we construct a hypothesis
              transformation, $\AffMatij$ using the matrices $\rvmat_{i}$ and $\inv{\rvmat_{j}}$, which where
              defined in~\cref{eqn:RVTConstruct} and~\cref{eqn:invTVRConstruct}.
            The first transformation $\rvmat_{i}$, warps points from $\annotI$-space into a normalized reference
              frame.
            Then the second transformation, $\inv{\rvmat_{j}}$, warps points in the normalized reference frame
              into $\annotII$-space.
            Formally, the hypothesis transformation is defined as $\AffMatij \eqv \inv{\rvmat_{j}}\rvmat_{i}$,
              and the set of hypothesis transformations is:
            \begin{equation}
                \HypothSet \eqv \curly{ \AffMatij \where (i, j) \in \Matches_{\annotII} }
            \end{equation}

        \subsubsection{Measuring the affine transformation error}
            The transformation $\AffMatij$ perfectly aligns the corresponding $i$\th{} query feature with the
              $j$\th{} database feature in the space of the database annotation ($\annotII$).
            If the correspondence is indeed correct, then we can expect that other corresponding features will be
              well aligned by the transformation.
            The next step is to determine how close the other transformed features from the query annotation
              ($\annotI$) are to their corresponding features in database annotation ($\annotII$).
            This can be measured using the error in distance, scale, and orientation for every correspondence.
            The following procedure is repeated for each hypothesis transform %
            $\AffMatij \in \HypothSet$.
            Note that the following description is in the context of the $i$\th{} query feature and the $j$\th{}
              database feature, and the $i,j$ suffix is omitted for clarity.
            In this context, the suffixes $\idxI$ and $\idxII$ will be used to index into features
              correspondences.

            Let $\set{B}_{\idxI} = \curly{\invvrmatI \where (\idxI, \idxII) \in \Matches_{\annotII}}$ be the set
              of keypoint matrices in the query annotation with correspondences to database annotation $\annotII$.
            Given a hypothesis transform $\AffMat$, each query keypoint in the set of matches
            %(mapping from the normalized reference frame to feature space)
            $\invvrmatI \in \set{B}_{\idxI}$, is warped into $\annotII$-space:
            \begin{equation}
                \warp{\invvrmatI} = \AffMat \invvrmatI
            \end{equation}
            %---
            The warped position $\warp{\ptI}$, scale $\warp{\scaleI}$, and orientation $\warp{\oriI}$, can be
              extracted from $\warp{\invvrmatI}$ using~\cref{eqn:affinewarp}.
            The warped query keypoint properties in $\annotII$-space and can now be directly compared to the
              keypoint properties of their database correspondences.
            %Each warped point is checked for consistency with its
            %  correspondence's $\ptII$, scale $\scaleII$, and orientation
            %  $\oriII$, in $\annotII$.
            The absolute distance in position, scale, and orientation between the $\idxI$\th{} query feature and
              the $\idxII$\th{} database feature with respect to hypothesis transformation $\AffMat$ is measured as
              follows:
            \begin{equation}\label{eqn:inlierdelta}
                \begin{aligned}
                    \Delta \pt_{\idxI, \idxII}     & \eqv  \elltwo{\warp{\ptI} - \ptII}\\
                    \Delta \scale_{\idxI, \idxII}  & \eqv  \max(
                        \frac{\warp{\scaleI}}{\scaleII},
                        \frac{\scaleII}{\warp{\scaleI}}) \\
                    \Delta \ori_{\idxI, \idxII}    & \eqv  \min(
                        \modfn{\abs{\warp{\oriI} - \oriII}}{\TAU},\quad 
                        \TAU - \modfn{\abs{\warp{\oriI} - \oriII}}{\TAU})
                \end{aligned}
            \end{equation}

        \subsubsection{Selecting affine inliers}
            %Valid inliers are those matches that have all absolute differences
            %  within a certain spatial distance threshold $\xythresh$, orientation
            %  threshold $\orithresh$, and scale threshold $\scalethresh$.
            %  %$\xythresh$ is specified as a percentage of the matched chip's
            %  %  diagonal length.
            Any keypoint match $(\idxI, \idxII) \in \Matches_{\annotII}$  is considered an inlier \wrt{}
              $\AffMat$ if its absolute differences in position, scale, and orientation are all within a spatial
              distance threshold $\xythresh$, scale threshold $\scalethresh$, and orientation threshold
              $\orithresh$.
            This is expressed using the function $\isinlierop$, which determines if match is an inlier:
             %\begin{equation}
              %\label{eqn:inlierchecks}
              %    \begin{gathered}
              %    %\begin{aligned}
              %        \txt{isinlier}(\kp_1, \kp_2) \rightarrow \elltwo{\pt_1' - \pt_2} < \xythresh \AND \\
              %  %-----
              %        {\frac{\scale_1'}{\scale_2} < \scalethresh \txt{ if }
              %        \paren{\scale_1' > \scale_2} \txt{ else }
              %        \frac{\scale_2}{\scale_1'} < \scalethresh}  \AND\\
              %  %-----
              %        \txt{minimum}(
              %        \modfn{\abs{\ori_1' - \ori_2}}{\TAU},
              %        \TAU - \modfn{\abs{\ori_1' - \ori_2}}{\TAU}) < \orithresh
              %    %\end{aligned}
              %    \end{gathered}
            %\end{equation}
            \begin{equation}\label{eqn:inlierchecks}
                \begin{gathered}
                %\begin{aligned}
                    \isinlierop((\idxI, \idxII), \AffMat)  \eqv  
                        \Delta \pt_{\idxI, \idxII} < \xythresh \AND 
                        \Delta \scale_{\idxI, \idxII} < \scalethresh \AND 
                        \Delta \ori_{\idxI, \idxII} < \orithresh
                %\end{aligned}
                \end{gathered}
            \end{equation}
        The set of inlier matches for a hypothesis transformation $\AffMat$ can then be written as:
        \begin{equation}\label{eqn:affinliers}
            \Matches_{\AffMat} \eqv \curly{m \in \Matches_{\annotII} \where \isinlierop(m, \AffMat)}
        \end{equation}
        The best affine hypothesis transformation, $\HypothAffMat$, maximizes the weighted sum of inlier scores.
        % FIXME
        \begin{equation}
            \HypothAffMat \eqv \argmax{\AffMat \in \HypothSet} 
                \sum_{(\idxI, \idxII) \in \Matches_{\AffMat}} \fs_{\idxI, \idxII}
        \end{equation}

    \subsection{Homography refinement}
        Matches that are inliers to the best hypothesis affine transformation, $\HypothAffMat$, are used in the
          least squares refinement step.
        This step is only executed if there are at least $4$ inliers to $\HypothAffMat$, otherwise all
          correspondences between features in query annotation $\annotI$ to features in database annotation
          $\annotII$ are removed.
        The refinement step estimates a projective transform from $\annotI$ to  $\annotII$.
        To avoid numerical errors the $xy$-locations of the correspondence are normalized to have a mean of $0$
          and a standard deviation of $1$ prior to estimation.
        A more comprehensive explanation of estimating projective transformations using point correspondences can
          be found in~\cite[311--320]{szeliski_computer_2010}.

        Unlike in the affine hypothesis estimation where many transformations are generated, only one homography
          transformation is computed.
        Given a set of inliers to the affine hypothesis transform $\Matches_{\HypothAffMat}$, the least square
          estimation of a projective homography transform is:
        \begin{equation}
            \HmgMatBest \eqv \argmin{\HmgMat} \sum_{(i, j) \in
              \Matches_{\HypothAffMat}} \elltwosqrd{\HmgMat \pt_{i} - \pt_{j}}
        \end{equation}

        Similar to affine error estimation, we will identify the subset of inlier features correspondences
          $\Matches_{\HmgMatBest} \subseteq \Matches_{\annotII}$.
        A correspondence is an inlier if the query feature is transformed to within a certain spatial distance
          threshold $\xythresh$, orientation threshold $\orithresh$, and scale threshold $\scalethresh$ of its
          corresponding database feature.
        For convenience, let $\tohmg{\cdot}$ transform points into homogeneous coordinates.
        For each feature correspondence $(\idxI, \idxII) \in \Matches_{\annotII}$, the query feature position,
          $\ptI$, is warped from $\annotI$-space into $\annotII$-space.
        \begin{equation}
            \warp{\ptI} = \unhmg{\HmgMatBest \tohmg{\ptI}}
        \end{equation}
        However, because projective transformations are not guaranteed
          to preserve the structure of the affine keypoints, warped
          scales and orientations cannot be estimated with the method
          previously shown in~\cref{eqn:affinewarp}.

        \subsubsection{Estimation of warped shape parameters}
        Because we cannot warp the shape of an affine keypoint using a projective transformation, we instead
          estimate the warped scale and orientation for the $\idxI$\th{} query feature using a reference point.
        Given a single feature match $(\idxI, \idxII) \in \Matches_{\annotII}$, we associate a reference point
          $\refptI$ with the query location $\ptI$, scale $\scaleI$ and orientation $\oriI$.
        The reference point is defined to be $\scaleI$ distance away from the feature center at an angle of
          $\oriI$ radians in $\annotI$-space.
          \begin{equation}
            \refptI = \ptI + \scale_1 \BVEC{\sin{\oriI} \\ -\cos{\oriII}}
          \end{equation}

        To estimate the warped scale and orientation, first the reference
          point is warped from $\annotI$-space into $\annotII$-space.
        \begin{equation}
            \warp{\refptI} = \unhmg{\HmgMatBest \tohmg{\refptI}}
        \end{equation}
        %-----------
        Then we compute the residual vector $\ptres$ between the warped point and the warped reference point:
        \begin{equation}
            %\Delta \warp{\refptI} = \BVEC{\Delta \warp{\inI{x}} \\ \Delta \warp{\inI{y}}} = \warp{\ptI}- \warp{\refptI}.
            \ptres = \BVEC{\xres \\ \yres} = \warp{\ptI}- \warp{\refptI}.
        \end{equation}
        The warped scale is estimated using the length of the residual vector, and the warped orientation is
          estimated using the angle of the residual vector.
        In summary, the warped location, scale, and orientation of the $\idxI$\th{} query feature is:
        \begin{equation}\label{eqn:homogwarp}
            \begin{aligned}
                \warp{\ptI}      &\eqv \unhmg{\HmgMatBest \tohmg{\refptI}} \\
                 \warp{\scaleI}  &\eqv \elltwo{\ptres}\\
                %\warp{\oriI}    &= \atantwo{\yres, \xres} - \frac{\TAU}{4}.
                %\warp{\oriI}    &= \atantwo{\yres, \xres} - \frac{\pi}{2}.  % is this adjustment right?
                \warp{\oriI}     &\eqv \atantwo{\yres, \xres}
            \end{aligned}
        \end{equation}

        \subsubsection{Homography inliers}
        The rest of homography inlier estimation is no different than affine inlier estimation.
        \Cref{eqn:inlierdelta} is used to compute the errors $( %
        \Delta \pt_{\idxI, \idxII}, %
        \Delta \scale_{\idxI, \idxII}, %
        \Delta \ori_{\idxI, \idxII})$
        %
        between the warped query location, scale, and orientation, $(\warp{\ptI}, \warp{\scaleI}, \warp{\oriI})$, %
        and the corresponding database location, scale, and orientation, %
        $({\ptII}, {\scaleII}, {\oriII})$.
        The final set of homograph inliers is given as:
        \begin{equation}\label{eqn:homoginliers}
            \Matches_{\HmgMatBest} \eqv \curly{m  \in \Matches_{\annotII} \where \isinlierop(m, \HmgMatBest)}
        \end{equation}

        Spatial verification results in a reduced set of inlier feature correspondences from the query annotation
          to the database annotations.
        The \namescoring{} mechanism from~\cref{subsec:namescore} is then applied to these inlier feature
          correspondences.
        This final per-name score is the output of the identification algorithm and used to form a ranked list
          that is presented to a user for review.

     \sver{}



\subsection{Exemplar selection}\label{sec:exempselect}
    To scale one-vs-many matching to larger databases and to allow the LNBNN
    mechanism to find appropriate normalizers we restrict the number of
    examples of each individual in the database to a set of exemplars.

    Exemplars that represent a wide range of viewpoints and poses are
      automatically chosen using a modified version of the technique presented
      in~\cite{oddone_mobile_2016}.
    The idea is to treat exemplar selection as a maximum weight set cover
      problem.
    For each individual, the input is a set of annotations.
    A similarity score is computed between pairs of annotations.
    To compute covering sets we first choose a threshold, each annotation is
      assigned a covering set as itself and the other annotations it matches
      with a similarity score above that threshold.
    The maximum number of exemplars is restricted by setting a maximum weight.
    Searching for the optimal set cover is NP-hard, therefore we use the
      greedy %
    $(1 - \frac{1}{e})$-approximation algorithm~\cite{michael_guide_1979}.
    The algorithm is run for several iterations in order to find a good
      threshold that minimizes the difference between the weight of the set
      cover and the maximum weight limit.
    The similarity score between annotations can be computed using the LNBNN
      scores, but a better choice is the of the algorithm we will later describe
      in \Cref{chap:pairclf} to produce the probability that a pair of
      annotation correctly matches.

    
\section{Experiments}\label{sec:experiments}

    This section presents an experimental evaluation of the identification algorithm using annotated images of
      plains zebras, Grévy's zebras, Masai giraffes, and humpback whales.
    The input to each experiment is
    (1) a dataset,
    (2) a subset of query and database annotations from the database,
    (3) a pipeline configuration.
    The datasets are described in~\cref{sub:datasets}.
    The subsets of query and database annotations are carefully chosen to measure the accuracy of the algorithm
      under different conditions and to control for time, quality, and viewpoint.
    The pipeline configuration is a set of parameters --- \eg{} the level of feature invariance, the number of
      nearest neighbors, and the \namescoring{} mechanism --- given to the identification algorithm.
    We will vary these pipeline parameters in order to measure their effect on the accuracy of the ranking
      algorithm.

    For each query annotation, the identification algorithm returns a ranked list of \names{} with a score for
      each name.
    The accuracy of identification is measured using the cumulative match
      characteristic~\cite{decann_relating_2013} which can be understood as the probability that a query correctly
      finds a match at a specified rank under the assumption that a correct match exists in the database.
    We are primarily returned with only the first point in this curve --- the fraction of queries with a correct
      result at rank $1$ --- because often a user of the system will only review the first result of a query.

    %Additionally, we like to automatically review results returned by the ranking algorithm, so we will measure
    %  the separability of correct and incorrect results using the scores returned by the ranking algorithm.
    %The separability of scores is measured by first recording the score of the correct \name{} and the
    %  highest scoring incorrect \name{} for each query and then plotting the receiver operator characteristic
    %  (ROC) curve and measuring the area under the curve (AUC).
    %Other measures that are reported include the percentage of true positive
    %  and false positive results that fall within certain categories (\eg
    %  temporal windows).

    The outline of this section is as follows.
    First, \cref{sub:datasets} introduces and describes each dataset.
    Our first experiment in \cref{sub:exptbase} establishes the accuracy of our ranking algorithm on several
      datasets using a default pipeline configuration.
    We then compare our approach to an alternative \cref{sub:exptsmk} SMK approach.
    The next subsections perform in depth experiments on the parameter settings of our algorithm.
    \Cref{sub:exptfg} tests the effect of the foregroundness weight on identification accuracy.
    \Cref{sub:exptinvar} investigates the effect of the level of feature invariance and viewpoint.
    \Cref{sub:exptscoremech} compares the \csumprefix{} and the \nsumprefix{} \namescoring{} mechanism.
    \Cref{sub:exptk} varies the $\K$ parameter (the number of nearest neighbors used in establishing feature
      correspondences) and investigates the relationship between $\K$ and database size in terms of both the number
      of annotations and the number of exemplars per name.
    \Cref{sub:exptfail} discusses the failure cases of our ranking algorithm.
    %\Cref{sub:exptsep} presents an evaluation of the score separability for the pipeline configuration with the
    %  highest accuracy determined for each species.
    Finally,~\cref{sub:exptsum} summarizes this section.


    \subsection{Datasets}\label{sub:datasets}

        All of the images in the datasets used in these experiments were taken by photographers in the field.
        Each dataset is labeled with groundtruth in the form of annotations with name labels.
        Annotations (bounding boxes) have been drawn to localize animals within the image.
        A unique \name{} label has been assigned to all annotations with the same identity.
        Some of this groundtruth labeling was generated independently.
        However, large portions of the datasets were labeled with assistance from the matching algorithm.
        While this may introduce some bias in the results, there was no alternative because the amount of time
          needed to independently label a large dataset is prohibitive.

        There are two important things to note before we describe each dataset.
        First, in order to control for challenging factors in the images such as quality and viewpoint some
          experiments sample subsets of the datasets we describe here.
        Second, note that there do exist labeling errors in some datasets.

        \DatabaseInfo{}

        \timedist{}

        The number of names, annotations, and their distribution within each database are summarized in the
          following tables.
        In these tables we distinguish between \glossterm{singleton} and \glossterm{resighted} names.
        Singleton names are individuals sighted only once, \ie{} contain only a single encounter.
        Resighted names contain more than one encounter.
        We make this distinction because resighted names have known correct matches across a significant time
          delta.
        Note, that singleton names may still have more than one annotation, but those annotations are all from
          the same encounter.
        We have pre-filtered each database to remove annotations that are unidentified, are missing timestamps,
          or are labeled as ``junk'' quality.

        \Cref{tbl:DatabaseStatistics} summarizes the number of annotations and individuals in each database as
          well as the number of times (distinct encounters) each individual was sighted.
        \Cref{tbl:AnnotationsPerQuality} summarizes the quality labels of the annotations.
        \Cref{tbl:AnnotationsPerViewpoint} summarizes the viewpoint labels of the annotations.
        Distributions of when images in each dataset were taken are illustrated in \cref{fig:timedist}.
        The name and a short description of each dataset is given in the following list.

        \FloatBarrier{}

        \begin{itemln}
            \item \textbf{Plains zebras}.
            Our plains zebras dataset is an aggregate of several smaller datasets.
            There is variation in how the data was collected and preprocessed.
            Some of the images are cropped to the flank of the animal, while others are cropped to encompass the
              entire body.
            The constituent datasets were collected in Kenya at several locations including Nairobi National
              Park, Sweetwaters, and Ol Pejeta.
            More than $90\percent$ of the groundtruth generated for this dataset was assisted using the matching
              algorithm.
            This dataset contains many imaging challenges including occlusion, viewpoint, pose, quality, and time
              variation.
            There are some annotations in this dataset without quality or viewpoint labelings and some images
              contain undetected animals.
            This data was collected between 2006 and 2015, but the majority of the data was collected in
              2012--2015.

            \item \textbf{Grévy's zebras}.
            This is another aggregated dataset.
            The original groundtruth for this dataset was generated independently of the matching algorithm,
              however the matching algorithm revealed several groundtruth errors that have since been corrected.
            The Grévy's dataset was collected in Mpala, Kenya.
            Most of the annotations in this database have been cropped to the animal's flank.
            This dataset contains a moderate degree of pose and viewpoint variation as well as occlusion.
            This data was collected between 2003 and 2012, but the majority was collected in 2011 and 2012.

            \item \textbf{Masai giraffes}.
            These images of Masai giraffes were all taken in Nairobi National Park during the \GZC{} between
              February 20, 2015 and March 2, 2015.
            All groundtruth was established using the matching algorithm followed by manual verification.
            This dataset contains a high degree of pose and viewpoint variation, as well as occlusion.
            Because of their long necks, it is difficult to ensure that only a single giraffe appears in each
              annotation.
            This results in many \glossterm{photobombs} --- pairs of annotations where a background animal in one
              annotation matches the foreground animal in the other --- when matching.

            \item \textbf{Humpback whales}.
            The humpback dataset was collected by FlukeBook over nearly 15 years.
            Images were contributed by both marine and citizen scientists.
            The original groundtruth was established using both manual and automated methods that are disjoint
              from these techniques considered here, however our software was used to correct mistakes.
            The annotations in this dataset have not been manually reviewed.
            Some are cropped to the fluke while others encompass the entire image.
            Quality and viewpoint labels do not exist for this dataset.
        \end{itemln}

    \FloatBarrier{}
    \subsection{Baseline experiment}\label{sub:exptbase}

        This first experiment determines the accuracy of the identification algorithm using the baseline pipeline
          configuration.
        The baseline pipeline configuration uses affine invariant features oriented using the gravity vector,
          $\K\tighteq4$ as the number of feature correspondences assigned to each query feature, and \nscoring{}
          (\nsum{}).
        In this test we control for several biases that may be introduced by carefully selecting a subset of our
          datasets.
        We only use annotations that
        (1) are known (\ie{} have been assigned a name),
        (2) are comparable the species primary viewpoint (\eg{} left, front-left, and back-left for plains
          zebras),
        (3) have not been assigned a quality of ``junk''.
        Furthermore, to account for the fact that some \names{} contain more annotations than others, we
          constrain our data selection such that there is only one correct exemplar in the database for each query
          annotation.

        Of these annotations, we group them into encounters.
        For each encounter we sample one annotation with the highest quality.
        Names with only one encounter are added to the database as distractors.
        For the other names, we randomly sample two encounters --- regardless of quality --- one for the database
          and one to use as a query.
        This defines a set of query and database annotations that are separated by time, testing the ability of
          our system to match animals across gaps of time using only a single image per individual.
        The CMC curves for this baseline test are illustrated in~\cref{fig:BaselineExpt}.

        The results of this baseline experiment demonstrates that our algorithm is able to reliably find matching
          annotations in a database with many other images.
        The accuracy is over $60\percent$ for all species considered.
        Subsequent experiments will restrict our focus to Grévy's plains zebras in order to investigate detailed
          parameter choices of the ranking algorithm as well as alternative ranking algorithms.
      
        \BaselineExpt{}

    \FloatBarrier{}
    \subsection{SMK as an alternative}\label{sub:exptsmk}  

        Before we investigate the parameter choices of the LNBNN ranking algorithm, we briefly evaluate the
          performance of an alternative ranking algorithm, namely VLAD-flavored SMK.
        The SMK algorithm is a vocabulary based algorithm, that is representative of more traditional approaches
          to instance recognition problems.
        In contrast to the raw descriptors used in LNBNN, SMK assigns each descriptor to a visual word and builds
          a weighted histogram of visual words (accumulated residual vectors in the case of VLAD) to represent each
          annotation as a sparse fixed length vector.
        We have experimented with several configurations of VLAD and report our best results here.

        In our SMK implementation, we pre-trained an $8000$ word vocabulary using mini-batch k-means on the
          stacked descriptors from all database annotations.
        Note that typically the vocabulary is trained using a disjoint external dataset in order to prevent
          overfitting.
        However, we \naively{} train using the database annotations to be indexed, understanding that this will
          inflate the accuracy measurements.
        Each word in the vocabulary is weighted with its inverse document frequency.
        We use the vocabulary to compute an inverted index that maps each visual word to annotations containing
          that word in the database.
        Initial feature correspondences for a descriptor are computed using single assignment to a visual word
          and then creating a correspondence to every feature in that word's inverted index.
        We use spatial verification to filter spatially invalid correspondences, and re-score the remaining
          matches.

        The results of the SMK experiment are illustrated in \cref{fig:SMKExpt}.
        The query and database annotations are the same in each experiment.
        Despite the bias in the SMK vocabulary, our measurements show that LNBNN provides the most accurate
          rankings.
        For plains zebra's there is a difference of $8\percent$ in the number of correct matches at rank $1$, and
          for Grévy's zebras the difference is $6\percent$.

        \SMKExpt{}

    \FloatBarrier{}
    \subsection{Foregroundness experiment}\label{sub:exptfg}

        In this experiment we test the effect of our foregroundness weights --- weighting the score of each
          features correspondence with a foregroundness weight --- on identification accuracy.
        When foregroundness is enabled (\pvar{fg=T}), each feature correspondence is weighted using a
          foregroundness measure learned using a deep convolutional neural network~\cite{parham_photographic_2015}.
        When disabled (\pvar{fg=F}), the weight of every correspondence effectively becomes $1$.

        Running this experiment with using the query / database sample as outlined in the baseline experiment
          does not result in a noticeable difference in scores because the purpose of the foregroundness measure is
          to down weight matches between scenery objects (\eg{} trees, grass, bushes) that appear in multiple
          annotations.
        The baseline database sample contains only a single images from each encounter and only two encounters
          per individual.
        This means that it will be unlikely for an annotation in the query set and another annotation in the
          database set to have a similar background.

        To more clearly illustrate the effect of the foregroundness measure we use a different sampling strategy.
        We group all encounters by which occurrence they belong to.
        Annotations within the same occurrence are more likely to share background.
        We sample query and database annotations from within occurrences to simulate matching annotations within
          an encounter.
        We do not limit the number of exemplars in this test to ensure that annotation pairs that share common
          scenery exist.
        We perform this test over multiple occurrences and aggregate the results.
        Therefore, the reported database size will be an average, and the query size is the sum of all unique
          query annotations.
        %We sort the occurrences by size in descending order, and then iterate through each occurrence.
        %We sample the highest quality annotation in encounter in the occurrence as long as that name has not been
        %  sampled more than two times.
        %At the end of this process we have at most two annotations for each name.
        %We randomly choose annotations to be query and database annotations from each name with two annotations.
        %The rest are used as confusers.
        %We execute the ranking algorithm twice with foregroundness both enabled and disabled.
        %The rest of the pipeline configuration is the same as the baseline test.

        The accuracy of the foregroundness is illustrated in~\cref{fig:FGIntraExpt}.
        The results show that using foregroundness weights improve the number of correct results at rank $1$ by a
          significant margin for both species.
        In the higher ranks using using the \pvar{fg=T} line occasionally dips below the \pvar{fg=F} line because
          sometimes the foregroundness mask covers distinguishing keypoints, but this is neither significant nor
          common.
        Therefore we find it beneficial to always include foregroundness when a trained estimator is available.

        %\ForegroundExpt{}
        \FGIntraExpt{}

        %foreground weights occasionally dips below Occasionally the accuracy of using foreground weights Occasionaly toNote that towards higher ranks 
        %N
        %For plains zebras, using the foregroundness measure results in a significant $3.79\percent$ increase in
        %  identification accuracy.
        %For Grévy's zebras there is also a significant $3.3\percent$ increase.
        %This experiment clearly shows the importance of eliminating background feature correspondences.
     
    \FloatBarrier{}
    \subsection{Invariance experiment}\label{sub:exptinvar} %
        In this experiment we vary the feature invariance configuration.
        This influences the location, shape, and orientation of keypoints detected in each annotation, which in
          turn influences which regions in each annotation are matchable using SIFT descriptors extracted at each
          keypoint.
        The best invariance settings will be depend on properties of the data.

        In our experiments we test different settings by enabling (denoted as T) or disabling (denoted as F) the
          parameters affine invariance (AI), and our query-side rotation heuristic (QRH).
        Initially we also tested rotation invariance, but found that it provided the poorest results for all
          datasets by a significant margin, likely because the gravity vector assumption is mostly satisfied in all
          images.
        Therefore, we exclude rotation invariance from our experiments.

        In configurations where \pvar{AI=F}, keypoints are circular with a radius defined by the scale at which
          it was detected.
        When \pvar{AI=F}, the keypoint shape is adapted into an ellipse to normalize for small viewpoint changes.
        When \pvar{QRH=F}, each keypoint is assigned its orientation as normal, but when \pvar{QRH=T}, each
          keypoint in a query annotation is replaced by three keypoints, one rotated slightly to the left, another
          slightly to the right, and the last is the original keypoint.
        The four specific configuration that we test are outlined in the following list:

        \FloatBarrier{}
        \begin{itemln}

            \item \NoInvar{} (\pvar{AI=F,QRH=F}): % 
                This configuration uses circular keypoints and assumes the gravity vector.

            \item \AIAlone{} (\pvar{AI=T,QRH=F}): % 
                This is the baseline setting that assumes the gravity vector and where each feature's shape is skewed
                  from a circle into an ellipse.

            \item \QRHCirc{} (\pvar{AI=F,QRH=T}): %
                This is a novel invariance heuristic where each {database} feature assumes the gravity vector, but
                  {query} feature is $3$ orientations:
                the gravity vector and two other orientations at $\pm15\degrees$ from the gravity vector.
                Ideally, this will allow feature correspondences to be established between features seen from
                  slightly different orientations.

            \item \QRHEll{} (\pvar{AI=T,QRH=T}): %
                This is the combination of \QRHCirc{} and \AIAlone{}.

        \end{itemln}
        \FloatBarrier{}

        % Invar Conclusions
        The example in~\cref{fig:kptstype} illustrates the difference between \AIAlone{} and \QRHCirc{} features
          for plains and Grévy's zebras.
        The accuracy of the invariance experiment is shown in~\cref{fig:InvarExpt}.
        For plains zebras, the \QRHCirc{} scores are significantly better than all other invariance settings.
        Interesting, affine invariance results in worse performance when QRH is on, but if the QRH is off then
          affine invariance improves accuracy.
        This suggests that the QRH better handles matching the coarse patterns seen on the plains zebras across
          pose and viewpoint variations than using affine invariance, which can tend to adapt itself around
          non-distinctive diagonal stripes.
        Even though affine keypoints provide more precise localization, the area they describe is often smaller
          than a circular keypoint.
        It makes sense that affine keypoints would not describe coarse features as well as a circular keypoint
          covering a larger area.
        %
        The results for Grévy's zebras demonstrate similar levels of accuracy for \AIAlone{} and \QRHEll{}.
        Affine invariance seems to be the most important setting for matching Grévy's zebras.
        The distinctive details on Grévy's zebras are finer then plains zebras and are well captured by affine
          keypoints.
        While the QRH does improve accuracy for Grévy's zebras the density of the distinctive keypoints means
          that it is less important because it is more likely that a two annotations will have at least one
          distinctive region aligned an in common.

        \InvarExpt{}

        \kptstype{}

    \FloatBarrier{}
    \subsection{Scoring mechanism experiment}\label{sub:exptscoremech}  

        % TODO: change experiment so only one annotation per name is chosen for
        % each confuser

        % Database setup for name scoring
        The purpose of the scoring mechanism is to aggregate scores of individual feature correspondences across
          multiple annotations into a single score for each name --- \ie{} an annotation-to-name similarity, which
          is analogous to the image-to-class distance used in~\cite{boiman_defense_2008}.
        We test the identification accuracy of the two name scoring mechanisms that were described earlier
          in~\cref{subsec:namescore}:
        (1) \cscoring{} (denoted as \csum{}) and
        (2) \nscoring{} (denoted as \nsum{}).

        Because the scoring mechanism is meant to take advantage of multiple database annotations, we vary the
          number of \exemplars{} per database \name{} (\pvar{dpername}) between $1$, $2$, and $3$.
        Varying the number of \exemplars{} will cause each database to contain a different number of annotations.
        To normalize difference in database size we include additional confuser annotations (annotations that do
          not match any query) in the smaller databases to maintain a constant database size across experiments.
        Each \exemplar{} is chosen from a separate encounter.

        The accuracy of the scoring mechanism experiment is shown in~\cref{fig:NScoreExpt}.
        The results of this test does suggest that \nsum{} results in slightly more accurate ranking, but the
          overall difference in accuracy is relatively small (about $1-3\percent$).
        Intuitively, the \nsum{} scoring should produce better ranks because it can combine scores from multiple
          correspondences to different correct annotations.
        Note that when $\pvar{dpername}=1$, the \csum{} and \nsum{} scores might still be different because
          \nsum{} multiple correspondences to a name which may be generated when $\K > 1$ or when \pvar{QRH=T}.

        Perhaps the more interesting result of this experiment is the effect of increasing the number of
          exemplars in the database from $1$ to $2$.
        There is a drastic improvement in ranking accuracies in both species.
        The accuracy of plains zebras increases by $10\percent$ and for Grévy's zebras the gain is almost
          $20\percent$.
        It makes sense that this should be the case.
        If there are more examples of an individual in the database then the probability that the query is
          similar to at least one of them should increase as long as there is sufficient variation.
        This suggests that even if a new query new annotation initially fails to rank the correct result,
          subsequent annotations added to the system of the same individual will be more likely to correctly match
          a previous annotation.
        As more annotations of that individual are added the likelihood that the ranking algorithm will make a
          connection between all instances of that individual will increase.

        \NScoreExpt{}


    \FloatBarrier{}
    \subsection{K experiment}\label{sub:exptk}  

        % Introduce varied parameters
        In this experiment we investigate the effect of $\K$ (the number of nearest neighbors used in
          establishing feature correspondences, which was discussed in~\cref{sub:featmatch}) on identification
          accuracy.
        We vary $\K$ between the values $1, 2, 4$, and $6$.
        In all of these experiments we set the number of normalizing neighbors to be $\Knorm=1$.

        Two database factors that may influence the best choice of $\K$ are the number of annotations in the
          database and the number of annotation per name in the database.
        If there are more correct matches for a query annotation it would be beneficial to allow it to match more
          annotations.
        Likewise, if there are more overall annotation in the database, then it might be beneficial to search
          deeper into all of the database descriptors to find the correct matches.
        Therefore, in addition to varying $\K$ we also vary the number exemplars per name (\pvar{dpername}) and
          the overall number of annotations in the database (\pvar{dsize})

        We use a protocol similar to the one used in the scoring mechanism experiment to sample databases.
        The difference is that we use the extra confusers annotations to vary the total number of annotations in
          the database.
        However, controlling for these factors constrains the number of annotations we can use.
        For Grévy's, we can vary the total database size between $476$ and $774$.
        For plains we have more confuser annotations allowing us to test database sizes of $578$ and $1650$.
        We vary the number of \exemplars{} per name between $1$ and $2$

        The results of this experiment are illustrated in ~\cref{fig:KExptA,fig:KExptB}.
        Similarly to the previous experiment, the number of exemplars per name is the most significant variable
          impacting accuracy.
        Furthermore, when there are more exemplars in the database the choice of $\K$ starts become less
          significant.
        The results also show that accuracy does slightly decrease when the database becomes larger, but
          magnitude of the decrease is between $1\percent$ and $3\percent$.
        Interesting the optimal choice of $\K$ is not consistent between species when there is only one exemplar
          per name.
        For Grévy's zebras using a lower $\K$ results in better results, but for plains zebras there is a
          signifiant loss when $\K=1$ and the database size is large.
        This is likely due to the nature of the distinguishing patterns on the different zebras.
        When matching the detailed patterns of the Grévy's zebras, it is better to use a low $\K$ to reduce
          noise, but for coarser plains zebras patterns a low $\K$ might not find a correct match immediately.
        Thus, the choice of $\K$ is a trade-off between precision and recall that depends on the type of texture
          patterns that are being matched.
        
        %Our second test varies the size of the database as well as the value of $\K$.
        %The effect of $\K$ and database size on matching accuracy is shown in~\cref{fig:DBSizeExpt}.
        %The results for all species show that the number of \exemplars{} per \name{} is the most important factor
        %  in this experiment.
        %Interestingly, the number of annotations in the database is only a minor factor in identification
        %  accuracy.
        %% plains
        %The results for plains zebras show a small positive relationship between the number of annotations in the
        %  database and $\K$.
        %This may be because many plains zebra features are not globally distinctive in a large database and a
        %  feature's correct correspondence may not be the nearest neighbor.
        %For smaller database sizes lower values of $\K$ produce more accurate results.
        %% Grévy's
        %For Grévy's zebras lower values of $\K$ seem better for all database sizes.
        %It also seems that the best values of $\K$ should be set to the number of \exemplars{} per \name{}, which
        %  is expected when there is little confusion between features.
        %% giraffes
        %For Masai giraffes, the amount of data again makes it difficult to draw conclusions.

        %\DBSizeExpt{}

        Overall the experiments on the setting of $\K$ does not yield definitive choice for this parameter.
        However, it appears that $\K$ only has a small influence on identification accuracy.
        This section does shows that the number of exemplars per annotation has a significant impact on
          identification accuracy.

        \KExptA{}
        \KExptB{}

    \FloatBarrier{}
    \subsection{Failure cases}\label{sub:exptfail}  
        
        We no investigate the causes of identification failure and consider example failure cases.
        When investigating the cause of a failure case we consider both
        (1) the matches between the query annotation and the incorrect \name{}  at rank $1$ and
        (2) the matches between the query annotation and the correct \name{} that appears further down the ranked
          list.
        We identify the main 3 failure cases as:
        (1) unaligned annotations,
        (2) quality factors, and
        (3) non-primary correspondences.
        The remainder of this subsection defines, discusses, and provides examples of these failure cases.

        %The first three types of failure cases denote type $2$ errors (false negatives) where the correct match
        %  fails to produce a high score.
        %The last three types of failure cases denote type $1$ errors (false positive) where the incorrect match
        %  produces a score that is too high.

        \FloatBarrier{}
        \subsubsection{Alignment}
            
            When two annotations are not aligned (ignoring translation and small scale differences), there can be
              significant differences in appearance that can cause inconsistency in feature localization and
              description.
            There are two major causes of alignment error:
            (1) viewpoint variations which cause out-of-plane rotations and
            (2) pose variations which can cause local non-rigid non-linear transformations of distinguishing
              features.
            These issues cause variations in feature description which renders the approximate nearest neighbor
              algorithm unable to establish the correct correspondence.
            Furthermore, non-projective transformations between annotations can cause homography-based spatial
              verification to discard correctly established correspondences.
            Failing to establish correspondences and incorrectly discarding them ultimately results in
              identification failure.

            The example in~\cref{fig:FailViewpoint} illustrates a failure case due to a difference in viewpoint
              and pose.
            To address matching across different viewpoints and poses it helps to choose an appropriate level of
              feature invariance (like affine invariance and the query-side rotation heuristic), but these only
              work up to a point.
            However, in practice the animal identification problem is not a one-shot identification challenge.
            Given multiple annotations of an individual we expect that the matching algorithm will be able to
              overcome viewpoint and pose differences by matching annotations with intermediate positions.

            \FailViewpoint{}

        \FloatBarrier{}
        \subsubsection{Quality factors}
            Factors such as low resolution, blurring, poor exposure, lighting (shadows / non-uniform
              illumination), and occlusion can significantly reduce the density of distinctive features on an
              annotation.
            Note that lighting and occlusion (scene quality factors) should be distinguished from the other
              factors (capture quality factors) because they are related to the scene itself rather than a poor
              capturing of that scene.
            Annotations with low capture quality tend to generate fewer, larger, less distinct, and distorted
              features.
            Annotations with low scene quality tend to have their distinguishing features distorted or masked by
              grass, tree branches, shadows, and other animals.
            This is a significant problem for species with relatively few distinctive features like plains
              zebras.
            The example in~\cref{fig:FailOcclusion} illustrates a failure case due to occlusion, and
              \cref{fig:FailQuality} illustrates failure case due to low resolution.

            In some cases low quality annotations can be still be matched, but in the worst case all distinctive
              features are missing and there is no way to visually identify the individual.
            Therefore, the best way to handle these annotations is either to ignore them entirely, or to first
              attempt to match them, but then discard them if they cannot be matched.

            \FailOcclusion{}
            \FailQuality{}

        \FloatBarrier{}
          \subsubsection{Non-primary correspondences}
            Sometimes an annotation bounding box cannot be placed tightly around an animal (this happens often
              for some species like giraffes), which means that other objects will appear in the background.
            Similarly, objects that occlude the animal will be in the foreground.
            Ideally, the primary animal in each annotation would be segmented, but when simply matching raw
              annotations non-primary correspondences may be formed.
            This results in the photobomb and scenery-match failure cases.

            Photobombs are caused by correct correspondences a non-primary animal (seen either in the foreground
              or background) in an annotation.
            Likewise, scenery matches are caused by matches in the background landscape.
            Both cases are most commonly caused by pairs of annotations with the same occurrence, but photobombs
              can occur over larger time deltas.
            The example in~\cref{fig:FailPhotobomb} illustrates a failure case due to a photobombing background
              animal, and \cref{fig:FailScenery} illustrates a scenery match.
            For plains and Grévy's zebras, most scenery matches are be eliminated using the foregroundness
              measure, but the problem remains in databases without a trained foregroundness estimator.
            Accounting for photobombs is a more challenging problem because a simple patch based classifier
              cannot distinguish a primary feature from a  secondary feature without having information about the
              animal identity.
            However, there are some patterns that photobomb matches present that we seek to take advantage of
              later in \cref{sec:learnpb}.

            \FailScenery{}
            \FailPhotobomb{}

    \FloatBarrier{}
    \subsection{Experimental conclusions}\label{sub:exptsum}  

        In this section we have evaluated our ranking algorithm on multiple species, compared it to an
          alternative ranking algorithm, and evaluated detailed parameter choices.
        Our experiments were performed under restrictive conditions to control for the effect of time, database
          size, and number of exemplars.
        Based on the results of these experiments we are able to make several observations and conclusions.

        Our experiments with comparing the SMK and LNBNN ranking algorithm demonstrated that LNBNN achieved
          better ranking accuracy.
        LNBNN does not quantize descriptor and therefore it is able to distinguish more subtle descriptor
          details.
        Because LNBNN does not require an expensive pre-training phase it makes it make it ideal to rank the
          databases on the scales considered in this thesis.
        However, we note that SMK is more efficient on larger scales, and it may be necessary to consider when
          databases become very large.

        %\item \textbf{Identification accuracy improves with more exemplars}:
        % NUM EXEMPLARS MATTERS
        In most experiments we evaluated our ranking algorithm as if it were addressing a single-shot
          identification problem.
        This was to establish the performance of the algorithm when an individual has only been seen once
          before.
        However, in practice this will not be the norm.
        The name scoring and $\K$ experiments demonstrated that the ranking accuracy significantly increases
          with the number of exemplars per database name.
        We will use this observation to address the challenges of matching through viewpoint and occlusion by
          taking advantage of multiple images of an individual in~\cref{chap:graphid}.

        %\item \textbf{Foregroundness weighting reduce scenery matches}:
        %% FOREGROUNDNESS GOOD
        %Identification accuracy significantly improves by a few percentage points when using foregroundness
        %  weighting.
        %We have found that enabling foregroundness weighting eliminates nearly all failure cases due to
        %  scenery matches without significantly affecting other results.

        %\item \textbf{Invariance settings are data dependent}:
        We also saw that the best choice for feature invariance is data dependant.
        The invariance experiment demonstrated that affine invariance produces better results for Grévy's zebras,
          whereas circular keypoints lead to more accurate results for plains zebras.
        This experiment also showed that the query-side rotation heuristic improves accuracy by adding a small
          amount of orientation invariance to feature localization.
        Likewise, the $\K$ experiment shows that identification accuracy is not significantly influenced by the
          choice of $\K$ for plains zebras, but for Grévy's zebras the most accurate results were obtained with
          $\K\tighteq1$.
        This is likely because the features from plains zebras are less distinguishing than features from Grévy's
          zebras, hence the correct match of a plains zebra feature is less likely to be its closest neighbor.
        Both the choice of $\K$ and invariance settings should be evaluated on a per-dataset basis and there is
          likely benefit to performing identification using multiple parameter choices.
        %Furthermore, the size of the database does not seem to strongly influence the optimal choice of $\K$.
        %However, note that most tests were run on different sizes of large databases, and the choice of $\K$
        %  using small databases was not investigated.




\section{Rank-based identification summary}\label{sec:staticsum}

    In this chapter we have addressed the problem of animal identification
      using a computer-assisted algorithm that ranks a labeled database of
      \names{} by their similarity to a single query annotation.
    %In this section we have introduced an algorithm that ranks known
    %  database \names{} by their similarity to a single query annotation.
    This algorithm beings by extracting local patch-based features from
      cropped and normalized chips.
    Features from database annotations are indexed for fast nearest neighbor
      search using a kd-tree.
    An mechanism based on LNBNN is used to compute a matching score for each
      database annotation.
    Based on these scores potential matches have their feature correspondences
      spatially verified and then are re-scored.
    We have shown how this algorithm can be applied to individual animal
      identification and demonstrated that in a majority of cases the correct
      match is ranked first by our algorithm.

    %This remainder of this section summarizes the conclusions drawn about the
    %  static identification algorithm in a broad sense and discusses possible
    %  directions for future work.

    Because we have used the algorithm to curate the groundtruth we do
      not claim the reported accuracies in our experiments to be
      quantitatively absolute.
    However, the fact that we were able to use the algorithm to
      identify a significant number of individuals from different species
      is qualitative evidence for the algorithm's overall success.
    From our experiments we conclude that the algorithm is effective at
      identifying medium to high quality images of animals with
      distinguishing patterns when taken from the same viewpoint.
    %In addition we have comparative evidence that this problem is more
    %  difficult than location recognition.


    %Plains zebras:
    %LNBNN: 75.6% @ rank 1
    %ASMK: 69.69% @ rank 1

    %Grevy's zebras: 
    %LNBNN: 84.94% @rank 1
    %ASMK: 70.87% @rank 1
    In our ranking experiments we only did extensive testing of the LNBNN algorithm.
    We have briefly experimented with using the vocabulary based SMK ranking algorithm (using the VLAD variant)
      and found LNBNN to provide superior results.
    In our preliminary experiments we found that the SMK algorithm was able to correct rank $69.69\percent$ of
      plains zebras and $70.87\percent$ of Grevy's zebras correctly at rank $1$.
    Comparable versions of LNBNN solutions achieved $75.6\percent$ and $69.7\percent$.

    While we have demonstrated that the ranking algorithm accurately ranks
      correct matches when they exist, there are several limitations to this
      approach.
    \begin{itemln}
        \item All results must be manually verified, which can be a time
          consuming process for large datasets.
            %We introduced a mechanism for ranking the individuals in a database
            %based similarity to a single query, but provided no means of verifying
            %which --- if any --- of the top ranked results matched.
        \item There is no mechanism for recovering from errors once they
          occur.
        \item There is no mechanism to determine when identification is
          complete.
    \end{itemln}
    In the following chapters we seek to address these issues.
    In \Cref{chap:pairclf} we introduce an algorithm to make automatic
      decisions based on results from this algorithm, and in \cref{chap:graphid}
      we introduce a graph-based framework that determines identification
      confidence and introduces error recovery mechanisms.



% L___ CHAPTER ___


\begin{comment}
fixtex --reformat --fpaths chapter4-pairclf.tex --print
fixtex --fpaths chapter4-pairclf.tex --outline --asmarkdown --numlines=999 

fixtex --fpaths chapter4-pairclf.tex --outline --asmarkdown --numlines=999 --shortcite -w && ./checklang.py outline_chapter4-pairclf.md
https://www.languagetool.org/
\end{comment}

\newcommand{\nan}{\text{nan}}

\chapter{Pairwise classification}\label{chap:pairclf}

In this chapter we consider the problem of verifying if two annotations are from the same animal or from different
animals. By addressing this problem we improve upon the ranking algorithm from \cref{chap:ranking} --- which ranks
the \names{} in a database based on similarity to a query --- by making semi-automatic decisions about results
returned in the ranked lists. The algorithms introduced in this chapter will assign a confidence to results in the
ranked list, and any pair above a confidence threshold can be automatically reviewed. We will demonstrate that our
decision algorithms can significantly reduce the number of manual interactions required to identify all individuals
in an unlabeled set of annotations.

To make semi-automatic decisions up to a specified confidence we develop a \emph{pairwise probabilistic
  classifier} that predicts a probability distribution over a set of events given two annotations (typically a
  query annotation and one of its top results in a ranked list).
Given only the information in two annotations,  there are three possible decisions that can be made.
A pair of annotations is either:
\begin{enumln}
    \item incomparable --- the annotations are not visually comparable,

    \item positive --- the annotations are visually comparable and the same individual, or

    \item negative --- the annotations are visually comparable and different individuals.
\end{enumln}
Two annotations can be incomparable if the annotations show different parts or sides of an animal, or if the
  distinguishing information on an animal is obscured or occluded.
The positive and negative states each require distinguishing information to be present.
These mutually exclusive ``match-states'' are illustrated in \cref{fig:MatchStateExample}.
The multi-label classifier then predicts the probability of each of the three states, with the probabilities
  necessarily summing to $1$.

\MatchStateExample{}

To construct a pairwise probabilistic classifier we turn towards supervised machine learning, which requires that
  we:
\begin{enumin}
    \item determine a set of labeled annotation pairs for training, 

    \item construct a fixed-length feature vector to represent a pair of annotations,  and

    \item choose a probabilistic learning algorithm.
\end{enumin}
The first requirement can be satisfied by carefully selecting representative annotations pairs, and the last
  requirement is satisfied by many pre-existing algorithms (\eg{} random forests and neural networks).
It is the second requirement --- constructing an appropriate fixed-length feature vector --- that is the most
  challenging.
If given enough training data and a technique to align the animals in two annotations, using image data with a
  Siamese or triplicate network~\cite{taigman_deepface_2014,schroff_facenet_2015} might appropriate, but without
  both of these pre-conditions we must turn towards more traditional methods.
Recall from \cref{sec:annotrepr} that our annotation representation is an unordered bag-of-features, which cannot
  be directly fed to most learning algorithms.
Therefore, we develop a method for constructing a fixed length \glossterm{pairwise feature vector} for a pair of
  annotations.
This novel feature vector will take into account local matching information as well as more global information
  such as GPS and viewpoint.
A collection of these features from multiple labeled annotation pairs are used to fit a random
  forest~\cite{breiman_random_2001} which implements our pairwise classifier.
We choose to use a random forest classifier, in part because they are fast to train, robust to overfitting, and
  naturally output probabilities in a multiclass setting, and in part because they can handle (and potentially take
  advantage of) missing data --- \ie{} \nan{} values in feature vectors --- using the ``separate class''
  method~\cite{ding_investigation_2010}.
  

A final concern investigated in this chapter is the issue of image challenges that may confound the match-state
  pairwise classifier.
Recall from~\cref{sub:exptfail}, {photobombs} --- pairs of annotations where feature correspondences are caused
  by a secondary animal --- which are the most notable cause of such a challenge.
By most accounts photobombs appear very similar to positive matches, and this similarity could confuse the
  match-state classifier.
However, because photobombs are inherently a pairwise property between annotations, it should be possible to
  learn a separate classifier explicitly tasked with the challenge.
Therefore, we also learn a photobomb classifier using the same sort of pairwise feature vector and random forest
  classifier.
This supporting classifier will allow us to increase the accuracy of our identification by restricting automatic
  classification to pairs where the decision is straightforward.


This outline of this chapter is as follows.
\Cref{sec:pairfeat} details the construction of the feature vector that we use as input to the pairwise
  classifier.
\Cref{sec:learnclf} describes the process of collecting training data and learning the match-state pairwise
  classifier.
\Cref{sec:learnpb} extends this approach to predict secondary attributes (\eg{} is a pair a photobomb) beyond
  just the matching state.
\Cref{sec:pairexpt} presents a set of experiments that evaluate the pairwise classifier.
\Cref{sec:pairconclusion} summarizes and concludes this chapter.


\section{Constructing the pairwise feature vector}\label{sec:pairfeat}

In order to use the random forest learning algorithm to address the problem of pairwise verification, we must
  construct a feature vector to representing a pair of annotations that contains information able to differentiate
  between each class.
In contrast to the unordered bag-of-features used to represent an annotation, the dimensions in this feature
  vector must be ordered and each dimension should correspond to a specific measurement.
In practice this means that the feature vector must be ordered and have a fixed length.

We construct this feature vector to contain both global and local information.
Global information is higher level and serves to augment visual information.
The local information aggregates statistics about feature correspondences between the two annotations.
The local and global vectors are constructed separately and then concatenated to form the final pairwise feature
  vector.
The remainder of this section discusses the construction of these vectors.

\subsection{The global feature vector}

The global feature vector contains information that will allow the classifier to take advantage of semantic
  labels and non-visual attributes of our data to solve the verification problem.
Semantic labels such as quality and viewpoint are derived from visual information and can provide abstract
  knowledge to help the classifier make a decision.
Non-visual attributes such as GPS and timestamp can be extracted from EXIF metadata and may help determine facts
  not discernible from visual data alone.
The global feature vector is derived from the following attributes extracted from each annotation:
\begin{enumln}

    \item Timestamp, represented in POSIX format as a float.

    \item GPS latitude and longitude, represented in radians as two floats. 

    \item Viewpoint classification label, represented as a categorical integer ranging from $1$ to $8$.

    \item Quality classification label, represented as a categorical integer ranging from $1$ to $5$.
\end{enumln}
We gather the GPS and timestamp attributes from image EXIF data, and the viewpoint and quality labels are outputs
  of the deep classifiers discussed in \cref{subsec:introdataprocess}.
The GPS and timestamp attribute inform the classifier of when its not possible for two annotations to match
  (\eg{} when a pair of annotations is close in time but far in space).
The viewpoint and quality attributes should help the classifier predict when pairs of annotations are not
  comparable --- forcing there to be stronger evidence to form a match, such as strong correspondences on a face
  turned toward the camera in both a left and right side view.
An example illustrating such a case where two annotations with different viewpoints are a positive match is
  illustrated in \cref{fig:LeftRightFace}.

\LeftRightFace{}

These four ``unary'' attributes are gathered for each annotation.
Thus, for each attribute we have two measurements, but we do not use them directly because the ordering of the
  annotations in each pair is arbitrary.
For each unary attribute, we either ignore it (as in the case of GPS and time) or record the minimum of the two
  values one feature dimension and the maximum in another (as is done with viewpoint and quality).
This results in $4$ unary measurements, $2$ for viewpoint and $2$ for quality.

The remaining dimensions of the global feature vector are constructed by encoding relationships between pairs of
  unary attributes using distance measurements.
In the case of GPS coordinates we use the haversine distance (as detailed in \cref{app:occurgroup}), but for all
  other measures we use the absolute difference of their values.
In the case of viewpoint this absolute difference is cyclic  --- \ie{} the distance between viewpoint $x$ and $y$
  is $\min(|x - y|, 8 - |x - y|)$.
This results in $4$ pairwise measurements, one for each global attribute.
Lastly, we include the ``speed'' of the pair, which is the GPS-distance divided by the time-delta.
Thus, there are a total of $4 + 4 + 1 = 9$ global measurements.

In the event that an attribute is not provided or not known (\eg{} the EXIF data is missing) a measurement cannot
be made, so a \nan{} value is recorded instead. To apply random forests learning, these \nan{} values must be
handled by either modifying the learning algorithm or replacing them with a number. Ding and Simonoff investigate
several methods for doing this in~\cite{ding_investigation_2010}, and they conclude that the best choice is
application dependent. For our application we choose the ``separate class'' method because it is simple and in
their experiments it performs the best when \nan{} values are in both the training and testing data.

The separate class method simply replaces all \nan{} measurements with an extremely large number.
This means that whenever a decision node applies its test to a \nan{} value, the result will always be the same.
In this way the \nan{} values are essentially treated as a separate category because a test can always be chosen
  that separates the measured and unmeasured data.
In the case that a \nan{} measurement in a feature dimension is informative (\eg{} if images without timestamps
  are less likely to match other annotations), the random forest can take advantage of that dimension.
However, in the more likely case that the same \nan{} measurement is uninformative, the dimension can still be
  used, but it is penalized proportional to the fraction of samples where it takes a \nan{} value.
This captures the idea that a feature dimension is less likely to be informative if it cannot be measured
  reliably.
However, if that feature dimension is highly informative for samples where it has a numeric value, then the
  random forest can still make use of it, and the samples with \nan{} values can be split by later nodes.

\subsection{The local feature vector}
The local feature vector distills two orderless bag-of-features representations into a fixed length vector
  containing matching information about a pair of annotations.
Three steps are needed to construct the local feature vector.
First we determine feature correspondences between the two annotations.
Then for each correspondence we make several measurements (\eg{} descriptor distance and spatial position).
Then we aggregate these measurements over all correspondences using summary statistics (\eg{} mean, sum, std).
Thus, the total length of the feature vector is the number of measurements times the number of summary statistics
  used.

To determine feature correspondences between two annotations, $\qaid$ and $\daid$, we use what we refer to as a
  one-vs-one matching algorithm.
Each annotation's descriptors are indexed for fast nearest neighbor search~\cite{muja_fast_2009}.
Keypoint correspondences are formed by searching for the reciprocal nearest neighbors between annotation
  descriptors~\cite{qin_hello_2011}.
For each feature in each correspondence, the next nearest neighbor is used as a normalizer for Lowe's ratio
  test~\cite{lowe_distinctive_2004}.
Because matching is symmetric, each feature correspondences is associated with two normalizing neighbors.
The feature / normalizer pair with the minimum descriptor distance is used as a normalizing pair.
If the descriptor distance between correspondences is divided by the distance between the normalizing pair is
  above a threshold, the correspondence is regarded as non-distinct and removed.
For the simplicity of the description we consider just one ratio threshold for now, but later we will describe
  this process using multiple thresholds.
Spatial verification~\cite{philbin_object_2007} is applied to further refine the correspondences.
This results in a richer set of correspondences between annotations $\qaid$ and $\daid$ than would be found using
  the ranking algorithm.
After the one-vs-one matching stage, measurements are made at each feature correspondence.

Consider a feature correspondence between two features $i$ and $j$ with descriptors $\desc_i$ and $\desc_j$.
Let $\descnorm_i$ be the normalizer for $i$, and let $\descnorm_j$ be the normalizer for $j$.
Note that while $i$ is from $\qaid$, its normalizer, $\descnorm_i$, is a descriptor from $\daid$.
Let $c \in \curly{i, j}$ indicate which feature / normalizer pair is used in the ratio test, %
$c = \argmin{c \in {i, j}}{\elltwo{\desc_c - \descnorm_c}}$.
Given these definitions, the measurements we consider are:

\begin{itemln}

    \item Foregroundness score:
    This is the geometric average of the features' foregroundness measures, $\sqrt{w_i w_j}$.
    %along with the individual foregroundness weights $w_i$ and $w_j$.
    This adds $1$ measurement, denoted as $\tt{fgweight}$, for each correspondence.

    \item Correspondence distance:
    This is the Euclidean distance between the feature correspondence, $\elltwo{\desc_i - \desc_j} / Z$.
    This serves as a measure of visual similarity between the features.
    (Recall $Z\tighteq\sqrt{2}$ for SIFT descriptors).
    This adds $1$ measurement, denoted as $\tt{match\_dist}$, for each correspondence.

    \item Normalizer distance:
    This distance between a matching descriptor and the normalizing descriptor, %
        $\elltwo{\desc_c - \descnorm_c} / Z$.
        This serves as a measure of visual distinctiveness of the features.
    We also include a weighted version of this measurement by multiplying it with the foregroundness score.
    This adds $2$ measurements, denoted as $\tt{norm\_dist}$ and $\tt{weighted\_norm\_dist}$, for each correspondence.

    \item Ratio score:
    This is the one minus the ratio between the distance between the correspondence, and the distance to the
      normalizer, %
    $1 - \elltwo{\desc_i - \desc_j} / \elltwo{\desc_c - \descnorm_c}$.
    This weights the similarity against the distinctiveness.
    Note that this is one minus the measure used to filter correspondences in the ratio test.
    We subtract the original ratio value from one in order to obtain a score that varies directly with
      distinctiveness.
    We also include a weighted version of this measurement by multiplying it with the foregroundness score.
    This adds $2$ measurements, denoted as $\tt{ratio\_score}$ and $\tt{weighted\_ratio\_score}$, for each correspondence.
    %Note, we subtract the ratio from one to ensure that larger score are better.
    %We also include a weighted version of the ratio and normalizer distance by multiplying this weight by the
    %  respective value.
        
    \item Spatial verification error:

        This is the error in location, scale, and orientation, $( %
            \Delta \pt_{i, j}, %
            \Delta \scale_{i, j}, %
            \Delta \ori_{i, j})$, as measured using~\cref{eqn:inlierdelta}.
            This adds $3$ measurements, denoted as $\tt{sver\_err\_xy}$, $\tt{sver\_err\_scale}$, and
              $\tt{sver\_err\_ori}$, for each correspondence.

    \item Keypoint relative locations:

        These are the xy-locations of the keypoints divided by the width and height of the annotation  %
        $x_i / w_{\qaid}, y_i / h_{\qaid}$ and $x_i / w_{\daid}, y_j / h_{\daid}$.
        This adds $4$ measurements, denoted as $\tt{norm\_x1}$, $\tt{norm\_y1}$, $\tt{norm\_x2}$, and
          $\tt{norm\_y2}$, for each correspondence.
        Note that unlike the global quality and viewpoint measures, we do not make an effort to account for the
          arbitrary ordering of annotations when recording these local features.
        This is to preserve the association between the spatial dimensions of each annotation.
        The same is true for the next feature.

    \item Keypoint scales:

        These are the keypoint scale parameters $\sigma_i$ and $\sigma_j$ that indicate the size of each keypoint
          with respect to its annotation.
        This adds $2$ measurements, denoted as $\tt{scale1}$ and $\tt{scale2}$, for each correspondence.

\end{itemln}

Once these $15$ measurements have been made for each keypoint correspondence we summarize them using summary
  statistics.
We consider the sum, median, mean, and standard deviation over all correspondences.
%We consider the sum, inverse-sum, mean, and standard deviation over all correspondences.
We have also considered taking values from individual correspondences based on rankings and percentiles with
  respect to a particular attribute (\eg{} ratio score), however we found that these provided little information in
  practice.
The resulting measurements are stacked to form the local feature vector.
This results in $15 \times 4 = 60$ measurements.
A final step we have found useful is to append an extra dimension simply indicating the total number of feature
  correspondences.
So, in total there are $61$ summary statistics computed for a set of feature correspondences.

As previously noted this process is repeated for multiple threshold values of the ratio test.
In our implementation we use values of $0.5$, $0.6$, $0.7$, and $0.8$.
Once we have assigned feature correspondences, we filter these correspondences using the ratio test with the
  maximum value of the ratio threshold.
Note that these threshold points are with respect to the original ratio values, and not the ratio scores used in
  the feature vector.
Then the remaining feature correspondences are spatially verified as normal.
At this point, we create a group of correspondences for each value of the ratio threshold.
The member of each group are all correspondences with a ratio score less than that value.
The local feature vector is constructed by applying summary statistics to each of these groups and then
  concatenating the results, so in essence the size of the local feature vector is multiplied by the number of
  ratio thresholds used.
Thus the total number of local measurements is $4 \times 61 = 244$.

% NOTE:
%Once we have assigned feature correspondences, we create a group of correspondences for each ratio value.
%The member of each group are all correspondences with a ratio score less than that value.
%Each group is then spatially verified, and the union of the groups is the final set of correspondences.
%When measuring spatial verification errors, each keypoint may be associated with more than one.
%Therefore, we use the minimum error over all values of the ratio threshold.

The reason we decided to use multiple ratio thresholds is because we noticed that some positive annotation pairs
  had all of their correspondences filtered by the ratio test, but if we increased the threshold then the overall
  classification performance decreased.
Including larger values of the threshold ensures that most pairs of annotations generate at least a few
  correspondences, while smaller threshold ensure that information in highly distinctive correspondences are
  separated out and considered by the classifier.
This softens the impact of the ratio test's binary threshold and adds robustness to viewpoint and pose variations
  that may cause correspondences to appear slightly less distinctive.

\FloatBarrier{}
\subsection{The final pairwise feature vector}

The final pairwise feature vector is constructed by concatenating the local and the global vector.
This results in a $253$ dimensional vector containing information that a learning algorithm can use to predict if
  a pair of annotations is positive, negative, or incomparable.
Of these dimensions, $9$ are from global measurements and $244$ are from local measurements.
The example in~\cref{fig:PairFeatVec} illustrates part of a final feature vector.

%\PairFeatVec{}
\begin{figure}
\begin{minted}[gobble=4]{python}
    OrderedDict([('global(qual_min)',    3),
                 ('global(qual_max)',    nan),
                 ('global(qual_delta)',  nan),
                 ('global(gps_delta)',   5.79),
                 ('len(matches[ratio < .8])',        20),
                 ('sum(ratio[ratio < .8])', 10.05),
                 ('mean(ratio[ratio < .8])',         0.50),
                 ('std(ratio[ratio < .8])',          0.09)])
\end{minted}
\caption[\caplbl{PairFeatVec}A pairwise feature vector]{\caplbl{PairFeatVec} %
% ---
Example of a small pairwise feature vector containing local and global information.
In the constructed pairwise feature contains $253$ dimensions.
Note the summary statistics in this example are all computed for correspondences with a ratio measurement that is
  less than $0.8$.
% ---
}
\label{fig:PairFeatVec}
\end{figure}

%This results in a final descriptor vector where  that can be quite large (thousands of dimensions) and some dimensions
%  might not be useful.
%Therefore, we prune the feature vector using only the most useful of the proposed dimensions.


\section{Learning the match-state classifier}\label{sec:learnclf}

    Having defined the pairwise feature vector the only remaining steps to constructing the pairwise classifier
      are to select a sample of labeled training data and choose a probabilistic learning algorithm.
    We will use the random forest learning algorithm, which is implemented in Scikit
      Learn~\cite{pedregosa_scikit_learn_2011}.

    The random forest learning algorithm is well understood, so we only provide a brief overview.
    A random forest is constructed by learning a forest of decision trees.
    Learning begins by initializing each decision tree as a single node.
    Each root node is assigned a random sample of the training data with replacement, and then a recursive
      algorithm is used to grow each root node into a decision tree.
    Each iteration of the recursive algorithm is given a leaf node, and will choose a random test to split the
      training data at the node into two child nodes.
    The random test is constructed by choosing a random threshold and a random subset of feature dimensions as
      candidates.
    Each candidate feature dimension splits the training data, and the one that results in the largest decrease
      in class-label entropy is chosen as the test for this node.
    The algorithm is then recursively executed on the right and left node until a leaf is assigned fewer than a
      minimum number of training examples.
    To learn our random forest classifiers we use $256$ decision trees.
    To select a test for a node, the number of candidate features dimensions we choose is the square root of the
      total number of feature dimensions.
    We stop growing a branch if a leaf node contains $5$ or fewer samples.
    Each decision tree predicts a probability distribution over all classes for a testing example by descending
      the tree, choosing left or right based on the test chosen at the node until it reaches a leaf node.
    The predicted probabilities are the proportions of training class-labels at that leaf node.
    The probability prediction of the random forest is the average of the probabilities predicted by all decision
      trees.

    Now that we have chosen a learning algorithm, the last remaining step to explain is the selection of training
      data and generation of labels.
    Recall that our the purpose of our classifier is to output a probability distribution over three labels:
    positive, negative and incomparable.
    Given a pair of annotations we need to assign one of these three labels using ground-truth data.
    In recent versions of our system, this ground-truth label is stored along with each unordered pair of
      annotations that has been manually reviewed, but because this is a new feature there does not exist many
      pairs with the explicit three-state label.
    Therefore, we must make use of data contained in previous versions of the system that simply assign a name
      label to each annotation.
    This allows us to determine if an annotation pair are the same or different animals, but it does not allow us
      to determine if the pair is comparable.
    To account for this we use heuristics to assign the incomparable label using the viewpoints, and if either
      annotation is not assigned a viewpoint it is assumed that they are comparable because most images in our
      datasets are taken from a consistent viewpoint (\ie{} collection events were designed to reduce
      incomparability).
    Thus, training labels are assigned to a pair as follows:
    use the explicit labels if they exist, otherwise heuristically decide if the pair is comparable based on
      viewpoint information, if not then return incomparable, otherwise return positive if the annotations share a
      name label and negative if they do not.

    In order to select pairs from our ground truth dataset, we sample representative pairs of annotations guided
      by the principal of selecting examples that exhibit a range of difficulties~\cite{shi_embedding_2016} (\eg
      hard-negatives and moderate positives).
    We use the LNBNN ranking algorithm to estimate how easy or difficult it might be to predict the match-state
      of a pair.
    Pairs with higher LNBNN ranks will be easier to classify for positive examples and will be more difficult for
      negative examples, and lower ranks will have the reverse effect.

    Specifically, to sample a dataset for learning, we first rank the database for each query image using the
      ranking algorithm.
    We partition the ranked lists into two parts:
    a list of correct result matches and a list of incorrect matches.
    We select annotations randomly and from the top, middle, bottom of each list.
    For positive examples we select $4$ from the top, $2$ from the middle, $2$ from the bottom, and $2$ randomly.
    For negative examples we select $3$ from the top, $2$ from the middle, $1$ from the bottom, and $2$ randomly.
    If there are not enough examples to do this, then all are taken.
    We included all pairs explicitly labeled as incomparable because there is only a small number of such
      examples.
    If this was not the case, then we would include an additional partition for incomparable cases.


\section{Secondary classifier to address photobombs}\label{sec:learnpb}
    It is useful to augment the primary match-state pairwise classifier with a secondary classifier able to
      determine if a pair of annotations contains information that might confuse the main classifier.
    These confusing annotation pairs should not be considered for automatic review.
    In annotation one of the most challenging of these secondary states is one that we refer to as a {photobomb}.
    A pair of annotations is a photobomb if a secondary animal in one annotation matches an animal in another
      annotation (\eg see Figure~\ref{fig:PhotobombExample}).
    Only the primary animal in each annotation should be used to determine identity, but photobombs provide
      strong evidence of matching that can confuse a matching algorithm.

    \PhotobombExample{}

    During events like the \GZC{} we labeled several annotation pairs as photobombs.
    Using these labels we are able to construct a classifier in the same way that we constructed the primary
      match-state classifier.
    Here we can select all pairs of annotations labeled as photobombs for positive training data.
    For negative training data we must be careful only to choose pairs that have been explicitly reviewed as not
      a photobomb.
    We attempt to balance the ratio of positive, negative, and incomparable examples in the negative training
      data.

\section{Pairwise classification experiments}\label{sec:pairexpt}

    We evaluate the pairwise classifiers on two datasets, one of plains zebras and the other of Grévy's zebras.
    These datasets were previously described in \cref{sub:datasets}.
    We will use these datasets to sample a set of annotation pairs, from which we will train and test our
      classifiers.
    The plains zebra dataset has $1202$ names with $5720$ annotations to sample from, and the Grevy's zebras
      dataset has $771$ names with $2283$ annotations.
    For each dataset we choose a sample of annotation pairs as detailed in \cref{sec:learnclf}.
    This results in a sample of $47312$ pairs for plains zebras, with $16583$ positive pairs, $30376$ negative
      pairs, and $353$ incomparable pairs.
    For Grévy's zebra we sample $18010$ pairs, where $5002$ are positive, $13008$ are negative, and $0$ are
      incomparable.
    Because our datasets only have a small number of labeled incomparable and photobomb cases, our experiments
      will primarily focus on the important question of separating positive from negative matching states.
    %When interpreting the results of these experiments it is important to note that the plains zebra dataset
    %  contains only $54$ explicitly labeled incomparable cases and $286$ labeled photobombs, while the Grévy's
    %  zebra dataset contains $0$ labeled incomparable cases and $79$ labeled photobombs.
    %For plains zebras we can increase the number of incomparable cases to $353$ using by using viewpoint
    %  heuristics.
    %Therefore, our experiments will primarily focus on the important question of separating positive from
    %  negative matching states.

    After sampling, we have a set of annotation pairs and each is associated with a ground-truth matching state
      label of either positive, negative, or incomparable.
    Additionally, each pair is also labeled as either a photobomb or not a photobomb.
    For each pair we construct a pairwise feature vector as described in \cref{sec:pairfeat}.
    We have found that, for plains zebras, it is important to use orientation augmentation when computing
      one-vs-one matches.
    In this case we should not use the second nearest neighbor as the normalizer for the ratio test, because the
      augmented keypoints will have similar descriptors.
    We account for this by using the $3\rd$ nearest neighbor as the normalizer instead.

    We split this dataset of labeled annotation pairs into multiple disjoint training and testing sets using
    grouped stratified $k$-fold cross validation (we use $3$ folds). First, we enforce that each sample (a pair of
    annotations) within the same name or between the same two names must be placed in the same group. Then, the
    cross validation is constrained such that all samples in a group are either in the training or testing set for
    each split. In other words, this means that a specific individual cannot appear in both the training and
    testing set, which helps ensure that our results will generalize to new individuals.

    For each cross validation split, we train the matching state and photobomb-state classifier on the training set
    and then predict probabilities on each sample in the testing set. Because the cross validation is $k$-fold and
    the splits are disjoint, we make a single prediction for each sample for each classifier. The result is that
    each sample in our dataset will be assigned unbiased probabilities.

    We will compare these predictions with the scores generated by LNBNN, so we must additionally generate an
      LNBNN score for each pair.
    This is done by issuing each annotation as a query.
    Any (undirected) pair in our dataset that appears as a query / database pair in the ranked list is assigned
      that LNBNN score.
    If a pair appears twice in the ranked list, then the maximum of the two scores is used.
    Any pair that does not appear in the ranked list (because LNBNN failed to return it) is implicitly given a
      score of zero.

    We have now predicted match-state probabilities, photobomb-state probabilities, and one-vs-many LNBNN scores
      for each pair in our dataset.
    We now analyze the results for each classifier in the following subsections.
    Our tests will compare the match-state classifier and the LNBNN ranking algorithm.
    However, note that the scores from the LNBNN ranking algorithm can only be used to distinguish positive cases
      from non-positive pairs.
    Unlike the match-state classifier, the LNBNN scores cannot be used to tell if a non-positive case is negative
      or incomparable.
    Later in \cref{chap:graphid} it will be vitally important to distinguish these cases, but for now, in order
      to fairly compare the two algorithms, we only consider positive probabilities from the match-state
      classifier.

    In our experiments we will measure the raw number of classification errors and success in a confusion matrix,
      and then summarize these numbers using standard classification metrics.
    We will then compare our classifier to LNBNN in two ways.
    First, we will compare the original LNBNN ranking against a re-ranking using the positive probabilities.
    Second, we will compare the ability of each algorithm to predict if a pair is positive or not.
    We will look at the distribution of positive and non-positive scores as well as the ROC curves.
    After comparing to LNBNN, we inspect the importance of each feature dimension of our pairwise feature vector.
    Finally, we will present the several examples of failure cases to illustrate where improvements are can be
      made.

    \FloatBarrier{}
    \subsection{Matching state}

        The primary classifier predicts the matching state (positive, negative, incomparable) of a pair of
        annotations. To demonstrate the effectiveness of our multiclass classifier we report the confusion matrix
        of class predictions for plains and Grévy's zebras in \cref{tbl:ConfusionMatch}.

        Using this confusion matrix we compute the average precision, recall, and Matthews correlation
          coefficient (MCC)~\cite{powers_evaluation_2011} for each class.
        These numbers are reported in \cref{tbl:EvalMetricsMatch}.
        The MCC provides a measurement of overall multiclass classification performance unbiased by class
          distribution.
        An MCC ranges from $+1$, indicating perfect predictions, to $-1$, indicates pathological inverse
          predictions, with $0$ being random uninformed predictions.
        % FIXME: ensure these numbers match the tables
        The MCC of $0.86$ for Grévy's zebras and $0.91$ for plains zebras, indicating that our classifiers have
          strong predictive power.

        \ConfusionMatch{}

        \EvalMetricsMatch{}

        In addition to being strong predictors of match state, the positive probabilities from our classifiers
          can be used to re-rank the ranked list produced by LNBNN.
        Using the procedure from our experiments in~\cref{sec:rankexpt}, we issue each annotation in our testing
          set as a query and obtain a ranked list.
        Using the fraction of correct results found at each rank, we construct a cumulative match characteristic
          (CMC) curve~\cite{decann_relating_2013}.
        We denote this CMC curve as \pvar{ranking}.
        Then we take ranked lists and compute match-state probabilities for each pair of query/database
          annotations.
        We use the positive probabilities to re-rank the lists and construct another CMC curve denoted as
          \pvar{rank+clf}.
        The results of this experiment are illustrated in~\cref{fig:ReRank}, and clearly demonstrate that the
          number of correct matches returned at rank $1$ is improved by re-ranking with our pairwise classifier.
        

        \ReRank{}
        
        \FloatBarrier{}
        \paragraph{Binary positive classification}
        Although the accuracy of multiclass predictions is important, we are most concerned with distinguishing
          positive pairs from the other classes.
        Ideally our learned classifiers will result in superior separation when compared to using just the LNBNN
          scores.
        To test this we plot histograms of scores (LNBNN vs positive probability) in \cref{fig:PositiveHist}.
        It is immediately noticeable that the pairwise scores seem to have a much better separating in addition
          to being in the interpretable range of zero to one.

        %These results demonstrate the advantage of using the pairwise classifier
        %of the pairwise classifier
        %that the pairwise classifier results in
        %interpretable scores

        We can make a more direct and precise comparison by considering both LNBNN scores and pairwise
          probabilities as binary classifiers.
        The separability of each method is measured using the area under an ROC curve.
        The ROC curves for plains and Grévy's zebras are illustrated in \cref{fig:PositiveROC}.
        For plains zebras, the pairwise AUC of $0.96$ is better than the LNBNN AUC of $0.94$.
        For Grévy's zebras the pairwise AUC of $0.99$ is convincingly better than the LNBNN AUC of $0.90$.

        \PositiveHist{}

        \PositiveROC{}


        %\FloatBarrier{}
        \paragraph{Feature importance}

        To gain an intuition for which features are most important we create a word cloud where the size of the
          text is proportional to the importance of the feature as measured using the ``mean decrease impurity''
          (MDI)~\cite{louppe2014understanding}.
        The MDI is a measure of feature importance that is computed during the training phase.
        As a decision tree is grown, the number of training samples of each class that reach a particular node
          are recorded.
        This is used to compute the impurity of each node, which is the entropy of class labels.
        The fraction of total samples that reach a node is its weight.
        The weighted impurity decrease of a node is its weighted impurity minus the weighted sum of its
          children's impurity.
        The MDI for a single feature dimension in a single tree is computed as the weighted impurity decrease of
          all nodes using that feature.
        The overall MDI for the forest is obtained by averaging over all trees.

        The word cloud is illustrated in \cref{fig:MatchWordCloud}.
        The numeric importance of the top features is recorded in \cref{tbl:ImportantMatchFeat}.

        For plains zebras, the global viewpoint measures and the local scales of corresponding keypoints are the
          most important features.
        This makes sense because the global viewpoint helps distinguish negative from incomparable cases, and the
          keypoint scale helps determine if matches are being made on a coarse level (\eg{} matching general zebra
          shapes) or a fine detailed level (\ie{} capturing the smaller features that actually distinguish
          individuals).
        For Grévy's zebras, different statistics about the ratio measures dominate.

        \MatchWordCloud{}
        \ImportantMatchFeat{}

        We also consider if removing the least important feature dimensions might have a positive impact on
          classification performance.
        To measure this, we consider a greedy algorithm.
        First we learn a random forest on a training set and compute the MCC on a test set.
        Then we measure the least important feature dimension as the one with the lowest MDI and remove it from
          the dataset.
        We repeat these two steps until there is only a feature dimension remaining.
        In~\cref{fig:Prune} we plot the MCC as a function of the number of features.

        The results of this experiment indicate that there is little to no increase in classification accuracy
          from pruning features dimensions.
          However, once the number of feature dimensions falls below ${\sim}50$, the performance starts to noticeably
          degrade.
        This suggests that removing any of these top $50$ features would degrade performance. 

        For Grévy's zebras these $50$ dimensions are composed of $2$ global measures:
        $\tt{delta\_gps}$ and $\tt{speed}$, and statics about $9$ local measurements:
        $\tt{match\_dist, norm\_dist, norm\_y1, norm\_y2, ratio, scale1, sver\_err\_ori, sver\_err\_scale,}$ and
          $\tt{sver\_err\_xy}$.

        However, one weakness of this test is that the MDI does not measure correlation between features.
        It might be the case that it if two feature are highly correlated (\eg{} the mean and median ratio
          values), it might not be necessary to include both.
        Correlated feature dimensions are also known to decrease the accuracy of individual decision trees.
        However, it is also known that this effect is alleviated by the randomization process and when averaging
          over multiple decision trees~\cite{louppe2014understanding}.
        It is for this reason and the fact that the $MCC$ does not significantly improve that we chose to use all
          $253$ dimension of our feature vector in the other experiments.

        \MatchPrune{}
        
        \paragraph{Failure cases}

        Lastly we investigate several failure cases.
        First, it is not surprising that incomparable pairs like the one illustrated in \cref{fig:PairFailIN}
          might be labeled as incomparable because of the lack of incomparable training data.
        The previously discussed failure ranking cases from ~\cref{sub:exptfail} such as viewpoint and quality
          are inherently challenging and also cause the pairwise classifier to fail as illustrated
          by~\cref{fig:PairFailPN}.
        Lastly, sometimes the pairwise classifier is simply unable to key in on cues that a set of one-vs-one
          correspondences is not distinctive like when it predicts that the pair in ~\cref{fig:PairFailNP} is
          positive instead of negative.
        However, note that in all of these examples that the non-extreme probabilities assigned to each state
          demonstrate that the classifier is not confident in these predictions.
        We will take advantage of this in the next chapter.
        

        %illustrate failure cases: TODO Match State Failure Cases
        \PairFailIN{} 

        \PairFailPN{}

        \PairFailNP{}

        \FloatBarrier{}


    %---------
    \FloatBarrier{}
    \subsection{Photobomb state}
        We perform a subset of similar experiments to demonstrate the effectiveness of our photobomb classifier.
        The overall performance is given in the classification confusion matrix illustrated in
          \cref{tbl:ConfusionPhotobomb} and evaluation metrics are given in \cref{tbl:EvalMetricsPhotobomb}.

        Due to the small amount of available training data the performance of the photobomb-state classifier is
          weaker than the match-state classifier.
        For plains zebras the MCC of $0.64$ demonstrates that the photobomb classifier has moderate predictive
          power.
        For Grévy's zebras the MCC of $0.31$ demonstrates weak but significant predictive power.

        While the small number of training examples makes it difficult to draw conclusions, there are two
          observations we can make.
        First, plains zebras have more photobomb examples than Grévy's zebras, and the photobomb-state MCC is
          also much higher for that dataset.
        This suggests, that with more training data the quality of the classifier could improve considerably.
        The second observation is that the number of false negatives is lower than both the number of true
          positives and false positives --- \ie{} the precision of the classifier is high.
        This means that the classifier correctly flags most photobomb cases it considers.
        Therefore, it can be used to prevent the match-state classifier from automatically classifying these
          pairs as positive.
        However, in its current state, it also incorrectly prevents automatic classification of a significant
          number of pairs.

        %The overall quality of the secondary classifier is not ideal for both species, but the small amount of
        %  training data makes it difficult to draw conclusions from these experiments.

        %As the important features in~\cref{tbl:ImportantPBFeat} illustrate, the photobomb classifier seems to
        %  rely on global information such as time delta, GPS delta, and speed, because photobombs are more likely
        %  when taking images of several animals in a small time period.
        %The classifier also makes use of the spatial position of the feature correspondences, which can be
        %  indicative of a photobomb (\eg{} when all the matches are in the top left corner of an annotation).
        %This suggests that with more training data the classifier could be trained reliably used to predict when
        %  a match is a photobomb.

        \ConfusionPhotobomb{}

        \EvalMetricsPhotobomb{}

        %\ImportantPBFeat{}

    \FloatBarrier{}
    \subsection{Classifier experiment conclusions}
        In these experiments we have demonstrated that our pairwise match-state classifier is able to reliably
          separate positive from negative and incomparable cases.
        When compared to the LNBNN scores from~\cref{chap:ranking}, not only do these probabilities have more
          predictive power, they are interpretable, always ranging between $0$ and $1$.
        In practice, we have found that a validation dataset can be used to select a threshold --- to
          automatically classify pairs --- where the false positive rate is sufficiently low.
        %In practice, we have found these probabilities to be well calibrated, meaning that a validation dataset
        %  can be used to select a threshold --- to automatically classify pairs --- where the false positive rate
        %  is sufficiently low.
        Because the positive cases are well separated from the negative and incomparable cases, a significant
          number of automatic reviews is possible.
        Furthermore, because the classifier was trained using hard, moderate, and easy training examples, it is
          able to correctly re-rank results of the ranking algorithm, where the correct match was ranked in highly
          but not rank $1$.
        Lastly, because the classifier predicts probabilities independent of their position in the ranked list,
          it can be used to determine when a query individual is new --- \ie{} does not have a correct match in the
          database.

        The performance of secondary photobomb classifier is weaker, but this is likely due to a small amount of
          training data.
        Even in its weak state, it can be used to prevent automatic review of some photobomb cases.
        Because the photobomb classifier can only prevent automatic decision and does not make them, the cost of
          including it in our algorithms is small, and by doing so we will increase the amount of labeled training
          data from which a stronger photobomb classifier can be bootstrapped.
        %Therefore, the pairwise algorithm can be used to re-rank the of the ranking algorithm.


\section{Summary of pairwise classification}\label{sec:pairconclusion}

    In this chapter we have constructed a verification mechanism that can predict the probability that a pair of
    annotations is positive, negative, or incomparable. We have also constructed a secondary classifier that can
    predict when --- namely in the case of photobombs --- a pair of annotations might confuse the primary
    match-state classifier. This was done by constructing a feature vector that contains matching information about
    a pair of annotations. We have constructed a representative training set by selecting hard, moderate, and easy
    training examples. Our classifiers were learned using the random forest learning algorithm. Our experiments
    demonstrate that the match-state classifier is able to strongly separate positive and negative cases. The
    performance of the photobomb classifier was weaker, but could likely be improved with more training data.

    Based on our experiments, it is clear that the ranking algorithm is improved by an automatic verifier, but by
      themselves ranking and verification are not enough to robustly address animal identification.
    There is no mechanism for error recovery, nor is there a mechanism for determining when identification is
      complete.
    This means, the automatic review threshold must be set conservatively to avoid any errors, which results in
      more work for a human reviewer.
    These issues are addressed in \cref{chap:graphid} using a graph-based framework to manage the identification
      process.
    This framework will detect when errors have occurred, recover from the errors, and stop the identification
      process in a timely manner.

\begin{comment}
fixtex --fpaths chapter5-graphid.tex --outline --asmarkdown --numlines=999 --shortcite

fixtex --fpaths chapter5-graphid.tex --outline --asmarkdown --numlines=999 --shortcite -w && ./checklang.py outline_chapter5-graphid.md

fixtex --fpaths chapter1-intro.tex --outline --asmarkdown --numlines=999 --shortcite -w && ./checklang.py outline_chapter1-intro.md
\end{comment}

% TODO: relate back to GZC, and talk about non-rally scenarios


\chapter{Animal identification using connectivity in a decision graph}\label{chap:graphid}
\newcommand{\nT}{N}

In this chapter we frame the problem of animal identification in terms of constructing a %
\glossterm{decision graph}.
In this graph, each node is an annotation, and each edge represents a decision made between two annotations.
Edges determine if two annotations are the same (positive) or different (negative) individuals or if they cannot
  be compared (incomparable).
Using the connectivity of the decision graph, we naturally address the problem of animal identification.
Assuming no edges are incorrectly labeled, each connected component of positive edges are all annotations from
  the same individual animal.
We will refer to these as \glossterm{positive connected components} (PCCs).
We call a graph \glossterm{id-complete} if there is at least one negative edge between all pairs of PCCs.
In this case all labeled individuals must be distinct, and we have therefore completed identification.
Alternatively, if every pair of PCCs either has a negative edge between them except for pairs of PCCs where all
  possible edges between them are incomparable, then we know that the data cannot be used to determine if those
  incomparable PCCs are the same or different.
Removing the assumption that all edges are correctly labeled, we call a decision graph \glossterm{inconsistent}
  if any PCC contains a negative edge because it implies that an edge has been mislabeled.
Therefore, stated abstractly, goal of graph identification is to determine a correct, consistent, and id-complete
  set of edges in the decision graph.


In the most general case, the decision graph is initialized with an empty set of edges.
This captures the challenge posed by events like the \GZC{} from~\cref{sec:introgzc}, where we are given a set of
  annotations without name labels.
In essence, each annotation starts by itself as an individual animal, but because there are no negative edges, we
  cannot be sure that this is the correct name labeling.
In order to refine the name labeling, we add edges to the decision graph, gradually moving from a state of zero
  confidence that the name labelings are correct to a state of high confidence.
However, it is important to note that the graph algorithm does not require that we start in an empty state.
Given a known set of annotations with name labels and one or more new annotations with unknown name labels, we
  can add these new annotations to the existing database simply by labeling the graph with edges that captures our
  current knowledge.
Simply put, this means each known individual is a PCC, there is one negative edge between each pair of PCCs, and
  the new annotations are added as nodes without any edges.
Regardless of the initial state, graph identification proceeds to complete the decision graph.
For the remainder of this chapter, without loss of generality, we can assume that the graph starts in an empty
  state.

%The name labelings --- \ie{} the set of PCCs --- are refined as new edges are added to the graph.

%Akin to a segmentation algorithm~\cite{fulkerson_class_2009} that starts with an over-segmentation of an image,
%  the identification graph starts with an empty set of edges, $G = (V, \{ \})$, so in essence each annotation
%  starts by itself as an individual animal.
%we must discard all of those PCCs except those in an independent
%set of a meta-graph where the PCCs are nodes and edges are formed between incomparable PCCs.

To construct the decision graph, we develop a semi-automatic review procedure that combines the ranking and
  verification algorithms presented in \cref{chap:ranking,chap:pairclf}.
The ranking algorithm will be used to suggest candidate edges to be placed in the graph, and the verification
  algorithm will be used to automatically review as many edges as possible.
The key reason for combining these algorithms with a decision graph is to take advantage of its connectivity
  information.
Connectivity not only identifies the individuals, but it can also be used to develop graph measures of
  \emph{redundancy}, \emph{completeness}, \emph{consistency}, and \emph{convergence}.
By combining these graph measures with the ranking and verification algorithms we can prioritize edges for review
  based on both their pairwise probabilities and their ability to affect the consistency of the graph, which in
  turn allows us to:
\begin{enumin}
\item increase confidence that the identifications are correct, %
\item reduce the number of manual reviews,  % 
\item detect and recover from review errors, and %
\item determine when identification is complete. %
\end{enumin}

An important property of the graph identification framework is that it is agnostic to the underlying computer
  vision procedures, which are abstracted into three components:
\begin{enumin}
\item a ranking algorithm used to search for candidate positive edges, %
\item a verification algorithm used to automatically review edges, and %
\item a probability algorithm used assign probabilities to edges (note this is typically a by-product of the
  ranking or verification algorithm).
\end{enumin}
In this thesis we use ranking algorithm from \cref{chap:ranking}, and the verification algorithm from
  \cref{chap:pairclf} to define these components because these are suitable for identifying textured species.
However, while graph identification benefits from accurate computer vision subroutines, it can stand alone
  without them.
This means that existing identification algorithms that only define a subset of these procedures (\eg{}
  contour-based rank-only identification of humpback whales and bottlenose dolphins) could be seamlessly
  incorporated into our framework and realize the benefits of graph identification (\eg{} a reduced number of
  manual reviews and error recovery mechanisms).
Furthermore, because pairwise decisions are gathered and maintained by this framework, verification algorithms
  can be retrained and improved, moving closer to a fully-automatic algorithm.


The first section (\Cref{sec:decisiongraph}) of this chapter formalizes the decision graph and summarizes the
  priority-based review procedure used to construct it.
This provides an overview of each component of the processes.
Each of these components is then discussed in further detail
  in~\cref{sec:redun,sec:incon,sec:cand,sec:decision,sec:converge}.
\Cref{sec:graphexpt} experimentally demonstrates the ability of the graph identification algorithm to reduce the
  number of manual reviews and recover from decision errors.
\Cref{sec:graphconclusion} concludes and summarizes the chapter.


%%%%%%%%%%%%%%%%%%%%%%%%%%%%%%%%%%%%%%%%%%%%%%%%%%%%
\FloatBarrier{}
\section{The decision graph}\label{sec:decisiongraph}

The graph identification algorithm is a review procedure formalized around the notion of a \glossterm{decision
  graph} $G = (V, E)$ whose nodes are annotations and whose edges are suggested by a ranking algorithm (LNBNN in
  our case) and decided upon by a combination of the probabilities output by a verification algorithm and by manual
  review.
The edge set $E = E_p \cup E_n \cup E_i$ is composed of three disjoint sets.
Throughout this chapter we will refer to the set that an edge belongs to as the label of that edge.
Each edge in $E_p$ is \emph{positive}, meaning that it connects two annotations determined to be from the same
  individual.
Each edge in $E_n$ is \emph{negative}, meaning that it connects annotations determined to be from different
  individuals.
Finally, each edge in $E_i$ is \emph{incomparable}, meaning that it connects two annotations where it has been
  determined that there is not enough information to tell if they are from the same individual (\eg{} when one
  annotation shows the left side of an animal and another other shows the right side).
An example of a decision graph with all three edge types is illustrated in \cref{fig:decisiongraph}.
The goal of graph identification is to construct these edges.

\decisiongraph{}

The most important task is to determine the positive edges $E_p$.
This is because each connected component in the subgraph $G_p = (V, E_p)$ corresponds to a unique individual.
Producing an accurate set of these \glossterm{positive connected components} (PCCs) addresses the larger problem
  of animal identification.
However, an algorithm that only determines positive edges is not enough.
This is because the algorithm may have failed to find all positive edges, resulting in two unconnected PCCs that
  should be \emph{merged} into one.
To ensure that this is not the case we must turn towards negative edges.

We can gain confidence that all positive edges have been found by using negative edges $E_n$, which provide
  direct evidence that two annotations are different individuals.
A correctly labeled negative edge between two PCCs means that no other unreviewed edge between those PCCs can be
  positive.
Another important case is when a negative edge is contained within a PCC.
When this happens, the PCC is \emph{inconsistent}, and it implies that the PCC contains at least one mislabeled
  edge.
Whenever an inconsistency is detected, we resolve it using the algorithm that we will define in~\cref{sec:incon}.

Lastly, incomparable edges, $E_i$, simply signify that a positive or negative decision cannot be made.
Whenever an edge is not labeled as positive, it is critically important to the construction of the decision graph
  that this non-positive edge is distinguished as either negative or incomparable.
Negative edges restrict what new edges can be added because they carry information about the completeness and
  consistency of the graph.
In contrast, incomparable edges do not.
Incomparable edges can exist internally in a PCC without causing inconsistencies or between two PCCS without
  precluding them from being matched at a later point.
%and , whereas incomparable edges do not.
%To be able to use this negative information, we must recognize the distinction between non-positive and negative
%  edges.
%Recall, that this was the basis of our $3$-state classification algorithm in \cref{chap:pairclf}.
%Algorithms like the ranking algorithm from~\cref{chap:matching} do not make this distinction, and therefore
%  cannot make use of the negative information carried by negative edges.
In the case where all edges between two PCCs are incomparable and those PCCs are complete, then we know the
  current data is not enough to determine if those PCCs are the same or different.
In our datasets most images were taken to reduce the number of incomparable PCCs in order to simplify the
  sight-resight analysis, where all PCCs must be comparable to each other, but incomparable cases do exist.
Thus, in our context incomparable edges play a minor but necessary role.
%Note that sight resight analysis can be accomplished on sets with incomparable edges using the algorithm
%  in~\cref{app:markrecapincomp}.


%datasets where 
%play a minor but necessary role by 

Using the connectivity of these edges, we can reduce the number of manual reviews needed.
We now make several observations assuming that each edge is reviewed correctly.
To reduce the number of potential reviews, notice, that once a group of nodes is connected by (a tree of)
  positive edges, all those nodes in that PCC can be inferred to belong to the same individual, and it is not
  necessary to consider any other edge internal to the PCC for review.
Likewise, once a negative edges has been placed between two PCCs, all edges between those PCCs can be ignored.
By ignoring these redundant edges we can reduce the number of reviews.
Furthermore, we can construct a \glossterm{deterministic termination criterion}; if a negative edge is placed
  between every pair of PCCs, then all individuals must have been discovered and identification has converged.
We call such a graph id-complete because when all vertices in a PCC are collapsed into one, the resulting
  meta-graph is complete.

Unfortunately, there are two issues with these observations.
First, they depend on the condition that each edge was correctly reviewed.
In fact, we know that both the verification algorithm and human reviewers are sometimes wrong.
We therefore will introduce redundancy into our graph that allows the algorithm to detect and correct errors,
  trading off the level of redundancy with the sophistication of the errors that may be caught.
A small amount of redundancy is desirable because:
\begin{itemln}
    \item PCCs with redundant edges are less likely to contain errors.
    \item Redundancy can potentially introduce inconsistency into a PCC, which signifies that an error has
      occurred.
\end{itemln}
Therefore, we will define a redundancy criterion in~\cref{sec:redun} which ignores edges within and between PCCs,
  but only after they meet a minimum level of redundancy.

Additionally, the deterministic termination criterion would require that each pair of PCCs has a redundant set of
  negative edges between them before the algorithm stop.
However, the number of edges that need review grows quadratically.
Therefore, unless the automatic algorithm is perfect or the dataset is small, the number of negative reviews
  needed to converge will be too large for a human to handle.
We address this concern in~\cref{sec:converge} using a probabilistic termination criterion.
%While this criterion will not guarantee ,
%that the graph is complete

%it will ensure that the algorithm terminates 


\FloatBarrier{}
\subsection{The review algorithm}\label{sub:graphalgo}


% Algorithm overview
The review algorithm that produces the edges of a decision graph is outlined in Algorithm~\ref{alg:AlgoOverview}.
Akin to a segmentation algorithm~\cite{fulkerson_class_2009} that starts with an over-segmentation of an image,
  the identification graph starts with an empty set of edges, $G = (V, \{ \})$, so in essence each annotation
  starts by itself as an individual animal, but because there are no negative edges we cannot be confident that any
  pair of annotations should indeed be given different labels.
Therefore, the algorithm proceeds prioritize and add positive edges that merge multiple annotations into the same
  PCC, negative edges that indicate that two annotations are different individuals, and incomparable edges that
  prevent two annotations from being labeled as positive or negative.
Throughout the main algorithm, the graph is maintained in a \emph{consistent} state, which means that each PCC
  has no internal negative edges.
Note, that while we describe the algorithm starting from an empty graph, without loss of generality, the edges in
  the graph can be initialized to reflect a previously known labeling.
Thus, the algorithm can address the case where nothing is known, new images are being added to an existing
  dataset, or multiple datasets are being combined.


\begin{algorithm}
        While the graph has not converged:
        \begin{enumln}
            \item Generate and prioritize candidate edges 
            \item Insert candidate edges into a priority queue 
            \item While the priority queue is not empty:
            \begin{enumln}
                \item Pop an edge from the priority queue
                \item Make a decision and add the edge to the graph
                \item If the edge causes an inconsistency drop into inconsistency recovery mode
                \item Update the priority queue based on the new edge
                \item If candidate edges require refresh, break
            \end{enumln}
        \end{enumln}
\caption[Algorithm Overview]{Overview of the graph identification review procedure}
\label{alg:AlgoOverview}
\end{algorithm}

The first step of the algorithm is to generate candidate edges predict probability measures (positive, negative,
  or incomparable) for each candidate edge.
In the next step each edge is then entered into a priority queue with a priority based first on its ability to be
  automatically reviewed and then on its positive probability.
Next, the algorithm enters a loop where the next candidate edge is selected, a decision is made about this edge
  --- either automatically (as much as possible) or by the user --- and it is added to the graph.
The algorithm proceeds toward convergence by removing candidate edges from the priority queue, either directly
  from the top of the queue or indirectly by eliminating candidate edges that are no longer needed.
A candidate edge is no longer needed when there are sufficient redundancies in the edge set within or between its
  PCCs.
A pair of PCCs is \emph{complete} when there are enough negative edges between them.

Each new edge addition could trigger two important events:
\begin{enumin}
    \item a \emph{merge} --- addition of a positive edge between different
      PCCs combines them into one PCC, and

    \item an \emph{inconsistency} --- addition of either a negative edge within a PCC or a positive edge between
      PCCs that already have a negative edge between them creates an inconsistent PCC.
\end{enumin}
Handling a merge is largely a matter of bookkeeping and can be done efficiently using a data structure that can
  dynamically maintain connected components~\cite{jacob_holm_poly_logarithmic_1998}.
Finding an inconsistency, however, drops the user into inconsistency recovery mode which alternates between
  hypothesizing one or more edges to fix and manually verifying these edges with the user until consistency is
  restored.

Finally, the outer loop of the overall algorithm allows the ranking algorithm to generate additional candidate
  edges --- this allows the ranking algorithm to take advantage of more subtle matches as the PCCs begin to form.
The priority queue will gradually be emptied as each PCC obtains a sufficiently redundant set of positive edges
  and enough negative edges to be complete.
%Ensuring completeness requires examining $O(|V|^2)$ edges, so in practice we
%  develop a learned probabilistic completeness measure.
%If sufficient training data is not available simple heuristics can be used to
%  terminate.

Details of each step in the review algorithm are described in the following sections.
First we describe the redundancy criterion in~\cref{sec:redun}.
Then we will define candidate edge generation in~\cref{sec:cand} and decision-making in~\cref{sec:decision}.
Inconsistency recovery is covered in~\cref{sec:incon}.
Finally, we describe the refresh and termination criteria in~\cref{sec:converge}.

% will be emptied when each PCC is sufficiently re
% Obtaining sufficient redundancy and completeness in order to empty the priority queue can,
% Ensuring all PCCs are complete leads to the need to ,
%  so to prevent this we develop a probabilistic measure (\cref{sec:converge})
%  that triggers much earlier convergence when positive edges are no longer
%  likely to be found.

\FloatBarrier{}
\section{Positive and negative redundancy}\label{sec:redun}
%One paragraph on notion.
%One paragraph on algorithm.
%One paragraph on book-keeping and elimination from priority queue.

In this section we define criteria that
(1) increases our confidence that PCCs are correct by enforcing a minimum level of redundancy and
(2) prevents edges that exceed this redundancy from being reviewed.
At a minimum each PCC must be a tree of positive edges, but when errors can occur, it's difficult to be confident
  that all nodes in the PCC are really annotations from the same individual.
By adding a redundant edge we either increase the confidence that other edges are correct or detect an
  inconsistency which can be resolved with the algorithm in ~\cref{sec:incon}.
However, the gains in confidence from adding each additional edge are diminishing.
Therefore, it is desirable to achieve a minimum level of redundancy, but once this has been achieved we should
  prevent additional redundant edges from being reviewed.
We formalize this minimum level of redundancy in two forms.
The first is for positive edges within PCCs and the second is for negative edges between PCCs.
\begin{enumln}

    \item positive-redundancy --- % 
        A PCC is $k$-positive-redundant if its positive subgraph is $k$-edge-connected (contains no cut-sets
          involving fewer than $k$ positive edges~\cite{eswaran_augmentation_1976}), or if the PCC has $k$ or fewer
          nodes and the union of positive and incomparable edges is a complete graph.

    \item negative-redundancy --- % 
        A pair of PCCs $C$ and $D$ is $k$-negative-redundant if there are $k$ negative edges between $C$ and $D$,
          or if there are $|C| \cdot |D|$ negative or incomparable edges between them.
          %or if both PCCs have fewer than $k$ nodes and there are at least $\mathop{max}(|C|, |D|)$ negative edges
          %between them.
        
    %which can be determined in $O(n_1 n_2)$ time.
    %(by looping over adjacency sets of nodes
    %in $C$ and performing set intersection with nodes in $D$ to get the edges
    %between $C$ and $D$).

    %$k$-negative-redundant if there are $k$ negative edges between them.
\end{enumln}
\kredun{}

The example in \cref{fig:kredun} illustrates different levels of redundancy.
To understand these criteria better, consider what it means for a PCC that has been determined to be
  $k$-positive-redundant to have an undiscovered error.
The error means that the PCC really should be split into (at least) two separate PCCs.
Suppose these PCCs correspond to animals $C$ and $D$.
If the combined PCC is $k$-positive-redundant then are $k$ separate undiscovered mistakes connecting $C$ and $D$,
  and there must also be no negative edges connecting $C$ and $D$.
This may be plausible if $C$ were identical twins, but these tend not to occur for species where the
  distinguishing markings (\eg{} hip and shoulder of zebras) are mostly random.
In other words, an error only becomes undiscoverable if a reviewer makes the same mistake with different
  annotations from the same individual $k$ times.
Note that $k$ can be different for positive and negative redundancy, but in our current implementation we use
  $k=2$ for both positive and negative redundancy.


\subsection{Checking redundancy}

We now describe how to check if a consistent PCC with $n$ nodes and $m$ edges is $k$-positive-redundant.
An PCC that contains a negative edge is inconsistent and is never considered as positive-redundant.
In the case where $n \leq k$, the PCC is positive redundant if all possible edges are either positive or
  incomparable.
This can be determined in $O(m)$ by checking if the sum of the number of positive and incomparable edges between
  the nodes in the PCC is equal to $\binom{n}{2}$.
Otherwise, in the more interesting case, when $n > k$, a PCC is positive-redundant iff it is $k$-edge-connected
  --- \ie{} it is impossible to disconnect the nodes in the PCC by removing any set of $k - 1$ edges.
In practice, we are primarily concerned with the case when $k=2$.
The special case of $2$-edge-connectivity is also called bridge-connectivity, and this can be determined in %
$O(n + m)$~\cite{eswaran_augmentation_1976,wang_simple_2015}.
The basic idea is to trace cycles encountered during a depth-first-search of the graph and mark all edges that
  are part of a cycle.
Any unmarked edge is not part of any cycle, and is called a bridge.
Removing any bridge edge would disconnect the PCC.
Thus, if no bridge exists then the PCC is $2$-edge-connected.

In the general case when $k>2$, edge-connectivity can be determined in $O(m n)$ amortized
  time~\cite{esfahanian_connectivity_2017}.
This involves first computing a small (not necessarily the smallest) dominating set.
A small dominating set can be using a greedy algorithm that starts with an empty set, and iteratively adds an
  arbitrary node that is not in or adjacent to the current set until all nodes are in or adjacent to that set.
Then an arbitrary node in the dominating set is chosen.
The local-edge-connectivity is computed between the chosen node and every other node in the dominating set.
The local-edge-connectivity between two nodes is simply the maximum flow between those nodes, if all edge
  capacities are equal to $1$.
The edge-connectivity of the entire PCC is the minimum of
(1) the minimum degree of the PCC and
(2) the minimum computed local-edge-connectivity.


We now describe how to check if two PCCs are a consistent PCC  $C$ and $D$ with sizes $n_1$ and $n_2$ are
  $k$-negative-redundant.
This can be done in $O(n_1 n_2)$ time using adjacency lists and set intersections to check if the number of
  negative edges between the nodes in $C$ and $D$ is greater than $k$.
For each node in $C$ we simply check if any node in $D$ is in the adjacency list (stored as a set) of that node.
The number of times this is true is the number of negative edges between $C$ and $D$.

%Whenever a positive edge is added inside a PCC, we can check positive-redundancy, and if it is, we can remove all
%  other edges inside that PCC from the priority queue.
%Similarly, when a negative edge is placed between two PCCs, we can check negative redundancy and, if it is
%  satisfied, remove all other edges between those PCCs.

Using this redundancy criterion we are able to find and remove edges from the priority queue that are no longer
  needed.
When a positive edge is added within a single PCC, we check for positive-redundancy.
If this passes, all remaining internal edges for that PCC may be removed from the priority queue.
When a negative edge is added between a pair of PCCs, we run the negative-redundancy check on the pair, and if
  this passes, all remaining edges between the PCCs may be removed from the priority queue.
When a positive edge is added between a pair of PCCs, the two PCCs are merged into a single new PCC $C'$, and the
  above negative-redundancy check must be run between $C'$ and all other PCCs having a negative edge connecting to
  $C'$.
It can be shown that if the graph is in a consistent state, that these are the only updates required.

As a final note, consider the case where a PCC is composed of two positive $k$-edge-connected subgraphs joined by
  a single positive edge.
While the entire PCC is not positive redundant, much of it is.
In this case, we do not need to review any edge within any $k$-edge-connected component of the PCC.
We can dynamically remove these from the priority queue by checking when a new edge popped off of the priority
  queue is within an existing PCC.
We can check if the local-edge-connectivity~\cite{esfahanian_connectivity_2017} --- \ie{} the maximum flow ---
  between the edge's endpoints is at least $k$.
If local-edge-connectivity between those nodes is at least $k$, then that edge is part of a $k$-edge-connected
  component and can be ignored.


\subsection{Redundancy augmentation}\label{subsec:augredun}

In addition to determining if existing edges are redundant, it would be useful determine a small set of edges
  that would make an existing PCC positive-redundant or two PCCs negative-redundant.
Reviewing these edges would help expose any undiscovered errors in the graph.

To find edges that would complete positive-redundancy for a PCC, we compute a $k$-positive-augmentation.
This is equivalent to the problem of finding a minimum $k$-edge-augmentation.
When $k=2$ the problem is called bridge augmentation and can be computed $O(m)$~\cite{eswaran_augmentation_1976}
  as long as all edges in the complement of the PCC can be used in the augmentation.
However, if the PCC contains incomparable edges, then these edges cannot be used.
We can address this by using a weighted variant of the problem, setting the weights of the incomparable edges to
  $\inf$, and searching for a minimum cost augmentation.
However, the weighted variant of this problem is NP-hard, even for $k=2$, but can be approximated within a factor
  of $2$~\cite{khuller_approximation_1993}.
We make use of this algorithm later in~\cref{sec:cand}

A set of edges that would complete negative-redundancy for a pair of PCCs, we compute a
  $k$-negative-augmentation.
Computing this set of augmenting edges is trivial.
Initialize an empty augmenting set.
Iterate through all combinations of nodes between the two PCCs.
For each combination of nodes, if there is no existing edge between those nodes in the graph, add it to the
  augmenting set.
Once $k$ edges have been added to the augmenting set or there are no more node combinations, stop and return the
  augmenting set.
Note, that we do not use this algorithm in practice because we use a probabilistic termination criteria instead
  of enforcing that all pairs of PCCs are negative-redundant.

%If the PCC only contains positive edges, edge augmentation with the fewest edges can be computed in linear time
%  using~\cite{eswaran_augmentation_1976}.
%However, if some edges in the complement of the positive PCC edges are incomparable then we must compute a
%  minimum weight edge augmentation (which is NP-hard) using a $2$-approximation algorithm
% ~\cite{khuller_approximation_1993}.


\section{Candidate edge generation and priorities}\label{sec:cand}

In this section we describe the first step of the algorithm where candidate edges are generated and then
  prioritized for review.
There many ways that candidate edges can be generated.
Different sets and orderings of candidate edges will impact different properties of the graph at different rates.
Therefore, we choose candidate edges to depend on what properties of the graph we want to manipulate.
In the context of identifying individual animals, the most obvious manipulations is to reduce the number of PCCs
  in the graph by adding positive edges between existing PCCs, thus merging them into one.
A less obvious property that we can manipulate is the fraction of PCCs that are positive redundant.
By increasing this fraction to $1$, we discover mistakes that have been made, which can be fixed using the
  algorithm in~\cref{sec:incon}.
This ultimately decreases the likelihood that any PCC contains a mislabeled positive edge.
In each iteration outer loop of our main algorithm, we alternate between first generating candidate edges to
  merge PCCs, and then generating candidate edges to ensure that all PCCs are positive redundant.

To generate candidate edges that may merge existing PCCs we use the LNBNN ranking algorithm
  from~\cref{chap:ranking}.
We issue each annotation as a query to the ranking algorithm, and form edges from the top results of the ranked
  lists.
We then assign a priority to each new candidate edge.
We use the pairwise algorithm from~\cref{chap:pairclf} to estimate the positive, negative, and incomparable
  probabilities of each edge.
Any edge whose maximum positive, negative, or incomparable probability is above the threshold for automatic
  decision-making is ranked according to this probability.
All other edges are ordered by their positive probability.
This ensures automatic decision-making is first, followed by an ordering of the edges needed for manual review
  that are most likely to be positive and therefore add the most information to the graph.
It is desirable to add positive decisions to the graph first because
(1) they are the most important edges with respect to determining the animal identities, and
(2) larger PCCs increase the number of edges that can be skipped using the redundancy criterion.

The first iteration of the outer loop the algorithm generates edges using the ranking algorithm.
Review of these edges continues until the priority queue is empty, or we determine that the candidate edges
  should be refreshed using the algorithm we will describe in~\cref{sec:converge}.
This ends the current inner loop, and as long as the algorithm has not converged, it proceeds to the next
  iteration of the outer loop.
In this next iteration, we generate candidate edges to ensure that all PCCs are positive redundant.

To generate candidate edges that will make PCCs positive redundant, we use the positive-augmentation algorithm
  from~\cref{subsec:augredun} to generate edges.
These edges are assigned prioritized in the same way, however in this iteration the refresh criterion is
  disabled.
This enforces that if each edges is reviewed as positive, then all PCCs will be positive-redundant at the end of
  this inner loop.
However, sometimes an edge will not be reviewed as positive.
If it is reviewed as negative, then it will introduce an inconsistency and be handled by the algorithm
  in~\cref{sec:incon}.
If it is reviewed as incomparable, then the PCC will only be positive-redundant if all edges between the nodes in
  the PCC have been reviewed.
Because we want to ensure that all PCCs are positive-redundant, we delay alternating back to the ranking
  algorithm.
Instead, we simply regenerate candidate edges using the positive-augmentation algorithm in the next iteration of
  the outer loop until all PCCs are positive-redundant.
Therefore, at the end of this process all PCCs are positive-redundant by construction.

After we have ensured positive-redundancy, the outer loop alternates back to generating candidate edges using the
  ranking algorithm.
In this way our algorithm alternates between two modes, first searching for merges, and then ensuring redundancy.
This continues until the termination criterion that we will describe in~\cref{sec:converge} is satisfied.
At the end of this process it is guaranteed that all PCCs are consistent and positive-redundant.


\section{Making decisions}\label{sec:decision}

Now that we have generated and prioritized a set of candidate edges we come to the core of the inner loop ---
  decision-making.
Because of the surrounding structure of the graph framework, this step is quite simple.
Given a popped edge from the priority queue, we check if any of the positive, negative, and incomparable state
  probabilities produced by the pairwise algorithm is above their automatic decision threshold (set externally as a
  hyperparameter).
If the edge cannot be automatically reviewed we issue a request for user feedback.
Once we have obtained feedback for an edge --- either automatically or manually --- the edge is added to the
  appropriate edge set.
After the new edge is added, we update candidate edge priorities discussed in~\cref{sec:redun}.
If the new edge causes an inconsistency, then we drop into inconsistency mode we will discuss
  in~\cref{sec:incon}.

For each decision we record a user-id to identify the reviewer or algorithm making the decision.
We also follow the approach of~\cite{branson_visual_2010} and store a user-specified categorical confidence value
  of unspecified, guessing, not-sure, pretty-sure, and absolutely-sure (with associated integer values $0$, $1$,
  $2$, $3$, and $4$).
The user-id allows us to differentiate between edges that were automatically from those that were manually
  reviewed.
While this is not directly used in this algorithm description, it enables a variety of possible post-processing
  techniques, \eg{} manually reviewing automatically reviewed edges between PCCs containing only two annotations.
However, the user confidence contributes to the edges weights in the error detection and recovery algorithm
  from~\cref{sec:incon}.


\section{Recovering from inconsistencies}\label{sec:incon}

In~\cref{sec:redun} we described a redundancy criterion that exposes errors by introducing inconsistencies.
In this section we describe an algorithm for fixing these errors and recovering from these inconsistencies.

Whenever a decision is made that either adds a negative edge within a PCC or adds a positive edge between two
  PCCs with at least one negative edge between them, the graph becomes inconsistent.
In both cases the result is a single PCC $C$ with internal negative edges.
The goal of inconsistency recovery mode is to change the labels of edges in order to make the subgraph formed by
  the nodes of $C$ and all of their edges.
An inconsistency implies that a mistake was made, but does not necessarily determine which edge has the wrong
  label.
Therefore, we develop an algorithm to hypothesize the edge(s) most likely to contain the mistake(s) using a
  minimum cut.
If the hypothesis is correct, and all the labels on these edges were changed, then we show that the PCC would
  either become consistent or be split into multiple consistent PCCs.
Because the hypothesis might not be correct, we present these edges to a user for manual review, and if the user
  agrees with the hypothesis, the algorithm completes.
Otherwise, the new information received by the user it taken into account, and the hypothesis is recomputed.
An example of an inconsistent PCC with hypothesized edges is illustrated in \cref{fig:inconpcc}.

\inconpcc{}

\paragraph{Hypothesis generation}
The procedure alternates between steps of generating ``mistake hypothesis'' edges, and presenting these to the
  user for review.
The ``hypothesis generation algorithm'' returns a set of negative edges or a set of positive edges, which is
  re-labeled as positive or negative respectively would cause $C$ to become consistent.
For simplicity, we focus first the case where $C$ contains exactly one negative edge.
It will not be hard to extend to the general case.


First consider a cut of $C$ that disconnects the endpoints of the negative edge.
If the labels on these edges were changed from positive to negative or incomparable, then the inconsistent PCC
  would be split into multiple consistent PCCs.
Alternatively, if the label on the negative edge was change to positive or incomparable, the PCC would no longer
  be inconsistent.
Therefore, the algorithm starts by creating a minimum $s$-$t$-cut using the subgraph of $C$ containing only
  positive edges.
The endpoints of the single negative edge are the terminal nodes $s$ and $t$.
Unlike in our preceding discussion where the edges only have positive/negative/incomparable labels the edges will
  now have weights that reflect a measure of confidence in their current label.
In the case where the negative edge is correct, this will encourage the minimum cut to return the positive edges
  that are most likely to be mislabeled.

The weight of edge in this cut problem is the sum of three values:
\begin{enumln}
    \item its positive probability previously computed using the pairwise classifier,

    \item the number of consecutive times that edge was manually reviewed with
        its current label, and

    \item an integer ranging from $0$ to $4$ indicating the confidence of the most recent review (see
      \cref{sec:decision}).
\end{enumln}
Because we are using a minimum cut, the algorithm will find the lowest confidence cut set of these edges.
The higher each of these values is, the more likely that the edge is correctly labeled.
The positive probability offers a baseline estimate of this confidence.
Using the number of manual reviews means that if a user disagrees with a hypothesis, the weight on that edge will
  be increased and the algorithm will be encouraged to select a different edge.
The confidence performs a similar role, speeding up the pace at which the algorithm tries different edges.

Using this weighting scheme we find the minimum $s$-$t$-cut, which returns a cut set of positive edges.
We now need to decide if it is more likely that the positive cut-edges should be relabeled, or the negative edge
  should be relabeled.
We compare the total weight of cut positive edges with the weight of the terminal negative edge (weighted using
  the same scheme).
If the positive weight is smaller, the algorithm suggests that the cut positive edges should be relabeled as
  negative.
Otherwise, it suggests that the negative edge should become positive.
%The minimum cut returns a set of edges that disconnects the terminal nodes.
%In other words, if the label on these edges is changed from positive to negative or incomparable, then $s$ and
%  $t$ would no longer be part of the same PCC and the inconsistency would be removed.

\paragraph{Generalization}
In a restricted scenario where each edge is reviewed exactly to the specifications of the algorithm, then it is
  only possible for one negative edge to exist in a PCC when $k=2$.
However, in practice it is possible for edges to be added outside of these constraints.
In this case we can slightly modify inconsistency recovery to apply to the case when there are any number of
  negative edges between the nodes of a PCC.
The modification is simple.
Instead of using a minimum $s$-$t$-cut, we use a multicut to disconnect the endpoints in all pairs of negative
  edges.
Multicut is NP-hard, but a simple approximation algorithm is to perform a minimum $s$-$t$-cut for each set of
  terminal nodes, and then take the union of the resulting cut~\cite{vazirani_approximation_2013}.
When considering which edges to return we consider all negative edges together, comparing the sum of their
  weights to the sum of the positive edge weights.

\paragraph{Hypothesis review}
The user iterates through each hypothesis edge and chooses
(1) to agree with the hypothesis and change the label of the edge, or
(2) to disagree with the hypothesis and keep the edge label.
In the case where the user agrees that a positive label should be changed, it does not matter if the label is
  changed to negative or incomparable, it will still remove the positive connection.
A similar argument is true when the user agrees to change a negative label to either positive or incomparable;
  either way, the negative edge between the nodes in the PCC is removed.
As long as the user agrees with the hypothesis and the hypothesis set is non-empty, the iteration continues.
If there are no more hypothesis edges, then either all inconsistencies were removed or all PCCs were split into
  multiple consistent PCCs.
In the case where the reviewer disagrees with the algorithm, the new review is recorded and we generate a new set
  of hypothesis errors.
Because reviewing the edge increases its weight, the algorithm will be forced to look elsewhere for a cut.
This alternation between hypothesis generation and user review repeats until there the reviewer agrees with all
  hypothesis edges, which means that all inconsistencies have been eliminated

Fixing inconsistencies can result in splitting $C$ into multiple PCCs.
This may invalidate implicit reviews inferred from redundancy either within or incident to this subgraph.
Therefore, we recompute positive-redundancy within each new PCC, ignoring edges where the criterion is satisfied
  and re-prioritizing unreviewed edges where it is no longer valid.
A similar process happens for negative-redundancy between each pair of new PCCs as well as between each new PCC
  and all other PCCs previously negative-redundant with $C$.

\paragraph{Implementation details}
While it might be conceptually simpler to think of inconsistency recovery as a separate mode, in practice it is
  actually integrated as part of the algorithm's inner loop.
The algorithm dynamically maintains a list of all PCCs that contains errors.
We disable redundancy checks for any PCCs that contain errors.
Hypothesis edges are computed whenever a new inconsistency is created, and these edges are entered into the queue
  with their normal priority plus $10$ to ensure they are reviewed first.
It can be shown that as long as the user agrees with the current hypothesis edges computed so far, recomputing
  the new hypothesis edges will always be the same as the remaining hypothesis edges.
This implementation is important first step for generalizing the graph algorithm into a distributed setting with
  multiple reviewers.


%\section{Refreshing candidate edges}\label{sec:refresh}


%The goal is to refresh if there has been a significant number of positive reviews, but new results are consistently
%negative. If we have not found any positive edges then we do not want to refresh. We keep track of the fraction of
%positive review decisions as a moving average of manual decisions. We also maintain the total number of positive reviews
%made since the last candidate edge generation. Thus the candidate edges are refreshed whenever the number of positive
%reviews is above a threshold and the positive review fraction is below a threshold

%As the last outer iteration of the overall algorithm before convergence, triggered when the LNBNN ranking algorithm
%fails to produce positive edges, candidate edges between untested pairs of annotations are added within PCCs that are
%not positive-redundant and between PCCs that are not negative-redundant. This is because the ranking algorithm itself is
%imperfect and the missed matches tend to affect small PCCs disproportionately, which are the last to satisfy redundancy
%tests.


%\subsection{Probabilistic convergence}
%The goal of probabilistic convergence is to determine if a PCC $C$ is negative-redundant with all other components with
%high probability. When all components are positive-redundant and satisfy this, then all edges will be removed from the
%priority queue and the algorithm will converge. We consider the probability $\Pr{E_c \given \nT_C}$ that an undiscovered
%positive edge exists ($E_C$) given $C$'s existing set of outgoing negative edges ($\nT_C$). Under mild conditions (if we
%assume that $\Pr{E_c \given \nT_C} < 0.5$), we can show that the probability $\Pr{\nT_C \given E_C}$ of observing the
%negative edges bounds this p given that an undiscovered match exists can be used as a surrogate. We can learn this
%probability offline by measuring the frequency that correct results are at a given rank in a PCC's ranked list
%(constructed by aggregating the ranked lists of all annotations in the PCC).

%To predict $\Pr{\nT_C \given E_C}$ we issue all queries as a single LNBNN query to obtain a single ranked list for the
%entire PCC. This can be done by treating all query descriptors as if they were the from the same annotation except
%during the spatial verification stage. Let $R_C$ denote the ranks of every PCC marked as negative with $C$. In an
%offline step we learn a probability mass function $\phi$ that predicts the probability that a correct match appears at a
%given rank for the PCC $C$. The predicted probability is %
%$\Pr{\nT_C \given E_C} = 1 - \sum_{r \in R_C} \phi(r)$.

%To learn $\phi$, we measure the probability that a correct match appears at a given rank, given a correct match exists.
%To do this initialize an histogram. For each $C$ in the training set, divide it into a query $C_q$ and target $C_t$. The
%target and the rest of the PCCs in the training set become database PCCs. Use LNBNN to score each annotation in $C_q$
%against the database PCCs. Determine the best rank that $C_t$ appears in each ranked list, and increment the
%corresponding index in the histogram. Repeat this process for all PCCs in the training set and for multiple partitions
%of each PCC. Normalizing the histogram array results in the PMF $\phi$. In order to prevent marginalization across
%important attributes (such as the number of exemplars in a PCC), construct multiple PMFs for different numbers of
%exemplars in a query.


\section{Refresh and termination criteria}\label{sec:converge}


In this section we discuss the criteria we use for both determining when to refresh candidate edges and when to
  stop the algorithm altogether.
We first consider the need for a refresh criterion.
As the review algorithm proceeds, we should only continue manually review as long as the algorithm is
  consistently generating edges that --- once reviewed --- change the PCCs.
This happens whenever a review merges two PCCs into one, or splits one PCC into two.
Because splits only happen during inconsistency recovery, we are primarily concerned with searching for new
  positive edges.
However, at some point the candidate edges may no longer contain positive results, but undiscovered positive
  matches may still exist.
This is because LNBNN, working initially with each annotation having a separate label, can miss more subtle but
  correct matches, especially when there are several annotations for an animal and subtle viewpoints.
As the labeling improves, so does the reliability of LNBNN.
Recall that the verification algorithm is only run after candidate edges are generated.
If the edges required to complete the underlying real PCCs are missing from the candidate set, then nothing can
  be done.
We therefore must develop a ``refresh criterion'' to determine when to stop and update the current priority queue
  with new LNBNN matches.
This will be done by predicting if none of the remaining edges would change the PCCs, and then recomputing
  candidate edges when this happens.
However, before we describe this process, we consider the problem of termination, which will turn out to have a
  similar solution.

Similar to the refresh criterion, we must be able to determine when to terminate the algorithm.
To ensure that the identification is perfect, the algorithm would need to use all $\binom{|V|}{2}$ edges as
  candidates and then terminate using the deterministic convergence criterion explained in the introduction of this
  chapter.
Recall that the deterministic convergence criterion only stops the algorithm once each PCC is positive redundant
  and each pair of PCCs is negative-redundant.
This essentially results in a brute-force search, that requires $O(|V|^2)$ reviews, and is only feasible if
(1) the number of annotations is very small, or
(2) all edges can be automatically reviewed.
However, even if all edges can be automatically reviewed the quadratic computation required to run the
  verification algorithm on all pairs of annotations might be too computationally expensive for very large
  databases.
Therefore, in practical circumstances, we turn towards probabilistic methods to determine when to stop.
Like, the refresh criterion, this can be determined --- in part --- by predicting if new reviews will change the
  PCCs.

\subsection{Convergence as a Poisson process}

%\newcommand{\meaningful}{meaningful}
\newcommand{\meaningful}{label-changing}

Both the review and termination criteria can be addressed by considering the question:
``Will there be a \meaningful{} review anytime soon?''.
A \glossterm{\meaningful{} review} is one that changes the name labeling of the annotations, \ie{} it is either a
  positive edge that merges two PCCs or a non-positive edge that splits one.
Correct and \meaningful{} reviews improve the accuracy of the identification by increasing the similarity --- in
  terms of the name labeling --- between the predicted PCCs and the real underlying PCCs.
However, producing a \meaningful{} review has a cost:
the number of manual reviews since the last \meaningful{} review.
During the inner loop of the algorithm, when edges popped from the priority queue no longer consistently result
  in \meaningful{} reviews, the marginal gains in identification accuracy that could be made from continuing are
  outweighed by the cost of manual review.
In this circumstance it is best to break out of the inner loop.
Note that if any \meaningful{} reviews were made during that loop, we should refresh candidate edges and start a
  new loop because a refresh could result in new high priority \meaningful{} edges.
On the other hand, if no \meaningful{} reviews were made in the loop, then refreshing will have no benefit, and
  the algorithm should terminate.

Thus, the task is to construct a criterion that determines when edges on the top of the priority queue are no
  longer \meaningful{}.
In this way we directly address the refresh criterion and indirectly address termination criterion.
This is direct in the case of the refresh criterion because when the name labeling is more accurate, the LNBNN
  ranking will improve.
When we directly measure that the next reviews in the current priority queue are unlikely to be \meaningful{}, and
  the name labeling of the decision graph has changed, then it is more likely that we would review a \meaningful{}
  edge on the top of a new set of candidate edges sorted by priority.

This task indirectly addresses the termination criterion because instead of stopping once the probability that
  identification is complete is high, the algorithm simply stops when the cost in terms of manual labor is too
  high.
However, if we were to directly address the termination criterion, we would have to estimate the probability that
  undiscovered merge and split cases exist.
This probability depends on the effectiveness of the ranking algorithm.
Even if our estimate of this probability was perfect, once the ranking algorithm starting producing \meaningful{}
  reviews at a rate no better than random edge generation, it would take an enormous amount of manual effort to
  push this probability passed a desired threshold.
Thus, instead of providing guarantees about identification accuracy our termination criterion achieves a
  trade-off between the number of manual reviews and the cost of identification.


%\newcommand{\M1}{\Pr{M\teq1}}
%\newcommand{\M0}{\Pr{M\teq0}}
%\newcommand{\Mi}{\Pr{M_i\teq1}}
%\newcommand{\PT}{\Pr{T\teq1}}
%\newcommand{\T}{\Pr{T\teq1}}

Based on these observations we estimate the probability that ``there will be a \meaningful{} review soon''.
We define ``soon'' using a patience parameter $a$, defined as the maximum number of consecutive reviews that a
  manual reviewer is willing to do between \meaningful{} reviews.
Let $M\teq1$ be the event that a review is \meaningful{} and $M\teq0$ otherwise.
Because reviews are ordered, denote if the $i\th$ review is \meaningful{} as $M_i$, and denote the index of the
  next review as $n$.
Let  %
%$T\teq1 \equiv (\M_i\teq1, \exists i \in [n, n + a])$ 
$T = \Or_{i=n}^{n+a} \M_i$
%
be a binary random variable that takes the value $T\eq=1$ in the event that any of the next $a$ reviews will be
  \meaningful{}.
Thus, the aforementioned question can be addressed by measuring the probability of the event $T\teq1$ The goal is
  to measure the probability the event $T\teq1$.
We can periodically check if $\Pr{T\teq1}$ is less than a threshold, and if so, we stop the current loop and
  either refresh or terminate.

To estimate $\Pr{T\teq1}$, we model the event $M_i\teq1$ as a Poisson processes, but for this to be appropriate,
  $M_i$ must follow a uniform distribution.
This would be true if the edges were reviewed in a random order.
However, the priority queue orders edges more likely to be \meaningful{} first, causing $M_i$ to follow a right
  skewed long tail distribution and violate Poisson assumptions.
Even so, the use of a Poisson model can be justified by considering a sliding window along the distribution of
  $M_i$.
Recall that we only need to make predictions about the next $a$ reviews in the future, thus we are only concerned
  with a small window to the right on the distribution.
Assuming the long-tailed distribution is monotonic decreasing, we can use a small window in the past to estimate
  an upper bound on probability of $T\teq1$ in the future.
As the window moves to the right, the interval on the distribution becomes increasingly approximately uniform and
  the tightness of the bound improves and eventually becomes tight.
This is because once the prioritization algorithm cannot distinguish positive from negative cases, the order of
  the remaining reviews becomes random and the Poisson model becomes exactly appropriate.
Thus, the use of a Poisson model with a sliding window allows us to approximate an upper bound on $\Pr{T\teq1}$,
  and the smaller $\Pr{T\teq1}$ is, the more accurate our estimate will be.


\subsection{Details of Poisson convergence}

Having justified its use, we model $M$ as a Poisson process, which is determined by a single parameter $\mu$.
We can measure $\mu$ as the fraction of recently observed manual reviews that were \meaningful{} using an
  exponentially weighted moving window.
We initialize $\mu=1$ to denote that it is likely that the first review will be \meaningful{}.
Then, after each new review we update the parameter as %
$\mu \leftarrow m \alpha + (1 - \alpha) \mu$, where $m\teq1$ if the review was \meaningful{} and $0$ otherwise.
The exponential decay $\alpha = 2 / (s + 1)$ is determined by a span parameter $s$, which roughly represents the
  number of previous reviews that are significant.
Using this model, the desired probability that any of the next $a$ reviews will be \meaningful{} is %
$\Pr{T\teq1} = 1 - \exp{-\mu  a}$.
The example in \cref{fig:poisson} illustrates the behavior of the criterion using a synthetic dataset.
%In this example we use a window span of $s=20$, a patience of $a=20$, and a threshold of $t=\exp{-2}\approx 0.135$.

%To achieve a certain threshold with a specified span the number of iterations required is: 
%$\lceil{\frac{\log{\left (t \right )}}{\log{\left (\frac{s - 1}{s + 1} \right )}}}\rceil$

%If we set the threshold to $\exp{-2}\approx0.135$, then the window span parameter also roughly represents how
%  many successive consecutive non-\meaningful{} decisions are necessary before reaching the threshold if starting from
%  initial state.

\begin{comment}
Another alternative might be measuring once the advantages from the ranking
algorithm become indistinguishable from a brute force search.


Let $D = \binom{|V|}{2}$ be the number of edges in the dataset and let $N=\sum_{C \in \set{C}} (|C| - 1)$ the
  number of edges in the MSTs of all PCCs.
We can measure the density of meaninful reviews in the current labeled dataset as $N/D$.

After our estimated mu gets close enough to to $N/D$, we can terminate.

\end{comment}

\poisson{}

%As the last outer iteration of the overall algorithm before convergence, triggered when the LNBNN ranking algorithm
%fails to produce positive edges, candidate edges between untested pairs of annotations are added within PCCs that are
%not positive-redundant and between PCCs that are not negative-redundant. This is because the ranking algorithm itself is
%imperfect and the missed matches tend to affect small PCCs disproportionately, which are the last to satisfy redundancy
%tests.

%Re-estimating the $k\mu$ at each time-step should


\section{Experiments}\label{sec:graphexpt}

    In this section we design an end-to-end experiment where we start with a set of annotations without any name
      labels, and then we proceed to construct all the names.
    This simulates identification events like the \GZC{} and provides insight into how the algorithms behave in
      practice.
    Because our algorithms are semi-automatic, we simulate a noisy user response using groundtruth data.
    Given a pair of annotations, the simulated user returns the groundtruth classification $99\percent$ of the
      time, making errors $1\percent$ of the time uniformly at random.
    These assumptions may be simple and too inaccurate to model an expert reviewer, but it will serve to
      demonstrate our graph algorithm's ability to recover from errors.
    We will measure the accuracy and error as a function of the number of manual reviews.
    Using these measures we will compare our graph algorithm to alternative techniques to demonstrate that it
      produces accurate identifications with fewer errors using significantly fewer manual reviews.

    For this experiment, we will use the plains and Grévy's zebras datasets previously described in
      \cref{sub:datasets}.
    Because some of our algorithms will require training, we split each dataset into a training set and a testing
      set each containing half of the names.
    The training set will be used to learn the pairwise classifiers and determine thresholds for automatic
      classification.
    The details of training and testing sets are summarized in \cref{tbl:TestTrainDBStats}.
    \TestTrainDBStats{}
    \FloatBarrier{}

    This experiment will compare the three different methods of determining the identifies of annotations based
      on the algorithms defined in this \thesis{}.
    The first uses just the ranking algorithm from We denote these algorithms as:
    These algorithms are:
    \begin{enumin}

    \item \pvar{ranking} --- the ranking algorithm from \cref{chap:ranking}, 

    \item \pvar{rank+clf} --- the same
        ranking algorithm but augmented with the automatic classification algorithm from \cref{chap:pairclf}, and
        finally

    \item \pvar{graph} --- the graph identification algorithm introduced in this chapter.

    \end{enumin}
    In the case \pvar{graph}, the procedure from~\cref{sub:graphalgo} can be directly applied.
    However, in order to compare \pvar{graph} to \pvar{ranking} and \pvar{rank+clf} we must define baseline
      methods to determine the ordering of the reviews and when to stop.
    Details of these identification procedures are given in the next subsection.

    \FloatBarrier{}
    \subsection{Identification procedures}

    We design the procedure for \pvar{ranking} to be similar to the approach described in \cref{sec:introgzc}
      that was used in the \GZC{}.
    This algorithm does not require a pre-training phase and thus only the testing set is used.
    Given the unlabeled annotations, we index the database, issue each annotation as a query, and collect the top
      $5$ results from each ranked list.
    The resulting pairs of query and database annotations are stacked and sorted by LNBNN score.
    The user reviews each pair in the list sequentially.
    As was done in the \GZC{}, all pairs are all manually considered and verified, and all inconsistencies are
      ignored.
    In the \GZC{} the choice to re-run the ranking algorithm was made manually, making it difficult to reproduce.
    Therefore, we simplify the experiment by choosing to run the ranking algorithm once.
    Consistency checks are not applied because developing these is the point of the graph algorithm.
    %When designing this protocol we considered alternatives

    %This is primarily due to simplicity.
    %--- in part for simplicity and in part because it
    %  does not change how we interpret the results.
    %We also do not apply the consistency checks used in during the \GZC{} because these were mainly based on
    %  heuristics that are difficult to reproduce and splitting names is difficult without connectivity
    %information.

    The procedure for \pvar{rank+clf} is similar to the one used for \pvar{ranking}.
    The main difference is that we use the automatic classification algorithm to predict pairwise probabilities
      for each pair of top ranked query and database annotations.
    If the probabilities of a class are above a threshold, then the pair is automatically reviewed.
    This has the effect of reducing the total number of manual reviews.
    The rest of the procedure is unchanged.
    The pairwise classifier is trained on the training set, and the algorithm is evaluated on the test set.
    We choose automatic thresholds by finding the thresholds that result in a specified false positive rate on a
      validation dataset.
    Because \pvar{rank+clf} has no mechanisms for error recovery, we choose conservative automatic thresholds by
      specifying an acceptable false positive rate of $0.001$.
    This results in a positive, negative, and incomparable threshold of $0.976$, $0.991$, and $0.5$ respectively
      for plains zebras, and $0.997$, $0.998$, $1.0$ for Grévy's.
    %When choosing thresholds we enforce a warm-up period of $200$ examples, where the threshold is set to $1$.
    %This ensures that we do not automatically classify categories (\eg{} incomparable) with little supporting
    %  data.
    % GRAPH THRESH
    %We choose a false positive rate of $0.001$.

    %Because this approach has no mechanism for error recovery, we choose conservative classification thresholds
    %  that achieve a $0\percent$ false positive rate on a validation dataset.

    Finally, we quickly recap the procedure for \pvar{graph} which was defined in~\cref{sub:graphalgo}.
    The algorithm begins by using LNBNN to search for candidate edges, which are then assigned probabilities and
      inserted into a priority queue based on these probabilities.
    As candidate edges are removed from the queue they are either automatically or manually classified based on
      probability thresholds.
    Connectivity information is used to enforce a minimum level of redundancy, preventing extraneous redundancy,
      and ensure consistency.
    The convergence criterion determines when candidate edges should be refreshed and when the algorithm should
      terminate.
    We use the same pairwise classifier from \pvar{rank+clf}, but due to the graph algorithm's error recovery
      mechanisms we can choose more aggressive thresholds.
    However, we have found that the algorithm is sensitive to this parameter, therefore we only slightly increase
      acceptable false positive rate from $.001$ to $.0014$.
    % GRAPH THRESH
    This results in a positive, negative, and incomparable threshold of $0.969$, $0.986$, and $0.5$ respectively
      for plains zebras, and $0.989$, $0.992$, and $1.0$ for Grévy's.
    For the convergence criterion use a patience of $a=20$, a window span of $s=20$, and a termination threshold
      of $\exp{-2}\approx0.135$, which causes the criterion to trigger after at most $20$ consecutive reviews that
      are not \meaningful{}.

    \subsection{Results}

    We run the simulation for all combinations of datasets and algorithms.
    After each review decision is made we record two measurements pertaining to accuracy and error.
    The accuracy measurement is the number of merges remaining before all individuals have been identified.
    This is the number of edges in a spanning forest of the groundtruth positive subgraph minus the same
      measurement but applied to the subgraph of all correctly predicted positive edges.
    % better error measure?
    The error measurement is the total number of edges with a predicted label that differs from its groundtruth
      match-state.
    Additionally, for \pvar{graph}, after each review decision we record the probability of convergence estimated
      by the convergence criterion.
    These measurements are plotted against the number of manual reviews.
    \Cref{fig:Simulation} shows these accuracy and error plots, and \cref{fig:Refresh} shows the convergence
      criterion.

    \Simulation{}

    \Refresh{}

    The results illustrated in \cref{fig:Simulation} demonstrate that \pvar{graph} achieves the fewest manual
      reviews with the fewest errors while still correctly identifying almost all individuals.
    We first focus on the left part of this figure.
    The number of remaining merges slowly decreases for \pvar{ranking}, which is the only algorithm that requires
      manual review of each pair.
    For \pvar{graph} and \pvar{rank+clf} the initial decrease appears instantaneous due to the automatic
      classification algorithm, which does not cost manually reviews.
    Notice, once manual reviews begin the slope of \pvar{graph} is steeper than \pvar{rank+clf} due to the
      redundancy mechanisms that removes extraneous reviews.
    Furthermore, while \pvar{rank+clf} and \pvar{ranking} continue until their candidate edges are exhausted,
      while \pvar{graph} uses the convergence criterion to to terminate shortly after the curve flattens.

    We now turn our attention to the right of \cref{fig:Simulation}.
    The \pvar{ranking} and \pvar{rank+clf} algorithms have no mechanism for error recovery, and thus their error
      steadily increases over time.
    However, \pvar{graph} is able to recover from many of these errors and achieve a low error rate despite
      starting with more errors due to an aggressive auto-classification threshold.
      
    \subsection{Error cases}

    To further analyze the predictions of the graph the simulation, we compare the node groupings of the
      predicted PCCs to the real groundtruth PCCs.
    For convenience we will still refer to a group of nodes as a PCC.
    %Even though we are only concerned with the grouping of nodes
    In this analysis we categorize groups of predicted PCCs and their corresponding real PCCs as one of three
      types:
    correct, split, or merge.
    Consider an example where we are given a graph with $8$ nodes and the real PCCs are %
    $\curly{\curly{a, b, c}, \curly{d}, \curly{e, f}, \curly{g}}$, and we predict the PCCs%
    $\curly{\curly{a, b}, \curly{c}, \curly{d, e, f}, \curly{g}}$.
    Using this example we define the three group types:
    %correct, split We will categorize predictions as one of three types.
      %in terms of the nodes that belong to each group.
    %In this analysis we are only interested in the grouping of nodes, and we ignore the details of the edge
    %  labeling.
    %To analyze our predictions we define $3$ types of groups.
    \begin{enumln}
        \item A ``correct'' group is a predicted PCCs that contains all nodes in a real PCC.
        In the example the PCC $\curly{g}$ is a real group.
        %For a predicted PCC to be correct there must be a real PCC with the same nodes.
        
        \item A ``split'' error group consists of multiple real PCCs that are labeled as belonging to the same
          predicted PCC.
        Each split group contains at least one edge incorrectly labeled as positive.
        In the example, the real PCCs $\curly{\curly{d}, \curly{e, f}}$ are a split group.
        %Thus split groups are created when an edge is incorrectly labeled as positive.
        %A split cases is caused by when two real PCCs are connected.
        %This results in two real PCCs that are connected.
        %because that edge connects two real PCCs.
        %two nodes in two real PCCs are connected by
      
        \item A ``merge'' error group consists of multiple predicted PCCs that must be merged into a single real
          PCCs.
        Each pair of predicted PCCs in a merge group is missing a positive edge.
        This can happen if an edge is incorrectly labeled, if the ranking algorithm never generates this edge as
          a candidate, or the termination criterion stops the algorithm before the edge can be reviewed.
        In the example, the predicted PCCs $\curly{\curly{a, b}, \curly{c}}$ are a merge group.
        
        %that would have connected 
        %A merge cases is caused by the absence of a positive edge that would have connected two nodes in a real
        %  PCC.
    \end{enumln}
    As defined, the split and merge error groups do not cover all cases.
    We call the previously described cases ``pure'' split/merge error groups because there is a mapping between
      sets of real/predicted PCCs to exactly one predicted/real PCC.
    However, consider the real PCCs $\curly{\curly{x, y}, \curly{z}}$ and predicted PCCs $\curly{\curly{x},
      \curly{y, z}}$.
    This case cannot be cleanly categorized as a split or a merge group, but it contains elements of both.
    Thus, we call this a ``hybrid'' case.
    A slight modification to the above groups can incorporate hybrid cases.
    First, we change the definition of a split groups to be the set of maximal real PCC subsets that are labeled
      as belonging to the same predicted PCC.
    In essence this means we can treat $\curly{\curly{y}, \curly{z}}$ as a split group.
    Second, for merges, we simply change predicted PCCs to be the predicted PCCs after they have been split.
    This lets us treat $\curly{\curly{x}, \curly{y}}$ as a merge group.
    %In or graph simulation almost all error groups are either pure splits or pure merges.

    %In our analysis we will treat hybrids similar to split and merge cases. 
    %However, for the simplicity of analysis we will group hybrid cases with 
    %We can address a hybrid case by first splitting hybrid groups into maximal subsets of real PCCs and then
    %  merging these subsets into real PCCs.

    We analyze statistics of these groups and present the results in two tables.
    First, for each simulation we gather the set of predicted and real PCCs, and then we group them into correct,
      split, and merge groups.
    For each group type we count the number of predicted PCCs and the number of real PCCs they correspond to.
    We also measure the average number of annotations in each PCC.
    These number are reported in~\cref{tbl:ErrorSizeDetails}.
    Then, in~\cref{tbl:ErrorGroupDetails} we focus only on the results of \pvar{graph} and measure the average
      number of PCCs in each error group.
    Additionally, we consider the smallest and largest PCC in each error group, and we measure the average number
      of annotations in each.

    Looking at~\cref{tbl:ErrorSizeDetails} we observe that \pvar{graph} predicts the most correct PCCs and fewest
      splits in all cases.
    This is not surprising due to its error recovery mechanism.
    However, we also see that \pvar{graph} also has the most merge cases.
    The reason for this is that the other algorithms do not have termination criteria.
    They continued to review pairs, even long after the rankings were no longer consistently \meaningful{}, and
      essentially began to brute-force search the database.
    In the thousands of extra iterations, the other algorithms managed to find ${\sim}20$ extra merge cases that
      were not discovered by \pvar{graph}.

    Now moving to~\cref{tbl:ErrorGroupDetails},  we notice that almost all merge cases for \pvar{graph} are the
      result of failing to match a single annotation to a larger group of annotations.
    Similarly, most split cases involve splitting just one singleton annotation off of a larger PCC.
    To gain further insight into what is causing these errors we will visualize individual cases.
    %Mistakes involving individuals with only one annotation are more difficult to detect and highlight the
    %  importance of obtaining multiple images of each individua.
    %These error measures are reported
    %in~\cref{tbl:ErrorSizeDetails,tbl:ErrorGroupDetails}.

    %\SimDetails{}

    %\SimErrorSize{}
    \ErrorSizeDetails{}

    \ErrorGroupDetails{}

    \FloatBarrier{}

    We illustrate several individual error cases from \pvar{graph}
    in~\cref{fig:SplitErrorsPZ,fig:SplitErrorsGZ,fig:MergeErrorPZA,fig:MergeErrorPZB,fig:MergeErrorGZA,fig:MergeErrorGZB}.
    On the top of each figure we will show the subgraph corresponding to an error group.
    For a split case this will be the single PCC that must be broken apart, and for a merge case this will be
      multiple PCCs that should be connected.
    For merge cases, if no edge connecting the PCCs was ever added to the priority queue as a candidate, then we
      will insert a dashed edge between two arbitrary annotations.
    We will highlight the edges with labels that differ from their groundtruth.
    On the bottom we show the annotation pair corresponding to a selected highlighted edge.

    Upon inspection, we discover that the split cases in~\cref{fig:SplitErrorsPZ,fig:SplitErrorsGZ} are caused by
      groundtruth errors.
    In each of these cases the automatic verification algorithm made the PCCs positive redundant, thus the
      simulated reviewer --- which is driven by the groundtruth --- was unable to split the PCC.
    In fact, we discovered that all split cases are due to groundtruth errors.
    This means that \pvar{graph} did not predict any PCCs that were split cases.
    %This is encouraging because it means that 

    When inspecting merge cases we also found several caused by groundtruth errors, but most were due to
      challenging image conditions such as viewpoint, occlusion, and pose.
    These factors either prevented an edge from being generated as a candidate or caused the classifier to
      produce a low positive probability.
    Two merge cases for plains zebras are illustrated in ~\cref{fig:MergeErrorPZA,fig:MergeErrorPZB}, and two for
      Grévy's zebras are illustrated in~\cref{fig:MergeErrorGZA,fig:MergeErrorGZB}.

    %all split
    %  cases for the graph algorithm are due to groundtruth errors.
    % %$5$ split cases reported for plains zebras 
    %Two examples of these cases are shown in \cref{fig:SplitErrorsPZ,fig:SplitErrorsGZ}

    \SplitErrorsPZ{}

    \SplitErrorsGZ{}

    \MergeErrorPZA{}

    \MergeErrorPZB{}

    \MergeErrorGZA{}

    \MergeErrorGZB{}

 
\FloatBarrier{}
\section{Summary of graph identification}\label{sec:graphconclusion}

In this chapter we have described the final component in our approach to animal identification.
The graph-based framework takes advantage of the algorithms previously developed in
  \cref{chap:ranking,chap:pairclf} and uses them in a principled way to address the identification problem.
The ranking algorithm quickly identifies candidate edges that are likely to be \meaningful{}, and the pairwise
  classifier automatically verifies pairs of annotations, reducing the required manual interaction.
While these algorithms can be applied to address identification directly, our experiments have demonstrated that
  there is a considerable advantage to placing them in the context of the graph identification framework.

Our graph framework uses the connectivity of positive, negative, and incomparable edges in order to quickly
  converge on a final identification.
Edge connectivity is used to enforce that the graph contains a small exact amount of redundancy, which reduces
  the number of manual decisions while providing the means to detect inconsistencies.
When inconsistencies are detected, edge-connectivity is again used to quickly find and fix errors in edge
  labeling.
Our framework also includes a probabilistic convergence criterion based on a Poisson process.
This criterion stops identification once the ranking and verification algorithm are no longer able to find
  \meaningful{} edges.
This means that the algorithm will stop the identification process before it devolves into a brute force search.
The point at which this happens will depend on the power of the ranking and verification algorithms.

This brings us to our last point.
The graph framework is general.
It is not restricted to animal identification.
It could be applied to any instance recognition problem where one annotation corresponds to one individual
  object.
It does not depend on our specific ranking and verification algorithms and could easily incorporate other
  algorithms.
It is even possible to use it without any algorithms.
Even though this does result in a brute force search, the redundancy criterion still reduces the number of
  reviews, and error recovery still ensures that the graph is always consistent.
This case is not just theoretical.
To extend to new species and domains it is important to be able to construct a small labeled database from which
  learned ranking and verification algorithms can be bootstrapped.
As more pairwise training data is gathered and maintained using this framework, more sophisticated pairwise
  classifiers trained using deep learning could be applied, perhaps removing the need for manual verification
  interaction and resulting in a fully automatic identification algorithm.


% +--- CHAPTER --- 
\begin{comment}
    ./texfix.py --fpaths chapter6-conclusion.tex --outline --asmarkdown --numlines=99 -w
    fixtex --fpaths chapter6-conclusion.tex --outline --asmarkdown --numlines=999 --shortcite
\end{comment}

\chapter{Conclusion}\label{chap:conclusion} 

    In this \thesis{} we have addressed the problem of identifying individual animals from images.
    We have demonstrated that our approach is effective for identifying plains zebras, Grévy's zebras, Masai
      giraffes, and humpback whales.
    Our approach consists of three main components:
    (1) the ranking algorithm from \cref{chap:ranking} that uses a bounding box annotation around an animal to
      search a labeled database of annotations for likely matches,
    (2) the classification algorithm from \cref{chap:pairclf} that probabilistically verifies if a pair of
      annotations is positive, negative, or incomparable, and
    (3) the graph framework from \cref{chap:graphid} that harnesses the previous algorithms in a principled way
      to dynamically determine the identity of all animals in a dataset.
    Each of these algorithms was designed to build on the previous one(s), improving the overall accuracy and
      efficiency of the counting process.
    In \cref{sec:graphexpt} we demonstrated that this was indeed the case.

    By combining these algorithms we have made several meaningful contributions to the problem of animal
      identification.
    In \cref{sec:introgzc} we discussed the Great Zebra Count (\GZC{}), where the ranking algorithm was used in
      combination with the effort of citizen scientists to provide an estimate of the number of plains zebras and
      Masai giraffes in Nairobi National Park.
    In \cref{sec:rankexpt} we investigated several parameters and factors that can impact the performance of the
      ranking algorithm.
    We discovered that having multiple photos of each individual significantly improves the accuracy of the
      ranking algorithm and we designed a novel name scoring mechanism with this in mind.
    In \cref{sec:pairexpt} we demonstrated that a classification algorithm can be used to improve the separation
      of positive results from negative and incomparable results in a ranked list.
    In \cref{sec:graphexpt} we simulated the \GZC{} and demonstrated that our improvements to the ranking
      algorithm --- made by the classification and graph algorithm --- enable us to perform identification using
      less than $25\percent$ of the number of manual reviews required by the original event.

    \section{Discussion}\label{sec:discuss}

    The research that resulted in this \thesis{} began in $2010$ and was completed in $2017$.
    During that time, many significant developments were made in the fields of computer vision and machine
      learning, most notably the explosion of deep learning~\cite{lecun_deep_2015}.
    While some steps in our approach (\eg{} the foregroundness weights) do make use of deep convolutional neural
      networks (DCNNs), most do not.
    In some sense this is an advantage because the algorithms can be applied to different species without any
      need for pre-training, but this also means they do not obtain the level of accuracy shown to be achievable by
      these networks.
    Yet, the contributions of this \thesis{} are still relevant and complementary to DCNNs.
    This is trivially true in the case of the ranking and classification algorithms, in part due to the
      aforementioned reasons.
    However, the contribution of the graph algorithm is relevant, even in the era of deep learning.
    
    %\section{Discussion of the graph algorithm}
    The graph identification algorithm models the abstract constraints of the identification problem and provides
      a framework that can efficiently harness the power of any ranking or verification algorithm, whether it be
      deep or shallow.
    The framework dynamically manages the relationships between annotations.
    In most cases this means deciding if two annotations are the same (positive) or different (negative), but
      this also means handling cases like when the annotations are incomparable or when there is some other
      interesting connection between two annotations like scenery matches and photobombs.
    As new relationships are added, errors are discovered and corrected, and the identifications are updated.

    The framework also provides a means of prioritizing which edges need to be reviewed based on
    (1) the underlying computer vision algorithms,
    (2) the edge-augmentation needed to ensure minimum redundancy, and
    (3) the minimum cut needed to correct an error and split an inconsistent individual.
    Edge prioritization works in conjunction with a convergence criteria that determines when identification has
      been completed.
    A signal is emitted whenever manual interaction is needed, and the algorithm continues after the user returns
      with a response.
    The only time a user interacts with the algorithm after it begins is to label an edge as positive, negative,
      or incomparable.
    All other decisions are made internally.
    The algorithm stops once there is a high probability that the vast majority of identifications have been made
      correctly and consistently.
    This means that the graph algorithm requires little expertise to use and can be thought of as an
      ``identification wizard'' that simply guides the user through a set of simple questions.
    This design allows the graph algorithm to be run on a web server, where requests for manual interactions can
      be sent to remote users and quickly done in a web browser.

    %\section{Discussion of the ranking and verification algorithm}
    %pass
    \section{Contributions}\label{sec:contributions}

    A summary of the contributions made in this \thesis{} are as follows:

    \begin{enumln}
    \item {The ranking algorithm}:
        \begin{enumln}
        \item
        We have adapted LNBNN~\cite{mccann_local_2012} to the problem of individual animal identification.
        We have performed experiments that demonstrate the effect of several parameters at multiple database
          sizes.
        We have shown that tripling the number of annotations in a database can reduce the ranking accuracy at
          rank $1$ by $2\percent$.

        \item
        We have evaluated the effect of various levels of feature invariance in our experiments.
        We have introduced a heuristic that augments the orientation of query keypoints to account for pose
          variations.
        For plains zebras, this can improve the ranking accuracy at rank $1$ by $7\percent$.

        \item
        We have accounted for the influence of background features using a learned a foregroundness measure to
          weight the LNBNN scores of feature correspondences.
        We have empirically shown that this procedure can increase the ranking accuracy at rank $1$ by
          $5\percent$.


        \item We have demonstrated the impact of image redundancy and the importance of collecting more than one
          annotation in each encounter.
        Our experiments show that multiple exemplars per name can significantly increase the ranking accuracy at
          rank $1$ by $20\percent$.

        \item We have developed a \name{} scoring mechanism to take advantage of information in database names
          with multiple exemplars.
        We have shown that this can increase the ranking accuracy at rank $1$ by $1\percent$ when there are
          multiple exemplars per name.
        \end{enumln}

    \item {The pairwise classification algorithm}:
        \begin{enumln}

        \item We have developed a novel feature vector that represents local and global matching information
          between two annotations.
        Our experiments have shown that both the local and global feature dimensions are important for predicting
          if two annotations match.

        \item We have used this feature vector to learn a random forest that can predict the probability that two
          annotations are either positive, negative, or incomparable.
        We have shown that this learned pairwise classifier is a strong predictor of match-state by measuring an
          MCC of $0.83$ for plains zebras and $0.91$ for Grévy's zebras.

        \item We have compared the learned probabilities to LNBNN scores and demonstrated that re-ranking using
          the positive probabilities can improve the ranking accuracy at rank $1$ by $9\percent$ for plains zebras
          and $2\percent$ for Grévy's zebras.
        Additionally, the probabilities significantly improve the separation of positive and non-positive
          annotation pairs.
        For both species, an ROC AUC of less than $0.9$ is improved to an AUC greater than $0.97$.
          
        \end{enumln}

    \item {The graph identification algorithm}:
        \begin{enumln}

        \item 
            We have demonstrated that combining the graph algorithm with existing ranking and verification
              algorithms improves the accuracy and efficiency of semi-automatic animal identification.
            We have designed the framework to be agnostic to the specific ranking and verification algorithms so
              future DCNN-based algorithms can be swapped in.

        \item We have proposed a measure of redundancy based on edge-connectivity used to increase accuracy and
          reduce the number of reviews needed.

        \item We have developed an algorithm for fixing errors whenever inconsistencies in the graph are been
          discovered.

        \item We have developed a probabilistic termination criteria that determines when to stop identification.
        \end{enumln}
    \end{enumln}

    \section{Future work}\label{sec:futurework}

    We have shown that our ranking and match-state classification algorithms are both accurate and work well for
      identifying animals.
    However, the clearest direction for future research is to replace these algorithms with ones based on DCNNs.
    To replace the ranking algorithm, we believe that the approach in~\cite{arandjelovic_netvlad_2016} is a good
      starting point.
    We had briefly investigated replacing the pairwise classifier using the techniques
      in~\cite{taigman_deepface_2014}, but the results were poor because we did not have as much training data or
      an alignment procedure.
    Research into the geometric matching technique described in~\cite{rocco_convolutional_2017} may help address
      both of these issues.

    There are also improvements that can be made to the graph algorithm.
    First it would be useful to parallelize the algorithm so reviews could be distributed across multiple users.
    This can be obtained by popping multiple edges from the queue at a time, but this could add extraneous
      redundancy if one edge in the popped set would have been filtered by another.
    Second, the current prioritization of edges is based completely on the output of the pairwise classifier.
    In the best case, the ordering would first construct each PCC as a chain, and then only $1$ redundant review
      would be needed.
    In the worst case, this order would connect one annotation of an individual to all others causing a star
      shaped PCC.
    Then to make the PCC $2$-positive-redundant, it would take $n - 2$ reviews, where $n$ is the number of
      annotations in the PCC.
    Determining the best order in which to review edges depending on the specified level of redundancy is an
      interesting question, which is perhaps made more challenging if considered in a distributed setting.

% L___ CHAPTER ___



\cleardoublepage
\phantomsection
\addcontentsline{toc}{chapter}{BIBLIOGRAPHY}

\bibliographystyle{ieee}
\bibliography{My_Library_clean}

\begin{comment}
fixtex --fpaths appendix.tex --outline --asmarkdown --numlines=999 --shortcite -w && ./checklang.py outline_appendix.md
\end{comment}

\appendix    % This command is used only once!
%\addcontentsline{toc}{chapter}{APPENDICES}             %toc entry  or:
\addtocontents{toc}{\parindent0pt\vskip12pt APPENDICES} %toc entry, no page #
%\chapter{Appendix}

\crefalias{section}{appsec}
\crefalias{chapter}{appsec}
\chapter{OCCURRENCES}\label{app:occurgroup}

    In the identification workflow we separate groups of images into \glossterm{occurrences}.
    The Darwin Core defines an occurrence as a collection of evidence that shows an organism exists within
      specific location and span of time~\cite{wieczorek_darwin_2012}.
    For our purposes this amounts to a cluster of images localized in space and time.
    We outline an occurrence grouping algorithm performs agglomerative clustering on the GPS coordinates and time
      specified in the image metadata.
    We first describe the space-time measure of distance between images and then describe the clustering
      algorithm.

    \paragraph{Space-time image distance}
    To compute occurrences we define a space-time feature $\g_i$ for each image $i$, and a pairwise distance,
      $\Delta(\g_i, \g_j)$, between these features.
    This feature will a two-dimensional feature tuple, %
    $\g_i = \paren{\time_i, \gps_i}$, where the first component is the POSIX timestamp $\time_i$, and the second
      component is a GPS coordinate %
    $\gps_i = \brak{\lat_i, \lon_i}^{T}$, where the angles of latitude and longitude are measured in radians.
    To compute this distance between two images $\g_i$ and $\g_j$ we first compute the distance in each component
      of the feature tuple.
    The difference in time is the absolute value of the timedelta,  %
    $\Delta_t(\g_i, \g_j) = \abs{\time_i - \time_j}$, which is in seconds.

    % DISTANCE BETWEEN TWO IMAGES (space and final)
    Next, the distance in space is computed by approximating the Earth as a sphere.
    In general, the distance between two points on a sphere with radius $r$ is a function of inverse haversines,
      and is expressed as:
    \begin{equation}\label{eqn:geodistance}
        d(\gps_i, \gps_j, r) =
        2 r \asin{\sqrt{
            \haversine{\lat_i - \lat_j} +
            \haversine{\lon_i - \lon_j} +
            \cos\paren{\lat_i} \cos\paren{\lat_j}}}
    \end{equation}
    In the previous equation, $\haversine{\theta} = \haversineFULL{\theta}$ is the half vertical sine function.
    Thus, we arrive at the spatial distance between two images by estimating the radius of the earth to be
      $r=6367$ kilometers.
    \begin{equation}
        \Delta_s(\g_i, \g_j) = d(\gps_i, \gps_j, 6367).
    \end{equation}
    This results in distance in seconds and a distance in kilometers, which are in incompatible units.
    To combine these distances we convert kilometers to seconds by heuristically estimating the walking speed,
      $S$, of an animal (for zebras we use $S=2\sciE{-3}$ kilometers per second).
    This allows us to cancel kilometers from the expression and express GPS distance as a unit of time:
    $\frac{\Delta_s(\g_i, \g_j)}{S}$.
    This distance can be interpreted as the total amount of time it would take an animal to move between two
      points.
    The total distance between two images is the sum of these components.
    \begin{equation}\label{eqn:imgdist}
        \Delta(\g_i, \g_j) = \Delta_t(\g_i, \g_j) + \frac{\Delta_s(\gps_i, \gps_j)}{S}
    \end{equation}
    Notice that if there is no difference in GPS location, then this measure
      becomes to a distance in time.

    \paragraph{Clustering procedure}
    Having defined pairwise a distance between two images, we use the agglomerative clustering procedures
      implemented in SciPy~\cite{eric_jones_scipy_2001} to group the images.
    There are two inputs to the agglomerative clustering algorithm:
    (1) The matrix of pairwise distance between images, and
    (2) the minimum distance threshold between two images.
    The matrix of distances is computed using~\cref{eqn:imgdist}, and we set the distance threshold to $30$
      minutes.
    Any pair of images that is within this threshold connected via a linkage matrix.
    Connected components in this matrix form the final clusters that we use as occurrences{}.
    To improve the efficiency of the algorithm, we separate it into disjoint parts by sorting the images by
      timestamp and splitting the data wherever the difference in consecutive timestamps exceeds the threshold.
    Images that are missing either timestamp or GPS location are grouped by what data they do have and clustered
      separately.

    %\paragraph{Discussion of occurrences}
    %These computed occurrences are valuable measurements for multiple components of the IBEIS software.
    %At its core an occurrence describes \wquest{when} a group of animals was seen and \wquest{where} that group
    %  was seen.
    %However, to answer the questions like \wquest{how many} animals there were, \wquest{who} an animal is,
    %  \wquest{who else} is an animal with, and \wquest{where else} have these animals been seen, the \annots{} in
    %  the occurrence must be grouped into individual \encounters{} and then matched against the \masterdatabase{}.


\begin{comment}
\chapter{Converting existing datasets to decision graphs}\label{sec:rename}

In this section we briefly discuss the problem of applying graph identification from~\cref{chap:graphid} to
  existing databases.
Most datasets used in practice ignore detailed connectivity information and simply associate a name label with a
  database of cropped (or sometimes un-cropped) images.
Because graph identification relies on this detailed connectivity, we must reconstruct it before new images can
  be added.

To apply graph identification to a previously existing dataset where annotations have been assigned name labels
  and connectivity between the annotations is unknown use follow the following process.
First we compute the pairwise probabilities between each pair of annotations labeled with the same name.
Then, we automatically classify any edge above a threshold as positive, negative, or incomparable.
For any set of nodes originally labeled with the same name, we compute an edge augmentation to connect the PCC as
  detailed in~\cref{subsec:augredun} and insert these edges into the graph, labeling them as positive but assigning
  them the confidence of guessing.
Note that any edge labeled as negative in the classification step will result in an inconsistency because it will
  be a negative inside a PCC.

It will be common for such datasets to contain errors, we resolve any inconsistent PCCs using the algorithm
  from~\cref{sec:incon}, but then we search for additional split cases using the pairwise classifier.
The main idea is to re-review all edges where the pairwise classifier prediction disagrees with its assigned
  match-state.
Edges are sorted by the magnitude of the disagreement, but any edge with a confidence of absolutely-sure is
  ignored.
This will present edges labeled as guessing for the user to re-review.
At this point the dataset is in a legal state, where the name labels correspond to PCCs.
The final step is to execute normal graph review in order to find any merge cases and explicitly label hard
  negative edges.
In the case that a pairwise classifier does not exist, then one can be trained using the temporary edges defined
  by the maximum spanning trees of the PCCs.

It is sometimes desirable new PCCs to keep the old name labels from the original database (\eg{} sometimes
  ecologists encoded information in these names).
This is a simple matter when the original database contained no mistakes, but when the original database contains
  errors care must be taken.
We address this problem by seeking to minimize the number of annotations that have their name label changed from
  the original dataset.
This can be computing by finding a maximum linear sum assignment using the Munkres algorithm implemented in
  SciPy~\cite{eric_jones_scipy_2001}.
We create a matrix where each row represents a group of annotations in the same PCC and each column represents an
  original name.
If there are more PCCs than original names, then the columns are padded with extra values.
The matrix is first initialized to be negative infinity representing impossible assignments.
Then for each column representing a padded name, we set we its value to $1$ indicating that each new name could
  be assigned to a padded name for some small profit.
Finally, we encode both the profit of assigning a new name with an original name and the extra one ensures that
  these original names are always preferred over padded names.
Let $f_{rc}$ be the number of annotations in row $r$ with an original name of $c$, and set matrix value %
$(r, c)$ to $f_{rc} + 1$ if $f_{rc} > 0$.
The maximum linear sum assignment of this matrix results in the optimal consistent assignment of PCCs to original
  name labels.
\end{comment}


\begin{comment}
\chapter{Sight-resight analysis with incomparability}\label{app:markrecapincomp}

Given a completed decision graph, each PCC corresponds to an individual, and each pair of PCCs is either known to
  be different or known to be incomparable.
If all PCCs are known to be different, then sight-resight analysis is simple.
Each PCC can be grouped into a set of encounters.
The chronologically first encounter is a sighting, and the subsequent encounters are re-sightings.
However, if it cannot be determined that some pairs of PCCs are different, we must use only the first or second
  of these PCCs in our analysis, and the other must be discarded.

Finding the largest set of PCCs that can be used in sight-resight statistics, we must find the largest set of
  PCCs that are all comparable to each other.
This problem can be solved in the following steps:
\begin{enumln}
\item Find all pairs of PCCs that are incomparable.
\item Consider the meta-graph where each of these incomparable PCCs is a node and there is an edge between each
  pair.
\item Find the largest independent set in this meta-graph (note the is NP-hard).
\item Remove all nodes from the decision graph corresponding to the PCCs in the meta-graph that were not in the
  independent set.
\end{enumln}
Now, in the original graph there is no PCC that is incomparable with any other, otherwise there would have been
  two nodes in the independent set that had an edge between them, which is a contradiction.
Even though finding the largest independent set would use the most data, any independent set will do.
Thus, sight-resight statistics can now be performed on this graph.
\end{comment}


%\chapter{Properties of the graph algorithm}\label{app:graphprop}


%\begin{enumln}

%\item Finding the largest set of comparable PCCs is a generalization to the independent set problem.
%Using the meta-graph where PCCs are nodes, and an edge is drawn between any PCCs known to have no comparable
%  pairs of annotations, any independent set of PCCs will all be comparable to one another.
%The largest independent set will be the largest set of comparable PCCs.

%\item The graph identification procedure is agnostic to the underlying computer vision algorithms.
%    Any ranking and verification algorithm can be substituted in.

%\item The graph identification procedure can be used without computer vision algorithms to facilitate a more
%  efficient brute-force search, which can be useful to label small datasets.
%This is done by using all edges as candidate edges, and each edge is given a priority of $0$.
%The refresh criterion is not used, and the algorithm stops when the queue is empty.
%The positive and negative redundancy criteria will remove edges from the queue, so less than $|V|^2$ manual
%  reviews will have to be done.

%\item When $k=2$ and edges are only added to the graph exactly to the specifications of the graph algorithm, it
%  is impossible for a PCC to contain more than one negative edge at any time.

%\item When $k=1$, reviewing edges in order of positive match probability minimizes the expected number of total
%  reviews.

%\item At the end of the graph algorithm every PCC will be $k$-positive-redundant.

%\item In recovery mode, consider that the hypothesis algorithm has generated a set of edges.
%The user has reviewed some of these edges and agreed with each hypothesis so far.
%In this case the remaining edges are necessarily a minimum cut.
%This is because the edges reviewed so far have had their label changed, meaning that they no longer connect
%  terminal nodes.
%In this case, if hypothesis generation is recomputed, then the new hypothesis will be the same as the set of
%  remaining edges if the weights in the graph are unique.
%If the weights in the graph are not unique, then the implementation of min-cut can be chosen to enforce that the
%  new set is the same as the remaining set.

%\item A review can have one of the following mutually exclusive effects on the PCCs of the graph:
%    (1) merge exactly two PCCs into one PCC.
%    (2) split a single PCC into exactly two PCCs.
%    (3) do nothing to the PCCs, but potentially influence redundancy.
%\end{enumln}
  

%\end{appendices}


\end{document}
% chktex-file 17
