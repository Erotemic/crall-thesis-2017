\section{Fine-grained recognition}\label{sec:fgr}  

    Fine-grained recognition is a problem more general than instance
      recognition, but more specific than category
      recognition~\cite{parkhi_cats_2012, berg_poof_2013,
      gavves_local_2014}.
    Given an object of a known category, such as a bird, the goal of
      fine-grained recognition is to sub-classify the object into a
      fine-grained category such as a blackbird or a
      raven~\cite{berg_how_2013}.

    Algorithms for fine-grained recognition typically start by
      localizing the object and its parts with a detection
      algorithm~\cite{dalal_histograms_2005} and parts based models.
    Parts are segmented to remove background noise using algorithms
      like GrabCut~\cite{rother_grabcut_2004}.
    Classification is performed locally on aligned parts as well as
      globally on the entire body and aggregated to yield a final
      classification.

    Because fine-grained recognition lies on the same spectrum as
      instance recognition and category recognition it is not surprising
      that many of the same techniques --- like Fisher Vectors --- are
      used~\cite{gosselin_revisiting_2014}.
    %Also of note in~\cite{gosselin_revisiting_2014}, the $\ell_2$
    %  lengths of unnormalized SIFT descriptors are used to filter
    %  homogeneous regions when descriptors are densely sampled.
    Recently convolutional models have been successfully applied to
      fine-grained recognition~\cite{catherine_wah_similarity_2014,
      branson_bird_2014, zongyuan_ge_modelling_2015, zhang_weakly_2015,
      xiao_application_2015}.

    %  from In \cite{gosselin_revisiting_2014} they
    %  also filter out SIFT descriptors with low $\ell_2$ lengths (before
    %  the $\ell_2$ normalization step).
    %This removes helps homogeneous descriptors from their dense keypoint
    %  sampling.

    \subsection{Discussion --- fine grained recognition}
        %How should can all of this be done accurately?
        %How should image feature be matched?

        The goal of fine-grained recognition is somewhat similar to
          animal identification.
        Fine-grained recognition localizes subtle information to
          distinguish between two similar species, whereas animal
          identification localizes subtle information to distinguish
          between two similar individuals.
        However, the domains of species and individuals are dissimilar
          enough that off the shelf techniques for fine-grained
          recognition would need to be adapted before identification
          could be performed.
        One interesting avenue of research would be to use a parts
          model~\cite{felzenszwalb_object_2010} as
          in~\cite{gavves_local_2014}, to align
          individuals before they are compared.
