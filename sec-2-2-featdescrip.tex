\section{Image feature description}\label{sec:featuredescribe}  

    Once each feature has been localized its visual appearance must be
      described before it can be matched.
    The goal of feature description is to encode raw image data into a
      vector --- \ie{} a \glossterm{descriptor}.
    To represent the visual appearance of a keypoint a descriptor
      vector should have the following properties:
    (1) two visually similar patches produce vectors with a small
      metric distance and
    (2) visually dissimilar patches have vectors with large distances
      between them.

    Constructing such a descriptor vector has been a core problem
      throughout the history of computer vision.
    The first texture descriptor robust to small image transformations
      was the SIFT descriptor first published in
      1999~\cite{lowe_object_1999, lowe_distinctive_2004}.
    Since then, other hand-crafted algorithms have been proposed.
    However, results have always been at least comparable to the SIFT
      descriptor, and SIFT is still an effective and widely used
      hand-designed descriptors~\cite{mikolajczyk_performance_2005,
      calonder_brief_2010, bay_surf_2006, leutenegger_brisk_2011,
      alahi_freak_2012, jegou_triangulation_2014}.
    A promising direction for outperforming the SIFT descriptor is
      descriptor learning~\cite{simonyan_descriptor_2012,
      simonyan_learning_2014, winder_picking_2009}; specifically
      descriptor learning using deep neural
      networks~\cite{razavian_cnn_2014, bengio_representation_2013,
      russakovsky_imagenet_2014}.
    This section first describes the basic SIFT algorithm and then
      provides an overview of alternatives that have been proposed to
      SIFT{}.
    Work related to learning descriptor vectors using deep neural
      networks is discussed later in~\cref{sec:dcnn}.
      
    \subsection{SIFT}
        The \glossterm{SIFT descriptor} is a $128$ dimensional vector
          that summarizes the spatial distribution of the gradient
          orientations in an image patch~\cite{lowe_distinctive_2004}.
        To describe a keypoint with a SIFT descriptor, the keypoint's
          image data is warped using the affine transform of the scale
          space gradient image into a normalized reference frame
          (typically $41 \times 41$ pixels).
        For a descriptor to be useful in matching it is important that
          the keypoint is properly localized before a descriptor is
          computed~\cite{ke_pcasift_2004}.
        Because it is not always possible to perfectly localize a
          keypoint, the SIFT descriptor aggregates information into a
          soft-histogram.
        Allowing data to contribute to multiple bins helps the SIFT
          descriptor to be robust to small localization errors and
          viewpoint variations.
        Distance between two SIFT descriptors is typically computed
          using the Euclidean distance.
        The SIFT descriptor of a patch is visualized
          in~\cref{fig:vizfeatrow}.
        %To warp a keypoint to a patch
        %\[\pt, \vmat, \theta\]
        %\begin{equation}
        %\left[\begin{matrix}1 & 0 & \frac{S}{2}\\0 & 1 & \frac{S}{2}\\0 & 0 & 1\end{matrix}\right]
        %\left[\begin{matrix}3.0 \sqrt{S} & 0 & 0\\0 & 3.0 \sqrt{S} & 0\\0 & 0 & 1\end{matrix}\right]
        %\left[\begin{matrix}\cos{\left (\theta \right )} & \sin{\left (\theta \right )} & 0\\- \sin{\left (\theta \right )} & \cos{\left (\theta \right )} & 0\\0 & 0 & 1\end{matrix}\right]
        %\left[\begin{matrix}a & 0 & x\\c & d & y\\0 & 0 & 1\end{matrix}\right]^{-1}
        %\left[\begin{matrix}1 & 0 & - x\\0 & 1 & - y\\0 & 0 & 1\end{matrix}\right]
        %\end{equation}

        The structure of a SIFT descriptor is as follows:
        A $4\times4$ regular grid is superimposing over the normalized patch.
        Each of the $16$ spatial grid cells contains an orientation histogram
          discretized into $8$ bins.
        The SIFT descriptor is the concatenation of all orientation
          histograms, resulting in a single $16 \times 8 = 128$ dimensional
          vector.

        The patch information populates the SIFT descriptor as follows:
        For every pixel, the patch gradient (the derivative in the $x$
          and $y$ direction) is computed.
        Next, each pixel computes its gradient magnitude and
          orientation.
        Each pixel then casts a weighted vote.
        The bin that a pixel votes into is computed from its
          $xy$-location and gradient orientation.
        The weight of a pixel's vote is based on its gradient magnitude
          and Gaussian weighted distance to the patch center.
        To be robust to small localization errors, a pixel's vote is
          split via trilinear interpolation ($x$-location, $y$-location,
          and orientation) into the orientation histograms of the pixel's
          nearest grid cells as well as neighboring orientation bins in
          each grid cell's orientation histogram.

        Once voting is completed a SIFT descriptor is normalized to
          account for lighting differences between images.
        %The resulting $128$ orientation bins are concatenated into a single
        %  vector.
        First, the vector is L2-normalized to unit length, which makes
          the descriptor invariant to linear changes in intensity.
        Then, a heuristic --- that truncates each dimension to a
          maximum value of $0.2$ --- is applied to increases robustness
          to non-linear changes in illumination.
        Finally the vector is renormalized.

        For storage considerations the resulting $512$-byte floating
          point (float32) descriptor is typically cast as an array of
          unsigned $8$-bit integers (uint8), resulting in a $128$-byte
          descriptor.
        To reduce the impact of this quantization a trick is to
          multiply by $512$ instead of $255$ and then truncate values to
          $255$ before converting from a float to a uint8.
        Even though each component is $8$-bits and therefore can only
          store a maximum value of $255$, but because of truncation,
          L2-normalization, and properties of natural images value
          overflow is not likely to occur.

        %It is important for descriptors to be correctly localized in order to produce confident matches. 
        % interpolation schemes help it to be robust to small localization errors. 

        %Feature descriptors depend on correct localization to be successful.
        %SIFT describes the spatial distribution of intensity patterns 
        %while being robust to small localization errors or changes in intensity.
        %aggregates a large amount of information into a 
    
       \vizfeatrow{}

    \subsection{Other descriptors and SIFT extensions}
        Even a decade after its original publication, SIFT remains a
          popular descriptor for patch-based matching because it is
          versatile, unsupervised, widely available, and easy to use.
          %descriptor for patch-based matching because it is competitive with
          %all of its hand-crafted alternatives, versatile, unsupervised,
          %widely available, and easy to use \cite{lowe_distinctive_2004,
          %mikolajczyk_performance_2005, alahi_freak_2012}.
        The principles used to guide the construction of the SIFT
          descriptor --- particularly the use of aggregated gradients ---
          have inspired many variants, extensions, and new
          techniques~\cite{mikolajczyk_performance_2005,
          dalal_histograms_2005, bay_surf_2006}.
        Hand crafted alternatives to SIFT have been developed that are
          faster to compute and more efficient to store, but these
          alternatives do not significantly outperform SIFT's matching
          accuracy on general data~\cite{lowe_distinctive_2004,
          mikolajczyk_performance_2005, alahi_freak_2012}.
        %However, if training data is provided, machine learning can be used to
        %  compute descriptors that can achieve higher matching accuracy.
        This subsection provides a brief overview of these
          alternatives.

        % ALTERNATIVES FOR DETECTION
        The use of aggregated gradient information in SIFT has been
          adapted for use in other computer vision problems such as
          detection and scene classification.
        % GIST
        The GIST descriptor is a low dimension descriptor used for
          scene classification that coarsely summarizes rough appearance
          of an entire image~\cite{oliva_modeling_2001,
          douze_evaluation_2009}.
        % HOG
        The histogram of oriented gradients (HoG) descriptor is a high
          dimensional descriptor used in detection.
        The HoG descriptors describes the shapes of objects in an
          image~\cite{dalal_histograms_2005}.
        Like the SIFT descriptor, the HoG descriptor illustrates the
          value of gradient-based image descriptions and has inspired
          extensions such as the discriminatively trained parts
          model~\cite{felzenszwalb_object_2010}.

        As a general single-scale patch-based descriptor, the matching
          accuracy of SIFT has not been significantly outperformed on
          general datasets.
        % GLOH
        One attempt at an improved general descriptor is the Gradient
          Location-Orientation Histogram (GLOH)
          descriptor~\cite{mikolajczyk_performance_2005}.
        GLOH uses a similar structure to SIFT but replaces the
          rectangular-bins with log-polar bins.
        GLOH did achieve higher matching accuracy on some datasets, but
          it was not by a significant margin.
        %Despite However, domain specific modifications have shown promising
        %  results.
        %The Rotation Invariant Feature Transform (RIFT) descriptor
        %  \cite{lazebnik_sparse_2005} uses concentric circles to make a
        %  similar modification.
        %The RIFT descriptor are used in texture classification
        %\cite{lazebnik_sparse_2005}.
        Despite the lack of generic success, hand-crafted SIFT variants
          have been successful when applied to specific tasks.
        % COLORED SIFT
        Colored SIFT variants such as opponent-SIFT are valuable in
          category recognition tasks, where a color difference could be
          the distinguishing factor between
          categories~\cite{van_de_sande_evaluating_2010}
        % SCALE-LESS SIFT
        % FIXME: this does not flow well. Just said SIFT was not outperformed
        % THen I say that SLS outperforms it.
        Combining multiple SIFT descriptors over different scales has
          also shown moderate improvements.
        The scale-less SIFT descriptor combines SIFT descriptors
          computed at multiple scales into a single descriptor.
        It has been shown to produce more accurate dense
          correspondences than representing each scale with an individual
          descriptor~\cite{hassner_sifts_2012}.

        % SURF
        Efficiency is one area where SIFT has been significantly outperformed.
        ``Speeded Up Robust Features'' (SURF) is a fast approximation to SIFT
          based on integral images that achieves similar accuracy using a
          smaller $64$ dimensional descriptor~\cite{bay_surf_2006}.
        % DAISY
        The DAISY descriptor uses a similar binning structure to GLOH, but
          uses convolutions with Gaussian kernels to quickly aggregate gradient
          histograms~\cite{tola_fast_2008}.
        % BINARY PATTERNS
        Binary descriptors such as local binary patterns
          (LBP)~\cite{ojala_comparative_1996, zhang_local_2010}, local
          derivative patterns~\cite{heikkila_description_2009}, and their
          variants such as BRIEF~\cite{calonder_brief_2010},
          BRISK~\cite{leutenegger_brisk_2011}, and FREAK~\cite{alahi_freak_2012}
          also quickly compute compact distinctive descriptors.
        Binary descriptors are built using multiple pairwise comparisons of
          average image intensity at predetermined locations.
        This results in a small descriptor that effectively represents
          aggregated gradient information.

        Machine learning is able to outperform the matching accuracy of
          SIFT, however these techniques require training data to adapt
          to each new problem domain.
        Learned descriptors make use of the same aggregated gradient
          information used in the construction of SIFT descriptors.
        The Liberty, Yosemite, and Notre-Dame buildings datasets are
          standard datasets for descriptor
          learning~\cite{brown_discriminative_2011}.
        Error on these datasets is measured using false positive rate
          at $95\percent$ recall (FPR95).
        The baseline SIFT error on this dataset is $27.02\percent$.
        The configuration of a DAISY descriptor is learned
          in~\cite{winder_picking_2009} and achieves an error of
          $15.16\percent$ on the buildings datasets.
        In~\cite{simonyan_learning_2014}, large scale non-convex
          optimization is used to learn a spatial pooling configuration
          of log-polar bins, a dimensionality reduction matrix, and a
          distance metric to further reduce the FPR95 error to
          $10.98\percent$.
        The current state of the art error of $4.56\percent$ on the
          buildings dataset is achieved using a convolutional neural
          network~\cite{zagoruyko_learning_2015}.

    \subsection{Discussion --- descriptor choices}
        %Address these questions:
        %How should detected image features be represented?
        %In the previous two chapters we discussed various techniques for
        %  detecting and describing patch-based image features.
        %This work guides the answers to the questions presented at the
        %  beginning of this chapter.
        %Clearly a middle ground is needed.
        %We experiment on features with different degrees of invariance.
        %It is generally desirable for keypoint detection and description
        %  algorithms to be invariant to small changes the visual appearance of
        %  an image patch, but this must be balanced with a keypoints ability
        %  to be descriptive and distinguishing.
        %Assuming the gravity vector in keypoint detection removes a
        %  level of invariance that would otherwise cause a horizontal line
        %  to appear the same as a vertical line.
        %\paragraph{How should detected image features be represented?}
        In our application we use the scale invariant feature transform
          (SIFT)~\cite{lowe_distinctive_2004} as our baseline descriptor
          because it is one of the most widely used and well known
          descriptors.
        SIFT describes images patches in such a way that small
          localization errors do not significantly impact the resulting
          representation.
        Exploration of alternative convolutional descriptors is
          discussed later in~\cref{sec:dcnn}.
        %We have explored alternatives to SIFT that learned using deep
        %  convolutional neural networks.
        %We will also explore alternatives to SIFT that learned using deep
        %  convolutional neural networks.
        %Because we value accuracy over speed in our application, we learn 
        % all other descriptors used.
        %Learning descriptors with convolutional networks is further
        %discussed
        %in~\cref{sec:deepdesc}.
