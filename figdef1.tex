%Names: IBEIS-PZ-1348 viewpoint issue different and viewing conditions

%IBEIS-PZ-1421 Pose issues where the zebras are fighting

%\newcommand{\captionlabel}[2]{\caption{#1 - #2}\label{#2}}

\begin{comment}
    python -m ibeis.scripts.gen_cand_expts --exec-parse_latex_comments_for_commmands --fname figdef1.tex

    sudo apt-get install pngquant
    pngquant --quality=0-70 figures1/*.png
    pngquant --quality=0-70 figures2/*.png
    pngquant --quality=0-70 figures3/*.png
    pngquant --quality=0-70 figures4/*.png
    pngquant --quality=0-70 figures5/*.png
    pngquant --quality=0-70 *.png
    pngquant --quality=0-70 *.png
\end{comment}


\begin{comment}
python -m ibeis.viz.viz_name --test-show_multiple_chips --db NNP_Master3 --aids=6416,7458,13339,10170 --no-inimage --notitle --adjust=.05 --rc=1,4 --dpath ~/latex/cand/ --save "figures1/PlainsFigure.jpg" --figsize=9,4 --clipwhite --dpi=180 --diskshow --numlbl

python -m ibeis.viz.viz_name --test-show_multiple_chips --db NNP_Master3 --aids=6416,7458,13339,10170 --no-inimage --notitle --adjust=.05 --rc=1,4 --dpath ~/latex/cand/ --save "figures1/PlainsFigure.jpg" --figsize=9,4 --clipwhite --dpi=180 --diskshow --saveparts
\end{comment}
%\SingleImageCommand{PlainsFigure}{.9}{
\MultiImageCommandII{PlainsFigure}{.22}{A plains zebras}{
% ---
For plains zebras, most distinguishing features tend to be located on
  the upper shoulder.
Other distinguishing features are typically be found on the face and
  side.
Image pairs~\cref{sub:PlainsFigureA,sub:PlainsFigureB}
  and~\cref{sub:PlainsFigureC,sub:PlainsFigureD} depict the same
  individual.
% ---
}{figures1/PlainsFigureA.jpg}{figures1/PlainsFigureB.jpg}{figures1/PlainsFigureC.jpg}{figures1/PlainsFigureD.jpg}
%}{figures1/PlainsFigure.jpg}


\begin{comment}
python -m ibeis.viz.viz_name --test-show_multiple_chips --db NNP_Master3 --aids=13285,12598,6563,9332 --no-inimage --notitle --adjust=.05 --rc=1,4 --dpath ~/latex/cand/ --save "figures1/HardCaseFigure.jpg" --figsize=9,4 --clipwhite --dpi=180 --diskshow --saveparts
\end{comment}
\MultiImageCommandII{HardCaseFigure}{.22}{
Visually similar plains zebras}{
% ---
Plains zebras with visual similarities can be difficult to distinguish.
There are three individuals in these four images.
The images in~\cref{sub:HardCaseFigureB,sub:HardCaseFigureD} depict the
  same individual.
Dissimilarities can be seen on the lower thigh of
  images~\cref{sub:HardCaseFigureC,sub:HardCaseFigureD}, as well as on
  the front shoulder of
  images~\cref{sub:HardCaseFigureA,sub:HardCaseFigureB}.
% ---
}{figures1/HardCaseFigureA.jpg}{figures1/HardCaseFigureB.jpg}{figures1/HardCaseFigureC.jpg}{figures1/HardCaseFigureD.jpg}


\begin{comment}
python -m ibeis.viz.viz_name --exec-show_multiple_chips --aids=163,253,286,449 --db=NNP_MasterGIRM_core --dpath ~/latex/cand/ --save "figures1/GirMasaiFigure.jpg" --figsize=9,4 --clipwhite --dpi=180 --no-inimage --adjust=.05 --no-figtitle --notitle --diskshow --rc=1,4 --numlbl
%ib 
%./main.py --db NNP_MasterGIRM_core --query 164 --daids-mode=all -y
\end{comment}
\MultiImageCommandII{GirMasaiFigure}{.22}{
A Masai giraffe}{
% ---
Masai giraffes have an abundance of features distinctive to each
  individual.
There are two individuals seen in images
  pairs~\cref{sub:GirMasaiFigureA,sub:GirMasaiFigureB}
  and~\cref{sub:GirMasaiFigureC,sub:GirMasaiFigureD}.
Note that the numerous features make it initially difficult for a human
  to match giraffes.
In contrast, this is easy for algorithms.
% ---
}{figures1/GirMasaiFigureA.jpg}{figures1/GirMasaiFigureB.jpg}{figures1/GirMasaiFigureC.jpg}{figures1/GirMasaiFigureD.jpg}


\begin{comment}
python -m ibeis.viz.viz_name --exec-show_multiple_chips --aids=923,1013,823,960 --db=GZ_ALL --dpath ~/latex/cand/ --save "figures1/GrevysFigure.jpg" --figsize=9,4 --clipwhite --dpi=180 --no-inimage --adjust=.05,.05,.15 --no-figtitle --notitle --diskshow --rc=2,2 --numlbl
%ib 
./main.py --db GZ_ALL --query 923 --daids-mode=all -y
\end{comment}
%\SingleImageCommand{GrevysFigure}{.9}{A Grévy's zebra}{
\MultiImageCommandII{GrevysFigure}{.4}{A Grévy's zebra}{
% ---
Grévy's zebras most distinctive features are above the front and rear
  legs.
Useful, but less distinctive information can be seen on the side of the
  body.
Image pairs~\cref{sub:GrevysFigureA,sub:GrevysFigureB}
  and~\cref{sub:GrevysFigureC,sub:GrevysFigureD} depict the same
  individual.
% ---
}{figures1/GrevysFigureA.jpg}{figures1/GrevysFigureB.jpg}{figures1/GrevysFigureC.jpg}{figures1/GrevysFigureD.jpg}
%}{figures1/GrevysFigure.jpg}


\begin{comment}
python -m ibeis.viz.viz_name show_multiple_chips --aids=7,2,8 --db=humpbacks_fb --dpath ~/latex/crall-thesis-2017/ --save "figures1/HumpbackFig.jpg" --figsize=9,4 --clipwhite --dpi=300 --no-inimage --adjust=.05,.05,.15 --no-figtitle --notitle --diskshow --saveparts 
\end{comment}
\MultiImageCommandII{HumpbackFig}{.31}{HumpbackFig}{
    %
    A humpback whale can be identified by the texture patterns on the fluke or using the shape of the notches
      along the edge of the fluke.
    Note that some humpbacks (like the on seen in \ref{sub:HumpbackFigB}) do not have any texture patterns on
      their tail.
    Subfigures \ref{sub:HumpbackFigA} and \ref{sub:HumpbackFigC} are the same individual.
    %
}{figures1/HumpbackFigA.jpg}{figures1/HumpbackFigB.jpg}{figures1/HumpbackFigC.jpg}




\begin{comment}
python -m ibeis.viz.viz_name --test-show_multiple_chips --dpath ~/latex/cand --save 'figures1/OccludeFigure.jpg' --no-figtitle --notitle --db NNP_Master3 --figsize=9,4 --dpi=180 --adjust=.05,.05,.05,.15 --clipwhite --no-inimage --aids=13870,13740,7735,13233,13603,13906,7354,9776 --rc=2,4 --append temp_out_figure.tex --diskshow --saveparts
\end{comment}
\newcommand{\OccludeFigure}{
\begin{figure}[ht!]
\centering
\begin{subfigure}[h]{0.2\textwidth}
\centering
\includegraphics[width=\textwidth]{figures1/OccludeFigureA.jpg}\caption{}\label{sub:OccludeFigureA}
\end{subfigure}
~~% --
\begin{subfigure}[h]{0.2\textwidth}
\centering
\includegraphics[width=\textwidth]{figures1/OccludeFigureB.jpg}\caption{}\label{sub:OccludeFigureB}
\end{subfigure}
~~% --
\begin{subfigure}[h]{0.2\textwidth}
\centering
\includegraphics[width=\textwidth]{figures1/OccludeFigureC.jpg}\caption{}\label{sub:OccludeFigureC}
\end{subfigure}
~~% --
\begin{subfigure}[h]{0.2\textwidth}
\centering
\includegraphics[width=\textwidth]{figures1/OccludeFigureD.jpg}\caption{}\label{sub:OccludeFigureD}
\end{subfigure}
~~% --
\begin{subfigure}[h]{0.2\textwidth}
\centering
\includegraphics[width=\textwidth]{figures1/OccludeFigureE.jpg}\caption{}\label{sub:OccludeFigureE}
\end{subfigure}
~~% --
\begin{subfigure}[h]{0.2\textwidth}
\centering
\includegraphics[width=\textwidth]{figures1/OccludeFigureF.jpg}\caption{}\label{sub:OccludeFigureF}
\end{subfigure}
~~% --
\begin{subfigure}[h]{0.2\textwidth}
\centering
\includegraphics[width=\textwidth]{figures1/OccludeFigureG.jpg}\caption{}\label{sub:OccludeFigureG}
\end{subfigure}
~~% --
\begin{subfigure}[h]{0.2\textwidth}
\centering
\includegraphics[width=\textwidth]{figures1/OccludeFigureH.jpg}\caption{}\label{sub:OccludeFigureH}
\end{subfigure}
\caption[\caplbl{OccludeFigure}Examples of occlusion and distractors]{\caplbl{OccludeFigure}
% ---
Animals under varying degrees of occlusion from scenery and other animals.
Occlusions can obfuscate or remove distinctive feature entirely.
Other animals can introduce distinctive features that should not be there.
% ---
}
\label{fig:OccludeFigure}
\end{figure}
}



\begin{comment}
python -m ibeis.viz.viz_name --test-show_multiple_chips --db NNP_Master3 --aids=6524,6540,6571,6751 --no-inimage --notitle --adjust=.05 --dpath ~/latex/cand/ --save "figures1/BacksFigure.jpg" --figsize=9,4 --clipwhite --dpi=180 --rc=1,4 --diskshow --saveparts
\end{comment}
\MultiImageCommandII{BacksFigure}{.22}{Back viewpoints of plains zebras}{
% ---
The backs of plains zebras have very little distinguishing information.
All of the above images are different individuals.
% ---
}{figures1/BacksFigureA.jpg}{figures1/BacksFigureB.jpg}{figures1/BacksFigureC.jpg}{figures1/BacksFigureD.jpg}



\begin{comment}
python -m ibeis.viz.viz_name --test-show_multiple_chips --db PZ_Master0 --aids=5878,5885,5886,5888,5890,5904 --no-inimage --notitle --adjust=.05,.05,.05,.15 --dpath ~/latex/cand/ --save "figures1/ThreeSixtyFigure.jpg" --figsize=9,4 --clipwhite --dpi=180 --diskshow --saveparts
./dev.py --db PZ_Master0 --eval="','.join(list(map(str, ibs.search_annot_notes('360'))))"
\end{comment}
\newcommand{\ThreeSixtyFigure}{
\begin{figure}[ht!]
\centering
\begin{subfigure}[h]{0.44\textwidth}
\centering
%\includegraphics[width=\textwidth]{figures1/ThreeSixtyFigureA.jpg}\caption{}\label{sub:ThreeSixtyFigureA}
\includegraphics[height=120pt]{figures1/ThreeSixtyFigureA.jpg}\caption{}\label{sub:ThreeSixtyFigureA}
\end{subfigure}
~~% --
\begin{subfigure}[h]{0.24\textwidth}
\centering
\includegraphics[height=120pt]{figures1/ThreeSixtyFigureB.jpg}\caption{}\label{sub:ThreeSixtyFigureB}
\end{subfigure}
~~% --
\begin{subfigure}[h]{0.18\textwidth}
\centering
\includegraphics[height=120pt]{figures1/ThreeSixtyFigureC.jpg}\caption{}\label{sub:ThreeSixtyFigureC}
\end{subfigure}
~~% --
\begin{subfigure}[h]{0.26\textwidth}
\centering
\includegraphics[height=110pt]{figures1/ThreeSixtyFigureD.jpg}\caption{}\label{sub:ThreeSixtyFigureD}
\end{subfigure}
~~% --
\begin{subfigure}[h]{0.33\textwidth}
\centering
\includegraphics[height=110pt]{figures1/ThreeSixtyFigureE.jpg}\caption{}\label{sub:ThreeSixtyFigureE}
\end{subfigure}
~~% --
\begin{subfigure}[h]{0.26\textwidth}
\centering
\includegraphics[height=110pt]{figures1/ThreeSixtyFigureF.jpg}\caption{}\label{sub:ThreeSixtyFigureF}
\end{subfigure}
\caption[\caplbl{ThreeSixtyFigure}Viewpoint variations]{\caplbl{ThreeSixtyFigure}
% ---
Viewpoint variations of a individual Grévy's zebra.
It would not be possible to match
  image~\cref{sub:ThreeSixtyFigureA,sub:ThreeSixtyFigureF} without
  information from images showing intermediate views.
% ---
}
\label{fig:ThreeSixtyFigure}
\end{figure}
}


\begin{comment}
python -m ibeis.viz.viz_name --test-show_multiple_chips --aids=11081,13057,15897,15249,8081,13758 --db NNP_Master3 --dpath ~/latex/cand --save figures1/PoseFigure.jpg  --figsize=9,4 --clipwhite --dpi=180 --adjust=.05,.05,.05,.15 --no-figtitle --notitle --diskshow --no-draw_lbls --zoom=.5 --saveparts
\end{comment}
\newcommand{\PoseFigure}{
\begin{figure}[ht!]
\centering
\begin{subfigure}[h]{0.3\textwidth}
\centering
\includegraphics[width=\textwidth]{figures1/PoseFigureA.jpg}\caption{}\label{sub:PoseFigureA}
\end{subfigure}
~~% --
\begin{subfigure}[h]{0.3\textwidth}
\centering
\includegraphics[width=\textwidth]{figures1/PoseFigureB.jpg}\caption{}\label{sub:PoseFigureB}
\end{subfigure}
~~% --
\begin{subfigure}[h]{0.3\textwidth}
\centering
\includegraphics[width=\textwidth]{figures1/PoseFigureC.jpg}\caption{}\label{sub:PoseFigureC}
\end{subfigure}
~~% --
\begin{subfigure}[h]{0.3\textwidth}
\centering
\includegraphics[width=\textwidth]{figures1/PoseFigureD.jpg}\caption{}\label{sub:PoseFigureD}
\end{subfigure}
~~% --
\begin{subfigure}[h]{0.3\textwidth}
\centering
\includegraphics[width=\textwidth]{figures1/PoseFigureE.jpg}\caption{}\label{sub:PoseFigureE}
\end{subfigure}
~~% --
\begin{subfigure}[h]{0.3\textwidth}
\centering
\includegraphics[width=\textwidth]{figures1/PoseFigureF.jpg}\caption{}\label{sub:PoseFigureF}
\end{subfigure}
\caption[\caplbl{PoseFigure}Challenging pose variations]{\caplbl{PoseFigure}
% ---
Animals can appear in a wide variety of poses.
To clearly see the different poses, the images are shown with
  surrounding context.
When running identification the image is cropped to the bounding box
  shown around each animal.
%When matching an animal, the
% ---
}
\label{fig:PoseFigure}
\end{figure}
}


\begin{comment}
python -m ibeis.viz.viz_name --test-show_multiple_chips --dpath ~/latex/cand --save figures1/IlluminationFigure.jpg --no-figtitle --notitle --db NNP_Master3 --figsize=9,3 --no-inimage --append temp_out_figure.tex --aids=6466,10161,10634,9458,12472,14728 --clipwhite --diskshow  --saveparts --adjust=.05,.10,.05,.23
,10647 
\end{comment}
\newcommand{\IlluminationFigure}{
\begin{figure}[ht!]
\centering
\begin{subfigure}[h]{0.3\textwidth}
\centering
\includegraphics[width=\textwidth]{figures1/IlluminationFigureA.jpg}\caption{}\label{sub:IlluminationFigureA}
\end{subfigure}
~~% --
\begin{subfigure}[h]{0.3\textwidth}
\centering
\includegraphics[width=\textwidth]{figures1/IlluminationFigureB.jpg}\caption{}\label{sub:IlluminationFigureB}
\end{subfigure}
~~% --
\begin{subfigure}[h]{0.3\textwidth}
\centering
\includegraphics[width=\textwidth]{figures1/IlluminationFigureC.jpg}\caption{}\label{sub:IlluminationFigureC}
\end{subfigure}
~~% --
\begin{subfigure}[h]{0.3\textwidth}
\centering
\includegraphics[width=\textwidth]{figures1/IlluminationFigureD.jpg}\caption{}\label{sub:IlluminationFigureD}
\end{subfigure}
~~% --
\begin{subfigure}[h]{0.3\textwidth}
\centering
\includegraphics[width=\textwidth]{figures1/IlluminationFigureE.jpg}\caption{}\label{sub:IlluminationFigureE}
\end{subfigure}
~~% --
\begin{subfigure}[h]{0.3\textwidth}
\centering
\includegraphics[width=\textwidth]{figures1/IlluminationFigureF.jpg}\caption{}\label{sub:IlluminationFigureF}
\end{subfigure}
\caption[\caplbl{IlluminationFigure}Challenging outdoor lighting conditions]{\caplbl{IlluminationFigure}
% ---
The effects of outdoor and illumination.
Shadow and illumination can cause variations in the underlying image intensity
  and gradients.
This can makes it more difficult to localize repeatable keypoints and describe
  the underlying texture patterns.
% ---
}
\label{fig:IlluminationFigure}
\end{figure}
}



\begin{comment}
python -m ibeis.viz.viz_name --test-show_multiple_chips --db NNP_Master3 --dpath ~/latex/cand --save 'figures1/QualityFigure.jpg' --no-figtitle --notitle  --figsize=9,4 --dpi=220 --adjust=.05  --rc=2,5  --no-draw_lbls --doboth --diskshow  --trydrawline --qualtitle --aids=6416,8227,6262,10705,15417 --saveparts --grouprows

\end{comment}
\newcommand{\QualityFigure}{
\begin{figure}[ht!]
\centering
\begin{subfigure}[h]{0.18\textwidth}
\centering
\includegraphics[width=\textwidth]{figures1/QualityFigureA.jpg}\caption{}\label{sub:QualityFigureA}
\end{subfigure}
~~% --
\begin{subfigure}[h]{0.18\textwidth}
\centering
\includegraphics[width=\textwidth]{figures1/QualityFigureB.jpg}\caption{}\label{sub:QualityFigureB}
\end{subfigure}
~~% --
\begin{subfigure}[h]{0.18\textwidth}
\centering
\includegraphics[width=\textwidth]{figures1/QualityFigureC.jpg}\caption{}\label{sub:QualityFigureC}
\end{subfigure}
~~% --
\begin{subfigure}[h]{0.18\textwidth}
\centering
\includegraphics[width=\textwidth]{figures1/QualityFigureD.jpg}\caption{}\label{sub:QualityFigureD}
\end{subfigure}
~~% --
\begin{subfigure}[h]{0.18\textwidth}
\centering
\includegraphics[width=\textwidth]{figures1/QualityFigureE.jpg}\caption{}\label{sub:QualityFigureE}
\end{subfigure}
\caption[\caplbl{QualityFigure}Challenging image qualities]{\caplbl{QualityFigure}
% ---
An individual seen in images of different qualities.
The bottom row shows the cropped images that correspond to the ROIs in the top
  row.
Each column shows different qualities:
\Cref{sub:QualityFigureA} an excellent quality image taken from a short distance,
\Cref{sub:QualityFigureB} a good quality image with minor shadow and taken from a medium distance,
\Cref{sub:QualityFigureC} an ok quality image due to minor occlusion,
\Cref{sub:QualityFigureD} a poor quality image due to major occlusion,
\Cref{sub:QualityFigureE} a junk quality image due to considerable blur.
% ---
}
\label{fig:QualityFigure}
\end{figure}
}


\begin{comment}
python -m ibeis.viz.viz_name --test-show_name --name=08_106 --db PZ_MTEST --save "figures1/Age.jpg" --dpath ~/latex/cand/ --figsize=9,4 --clipwhite --dpi=180 --adjust=.05 --no-figtitle --notitle --diskshow --no-draw_lbls --no-inimage --saveparts
\end{comment}
%\SingleImageCommand{AgeFigure}{.9}{Visual differences caused by age}{
\MultiImageCommandII{AgeFigure}{.4}{Visual differences caused by age}{
% ---
The left and right images show the adult and juvenile appearance of the
  same individual.
As an animal ages its appearance changes mainly in color and texture
  with some minor shape and scale differences.
% ---
}{figures1/AgeA.jpg}{figures1/AgeB.jpg}
%}{figures1/Age.jpg}


\begin{comment}
python -m ibeis.viz.viz_name --test-show_multiple_chips --db PZ_Master0 --aids=4020,4839 --no-inimage --notitle --adjust=.05 --dpath ~/latex/cand/ --save "figures1/GashFigure.jpg" --figsize=9,4 --clipwhite --dpi=180 --diskshow --saveparts
\end{comment}
\MultiImageCommandII{GashFigure}{.4}{Visual differences caused by injuries}{
% ---
Injuries can obscure features on an animal as well as creating new ones.
The left image shows a wounded animal, and the right image shows an animal
  with a distinguishing scaring pattern.
% ---
}{figures1/GashFigureA.jpg}{figures1/GashFigureB.jpg}
%}{figures1/gash.jpg}


\begin{comment}
python -m ibeis.viz.viz_image --test-show_multi_images --db NNP_Master3 --gids=7409,7448,4670,7497,7496,7464 --adjust=.05 --dpath ~/latex/cand/ --save "figures1/DetectFigure.jpg" --figsize=9,4 --clipwhite --dpi=180 --diskshow --saveparts
7409,7448,4670,7497,7496,7464,7446,7442 
\end{comment}
%\SingleImageCommand{DetectFigure}{.9}{Detection of plains zebras}{
\newcommand{\DetectFigure}{
\begin{figure}[ht!]
\centering
\begin{subfigure}[h]{0.3\textwidth}
\centering
\includegraphics[width=\textwidth]{figures1/DetectFigureA.jpg}\caption{}\label{sub:DetectFigureA}
\end{subfigure}
~~% --
\begin{subfigure}[h]{0.3\textwidth}
\centering
\includegraphics[width=\textwidth]{figures1/DetectFigureB.jpg}\caption{}\label{sub:DetectFigureB}
\end{subfigure}
~~% --
\begin{subfigure}[h]{0.3\textwidth}
\centering
\includegraphics[width=\textwidth]{figures1/DetectFigureC.jpg}\caption{}\label{sub:DetectFigureC}
\end{subfigure}
~~% --
\begin{subfigure}[h]{0.3\textwidth}
\centering
\includegraphics[width=\textwidth]{figures1/DetectFigureD.jpg}\caption{}\label{sub:DetectFigureD}
\end{subfigure}
~~% --
\begin{subfigure}[h]{0.3\textwidth}
\centering
\includegraphics[width=\textwidth]{figures1/DetectFigureE.jpg}\caption{}\label{sub:DetectFigureE}
\end{subfigure}
~~% --
\begin{subfigure}[h]{0.3\textwidth}
\centering
\includegraphics[width=\textwidth]{figures1/DetectFigureF.jpg}\caption{}\label{sub:DetectFigureF}
\end{subfigure}
\caption[\caplbl{DetectFigure}Detection of plains zebras]{\caplbl{DetectFigure}
% ---
Images from the \GZC{} with detections of plains zebras.
Detections were automatically suggested and manually verified before
  being accepted.
% ---
}
\label{fig:DetectFigure}
\end{figure}
}

\begin{comment}
python -m ibeis.viz.viz_name --test-show_multiple_chips --dpath ~/latex/cand --save figures1/OccurrenceComplementFigure.jpg --no-figtitle --notitle --db NNP_Master3 --figsize=9,4 --clipwhite --dpi=180 --adjust=.03 --no-inimage --aids=15288,15333,15797 --diskshow --saveparts
\end{comment}
%\SingleImageCommand{OccurrenceComplementFigure}{.9}{
\MultiImageCommandII{OccurrenceComplementFigure}{.3}{Multiple views in an \occurrence{}
}{
% ---
Images taken within an \occurrence{} that demonstrate redundant and
  complementary features.
Features on the shoulders are somewhat redundant in
  images~\cref{sub:OccurrenceComplementFigureA,sub:OccurrenceComplementFigureB,sub:OccurrenceComplementFigureC}
  because they are all under approximately constant illumination and are
  seen from the same angle.
Images~\cref{sub:OccurrenceComplementFigureA,sub:OccurrenceComplementFigureC}
  have complementary features because the viewpoint of the animal has
  shifted slightly.
% ---
}{figures1/OccurrenceComplementFigureA.jpg}{figures1/OccurrenceComplementFigureB.jpg}{figures1/OccurrenceComplementFigureC.jpg}
%}{figures1/OccurrenceComplementFigure.jpg}


\begin{comment}
python -m ibeis.viz.viz_qres --test-show_qres --qaids=45 --top-aids=12 --db=PZ_MTEST --sidebyside --annot_mode=0 --notitle --no-viz_name_score --max_nCols=4 --adjust=.02 --figsize=9,4 --dpi=120 --clipwhite --dpath=~/latex/cand --save=figures1/RankFigure.jpg --diskshow
\end{comment}
\SingleImageCommand{RankFigure}{.9}{Examples of top ranked matches}{
% ---
A ranked list image pairs.
Each pair is a one-vs-one comparison.
All of the left images are the same.
The right image in each pair shows a candidate match.
The correct match is in the top-left.
% ---
}{figures1/RankFigure.jpg}
